\chapterimage{chapter_head_2.pdf}
\chapter{傅里叶级数}

\section{傅里叶级数求解}\index{傅里叶级数求解}



\subsection{傅里叶级数的一般求解形式}\index{傅里叶级数求解!傅里叶级数的一般求解形式}

1、简单三角信号:转换成复指数的形式和求解基波频率,直接求解傅里叶系数(三角信号的系数个数有限)

2、一般时域信号:根据分析公式求解复指数形式的傅里叶系数(也称频谱系数)

3、级数有限信号:可以将未知数$a_k$当作待求解的线性方程,求解$a_k$

4、注意傅里叶级数的性质的应用

5、利用傅里叶变换与傅里叶级数的关系求解傅里叶级数

注意:连续傅里叶级数对于0点的求值需要进行单独求(因为积分过程中出现了除$k$运算,而分母不能为0)

注意:对于离散事件信号求解出来的傅里叶级数系数标注循环特性(离散时间信号的项数是有限的)



\subsection{针对指定周期求傅里叶级数}\index{傅里叶级数求解!针对指定周期求傅里叶级数}

1、将信号表示成指数的形式

2、将周期固定(先把$\frac{2\pi}{T}$拎出来),对$k$进行配平(即选择$k$值使得与表示成的指数形式一致)



\subsection{连续信号傅里叶级数表示形式}\index{傅里叶级数求解!连续信号傅里叶级数表示形式}

1、指数形式:$x(t)=\sum_{k=-\infty}^{\infty}a_k e^{jk\omega_0t}$

2、三角形式:$x(t)=a_0 + 2\sum_{k=1}^{\infty}A_k cos(k\omega_0t+\theta_k)$(变换:$a_k=A_ke^{j\theta_k}$,就是将$a_k$用极坐标形式表出,注意极坐标的径是大于零的,所以$A_k=|a_k|$(取模,不是绝对值))(前提是实周期信号,有共轭对称性质)

3、笛卡尔形式:$x(t)=a_0 + 2\sum_{k=1}^{\infty}[B_k cos(k\omega_0t)-C_k sin(k\omega_0t)]$(变换:$a_k=B_k+jC_k$)(前提是实周期信号,有共轭对称性质)



\subsection{傅里叶级数逆变换}\index{傅里叶级数求解!傅里叶级数逆变换}

1、有限项的傅里叶级数:使用傅里叶级数指数形式展开

2、常见的时域信号的傅里叶级数表示(用于逆推):矩形信号($\frac{sin(k*)}{k*}$类似的形式)、常数信号(只有$a_0$项)、0-1奇偶交替离散信号(利用冲激(条线)表示傅里叶级数系数,值非相反对称形式利用$a_0$值进行配平)、1和-1奇偶交替离散信号$(-1)^n = e^{j\pi n}$(利用冲激(条线)表示傅里叶级数系数)

3、拆分和组合:傅里叶级数是多种常见的时域信号的傅里叶级数表示的组合的情况,利用组合获取到能够匹配目标傅里叶级数的信号(线性组合,LTI)

\section{傅里叶级数信号操作}\index{傅里叶级数信号操作}



\subsection{信号通过滤波器:(滤波器的系统频率响应$H(jw)$)}\index{傅里叶级数信号操作!信号通过滤波器:(滤波器的系统频率响应$H(jw)$)}

1、系数下标为$k$的项的频率为$kw_0$

2、通过系统时,如果系统频率响应在$kw_0$位置值为0,那么此频率就被过滤掉了

备注:对于复杂的信号,可以先将信号分解成卷积形式,可以分开送入到系统中(因为信号通过系统也是卷积运算,卷积运算具有可交换性)

注意:对于系统频率响应,可以将$H(jw)$表示为$H(jkw_0)$的形式求解(自变量由w换成了k,连续转为了离散)



\subsection{变换后的信号的傅里叶级数系数和基波频率}\index{傅里叶级数信号操作!变换后的信号的傅里叶级数系数和基波频率}

1、傅里叶系数:根据傅里叶级数的性质求解变换后的级数表示

2、基波频率:线性组合的信号基波频率不变



