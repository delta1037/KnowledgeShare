\chapterimage{chapter_head_2.pdf}
\chapter{信号系统}

\section{时域信号绘图}\index{时域信号绘图}



\subsection{绘制连续信号时域波形}\index{时域信号绘图!绘制连续信号时域波形}

1、横坐标的坐标标识为$t$,纵坐标标识看题目

2、冲激大小用括号加数字表示(表示冲激的强度),负冲激括号里也是正值(因为冲激强度只能为正)

注意:在对含有冲激的图像进行尺度变换时,冲激的大小也会跟着变化

注意:图像在间断点处也有连线(整个图像与$x$轴之间没有间断的地方)

备注:用闭式表达式表示图像时,自变量为$t$,有一次写成了$x$(具体自变量是什么参考题目)



\subsection{绘制离散信号时域波形}\index{时域信号绘图!绘制离散信号时域波形}

1、横坐标的坐标标识为$n$

2、纵坐标轴有没有都可以(如果题目有图形,可以参考题目是怎么画的)

3、端点用圆点表示,非等高的地方标注纵坐标的大小



\subsection{绘制时域波形的奇部和偶部}\index{时域信号绘图!绘制时域波形的奇部和偶部}

1、偶信号:y轴对称的点取平均值放右边,0处不变

2、奇信号:y轴对称的点右边减左边,取平均值,放右边;0处为0

3、对称:将求解后的右边的值按照就行对称到左边

注意:计算端点值,进行验证



\subsection{图形的变换与组合}\index{时域信号绘图!图形的变换与组合}

1、图形变换方式:时移,伸缩,反转(正向变换:(化简成$\alpha(x + \beta)$的形式)先伸缩和反转$\alpha$,再时移$\beta$;逆向变换:先做变量替换,再做正向变换)

2、图形组合方式:加减,求导(注意:连续信号求导时可能会产生冲激信号,看信号与$x$轴之间或者信号本身是否有间断的点)

注意:对特殊点(边界点,信号转折点)进行代入验证



注意波形图的要点:横坐标标注、端点值标注,端点标注

注意:复序列可以将实部和虚部分别表示成两个序列

\section{频域信号绘图}\index{频域信号绘图}



\subsection{连续周期信号的幅度频谱与相位频谱:(条线形式)(双边谱和单边谱)(对比:周期信号傅里叶变换频域是脉冲形式)}\index{频域信号绘图!连续周期信号的幅度频谱与相位频谱:(条线形式)(双边谱和单边谱)(对比:周期信号傅里叶变换频域是脉冲形式)}

1、将周期函数化简为指数信号和的形式$f(t)=\sum_{n=-\infty}^{\infty} F_{n} \mathrm{e}^{\mathrm{j} n \Omega t}$(傅里叶级数的形式)

2、频谱的横坐标为$n \Omega$,纵坐标为$|F_n|$,$\varphi_{n}$(分别时对应$F_n$的模和相位)

3、每一个指数信号的$n \Omega$部分对应为横坐标相应位置(傅里叶级数的$k$值与基波频率$\omega_0$相乘等于一个特定的$\omega$位置的幅度或者相位)

4、按照指数信号的系数求解幅度和相位

注意:如果是绘制单边谱,将函数化为标准傅里叶级数三角函数形式$f(t)=\frac{A_{0}}{2}+\sum_{n=1}^{\infty} A_{n} \cos \left(n \Omega t+\varphi_{n}\right)$,0位置的值为$\frac{A_{0}}{2}$(注意不是$A_{0}$,$A_{0}$是两倍的实际的直流分量)

单双边谱转换:从单边谱转为双边谱,除直流分量外(0位置的值),幅值减半,相位关于原点对称(实信号的傅里叶级数系数的共轭对称性);相应也可以从双边谱转换为单边谱(单双边转换过程中直流分量的幅度是不发生改变的)



\subsection{连续周期信号的傅里叶变换图:(冲激形式)(称为时域信号的频谱)}\index{频域信号绘图!连续周期信号的傅里叶变换图:(冲激形式)(称为时域信号的频谱)}

1、求解周期信号的傅里叶变换(冲激序列形式)

2、绘制各个点的冲激,注意冲激的表示方法



\subsection{离散傅里叶变换图}\index{频域信号绘图!离散傅里叶变换图}

1、图像在间断点是封闭的(间断点有垂线连接)

2、图像是关于$2\pi$是周期的



注意绘图要点:横坐标标注、端点值标注,端点标注

\section{信号性质之奇偶性}\index{信号性质之奇偶性}

1、奇信号的序列和为0(时域信号大小的总和)

2、奇信号与偶信号乘积为奇信号(用定义证明)

3、信号的每一项的平方和等于其奇奇部分的每一项平方和加上偶数部分的每一项的平方和(利用2和展开平方证明)

备注:对连续信号也成立

\section{信号性质之周期性}\index{信号性质之周期性}



\subsection{周期性判断与计算}\index{信号性质之周期性!周期性判断与计算}

1、使用定义判断周期性(特别是异形函数,自变量带平方的)

2、两个信号和:信号周期是两个信号周期的最小公倍数(对于连续信号,周期最小公倍数除以任一个周期的结果不是有理数则不是周期的;对于离散信号,两个整数周期一定有最小公倍数且是整数)

3、代入周期值判断(适用于选择题):根据周期的定义式,将选项代入,并代入$t=0$到定义式中,判断定义式是否成立



\subsection{通过时不变系统后的周期性}\index{信号性质之周期性!通过时不变系统后的周期性}

1、时不变的输入输出周期性:对于时不变的系统,如果输入是周期的,输出也是周期的(时不变的特性)(也可以将周期信号表示为频谱函数的形式来证明)

2、输入非周期,输出可能是周期:可以冲过冲激序列将输入函数扩增成全域内的函数;或者将输入函数中周期比值为无理数的其中一部分抹掉(按照无理数部分的频率特征构造系统的频率响应),使剩余的部分的周期有有理数公比

总结:对于LTI,输入是周期,输出一定是周期;输入是非周期,输出可能是周期



\subsection{信号伸缩变换后的周期性:(连续)}\index{信号性质之周期性!信号伸缩变换后的周期性:(连续)}

1、连续信号伸缩不会改变周期性,伸缩前后周期性一致,但是周期不同



\subsection{信号伸缩变换后的周期性:(离散)}\index{信号性质之周期性!信号伸缩变换后的周期性:(离散)}

1、原信号是周期的,对信号进行抽取和插值之后与原信号周期性一致(周期的计算注意区分之前周期的奇偶性(N为奇数时,可认为是多个周期合并再进行抽取操作))

2、抽取和插值之后的信号是周期的,原信号与信号进行抽取和插值之后的周期性一致不一定一致,离散信号的抽取之后是周期的不代表抽取前是周期的(可能抹掉了一些非周期特性),但是离散信号的插值之后是周期的代表插值之前是周期的

\section{系统的其它性质}\index{系统的其它性质}



\subsection{记忆性}\index{系统的其它性质!记忆性}

1、定义:系统的输出仅取决于该时刻的输入

2、证明:找特殊值,证明某个时间的信号取决于之前的输入

3、样例:累加器(或称为相加器)、延迟单元(都与之前的输入有关)

问题:对于与之后输入相关的是记忆性吗?答:是<奥本海姆 信号系统 第二版 P29 1.6.1 末尾>



\subsection{时不变}\index{系统的其它性质!时不变}

1、先时移再经过系统与先经过系统再时移进行对比,如果一样就是时不变的,如果不一样就是时变的

注意:理解先时移再经过系统,先时移是以原信号($x(t),x[n]$)为基础时移后再代入系统的(在系统外进行时移),不是基于系统中的输入信号(类似于$x(-t),x[-n]$之类的)直接时移的



\subsection{稳定性}\index{系统的其它性质!稳定性}

1、输入有界时输出也有界:设有界输入,判断输出是否是无界的(反证法)

2、LTI系统稳定:单位冲激响应的模在整个域上积分为有限值→稳定LTI(单位脉冲响应的模在整个域上求和为有限值→稳定LTI)

3、收敛域判断:连续系统收敛域包含$jw$轴;离散系统收敛域包含单位圆

4、$H(s)$有理式表示中,如果分母中缺少$s^0$项,则系统一定不稳定



\subsection{因果性}\index{系统的其它性质!因果性}

1、输出不早于输入:当前的输出只能与当前和过去输入有关,即使以未来的判断条件也不行(表达式中任何一处早于输入都不行)(注意:因果性只看输入和输出信号,不关注其它的因式)

2、LTI系统因果:当$t<0$时,$h(t)=0$(离散下,$n<0$时,$h[n]=0$),即具有初始松弛条件

3、其它判断条件:物理可以实现一定是因果的(连续 AND 离散);系统可以由微分方程表示,一定是因果的(连续);系统可以由差分方程表示,不一定是因果的(离散);系统可以由框图表示,不一定是因果的(连续 AND 离散)

4、$H(z)$的收敛域中,如果收敛域不包括无穷,则系统非因果

5、当$H(z)$与已知的信号相乘时,两者的收敛域需要有交集,此时可以推断出系统是因果的(推导过程示例:$H(z)$有两个极点$1/4$和$1/2$,没有明确说明收敛域,所以此时收敛域有三种情况,$X(z)$的收敛域为$|z| \gt 1$,所以此时两者相乘必须有交集,可以得到$H(z)$的收敛域为$|z| \gt 1/2$,也就得到了系统是因果系统)

6、一般对于未特别说明是非因果系统的,默认是因果系统(该条可以否定3和5,即高于3和5)

注意:对于级数$\sum_{i=m}^{n}$表示的输入输出关系的信号,级数上限可能小于下限($m>n$)



\subsection{可逆性}\index{系统的其它性质!可逆性}

1、预判断条件:$x(t)$是否有损值(逆时无法恢复)(损失类型:损失奇部或者偶部、求导损失常数部分、平方损失负数部分、忽略部分$x(t)$的值)

2、预判断条件:$x(t)$是否有多对一的情况(逆变换时无法区分)(损失类型:三角函数)

3、判断:可逆性判断时举反例,两个不同的输入对应同一个输出(输入可以取常数信号)(逆时无法区分的:找输入输出关系图上水平线上的两点输入即可)

\section{判断系统性质}\index{判断系统性质}



\subsection{常见的性质判断特征}\index{判断系统性质!常见的性质判断特征}

1、输入输出关系式带有非常系数的:线性、时变的

2、输入函数变量带有放缩变换的:线性、时变的(注意理解先时移再经过系统)、非因果的

3、输入函数变量带有负常系数的:时变的(注意理解先时移再经过系统,先时移是以原信号($x(t),x[n]$)为基础时移后再代入系统的,不是基于系统中的输入信号(类似于$x(-t),x[-n]$之类的)直接时移的)

4、系统转换函数的定义域与时间相关(非与输入相关):时变的(先经过系统后延时时定义域变了;先延时后经过系统时定义域没有变)(备注:如果定义域是和输入函数相关的(输入函数乘常系数,且变量无放缩),就是非时变的)

5、含有对数运算:非线性,不稳定(因为靠近y轴的地方虽然x是有限的,但是输出是无限的)

6、含有输入信号的高次方运算:非线性



\subsection{性质判断的注意点}\index{判断系统性质!性质判断的注意点}

1、对于输入信号自变量有放大系数(系数小于1)的信号:注意正值部分的记忆性;注意负值部分的非因果性(对于缩小的信号与上述相反)

2、使用导数定义时,注意Δx是可正可负的,正值会导致非因果

3、做时变和线性判断时,若定义域与$x(t)$相关,则分段范围需要与$x(t)$同步变换;定义域中的所有包含$t$的(包括$x(t)$中的$t$)需要与$y(t)$中的$t$同步变化(备注:对于离散信号也是如此)

4、对于未明确定义的信号取偶部:利用偶部表示公式($Ev\{x(t)\}=\frac{x(t)+x(-t)}{2}$)(将$x(t)/x[n]$代入公式即可,即直接令$t=-t/n=-n$,不改变其它的部分)

5、对于复杂的函数可以举反例进行说明

注意:如果是分段函数则注意范围表示是否变换

\section{单位冲激(脉冲)与单位阶跃}\index{单位冲激(脉冲)与单位阶跃}



\subsection{时域中表示}\index{单位冲激(脉冲)与单位阶跃!时域中表示}

1、连续:单位冲激信号和单位阶跃信号

2、离散:单位脉冲序列和单位阶跃序列



\subsection{频域中表示}\index{单位冲激(脉冲)与单位阶跃!频域中表示}

1、连续:单位冲激响应和单位阶跃响应

2、离散:单位脉冲响应和单位阶跃响应



\subsection{恒等系统:(输入输出为常数倍关系,并不是一倍)(注意与信号失真的条件区分)}\index{单位冲激(脉冲)与单位阶跃!恒等系统:(输入输出为常数倍关系,并不是一倍)(注意与信号失真的条件区分)}

1、离散恒等系统性质:$h[n]=k\delta[n]$

2、连续恒等系统性质:$h(t)=k\delta(t)$



\subsection{单位脉冲(冲激)响应与单位阶跃响应的关系}\index{单位冲激(脉冲)与单位阶跃!单位脉冲(冲激)响应与单位阶跃响应的关系}

1、单位脉冲与单位阶跃响应:$h[n]=s[n]-s[n-1]$(推导:$\delta[n]=u[n]-u[n-1]=u[n]-u[n]*\delta[n-1]$,两边同时与系统信号卷积即可得出上述结论);$s[n]=\sum_{k=0}^{\infty}h[n-k]$(由$u[n]=\sum_{k=0}^{\infty}\delta[n-k]$两边同时与输入卷积得到)(单位阶跃函数是单位冲激函数的累加积分)

2、单位冲激与单位阶跃响应:$h(t) = \frac {ds(t)}{t}$(单位冲激是单位阶跃的导数)(如果冲激函数没有突变类型的点(冲激),则说明阶跃信号是连续的)(拉氏变换中会添加或者消去0处的极点)

注意:真分式展开之后,每一项都是真分式,不包含常数项或者$s$的多项式,意味着冲激响应在$t=0$处不包含冲激函数及其导数



\subsection{冲激函数的运算性质}\index{单位冲激(脉冲)与单位阶跃!冲激函数的运算性质}

1、尺度性质:$\delta(at)=\frac{1}{|a|}\delta(t)$,$\delta(at-t_0)=\frac{1}{|a|}\delta(t-\frac{t_0}{a})$(⇒ 冲激函数是偶函数)(如果遇到带有非最简形式的冲激函数,都要化简成最简形式)

2、抽取性质:$x[n] \delta\left[n-n_{0}\right]=x\left[n_{0}\right] \delta\left[n-n_{0}\right]$,$\int_{-\infty}^{\infty} f(t) \delta\left(t-t_{0}\right) d t=f\left(t_{0}\right)$,$f(t) \delta\left(t-t_{0}\right)=f\left(t_{0}\right) \delta\left(t-t_{0}\right)$

3、导数(冲激偶)性质:$\int_{-\infty}^{\infty} \delta^{\prime}(t) d t=0$,$\int_{-\infty}^{t} \delta^{\prime}(\tau) d \tau=\delta(t)$,$\int_{-\infty}^{\infty} f(t) \delta^{\prime}(t) d t=-f^{\prime}(0)$,$\int_{-\infty}^{\infty} f(t) \delta^{\prime}\left(t-t_{0}\right) d t=-f^{\prime}\left(t_{0}\right)$,$f(t) \delta^{\prime}(t)=f(0) \delta^{\prime}(t)-f^{\prime}(0) \delta(t)$(推导:$(f(0)\cdot \delta(t))^\prime =(f(t)\cdot \delta(t))^\prime = f^\prime(x)\delta(t)+f(t)\delta^\prime(t) \Rightarrow f(t) \delta^{\prime}(t)=f(0) \delta^{\prime}(t)-f^{\prime}(0) \delta(t)$,即由乘积函数求导来推导)

4、冲激函数化简:若$f(t)=0$有$n$个互不相等的实数单根$t_{i}(i=1,2, \cdots, n)$,即$f^{\prime}\left(t_{i}\right) \neq 0$,则有$\delta[f(t)]=\sum_{i=1}^{n} \frac{1}{\left|f^{\prime}\left(t_{i}\right)\right|} \delta\left[\left(t-t_{i}\right)\right]$(具体的证明见page页面)(如果有重实根怎么办?放弃考研)

5、冲激函数与阶跃函数的关系:$\delta(t)= u^{\prime}(t)$

6、冲激偶的积分:冲激偶绝对值的积分是不存在的,即冲激偶不是绝对可积的;冲激偶不加绝对值的积分是0

备注:冲激函数只有积分才能消去(另外注意积分时注意积分点是否在积分范围内)

\section{信号平均功率和总能量}\index{信号平均功率和总能量}



\subsection{连续信号}\index{信号平均功率和总能量!连续信号}

1、平均功率公式:$P_{\infty} \triangleq \lim_{T \rightarrow \infty} \frac{1}{2 T} \int_{-T}^{T}|x(t)|^{2} d t$(无穷区间上函数模平方的积分,除以区间长度)(范围中T取无穷极限,范围从-T→ T)(备注:信号的平均功率可以与帕萨瓦尔定律相关联)

2、总能量公式:$E_{\infty} \triangleq \lim_{T \rightarrow \infty} \int_{-T}^{T}|x(t)|^{2} d t=\int_{-\infty}^{+\infty}|x(t)|^{2} d t$(无穷区间上函数模平方的积分)(范围中T取无穷极限,范围从-T→ T)



\subsection{离散信号}\index{信号平均功率和总能量!离散信号}

1、平均功率公式:$P_{\infty} \triangleq \lim_{N \rightarrow \infty} \frac{1}{2 N+1} \sum_{n=-N}^{+N}|x[n]|^{2}$(无穷区间上序列模平方的求和,除以区间长度)(范围中N取无穷极限,范围从-N→ N,项数一共是2N+1项)(备注:信号的平均功率可以与帕萨瓦尔定律相关联)

2、总能量公式:$E_{\infty} \triangleq \lim_{N \rightarrow \infty} \sum_{n=-N}^{+N}|x[n]|^{2}=\sum_{n=-\infty}^{+\infty}|x[n]|^{2}$(无穷区间上序列模平方的求和)(范围中N取无穷极限,范围从-N→ N,项数一共是2N+1项)



\subsection{信号分类}\index{信号平均功率和总能量!信号分类}

1、信号能量有限,平均功率为0(能量信号)

2、平均功率有限,信号能量无穷(功率信号)

2、信号能量无穷,平均功率无穷



\subsection{能量的计算}\index{信号平均功率和总能量!能量的计算}

1、对于能量的计算,首先想到帕萨瓦尔定理

\section{信号范围问题}\index{信号范围问题}



\subsection{根据指定信号值求解时间或者序列范围:<小蓝书 P6 1-2>}\index{信号范围问题!根据指定信号值求解时间或者序列范围:<小蓝书 P6 1-2>}

1、根据已有的方程确定

2、可以采用特殊值代入进行验证

注意:考虑无穷处的值是否满足



\subsection{输入$x[n]$界定值和输出$y[n]$界定值的关系}\index{信号范围问题!输入$x[n]$界定值和输出$y[n]$界定值的关系}

0、假设输入界定值为$B$,输出界定值为$C$(界定值是信号的范围小于该界定值,其它情况立即推)

1、通过输入与输出的关系,估算输出的用输入表示的界定值(用B表示的式子)

2、由于输出界定值是最接近输出的,所以C应该小于等于上述估算的用B表示的式子(注意取等号的条件,对输入的值的要求)

\section{离散信号的基波频率问题}\index{离散信号的基波频率问题}

<小蓝书 P27 1-26 1-27>(写的太复杂,理解即可)

0、假设:信号为$x[n]=e^{jm(2\pi / N)n}$

1、结论:$gcd(m,N)$是$m$和$N$中的最大公约数,则离散信号$x[n]=e^{jm(2\pi / N)n}$的基波周期为$N_0 = N / gcd(m,N)$

2、理解:对于信号$x_1[n]=e^{j2(2\pi / 3)n}$和$x_2[n]=e^{j(2\pi / 3)n}$,按照以上结论,这两个信号的基波周期都是3,区别在于$x_1[n]$的角频率比$x_2[n]$的角频率变化的快,当$x_1[n]$的角频率变化再快一些,变成$x_3[n]=e^{j3(2\pi / 3)n}$,这时$x_3[n]=e^{j(2\pi)n}$,基波周期变成了1,当$x_1[n]$的角频率变化再快一些,变成$x_4[n]=e^{j4(2\pi / 3)n}$,这时基波周期又变成了3(类似于人眼的只能按照指定频率扫描,当看风扇转动时,看到有扇叶静止的时候,可能是频率又轮转回来了)

3、应用:对于一连续信号$x(t)=e^{jw_0t}$,基波频率是$w_0$,基波周期$T_0$是$2\pi / w_0$,对该离散信号进行$T$的等间隔取样,即$x[n] = x(nT) = e^{jw_0nT}$,可以使用周期的定义(设周期是$N$,使用周期的定义进行证明,试着证一下)来证明只有$T/T_0$是有理数时,$x[n]$才是周期的,设周期是$N$,即$\frac{T}{T_0}=\frac{p}{q}$,其中$p$和$q$都是有理数(注意$\frac{p}{q}$不一定是最简形式),则$x[n] = e^{jw_0nT} = e^{j\frac{2\pi}{T_0}nT} = e^{jp\frac{2\pi}{q}n}$,由上述的结论部分可知,离散信号$x[n]$的基波周期为$N_0 = N / gcd(p,q)$(理解:当$\frac{T}{T_0}=\frac{1}{3}$时,即可简单地认为连续信号是以3为周期,离散信号以1来取样,那么离散信号肯定需要取三次才能取到一个周期;当$\frac{T}{T_0}=\frac{3}{1}$,即可简单地认为连续信号是以1为周期,离散信号以3来取样,那么离散信号取一次就能取到一个周期;当$\frac{T}{T_0}=\frac{2}{6}$时,即可简单地认为连续信号是以6为周期,离散信号以2来取样,那么离散信号需要取三次(3 * 2)才能取到一个周期(6));当$\frac{T}{T_0}=\frac{3}{7}$,即可简单地认为连续信号是以7为周期,离散信号以3来取样,那么离散信号需要取七次(7 * 3)就能取到一个周期(就是将连续信号的三个周期为21,然后被取样,刚好能取完整,续上下一个离散的取样)

