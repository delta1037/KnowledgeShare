\chapterimage{chapter_head_2.pdf}
\chapter{LTI系统}

\section{微分(差分)方程判定系统性质}\index{微分(差分)方程判定系统性质}



\subsection{<小蓝书 P52 2-22(一阶微分方程LTI的证明)P52 2-23(一阶差分方程LTI的证明)>}\index{微分(差分)方程判定系统性质!<小蓝书 P52 2-22(一阶微分方程LTI的证明)P52 2-23(一阶差分方程LTI的证明)>}

线性特性证明:

1、列出不同输入信号对应的系统微分(差分)方程(微分方程后需要标注输出信号为0时的$t$的范围)

2、将系统微分(差分)方程组合

3、说明输入的组合与输出的组合形式一致,并且输入与输出的信号存在的范围(信号非零;输入为0,输出也为0)一致

问题:如果输入和输出范围不一致是什么情况(输入为0,但是输出不为0)?答:这是初始松弛条件硬性规定的,输入为0,输出一定也是0(强制)



\subsection{时不变特性证明}\index{微分(差分)方程判定系统性质!时不变特性证明}

1、列出原输入信号对应的系统微分(差分)方程(微分方程后需要标注输出信号为0时的$t$的范围)

2、列出原输入信号时移之后的信号对应的系统微分(差分)方程(将时移之后的信号用原信号表示)(定义域:微分方程后需要标注输出信号为0时的$t$的范围)

3、对2中的系统微分(差分)方程做变量替换,与1中的输入信号的变量形式表示一致(定义域:微分方程后需要标注时移之后的输出信号为0时的$t$的范围)

4、将1、2(变量替换后的)方程做对比,找到时移信号输出的信号与原信号输出的信号的对应关系(定义域:时移之后的输入信号为0的范围和时移之后的输出信号为0的范围是一致的)



\subsection{因果特性证明}\index{微分(差分)方程判定系统性质!因果特性证明}

1、设定有初始松弛条件的系统<小蓝书 P52 2-22>:一定是因果的(LTI系统初始松弛等效于系统因果)(证明:初始松弛条件,即$x(t)=0,t<0$时,有$y(t)=0,t<0$,又因为系统是时不变的,所以说明系统输出不会早于输入,而系统输出不早于输入正是系统因果性的条件)(初始松弛应用:若系统满足初始松弛,即$x(t)=0,t<0$时,有$y(t)=0,t<0 $,可以得到$y(0)=0$,可以用该点的值求解系统未知参数,即可以作为一个求解辅助条件)

2、没有说初始松弛条件的系统<小蓝书 P52 2-22>:可以根据系统单位冲激响应推断,即单位冲激响应在$t<0$时,$h(t)=0$

\section{信号的卷积}\index{信号的卷积}



\subsection{卷积的性质}\index{信号的卷积!卷积的性质}

1、函数卷积后的微分:$\frac{\mathrm{d}}{\mathrm{d} t}[x(t) * h(t)]=x(t) * \frac{\mathrm{d}}{\mathrm{d} t} h(t)=\frac{\mathrm{d}}{\mathrm{d} t} x(t) * h(t)$(一边先进行微分)(可以推导至卷积后的二阶微分,由公式得两个信号都先微分再卷积)

2、函数卷积后的积分:$\int_{-\infty}^{t}[x(\tau) * h(\tau)] \mathrm{d} \tau=x(t) * \int_{-\infty}^{t} h(\tau) \mathrm{d} \tau=\left[\int_{-\infty}^{t} x(\tau) \mathrm{d} \tau\right] * h(t)$(一边先进行积分)(可以推导至卷积后的二阶积分,由公式得两个信号都先积分再卷积)

3、函数延时后再卷积:$x\left(t-t_{1}\right) * h\left(t-t_{2}\right)=y\left(t-t_{1}-t_{2}\right)$



\subsection{卷积求解类型}\index{信号的卷积!卷积求解类型}

1、窗函数卷积(两个窗信号)(连续信号端点重合/离散信号最少有一个点重合值才不为0,可以判断输出的长度)

2、函数与冲激信号的卷积(将冲激中的平移运算代入到函数中)(如果冲激中的自变量含有非1系数,注意化简)

3、使用卷积定义卷积(一般信号)

4、周期信号表示成级数和的形式(先计算级数的一段卷积,再加上冲激求和函数)

5、对于指定范围的信号(给出图像),使用$u(t)$对函数范围进行框定,转换成闭式表达式,用定义求解

6、利用频域求解:傅里叶变换,拉氏变换,$z$变换等



\subsection{卷积求解技巧}\index{信号的卷积!卷积求解技巧}

1、等腰三角形或者等腰梯形可以看作是两个矩形信号的卷积

2、计算离散信号的卷积时,将(一段)阶跃序列转换成脉冲序列的形式

3、求解卷积时可以将时移部分提出来,将别的计算完之后,再与时移进行卷积(利用了与冲激信号或者脉冲信号卷积,更好计算一些)

4、求解周期信号的卷积先求解其中一个周期内的卷积作为基础信号,然后周期信号卷积是以该基础信号,以原信号周期为扩展周期的信号

注意:结果的范围的表示用u(t)、u[n]表示

注意:当半边的时域信号与全域上的冲击序列卷积时,将求和符号挪到外边,求出里面的卷积后将级数展开,求出使得级数收敛的部分(利用前提条件说明的范围)<小蓝书 P42 2-10>



\subsection{窗信号卷积确定输出信号}\index{信号的卷积!窗信号卷积确定输出信号}

1、将系数拆出来单独计算(用来确定最大值,注意两个窗信号是两个系数,还有窗的边界值,最大值是系数*系数*窗长度)

2、对于相同的窗信号:卷积出来是三角形的,确定边界(窗函数边界的代数相加就是卷积后的边界)和顶点即可(图形求解)

3、对于不同的窗信号:卷积出来是梯形的,画个图看一下梯形关键点的范围,自求多福吧



\subsection{窗信号卷积确定输入与输出的范围关系:<小蓝书 P52 2-28>}\index{信号的卷积!窗信号卷积确定输入与输出的范围关系:<小蓝书 P52 2-28>}

1、稳妥方法:利用卷积的定义求解卷积信号(其中就包括卷积后信号的上下范围)

2、特殊技巧:画出图像,保持其中一个不变,对另一个进行反转后平移,当两个图像有交界时即卷积值不为零(窗函数边界的代数相加就是卷积后的边界,对于含有多个窗函数的,拆分为两两卷积然后线性组合即可)(当非窗函数图像时不要用这个)

\section{逆系统}\index{逆系统}

1、分离出$h[n]*g[n]=\delta[n]$的形式,$h[n],g[n]$互为逆系统

2、逆系统的拉氏变换的收敛域是一致的(why?)

注意:如果有第一问利用好第一问的结论(进行二次分解)

问题:$h[n],g[n]$在函数形态上是否需要满足什么条件才是互为逆系统?答:两个对称的冲激信号

