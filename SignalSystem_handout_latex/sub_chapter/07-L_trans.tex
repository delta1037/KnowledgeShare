\chapterimage{chapter_head_2.pdf}
\chapter{拉普拉斯变换}

\section{求解拉普拉斯变换}\index{求解拉普拉斯变换}



\subsection{拉普拉斯变换}\index{求解拉普拉斯变换!拉普拉斯变换}

1、有限指数信号:(带有三角(注意三角展开成指数形式是含有$j$的),指数,阶跃信号(指数为0))转换成指数形式,利用常见的变换对(注意是左半边还是右半边信号)

2、有限的非指数信号:利用拉普拉斯变换定义式求解(可利用拉普拉斯变换的微分和积分性质,转换成较简单信号)

3、无限的三角信号:利用变换对求解

4、无限指数信号:利用特征值法求解

5、单向周期延拓的信号:先求解一个周期内的拉氏变换,对于延拓相当于与冲激序列卷积,也就是相当于在频域频移,时域的卷积和在频域求解乘积和即可(注意标注收敛域)

6、逆系统的拉普拉斯变换:频域零点和极点互换,时域$x$与$y$互换位置

注意:标注拉普拉斯变换的收敛域($Re\{s\}$的形式)



\subsection{单边拉普拉斯变换}\index{求解拉普拉斯变换!单边拉普拉斯变换}

1、定义求解(利用拉普拉斯与单边拉普拉斯的关系)

2、时移或者微分后的信号的单边拉普拉斯:使用定义求解,可以使用原信号的积分表示其中的一部分<小蓝书 P237 7-36>



\subsection{拉普拉斯逆变换}\index{求解拉普拉斯变换!拉普拉斯逆变换}

1、部分分式展开:处理有理式(预处理:时移、频移)

2、围线积分法(利用留数定理):不考吧

注意:时移和频移性质的运用



\subsection{拉普拉斯变换收敛和收敛域的确定}\index{求解拉普拉斯变换!拉普拉斯变换收敛和收敛域的确定}

1、收敛:保证拉普拉斯分析式的实数部分积分值有限

2、收敛域:所有可以使拉普拉斯分析式收敛的值

\section{拉普拉斯收敛域}\index{拉普拉斯收敛域}



\subsection{拉普拉斯收敛域的性质}\index{拉普拉斯收敛域!拉普拉斯收敛域的性质}

1、频域相乘后的收敛域:取两个频域的$ROC$的交集;如果有零点和极点抵消的情况,$ROC$可能会扩大范围

2、右边信号、左边信号、有限信号和双边信号的收敛域特性:左右边信号收敛域与信号方向一致;有限信号收敛域为整个平面;双边信号收敛域为带状区域,或者不存在

3、因果信号收敛域的特性:$ROC$在右半平面(对于有限的因果信号,收敛域是全域)(即因果一定包含正无穷)

4、稳定系统(绝对可积)的收敛域特性:$ROC$包含$jw$轴

5、极点与收敛域的关系:收敛域中不包含任何的极点;极点是收敛域的边界

6、收敛域和有理式确定时域表达式:利用左右信号的$ROC$范围特性关系

7、时域有限的信号的收敛域特性:$ROC$为全平面

8、时域信号的积分为频域的零处的值:(频域和时域的对应关系,即傅里叶变换的对应关系)如果积分为有限值说明收敛域包括$s=0$(包含$jw$轴)

9、奇、偶信号的收敛域:由于偶函数信号为双边信号,所以偶函数的收敛域如果存在一定是带状,奇信号同理

注意:时移和频移性质的运用

注意:零点不影响拉普拉斯变换的$ROC$



\subsection{会改变收敛域的性质}\index{拉普拉斯收敛域!会改变收敛域的性质}

1、频域相乘:$ROC$取两个频域信号的交集,如果存在零点和极点抵消的情况,而且刚好是抵消的决定ROC边界的部分,$ROC$会扩大范围

2、S域平移:$ROC$也会随着平移

3、时域反褶:$ROC$发生关于$y$轴的对称变换

4、时域积分:引入新的极点,$ROC$变为$\{ ROC \cap R \gt 0 \}$

5、尺度变换:尺度变换会对现有的零点和极点的位置放缩,如果变换的参数小于0,那么还会进行类似反褶的$ROC$变换



\subsection{不会改变收敛域的性质}\index{拉普拉斯收敛域!不会改变收敛域的性质}

1、时域平移:时移不会引入新的极点,并且是原信号是同类型(左右边)的信号,不改变$ROC$

2、时域微分:引入新的零点,不改变$ROC$

\section{根据电路图求解系统函数/电路的分析}\index{根据电路图求解系统函数/电路的分析}

1、电容:电流与电压变化率的关系($v_{C}(t)=\frac{1}{C} \int_{-\infty}^{t} i_{c}(\tau) d \tau \Rightarrow i_{C}(t)=C \frac{d v_{C}(t)}{d t} \Rightarrow I_{C}(s)=s C V_{C}(s)-C v_{C}\left(0^{-}\right)$;$ V_{C}(s)=\frac{1}{s C} I_{C}(s)+\frac{1}{s} v_{c}\left(0_{-}\right)$)(在计算电流)(电容等效源电压的部分为$\frac{1}{s} v_{c}\left(0_{-}\right)$,方向与原电压$v_{c}\left(0_{-}\right)$方向相同)

2、电感:电压与电流变化率的关系($v_{L}(t)=L \frac{d i_{L}(t)}{d t} \Rightarrow V_{L}(s)=L (s I_{L}(s)- i_{L}\left(0_{-})\right)$;$I_{L}(s)=\frac{1}{s L} V_{L}(s)+\frac{1}{s} i_{L}\left(0^{-}\right)$)(电感等效源电压的部分为$- L i_{L}\left(0_{-}\right)$,方向与原电流$i_{L}\left(0_{-}\right)$方向相反)

3、电阻:电压与电流的线性关系($v_{R}(t)=R i_{R}(t)\Rightarrow V_{R}(s)=R I_{R}(s)$)

4、回路电流和回路电压方程:画出电路的S域模型,并将初始条件转换为源电压;列出来各个回路的KCL方程,联立求解

5、稳态情况:当存在电感的电路达到稳态时(具有恒压源之类的),电感相当于导线,不会对电路造成影响

\section{状态方程定义与建立}\index{状态方程定义与建立}



\subsection{相关定义}\index{状态方程定义与建立!相关定义}

1、状态变量:二阶系统状态变量$x_1(t),x_2(t)$,激励(输入变量)为$e(t)$,输出变量为$y_1(t),y_2(t)$(输出不一定是二阶)(状态变量是描述系统所需的最少的一组变量,根据这组变量在$t=t_0$时刻的值和系统的激励,就可以唯一确定系统在$t≥t_0$后任意时刻的响应)

2、状态方程:状态变量的一阶导数(离散时是状态变量的单位延时)与状态变量及输入变量之间的关系(方程左端是状态变量的一阶导数,右端是用状态变量和输入信号表示,方程不包含微分和积分运算)(状态方程描述了系统内部的状态)(状态方程描述了状态响应)

3、输出方程:输出变量与输入变量之间的关系(方程中不包含变量的微分和积分运算)(输出方程描述了系统的输入输出关系)(输出方程描述了输出响应)

4、方程表达式:$\begin{aligned}&\dot{x}(t)=A(t) x(t)+B(t) v(t) \\&y(t)=C(t) x(t)+D(t) v(t)\end{aligned}$,在状态方程中,$A(t)$是一个$N*N$的矩阵,$x(t)$是列状态变量,$B(t)$是一个$N$行的列向量,$v(t)$是关于$t$的输入函数变量,$C(t)$是一个$N$列的行向量,$D(t)$是一个关于$t$的实数函数变量,展开描述为$\begin{aligned}&\dot{x}_{1}(t)=a_{11} x_{1}(t)+a_{12} x_{2}(t)+\cdots+a_{1 N} x_{N}(t)+b_{1} v(t) \\&\dot{x}_{2}(t)=a_{21} x_{1}(t)+a_{22} x_{2}(t)+\cdots+a_{2 N} x_{N}(t)+b_{2} v(t) \\&\vdots \\&\dot{x}_{N}(t)=a_{N 1} x_{1}(t)+a_{N 2} x_{2}(t)+\cdots+a_{N N} x_{N}(t)+b_{N} v(t)\end{aligned}$和$y(t)=c_{1} x_{1}(t)+c_{2} x_{2}(t)+\cdots+c_{N} x_{N}(t)+d v(t)$



\subsection{由电路图建立状态方程}\index{状态方程定义与建立!由电路图建立状态方程}

1、状态变量选取:将电容电压,电感电流设为状态变量

2、建立状态方程:根据$KCL$和$KVL$建立电路的方程关系,化简得到状态方程(n个方程组成的方程组)

3、建立输出方程:根据电路的方程电压电流关系获取到输出和电容电流电压(即状态变量)的关系(1个方程)



\subsection{由系统框图或者信号流图建立状态方程}\index{状态方程定义与建立!由系统框图或者信号流图建立状态方程}

1、状态变量选取:选取积分器的输出为状态变量,对方框图或者信号流图进行标注

2、建立状态方程:建立状态变量的导数与其它状态变量的关系,得到状态方程(n个方程组成的方程组)

3、建立输出方程:由输出节点与状态变量之间的关系建立输出方程(1个方程)

注意:对于信号流图,从一个多输入节点取值,是取的该节点多输入相加后的值



\subsection{由微分方程建立状态方程:(或者说是由H(S)建立状态方程)}\index{状态方程定义与建立!由微分方程建立状态方程:(或者说是由H(S)建立状态方程)}

1、获取到系统的H(S):H(S)表示成有理式形式,分子分母同时乘X(s),拆分成分子分母两个微分方程

2、状态变量选取:按照x(t)从低阶到高阶,依次设状态变量$x_1(t)$、$x_1(t)$、$\cdots$、$x_n(t)$

3、建立状态方程:由设置状态变量的规则得到的有状态变量之间的导数关系(n-1个方程)和由分母建立的微分方程得到的输入与状态变量的关系(1个方程)得到状态方程(n个方程组成的方程组)

4、建立输出方程:由分子建立的微分方程得到的输出与状态变量的关系(1个方程)(需要化简至不存在状态变量的导数形式)

\section{状态方程基础与时域求解}\index{状态方程基础与时域求解}



\subsection{状态方程求解基础}\index{状态方程基础与时域求解!状态方程求解基础}

1、矩阵指数式:$e^{A t}=I+A t+\frac{A^{2} t^{2}}{2 !}+\frac{A^{3} t^{3}}{3 !}+\frac{A^{4} t^{4}}{4 !}+\cdots$,其中$A$是矩阵

2、矩阵指数的导数性质:$\frac{d}{d t} e^{A t} =A+A^{2} t+\frac{A^{3} t^{2}}{2 !}+\frac{A^{4} t^{3}}{3 !}+\cdots=A(I+A t+\frac{A^{2} t^{2}}{2 !}+\frac{A^{3} t^{3}}{3 !}+\cdots)  = A e^{A t}=e^{A t} A$(其实与指数的直接求导是一致的)

3、导数性质与解的关系:$\dot{x}(t)=A x(t), \quad t>0$的解就是$x(t)=e^{A t} x(0), \quad t \geq 0$(由矩阵指数导数的对应关系对比得到的,其中$x(0)$常系数是为了保证等号成立),其中$e^{A t}$被称为系统的状态转换矩阵

4、待解的方程表示:$\begin{aligned}&\dot{x}(t)=A x(t)+B v(t) \\&y(t)=C x(t)+D v(t)\end{aligned}$



\subsection{连续时间系统状态方程的求解:(时域法)}\index{状态方程基础与时域求解!连续时间系统状态方程的求解:(时域法)}

1、状态方程推导过程:对于方程$\dot{x}(t)=A x(t)+B v(t)$,两边同时左乘$e^{-A t}$,得到$e^{-A t}[\dot{x}(t)-A x(t)]=e^{-A t} B v(t)$,移位化简得到$\frac{d}{d t}\left[e^{-A t} x(t)\right]=e^{-A t} B v(t)$,两边同时积分得到$e^{-A t} x(t)=x(0)+\int_{0}^{t} e^{-A \lambda} B v(\lambda) d \lambda$,化简为$x(t)=e^{A t} x(0)+\int_{0}^{t} e^{A(t-\lambda)} B v(\lambda) d \lambda, \quad t \geq 0$,积分转成卷积形式可得$x(t)=e^{A t} x(0)+e^{A t} * B v(t), \quad t \geq 0$

2、状态方程求解结果:$x(t)=e^{A t} x(0)+\int_{0}^{t} e^{A(t-\lambda)} B v(\lambda) d \lambda, \quad t \geq 0$(卷积形式$x(t)=e^{A t} x(0)+e^{A t} * B v(t), \quad t \geq 0$)

3、输出方程推导过程:对于方程$y(t)=C x(t)+D v(t)$和状态方程解的推导过程,可以得到$y(t)=C e^{A t} x(0)+\int_{0}^{t} C e^{A(t-\lambda)} B v(\lambda) d \lambda+D v(t), \quad t \geq 0$,依据单位冲激响应,方程可以重写为$y(t)=C e^{A t} x(0)+\int_{0}^{t}\left\{C e^{A(t-\lambda)} B v(\lambda)+D \delta(t-\lambda) v(\lambda)\right\} d \lambda, \quad t \geq 0$,所以零输入响应和零状态响应分别是$y_{z i}(t)=C e^{A t} x(0)$和$y_{z s}(t)=\int_{0}^{t}\left\{C e^{A(t-\lambda)} B v(\lambda)+D \delta(t-\lambda) v(\lambda)\right\} d \lambda=\left\lfloor C e^{A t} B+D \delta(t)\right\rfloor * v(t)$,单位冲激响应为$h(t)=C e^{A t} B+D \delta(t), \quad t \geq 0$

4、输出方程求解结果:$y(t)=C e^{A t} x(0)+\int_{0}^{t}\left\{C e^{A(t-\lambda)} B v(\lambda)+D \delta(t-\lambda) v(\lambda)\right\} d \lambda, \quad t \geq 0$

4.1、零输入响应:$y_{z i}(t)=C e^{A t} x(0)$

4.2、零状态响应:$y_{z s}(t)=\int_{0}^{t}\left\{C e^{A(t-\lambda)} B v(\lambda)+D \delta(t-\lambda) v(\lambda)\right\} d \lambda=\left\lfloor C e^{A t} B+D \delta(t)\right\rfloor * v(t)$

4.3、单位冲激响应:$h(t)=C e^{A t} B+D \delta(t), \quad t \geq 0$

