\chapterimage{chapter_head_2.pdf}
\chapter{离散傅里叶变换}

\section{离散傅里叶变换求解}\index{离散傅里叶变换求解}



\subsection{非周期信号:(一般信号,差分信号,脉冲信号)}\index{离散傅里叶变换求解!非周期信号:(一般信号,差分信号,脉冲信号)}

1、一般信号:应用分析公式,级数求和(等比数列求和)

2、冲激信号:应用分析公式,求和

3、应用离散傅里叶变换的性质(注意形式的拆分和组合)



\subsection{非周期逆变换}\index{离散傅里叶变换求解!非周期逆变换}

1、应用综合公式积分

2、应用离散傅里叶变换的性质(注意形式的拆分和组合),对频域的式子进行拆分,分别求逆变换

3、应用常用的傅里叶变换变换对



\subsection{周期信号:(简单三角信号)}\index{离散傅里叶变换求解!周期信号:(简单三角信号)}

1、简单三角信号:展开成$e^{jw_0}$的形式,用展开公式($e^{jw_0n} \leftrightarrow \sum_{l=-\infty}^{\infty}2\pi \delta(w-w_0-2\pi l)$),常数信号是$w_0$为0的形式(特殊)

注意:如果遇到在周期边界的情况,两边在级数中会合成一个,所以在单项的表示中两边分别是一半(常用)或者抹除掉其中一边

注意:离散傅里叶变换的结果是级数形式(因为离散傅里叶变换是$2\pi$周期的),如果将结果限制到一个$2\pi$范围内(在结果后面标注$w$的范围),就不需要表示成级数的形式



\subsection{周期逆变换}\index{离散傅里叶变换求解!周期逆变换}

1、根据$e^{jw_0n} \leftrightarrow \sum_{l=-\infty}^{\infty}2\pi \delta(w-w_0-2\pi l)$逆推



\subsection{性质运用}\index{离散傅里叶变换求解!性质运用}

1、应用离散傅里叶变换的性质

2、共轭对称性质(特别是实信号的傅里叶变换之间的对应关系,时域信号的奇偶性与频域信号的实虚性的对应关系)

3、平移性质只有整数才能应用;非整数可以根据实际式子展开求解;猜测:非整数的平移可以将离散信号转换为连续信号,对连续信号平移之后,再对平移后的连续信号进行重新取样<小蓝书P176 5-23证明>

4、如果是周期函数应用平移性质,注意平移是循环平移的(不要忽略单个周期之外的项)



\subsection{离散傅里叶变换注意点}\index{离散傅里叶变换求解!离散傅里叶变换注意点}

1、变换后的重叠性!!!(因为离散傅里叶变换有周期性)

