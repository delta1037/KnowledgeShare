\chapterimage{chapter_head_2.pdf}
\chapter{采样}

\section{奈奎斯特频率}\index{奈奎斯特频率}



\subsection{时域信号的采样频率:(指定频率采样时样本点可以唯一确定)}\index{奈奎斯特频率!时域信号的采样频率:(指定频率采样时样本点可以唯一确定)}

1、一般信号:可以直接看出最大频率(对于加减的三角信号,其余的不赞成用这种,易出错),从而根据采样定理求得采样频率

2、复杂的信号:将时域信号转换为频域信号(参见连续信号的傅里叶变换),由频域的频率范围确定奈奎斯特频率和何时不会发生混叠

3、特性:如果频域是对称的,则指定的采样频率是边界频率的两倍

注意:奈奎斯特频率是针对带限信号来说的,非带限信号无论怎么采样都会混叠

注意:如果频域不是对称的,注意绘图进行辅助理解(不对称的频域,如果采样频率不是最大边界频率的两倍,可能不会混叠;严格来说采样频率应该大于最大频率和最小频率差值的绝对值)

注意:如果频域是离散的点,则采样频率应该大于(不包括等于,可以画图辅助理解)边界频率的两倍(对称情况下,非对称情况下绘图辅助理解)



\subsection{判断信号能否依据采样定理恢复}\index{奈奎斯特频率!判断信号能否依据采样定理恢复}

1、注意运用傅里叶变换的性质:

1.1、共轭对称性质:时域的实信号,偶数部分对应到频域有实偶特性,奇数部分对应到频域有虚奇特性,则如果当实信号的频域大于某一个值时为0,则小于某一个值时也为0

2、注意频率的判定是正负均有的



注意:对于信号频率看清是单边还是双边的,对于频率范围的表示有没有加绝对值

注意:采样频率应该加单位,$Hz$或者$rad/s$

\section{实际采样与恢复}\index{实际采样与恢复}



\subsection{采样:(针对连续信号进行采样)}\index{实际采样与恢复!采样:(针对连续信号进行采样)}

0、保持采样原因:实际系统产生和传输窄而幅度大的脉冲(近似于冲激)都相当困难

1、零阶保持采样:在一个给定的瞬时对信号采样并保持这一个样本值,直到下一个点被采到为止

2、一阶保持采样:将时域上各个采样点使用直线相连即可



\subsection{恢复:(内插)(针对冲激序列恢复信号)}\index{实际采样与恢复!恢复:(内插)(针对冲激序列恢复信号)}

1、零阶保持恢复:在时域上表现为一个冲击位置向$x$轴正方向作平行于$x$轴的线,直到下一次冲激位置,所构成的大致的图形

2、一阶保持恢复:在时域上表现为两个相邻的冲击位置连线,所构成的大致的图形

\section{采样与恢复计算}\index{采样与恢复计算}



\subsection{采样信号频谱}\index{采样与恢复计算!采样信号频谱}

1、计算采样信号的频域(采样信号一般为周期的冲击函数和;也有可能是方波信号,即非理想的采样)

2、计算待采样信号的频域(待采样信号的频域一般为有限长的)

3、计算采样信号与待采样信号的频域卷积(将有限信号频域以采样信号冲击的周期平移得到直观的采样后的信号),注意不要忘记参数$\frac 1{2\pi}$(采样为时域相乘,频域卷积,看到频域卷积就有$\frac 1{2\pi}$系数)

注意:注意卷积后$k$的取值,如果$k$不能取0,则原信号的位置是空的(即$a_0=0$时,或者说一个周期内的信号为奇信号)



\subsection{采样后信号特征:(序列及其周期、傅里叶级数等特征)}\index{采样与恢复计算!采样后信号特征:(序列及其周期、傅里叶级数等特征)}

1、求解序列:采样后的序列与采样前的连续函数的关系:$x[n]=x_c(nT)$

2、求解序列周期:已知输入信号是周期的,并且频域可以表示为有限的冲激序列(输入信号能够表示成傅里叶级数形式), 将$x[n]$的连续冲激的傅里叶级数(由$x[n]=x_c(nT)$表示式,从连续信号的傅里叶级数推导至$x[n]$的傅里叶级数表示)与离散傅里叶级数标准形式(对照式)对比,得到$N$<序列周期>与$w_0$<原信号频率>与$T$<采样周期>的关系(原信号频率是$w_0$,经过采样并归一化之后,频率变为$w_0T$,与$\frac{2\pi}{N}$对比(具有相等关系)得到$N$)<小蓝书 P200>

3、当$x[n]$的傅里叶级数表示式的项数大于序列周期N时,说明是有一部分是重叠的(说明采样的频率太小了,导致采样后的信号重叠了),在表示傅里叶级数系数时要把这部分进行合并<小蓝书 P200>



\subsection{恢复原信号设计}\index{采样与恢复计算!恢复原信号设计}

1、搬移:使用$cosw_0t$对没有处于中心的图像搬移到中心(搬移到中心之后注意信号进行了叠加,幅度变为了原来的两倍)

2、放大:定义滤波器取中心段的信号,并使得滤波器与中心信号段相乘等于输入信号(恢复为原来的信号)的傅里叶变换即可

\section{理想采样与恢复}\index{理想采样与恢复}

完整系统流程:

1、转换过程:输入信号→经过采样后的冲激串→转换成序列→转换成冲激串→滤波(通常取其中一个波)→输出信号

2、采样后的冲激串$x_p(t)$转换成序列$x[n]$:在频域表示为归一化(归成$2\pi$以内)问题,用冲激串的傅里叶变换转换($X(j\omega)$)成离散的序列的傅里叶变换($X(e^{j\Omega})$),$\Omega$和$\omega$之间的关系为$\omega = \frac {\Omega}{T}$(直接代入到冲激串的傅里叶变换)($T$一般是小于1的值,所以相当于将整个冲激串的频域图形做了压缩,并且是压缩到了$2\pi$以内;连续的冲激串转换成离散序列,只是将间隔为T的串,伸展成了间隔为1的序列,时域伸展,频域压缩)

3、$x[n],y[n]$之间的转换:完全离散的(符合大部分信号的实际运用场景,也是为什么这个系统很重要的原因)

4、序列$y[n]$转换成冲激串$y_p(t)$:在频域表示为延展问题(上述压缩后的信号恢复到原来应有的频域宽度),离散的序列的傅里叶变换($Y(e^{j\Omega})$)转换成冲激串的傅里叶变换转换($Y(j\omega)$),$\Omega$和$\omega$之间的关系为$\Omega=\omega T$(直接代入到离散序列傅里叶变换)($T$一般是小于1的值,所以相当于将整个离散序列的频域图形做了伸展,并且是回到原有的频域范围;离散序列转换成连续的冲激串,只是将间隔为1的序列,伸展成了间隔为T的串,时域压缩,频域伸展)

5、信号的频域损失:信号的损失是在采样过程就发生了的,主要表现为高频信号有部分叠加到一起导致无法区分

6、信号的频域幅度:频域信号的幅度产生$\frac 1{T}$的变化,该变化是在采样的时候引入的,其中分子中的$2\pi$是在频域相乘时消掉的(冲激串的傅里叶级数)

7、连续角频率与离散角频率之间的关系$\Omega=\omega T$推导:由于采样信号可以表示为$x_{p}(t)=\sum_{n=-\infty}^{+\infty} x_{c}(n T) \delta(t-n T)$,由于$\delta(t-n T)$的傅里叶变换是$\mathrm{e}^{-\mathrm{j} \omega n T}$(注意这里没有用时域信号相乘频域卷积的性质,因为这里需要得到一个级数形式与离散的傅里叶变换进行对比),所以得到级数表示式$X_{p}(\mathrm{j} \omega)=\sum_{n=-\infty}^{+\infty} x_{c}(n T) \mathrm{e}^{-\mathrm{j} \omega n T}$,又因为离散的傅里叶变换表示式为$X_{d}\left(\mathrm{e}^{\mathrm{j} \Omega}\right)=\sum_{n=-\infty}^{+\infty} x_{d}[n] \mathrm{e}^{-\mathrm{j} \Omega n}$,并且在时域上有$x_{d}[n]=x_{c}(n T)$,所以将采样后的信号的离散傅里叶变换与连续傅里叶变换对比可以得到$\Omega=\omega T$

注意:采样后的冲激串转序列时,$u(nT) \rightarrow u[n]$

