\chapterimage{chapter_head_2.pdf}
\chapter{连续傅里叶变换}

\section{信号的模和相位}\index{信号的模和相位}



\subsection{模与相位的定义}\index{信号的模和相位!模与相位的定义}

1、频域信号用模和相位表示:$H(jw) = |H(jw)|e^{jarg(H(jw))}$



\subsection{模和相位的拆分运用}\index{信号的模和相位!模和相位的拆分运用}

1、对于利用特征函数求解法,如果将$H(jw)$表示成$ |H(jw)|e^{jarg(H(jw))}$形式,如果$ |H(jw)|=1$,则可以求解对应频率的角度变换$e^{jarg(H(jw))}$(相当于对原始信号进行时域平移),从而得到对应指数信号的复数响应,然后得到系统变换结果<小绿书4-30>



\subsection{模与相位的求解}\index{信号的模和相位!模与相位的求解}

1、傅里叶变换法:求解信号的傅里叶变换,将信号傅里叶变换化简成模乘相位的形式

2、利用信号性质:例如实偶信号变换后还是实偶信号,实偶信号的相位为0,这就可以很快判断出信号的相位(如果信号并不是实偶的形式,可以通过平移变换之类的化简成实偶信号,然后再求解原信号的相位即可);对于实奇信号也是类似的

3、实数信号相位:实数信号相位可以为$0$也可以为$\pi$,对应到时域为系数$1$和$-1$;(易忽略点)

\section{连续傅里叶变换求解}\index{连续傅里叶变换求解}



\subsection{连续非周期信号:(一般信号,导数信号,冲激(阶跃)信号,抽象信号,有限三角信号)}\index{连续傅里叶变换求解!连续非周期信号:(一般信号,导数信号,冲激(阶跃)信号,抽象信号,有限三角信号)}

1、注意使用傅里叶变换的性质来简化计算(注意形式的拆分和组合)

2、冲激信号和一般信号:使用傅里叶变换对(分析公式)求解

3、有限三角信号:转换成指数形式,运用积分(分析公式)求解

4、抽象信号(给出已知变换参照的):一般用已知信号通过性质的变换求解

5、运用连续傅里叶变换的对偶性(理解和推导常用的变换对)(记忆矩形时域和矩形频域的变换)

注意:分段函数判定运用积分性质的条件(积分性质前提:负无穷为0,正无穷为常数)



\subsection{连续周期信号:(一般三角信号、常数信号、单位阶跃)(一般不收敛)}\index{连续傅里叶变换求解!连续周期信号:(一般三角信号、常数信号、单位阶跃)(一般不收敛)}

1、一般三角信号换成带系数的三角信号可以利用$sin$和$cos$的变换进行拼凑(记忆sin和cos的傅里叶变换)

2、常数信号→$1 \leftrightarrow 2\pi \delta(w)$

3、单位阶跃信号的变换:$\frac{1}{j \omega}+\pi \delta(\omega)$,公式记忆(推导:利用时域积分性质和单位冲激的傅里叶变换进行推导)

4、其它周期信号:利用傅里叶级数与傅里叶变换的关系式(对于级数$x(t)=\sum_{k=-\infty}^{+\infty} a_{k} \mathrm{e}^{j k \omega_{0} t}$的傅里叶变换为$X(\mathrm{j} \omega)=\sum_{k=-\infty}^{+\infty} 2 \pi a_{k} \delta\left(\omega-k \omega_{0}\right)$,其中$a_k$是傅里叶级数系数,对于该系数表示,如果含有各种参数,有可能需要对参数的值进行讨论)

5、复杂三角函数:积化和差的运用



\subsection{逆变换求解}\index{连续傅里叶变换求解!逆变换求解}

1、频域分解:利用性质化简,做部分分式展开

2、求解:利用常用的傅里叶变换对(上述的周期或者非周期的傅里叶变换)或者用综合公式求解

注意:因式分解分母可分解成式子乘积或者分母分解成平方和常数的和



\subsection{变换性质运用}\index{连续傅里叶变换求解!变换性质运用}

1、时域卷积:转换为频域相乘

2、时域相乘:转换为频域卷积(莫忘系数$\frac{1}{2 \pi}$!!!)



\subsection{傅里叶变换技巧}\index{连续傅里叶变换求解!傅里叶变换技巧}

1、时域积分形式求解:转换成卷积,使用频域求解(凡是遇到积分的求解,都要想到转成卷积的形式)

\section{希尔伯特变换}\index{希尔伯特变换}

0、希尔伯特变换:研究系统函数的约束特性,对于因果性的系统,系统函数的实部和虚部或者模与辐角之间具有某种制约的特性(即实部$R(w)$被已知的虚部$X(w)$惟一地确定,反过来也一样),这种特性以希尔伯特变换的形式体现(希尔伯特变换对:$R(\omega)=\frac{1}{\pi} \int_{-\infty}^{\infty} \frac{X(\lambda)}{\omega-\lambda} \mathrm{d} \lambda,X(\omega)=-\frac{1}{\pi} \int_{-\infty}^{\infty} \frac{R(\lambda)}{\omega-\lambda} \mathrm{d} \lambda$)(希尔伯特推导核心公式:$\mathscr{F}[h(t)]=\frac{1}{2 \pi}\{\mathscr{F}[h(t)] * \mathscr{F}[u(t)]\}$得到$R(\omega)+\mathrm{j} X(\omega)=\frac{1}{2 \pi}\left\{[R(\omega)+\mathrm{j} X(\omega)] *\left[\pi \delta(\omega)+\frac{1}{\mathrm{j} \omega}\right]\right\}$,实虚对应相等即可推导)

1、积分和卷积形式:$\int_{-\infty}^{\infty}\frac{x(\tau)}{\pi (t-\tau)}d\tau = x(t) * \frac{1}{\pi t}$

2、频域的表示:对于时域的$\frac{1}{\pi t}$,由$u(t)$的傅里叶变换为$\frac{1}{j \omega}+\pi \delta(\omega)$,得到$2u(t)-1 \leftrightarrow \frac{2}{j \omega}$,由傅里叶变换的对偶性,$\frac{1}{j t} \leftrightarrow \pi[2u(-\omega)-1 ]$,所以$\frac{1}{\pi t} \leftrightarrow j[2u(-\omega)-1 ]$(即正频域部分$X(jw)$系数为$-j$,负频域部分$X(jw)$系数为$j$,模在整个范围上都是1)

