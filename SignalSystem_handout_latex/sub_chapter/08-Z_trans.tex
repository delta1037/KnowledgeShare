\chapterimage{chapter_head_2.pdf}
\chapter{Z变换}

\section{Z变换求解}\index{Z变换求解}



\subsection{Z变换求解}\index{Z变换求解!Z变换求解}

1、单边级数信号:利用常见的变换对(注意是左半边还是右半边信号)

2、双边级数信号(项中没有绝对值,否则就不是无限级数信号,此时需要分开讨论):利用特征值法求解

3、有限级数信号:定义求解,收敛域可能不包括$z=0$或者$z=\infty$(除非定义求解的结果与$z$无关则是全平面收敛的)

注意:标注$Z$变换的收敛域($|Z|$的形式)

注意:$Z$变换性质的运用(注意平移性质可能会增加零点或者极点(无穷远和$(0,0)$处))

注意:(多阶)零点与(多阶)极点抵消的情况



\subsection{逆系统的Z变换}\index{Z变换求解!逆系统的Z变换}

1、频域:零点和极点互换

2、时域:$x$与$y$互换位置



\subsection{求解单边Z变换}\index{Z变换求解!求解单边Z变换}

1、定义求解(利用Z变换与单边Z变换的关系)

2、时移或者微分后的信号的单边Z变换:使用定义求解,可以使用原信号的级数表示其中的一部分(平移后从负域平移到正域所应该增加的部分,或者从正域平移到负域所应该减少的部分)



\subsection{求解Z逆变换}\index{Z变换求解!求解Z逆变换}

1、部分分式展开:处理有理式(预处理:时移、频移)

2、围线积分法(利用留数定理):(注意$z=0$这个点)大概不考

3、幂级数展开法:右边序列时,多项式按照z的降幂排列($z^{-1}$升幂排);左边序列时,多项式按照z的升幂排列($z^{-1}$降幂排)

4、长除法:求解时域的特定点的值

注意:时移和频移性质的运用



\subsection{收敛和收敛域的确定}\index{Z变换求解!收敛和收敛域的确定}

1、收敛:保证Z变换分析式中级数项的实数部分(取绝对值)值小于1

2、收敛域:所有可以使$Z$变换的值

注意:从负无穷到正无穷的级数分开讨论时不要忘记零点的值

注意:级数展开中是否含有正幂或者负幂项(会导致零点或者无穷远点不收敛)

\section{Z变换收敛域的性质}\index{Z变换收敛域的性质}



\subsection{Z变换收敛域的性质}\index{Z变换收敛域的性质!Z变换收敛域的性质}

1、频域相乘后的收敛域:取两个频域的$ROC$的交集;如果有零点和极点抵消的情况,$ROC$可能会扩大范围

2、右边信号、左边信号、有限信号和双边信号的收敛域特性:左右边信号收敛域与信号一致(左边为圆内,右边为圆外);有限信号收敛域为整个平面(可能不包括$z=0$或者$z=\infty$);双边信号的收敛域为一个圆环形区域,或者不存在

3、因果信号收敛域的特性:$ROC$在某个圆的外面

4、稳定系统(绝对可积)的收敛域特性:$ROC$包含单位圆

5、极点与收敛域的关系:收敛域中不包含任何的极点;极点是收敛域的边界

6、收敛域和有理式确定时域表达式:利用左右信号的$ROC$特性

7、时域有限的信号的收敛域特性:$ROC$为全平面(可能不包括$z=0$或者$z=\infty$)

8、时域信号的级数和为频域的角度为0,幅度为1处的值:如果级数为有限值说明收敛域包括单位圆

9、奇、偶信号的收敛域:由于偶函数信号为双边信号,所以偶函数的收敛域如果存在一定是圆环状,奇函数类似

注意:时移性质的运用

注意:零点不影响Z变换的$ROC$



\subsection{变换后的Z变换收敛域}\index{Z变换收敛域的性质!变换后的Z变换收敛域}

会改变收敛域的性质:

1、频域相乘:$ROC$取两个频域信号的交集,如果存在零点和极点抵消的情况,而且刚好是抵消的决定ROC边界的部分,$ROC$会扩大范围

2、$Z$域尺度变换:$z_0^{n} x[n] \leftrightarrow X\left(\frac{z}{z_0}\right)$,收敛域变为$z_0R$,即常数部分$|z_0|$相当于对原收敛域进行了放缩(对收敛域的范围有影响);指数部分相当于对原来的收敛域做了旋转(对极值和零值的位置有影响)

3、时域反褶:$ROC$发生关于单位圆的对称变换($R^{-1}$,即$R$中$z^{-1}$点的集合)

4、时域扩展变换:插值扩展变换公式$x_{(k)}[n]=\left\{\begin{array}{l}

x[n / k] \\

0

\end{array} \leftrightarrow X\left(z^{k}\right)\right.$,扩展变换会对现有的零点和极点的位置进行指数倍(指数小于0,即$\frac 1 k$)放缩

不会改变收敛域的性质:

1、$Z$域微分:引入新的零点,不改变$ROC$

2、共轭:共轭会对现有的零点和极点分别进行共轭变换

可能会改变收敛域的性质:

1、时域平移:时移可能会导致$z=0$或者$z=\infty$加入到ROC或者从ROC中去除

\section{频率响应与微分方程}\index{频率响应与微分方程}



\subsection{连续信号方程:(拉氏变换)}\index{频率响应与微分方程!连续信号方程:(拉氏变换)}

1、时域频域互转依据:$x^{\prime}(t) \leftrightarrow s X(s)$

2、微分方程和频率响应对照公式:$\sum_{k=0}^{N} a_{k} \frac{\mathrm{d}^{k} y(t)}{\mathrm{d} t^{k}}=\sum_{k=0}^{M} b_{k} \frac{\mathrm{d}^{k} x(t)}{\mathrm{d} t^{k}}$,$\left(\sum_{k=0}^{N} a_{k} s^{k}\right) Y(s)=\left(\sum_{k=0}^{M} b_{k} s^{k}\right) X(s)$,$H(s)=\frac{\left\{\sum_{k=0}^{M} b_{k} s^{k}\right\}}{\left\{\sum_{k=0}^{N} a_{k} s^{k}\right\}}$

3、关于逆系统求解:x与y的位置直接倒换即可(是否需要其它条件判断?)



\subsection{离散信号方程:(Z变换)}\index{频率响应与微分方程!离散信号方程:(Z变换)}

1、时域频域互转依据:$x[k \pm m] \leftrightarrow z^{\pm m} X(z)$

2、微分方程和频率响应对照公式:$\sum_{k=0}^{N} a_{k} y[n-k]=\sum_{k=0}^{M} b_{k} x[n-k]$,$\sum_{k=0}^{N} a_{k} z^{-k} Y(z)=\sum_{k=0}^{M} b_{k} z^{-k} X(z)$,$H(z)=\frac{Y(z)}{X(z)}=\frac{\sum_{k=0}^{M} b_{k} z^{-k}}{\sum_{k=0}^{N} a_{k} z^{-k}}$

3、关于逆系统求解:x与y的位置直接倒换即可



\subsection{连续信号方程:(连续傅里叶变换)}\index{频率响应与微分方程!连续信号方程:(连续傅里叶变换)}

1、时域频域互转依据:$x^{\prime}(t) \leftrightarrow j \omega X(j \omega)$

2、微分方程和频率响应对照公式:$\sum_{k=0}^{N} a_{k} \frac{\mathrm{d}^{k} y(t)}{\mathrm{d} t^{k}}=\sum_{k=0}^{M} b_{k} \frac{\mathrm{d}^{k} x(t)}{\mathrm{d} t^{k}}$,$Y(\mathrm{j} \omega)\left[\sum_{k=0}^{N} a_{k}(\mathrm{j} \omega)^{k}\right]=X(\mathrm{j} \omega)\left[\sum_{k=0}^{M} b_{k}(\mathrm{j} \omega)^{k}\right]$,$H(\mathrm{j} \omega)=\frac{Y(\mathrm{j} \omega)}{X(\mathrm{j} \omega)}=\frac{\sum_{k=0}^{M} b_{k}(\mathrm{j} \omega)^{k}}{\sum_{k=0}^{N} a_{k}(j \omega)^{k}}$

3、关于逆系统求解:x与y的位置直接倒换即可(是否需要其它条件判断?)



\subsection{离散信号方程:(离散傅里叶变换)}\index{频率响应与微分方程!离散信号方程:(离散傅里叶变换)}

1、时域频域互转依据:$x[n \pm n_0] \leftrightarrow e^{\pm j \omega n_0} X\left(e^{j \omega }\right) $

2、微分方程和频率响应对照公式:$\sum_{k=0}^{N} a_{k} y[n-k]=\sum_{k=0}^{M} b_{k} x[n-k]$,$\sum_{k=0}^{N} a_{k} \mathrm{e}^{-\mathrm{j} k \omega} Y\left(\mathrm{e}^{\mathrm{j} \omega}\right)=\sum_{k=0}^{M} b_{k} \mathrm{e}^{-\mathrm{j} k \omega} X\left(\mathrm{e}^{\mathrm{j} \omega}\right)$,$H\left(\mathrm{e}^{\mathrm{j} \omega}\right)=\frac{Y\left(\mathrm{e}^{\mathrm{j} \omega}\right)}{X\left(\mathrm{e}^{\mathrm{j} \omega}\right)}=\frac{\sum_{k=0}^{M} b_{k} \mathrm{e}^{-\mathrm{j} k \omega}}{\sum_{k=0}^{N} a_{k} \mathrm{e}^{-\mathrm{j} k \omega}}$

3、关于逆系统求解:x与y的位置直接倒换即可



