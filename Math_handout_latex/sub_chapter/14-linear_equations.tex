\chapterimage{chapter_head_2.pdf}
\chapter{线性方程组}

\section{方程组的解关系}\index{方程组的解关系}



\subsection{齐次方程判断矩阵秩与解的关系}\index{方程组的解关系!齐次方程判断矩阵秩与解的关系}

1、系数矩阵满秩→零解

2、系数矩阵不是满秩→无穷多解

3、基础解系向量个数 = n - r(A)



\subsection{非齐次方程判断矩阵秩与解的关系}\index{方程组的解关系!非齐次方程判断矩阵秩与解的关系}

1、系数矩阵与增广矩阵秩相等→有解

1.1 满秩→唯一解

1.2 不是满秩→无穷多解(齐次基础解系+特解)

2、系数矩阵与增广矩阵秩不相等→无解



\subsection{齐次和非齐次方程解之间的关系}\index{方程组的解关系!齐次和非齐次方程解之间的关系}

1、非齐次无穷解→齐次有非零解

2、非齐次唯一解→齐次只有零解

注意:反之均不成立(因为有可能无解,齐次只有零解非齐次无解的情况是系数矩阵非方阵时的情况)



\subsection{证明非齐次方程有解}\index{方程组的解关系!证明非齐次方程有解}

1、转换成证明秩关系:证明增广矩阵的秩与系数矩阵的秩相同

\section{求方程组的解}\index{求方程组的解}



\subsection{齐次方程,求解通解(基础解系)}\index{求方程组的解!齐次方程,求解通解(基础解系)}

0、参数矩阵为方阵:可以利用$|A| = 0$来讨论参数(防遗漏)

1、对齐次方程的系数矩阵进行初等行变换,求得阶梯矩阵

2、区分自由未知量和独立未知量,将自由未知量逐个赋1,其余自由未知量赋0,求得自由未知量个数的解(即为基础解系);若自由未知量个数为0,那么就只有0解

3、或者(与2对应)将自由未知量设不同字母表示,代入阶梯矩阵得到通解

注意:基础解系是一组向量;通解为基础解系组成的表达式(注意标注$k$是任意常数)

备注:克拉默法则求解非齐次方程(唯一解为$x_i = \frac {|A_i|}{|A|},i=1,2,\cdots, n$,其中$|A_i|$是$|A|$中的第i列元素替换成方程右端的常数项所构成的行列式)克拉默法则求解范围有限(强限制),要求系数行列式不能为0,方程个数需要与未知数个数相同



\subsection{非齐次方程,求解通解}\index{求方程组的解!非齐次方程,求解通解}

0、参数矩阵为方阵:可以利用$|A| = 0$来讨论参数(防遗漏)

1、对非齐次方程的增广矩阵进行初等行变换,求得阶梯矩阵

2、求得对应齐次方程的基础解系

3、将自由未知量取0值(简化计算),求得一个特解(特解的计算,特解只有一个)

4、对应齐次方程的基础解系和特解组成非齐次方程的通解

注意:通解为基础解系和特解组成的表达式(注意标注k是任意常数)



\subsection{抽象方程求解}\index{求方程组的解!抽象方程求解}

1、讨论秩,根据秩确定方程解的情况(通解形式)

2、再根据已知条件求解

\section{两个方程组}\index{两个方程组}



\subsection{求解两个方程组的公共解}\index{两个方程组!求解两个方程组的公共解}

情况一:已知两个方程组的结构

1、将两个方程组联立求解

情况二:已知其中一个方程组的解和另一个方程组

1、将其中一个方程包含多个未知量表示的解代入到另一个方程组,得到新的关于未知量的方程,求解其中未知量的值

情况三:已知两个方程的通解

1、将两个通解的对应部分相等,求得未知量的关系,从而求解未知量



\subsection{判定方程组同解:(这是啥)}\index{两个方程组!判定方程组同解:(这是啥)}

1、同解:方程组一的解是方程组二的解;方程组二的解是方程组一的解

2、利用方程组一的解构成的形式,构造转换成方程组二的形式

注意:反证法的运用

注意:同解方程组的系数矩阵秩相等



\section{方程基础解系}\index{方程基础解系}



\subsection{齐次方程-判断某个向量是否是解向量}\index{方程基础解系!齐次方程-判断某个向量是否是解向量}

1、方法一(简单):行变换求向量的线性表出:基础解系向量与待判断向量一起做初等行变换,判断初等行变换后的待判断向量能否由初等变换后的基础解析向量线性表出(原理是待判断的向量可以由基础解系表出,即非齐次方程的求解问题)

2、方法二:解原方程组(原方程组是具有该基础解系的一个方程组):根据基础解析向量求出原方程组,并代入待判断向量看是不是原方程组的解进行验证



\subsection{已知通解,求解原齐次方程组:(转置调换位置法)}\index{方程基础解系!已知通解,求解原齐次方程组:(转置调换位置法)}

1、构建新的齐次方程:利用矩阵转置的特性(交换解和矩阵的位置)获得以齐次方程组系数为未知量,通解为系数矩阵的新齐次方程,对新的齐次方程求解,得到的基础解系即是原齐次方程组的系数



\subsection{非齐次方程-判断某个向量是否是解向量}\index{方程基础解系!非齐次方程-判断某个向量是否是解向量}

1、转齐次方程判断:先将待判断向量与特解向量相减得向量A,就转换成了判断A是否是非齐次方程对应的齐次方程



\subsection{判断基础解系组合变换后的组合是否是基础解系}\index{方程基础解系!判断基础解系组合变换后的组合是否是基础解系}

1、变换矩阵行列式值:利用变换系数矩阵的行列式是否为0判定(为0说明组合线性相关,不为0说明组合线性无关)

2、反证法:将变换后的组合认定为线性相关的,求解线性系数(从出现较少的基础解入手),当求得的系数全是0时,组合就不是线性相关的

注意:基础解系必须是一组线性无关的向量

\section{方程组的解讨论方程未知参数}\index{方程组的解讨论方程未知参数}



\subsection{齐次方程:(带有未知参数的齐次方程,已知解的个数,判定未知参数的值(或者讨论解的情况对应的未知参数的值))}\index{方程组的解讨论方程未知参数!齐次方程:(带有未知参数的齐次方程,已知解的个数,判定未知参数的值(或者讨论解的情况对应的未知参数的值))}

1、行列式判断参数值:(防遗漏)判断解时,可以使用行列式(系数矩阵是n阶矩阵,否则使用秩来讨论),得到一个包含未知参数的表达式,讨论该表达式的值即可判定齐次方程只有零解(行列式值不为0)或者存在基础解系(行列式值为0)

2、秩判断参数值:将系数矩阵化简成阶梯矩阵形式(注意参数不要做变换的分母),讨论其中参数的值与秩的关系

3、求解具体的基础解系时,需要使用矩阵形式(将行列式为0时的参数条件代入)进行初等行变换化简成阶梯矩阵,求得基础解系

备注:齐次可以用行列式是因为使用行列式能更好地讨论参数(防遗漏)(使用行列式系数矩阵必须是n阶矩阵,否则得用秩来讨论了)

备注:只化简行(初等行变换)将矩阵化简成上三角或者下三角



\subsection{非齐次方程:(带有未知参数的非齐次方程,已知解的个数,判定未知参数的值(或者讨论解的情况对应的未知参数的值))}\index{方程组的解讨论方程未知参数!非齐次方程:(带有未知参数的非齐次方程,已知解的个数,判定未知参数的值(或者讨论解的情况对应的未知参数的值))}

1、将增广矩阵利用高斯消元法(即化简成阶梯矩阵)

2、根据系数矩阵与增广矩阵秩的关系判定有解或无解的情况;然后将其特定的未知参数值代入求得解(代入后可能需要继续进行行变换)

备注:将首列含有未知参数的进行行交换到末行(更好计算)

注意:未知参数不要用除法消去(防止分母为0)

注意:非齐次方程的求解不要当成齐次方程,每一行与解相乘不是等于0而是根据具体情况等于某个具体值的

\section{根据列向量的线性方程组和方程组的通解求解列向量线性组合之后的方程的解}\index{根据列向量的线性方程组和方程组的通解求解列向量线性组合之后的方程的解}



\subsection{观察判断法:(依赖于对现有的关系式观察组合)}\index{根据列向量的线性方程组和方程组的通解求解列向量线性组合之后的方程的解!观察判断法:(依赖于对现有的关系式观察组合)}

1、判断线性组合后的通解形式(应该有几个基础解系)

2、利用已知的通解,构造出所有的满足组合之后的方程组的基础解系和特解

3、将求得的基础解系和特解组合成通解(注意标注k是任意常数)



\subsection{代入求解法}\index{根据列向量的线性方程组和方程组的通解求解列向量线性组合之后的方程的解!代入求解法}

1、由已知方程的列向量和解,将解代入得到列向量之间的关系1

2、设列向量线性组合之后的解(设解),并得到解与列向量之间的关系2

3、对照关系1和关系2得到新解与已知方程解系数的关系(注意关系1或者关系2需要乘以一个系数,类似于平面束方程的构造)

\section{根据非齐次方程的特解(或者特解组合)求解非齐次方程的通解}\index{根据非齐次方程的特解(或者特解组合)求解非齐次方程的通解}



\subsection{求解方法:<880题 P67 三 1>}\index{根据非齐次方程的特解(或者特解组合)求解非齐次方程的通解!求解方法:<880题 P67 三 1>}

1、根据系数矩阵的秩判断线性组合后的通解形式(应该有几个基础解系)

2、线性无关解具有通解的形式,通过将组合(注意这些组合并不是方程组的解)进行再组合,消去其中的特解部分即可得到通解,消去到只保留一个特解(给出的组合可能包含多个特解)时就是方程组的特解

注意:对现有的解组合时,要考虑到有可能组合到零向量的情况(这时零向量无法作为通解存在)(<880题 P67 一 5>))



\subsection{组合方法(可以利用代入齐次表达式证明)}\index{根据非齐次方程的特解(或者特解组合)求解非齐次方程的通解!组合方法(可以利用代入齐次表达式证明)}

特解组合成特解:设$\alpha_1,\alpha_2,\cdots,\alpha_n$是方程组$Ax=b$的解,当$k_1+k_2+\cdots+k_n=1$时,$k_1\alpha_1+k_2\alpha_2+\cdots+k_n\alpha_n $也是$Ax=b$的解

特解组合成通解:设$\alpha_1,\alpha_2,\cdots,\alpha_n$是方程组$Ax=b$的解,当$k_1+k_2+\cdots+k_n=0$时,$k_1\alpha_1+k_2\alpha_2+\cdots+k_n\alpha_n $是$Ax=0$的解

注意:对现有的解组合时,要考虑到有可能组合到零向量的情况(这时零向量无法作为通解存在)

\section{线性方程组与空间几何的关系}\index{线性方程组与空间几何的关系}

1、多个面有一个交点:只有一个值使得方程成立,即非齐次方程有唯一解

2、多个面交于同一条直线:考虑直线的参数方程只有一个变量,即说明非齐次方程对应的齐次方程有一个线性无关的解

3、多点共线:则这些点的值是在同一条直线上的,将直线表示为参数方程的形式,就是一个同解加一个特解,所以这些个点是由两个线性无关的向量组成的,即秩为2(这些点构成的齐次方程组的解的秩也为2)

4、多个面重合:这些面构成的非齐次方程组都是由一个方程组变化(乘常系数)来的(也就是线性相关),所以组成的非齐次方程组秩为1

5、多个面平行:这些面构成的非齐次方程组无解(没有交点),这些面构成的齐次方程是线性相关的(保证面平行)

6、三个面两两相交:三个面没有公共交点说明构成的非齐次方程无解,但是任两个方程有解(通解和特解,即构成一条直线),说明任意两个面有解且是一个通解和一个特解,由于面的方程是具有三个未知数的,所以此时三个面构成的非齐次方程(增广矩阵)秩为3,任意两个面构成的非齐次方程的(系数矩阵和增广矩阵)秩为2

7、两个面平行,另一个面与这两个面相交:三个面没有公共交点说明构成的非齐次方程无解,其中两个面平行说明这两个面的方程的齐次部分是线性相关的,这两个面与另一个面的相交关系分析参考(三个面两两相交)的情况

\section{已知齐次线性方程组只有零解,判断系数矩阵列变换(列组合)之后是否有非零解<复习全书p392>}\index{已知齐次线性方程组只有零解,判断系数矩阵列变换(列组合)之后是否有非零解<复习全书p392>}

求解变换后系数矩阵秩的方法:

1、将变换后的系数矩阵用原齐次方程系数矩阵乘以变换矩阵表示出来(即求解变换矩阵)

2、原系数矩阵构成的齐次方程只有零解,则是可逆的;根据r(AB)=r(B)判断变换后的矩阵是否是可逆的,转换为求解变换矩阵是否是可逆的(对系数变换矩阵进行可逆性判断:判断系数变换矩阵的行列式是否为0)

\section{系数矩阵与系数伴随矩阵的解的关系}\index{系数矩阵与系数伴随矩阵的解的关系}

1、秩的关系:参考伴随矩阵与原矩阵秩的关系

2、解的个数:由$n - r(A)$,推断线性无关解的个数

3、伴随矩阵与行列式的关系:伴随矩阵的行乘矩阵的列是行列式的值或者0(取决于是否是同一行,即代数余子式的性质)

4、如果$A^T=A^*$,则对于伴随矩阵的某一行的解(如果解为$A$的列)的值有类似于$a_{11}A_{11}+a_{21}A_{21}+a_{31}A_{31} = a_{11}^2 + a_{21}^2+a_{31}^2\ge 0$的关系

\section{已知系数的方程组新加一个带有未知系数的方程,与原方程同解,求解未知系数的关系}\index{已知系数的方程组新加一个带有未知系数的方程,与原方程同解,求解未知系数的关系}

同解推出方程组系数之间有关系:

1、转换成非齐次方程求解:未知方程是已知方程的的组合,从而求得未知系数与已知系数之间的组合关系(原理:齐次方程组合之后的解与组合前的解相同,即同解)

\section{线性方程组界的奇葩}\index{线性方程组界的奇葩}

1、如果矩阵A的行元素之和为固定的值,那么设解向量为全1可能有惊喜

2、其它行或列为固定值的一般是加到同一行或者同一列上的思路

\section{向量组与矩阵构成的方程的基础解系等价,求解方程组的基础解系}\index{向量组与矩阵构成的方程的基础解系等价,求解方程组的基础解系}

1、等价:向量组必是矩阵构成的方程的解

2、求解:转换成了由一堆解找一些线性无关的解来组成方程的基础解系(判断方程组的解的结构)



\section{判断列向量是否能由某几个列向量线性表出}\index{判断列向量是否能由某几个列向量线性表出}

1、直接表出:根据现有的方程进行组合(依据通解),将未知向量表出

2、反证法:不能表出时反证法的应用(通常假设能线性表出后得出与现有的条件矛盾(通常是秩))

