\chapterimage{chapter_head_2.pdf}
\chapter{大数定律和中心极限定理}

\section{大数定律和中心极限定理的理解}\index{大数定律和中心极限定理的理解}

1、大数定律:各种类型的分布在随机变量较多时,收敛的值

2、中心极限定理:描述了当同分布的随机变量较多时,总体上会服从正态分布,正态分布的参数与随机变量的均值和方差有关,并且可以转换成标准正态分布(为了方便利用标准正态分布的分位数进行求解)

\section{求解随机变量与期望接近程度}\index{求解随机变量与期望接近程度}

随机变量与期望的接近程度:(粗略估计)

1、切比雪夫不等式:$ P\{|X-\mu| \geqslant \varepsilon\} \leqslant \frac{\sigma^{2}}{\varepsilon^{2}} $,正数 $ \varepsilon $任意,随机变量$ X $具有期望$ E(X)=\mu $,方差$ D(X)=\sigma^{2} $(该式子表示,越偏离中心范围越大,其概率越小)

注意:切比雪夫不等式主要用于估计概率,证明概率不等式

\section{依概率收敛的情况-中心极限定理}\index{依概率收敛的情况-中心极限定理}



\subsection{中心极限定理}\index{依概率收敛的情况-中心极限定理!中心极限定理}

1、棣莫弗一拉普拉斯定理(二项分布):$ \lim _{n \rightarrow \infty} P\left\{\frac{X_{n}-n p}{\sqrt{n p(1-p)}} \leqslant x\right\}=\int_{-\infty}^{x} \frac{1}{\sqrt{2 \pi}} \mathrm{e}^{-\frac{t^{2}}{2}} \mathrm{~d} t=\Phi(x) $(该定理说明二项分布在n充分大时,随机变量经过标准化之后(按照二项分布的期望和方差标准化),接近标准正态分布)

2、列维一林德伯格定理(独立同分布):$ \lim_{n \rightarrow \infty} F_{n}(x)=\lim_{n \rightarrow \infty} P\left\{\frac{\sum_{k=1}^{n} X_{k}-n \mu}{\sqrt{n} \sigma} \leqslant x\right\}=\int_{-\infty}^{x} \frac{1}{\sqrt{2 \pi}} \mathrm{e}^{-\frac{t^{2}}{2}} \mathrm{~d} t=\Phi(x) $,随机变量之间互相独立,服从同一分布,具有相同的均值$ \mu $和方差$ \sigma^2 $(均值和方差需要都存在),则随机变量的标准化分布$ Y_{n}=\frac{\sum_{k=1}^{n} X_{k}-E\left(\sum_{k=1}^{n} X_{k}\right)}{\sqrt{D\left(\sum_{k=1}^{n} X_{k}\right)}}=\frac{\sum_{k=1}^{n} X_{k}-n \mu}{\sqrt{n} \sigma} $成立上式(该定理说明任何同分布在n充分大时,随机变量经过标准化之后(按照同分布的期望和和方差和来进行标准化),接近标准正态分布)



\subsection{利用中心极限定理求解事件的概率}\index{依概率收敛的情况-中心极限定理!利用中心极限定理求解事件的概率}

1、判断事件的类型:二项分布或者独立同分布(问题:有些事件看似不是二项分布但是可以转换成二项分布?三项,求解一项的概率啥的);有可能需要利用和的关系转换为相反的事件的概率(反向操作)

2、求解二项分布或独立同分布的均值和方差

3、求解标准化随机变量的范围(原始范围与标准化之后的范围对照)

4、利用分位数和标准化后的分布的性质(正态分布性质等)求解

\section{随机变量序列依概率收敛的情况-大数定律}\index{随机变量序列依概率收敛的情况-大数定律}



\subsection{大数定律}\index{随机变量序列依概率收敛的情况-大数定律!大数定律}

1、依概率收敛的定义:$ \lim_{n \rightarrow \infty} P\{|X_n-A| \le \varepsilon \}=1 $,对任意$ \varepsilon > 0 $成立

2、切比雪夫大数定律:$ \lim_{n \rightarrow \infty} P\left\{\left|\frac{1}{n} \sum_{i=1}^{n} X_{i}-\frac{1}{n} \sum_{i=1}^{n} E\left(X_{i}\right)\right|<\varepsilon\right\}=1 $,其中$ X_{1}, X_{2}, \cdots, X_{n}, \cdots $是一列相互独立 的随机变量(或者两两不相关), 它们的期望与方差分别为$ E\left(X_{k}\right) $, $ D\left(X_{k}\right)(k=1,2, \cdots) $,并且存在常数$ C $, 使得$ D\left(X_{k}\right) \leqslant C(k=1,2, \cdots) $

3、辛钦大数定律(独立同分布的序列):$ \lim_{n \rightarrow \infty} P\left\{\left|\frac{1}{n} \sum_{k=1}^{n} X_{k}-\mu\right|<\varepsilon\right\}=1 $,数学期望 $ E\left(X_{k}\right)=\mu,(k=1,2, \cdots) $;任意 $ \varepsilon>0 $

4、伯努利大数定律($ n $次独立重复试验中事件发生的次数$ X_n $):$ \lim_{n \rightarrow \infty} P\left\{\left|\frac{X_n}{n}-p\right|<\varepsilon\right\}=1 $或者$ \lim_{n \rightarrow \infty} P\left\{\left|\frac{X_n}{n}-p\right| \geqslant \varepsilon\right\}=0 $,任意$ \varepsilon>0 $(这时随机变量就不是序列的形式了,但是可以将二项分布拆开看作0-1分布)

注意:切比雪夫大数定律并未要求同分布;切比雪夫大数定律是切比雪夫不等式的推论

注意:辛钦大数定律将切比雪夫大数定律推向了特殊,对趋向的值做了化简(要求同分布)

注意:伯努利大数定律将辛钦大数定律推向了特殊,对分布做了特殊要求(可以看作是0-1分布序列)



\subsection{求解事件依概率收敛的值}\index{随机变量序列依概率收敛的情况-大数定律!求解事件依概率收敛的值}

1、根据事件的类型套用相应的大数定律即可

