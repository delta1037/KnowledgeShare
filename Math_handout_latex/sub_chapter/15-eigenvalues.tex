\chapterimage{chapter_head_2.pdf}
\chapter{特征值/特征向量/相似矩阵}

\section{特征值/特征向量/相似矩阵-定义}\index{特征值/特征向量/相似矩阵-定义}

1、特征值,特征向量:特征值之和、特征值之积与矩阵的关系

2、特征方程,特征多项式,特征矩阵:

3、相似对角化,相似标准形:$A \sim B$,$A=P^{-1}BP$,相似标准形是一个对角阵

4、相似矩阵:$A \sim B$,具有反身性,对称性,传递性

5、实对称阵:元素是实数,并且对称$A=A^T$

注意:特征向量必须是非零向量,在大题中如果出现抽象特征向量需要对其进行讨论

\section{特征值和特征向量应用}\index{特征值和特征向量应用}



\subsection{由特征值和特征向量反求矩阵}\index{特征值和特征向量应用!由特征值和特征向量反求矩阵}

1、利用对角化公式$P^{-1}AP=\Lambda$,根据特征向量凑可逆矩阵$P$(注意一一对应和对应关系($A=P\Lambda P^{-1}$))

2、上述方法中构造可逆矩阵P时,如果可以构造成正交矩阵的形式,则可以利用$A=Q\Lambda Q^{-1}=Q\Lambda Q^{T}$,可以简化计算(利用了正交矩阵的性质)

备注:相同特征值的特征向量线性无关;不同特征值的特征向量具有正交的性质(该正交性质并不是实对称矩阵才有的)



\subsection{利用特征值,特征向量及相似矩阵确定参数}\index{特征值和特征向量应用!利用特征值,特征向量及相似矩阵确定参数}

1、定义:利用特征值的定义代入矩阵方程$A\alpha=\lambda \alpha$,使得对应分量相等



\subsection{求解$A^n$与特征向量或者特征向量组合相乘的问题}\index{特征值和特征向量应用!求解$A^n$与特征向量或者特征向量组合相乘的问题}

1、矩阵变换后的特征值:$\begin{array}{c|c|c|c|c|c|c}\hline \text { 矩阵 } & \boldsymbol{A} & k \boldsymbol{A}+\boldsymbol{E} & \boldsymbol{A}+k \boldsymbol{E} & \boldsymbol{A}^{n} & \boldsymbol{A}^{-1}(\boldsymbol{A} \text { 可逆 }) & \boldsymbol{A}^{*}(\boldsymbol{A} \text { 可逆 }) \\\hline \text { 特征值 } & \lambda & k \lambda+1 & \lambda+k & \lambda^{n} & \frac{1}{\lambda} & \frac{|\boldsymbol{A}|}{\lambda} \\\hline\end{array}$

2、由特征值的定义转换成$A\alpha=\lambda \alpha$形式

\section{实对称阵}\index{实对称阵}



\subsection{实对称阵的性质}\index{实对称阵!实对称阵的性质}

1、特征值全部都是实数,并且对应的对角阵的对角线上就是所有的特征值,非零特征值的个数等于矩阵的秩(对于非实对称矩阵的非零特征值个数不一定等于矩阵的秩)

2、属于不同特征值对应的特征向量互相正交(这是由于实对称性质带来的)(怎么证明)

3、实对称阵必相似于对角阵,即必存在可逆阵$P$,使得$P^{-1}AP=\Lambda$,且存在正交阵$Q$使得$Q^{-1}AQ=Q^{T}AQ=\Lambda$,即$P$就是正交阵(正交阵定义:$AA^T=A^TA=E$,即单位阵的各个向量是标准化之后的),所以$P^T=P^{-1}=Q^T=Q^{-1}$(这里将相似与合同关联了起来),所以由特征向量阵$Q$和对角阵求$A$时,$A=Q\Lambda Q^{-1}=Q\Lambda Q^{T}$(使用正交阵的变换可以将逆运算替换为转置运算)



\subsection{实对称阵正交相似于对角阵的步骤}\index{实对称阵!实对称阵正交相似于对角阵的步骤}

1、求解所有的特征值和特征向量(尽量使得各个向量刚好正交,不同特征值的特征向量已经正交,在求解同一个特征值对应的特征向量时就尽量保证不同的特征向量正交(避免正交化:求出第一个特征向量之后,其它特征向量设未知数,利用正交关系求解未知量即可))

2、对应特征值的所有特征向量正交化(不同特征值的特征向量已正交)(备注:取正交的基础解系,跳过施密特正交化,设自由变量让第二个解向量与第一个解向量正交,把第二个解向量代入方程,确定自由变量)

3、全部特征向量标准化

4、将n个单位正交向量合并成正交矩阵(即待求解的量),对应的特征值组成对角矩阵(即待求解的量)

注意:前提必须是实对称矩阵



\subsection{线性无关向量得到标准正交向量组的方法}\index{实对称阵!线性无关向量得到标准正交向量组的方法}

1、设向量组$\boldsymbol{\alpha}_{1}, \boldsymbol{\alpha}_{2}, \boldsymbol{\alpha}_{3}$线性无关

2、令$\boldsymbol{\beta}_{1}=\boldsymbol{\alpha}_{1}$;$\boldsymbol{\beta}_{2}=\boldsymbol{\alpha}_{2}-\frac{\left(\boldsymbol{\alpha}_{2}, \boldsymbol{\beta}_{1}\right)}{\left(\boldsymbol{\beta}_{1}, \boldsymbol{\beta}_{1}\right)} \boldsymbol{\beta}_{1}$;$\boldsymbol{\beta}_{3}=\boldsymbol{\alpha}_{3}-\frac{\left(\boldsymbol{\alpha}{3}, \boldsymbol{\beta}_{1}\right)}{\left(\boldsymbol{\beta}_{1}, \boldsymbol{\beta}_{1}\right)} \boldsymbol{\beta}_{1}-\frac{\left(\boldsymbol{\alpha}_{3}, \boldsymbol{\beta}_{2}\right)}{\left(\boldsymbol{\beta}_{2}, \boldsymbol{\beta}_{2}\right)} \boldsymbol{\beta}_{2}$

3、则$\boldsymbol{\beta}_{1}, \boldsymbol{\beta}_{2}, \boldsymbol{\beta}_{3}$两两正交

4、再将$\boldsymbol{\beta}_{1}, \boldsymbol{\beta}_{2}, \boldsymbol{\beta}_{3}$单位化, 取$\boldsymbol{\gamma}_{1}=\frac{\boldsymbol{\beta}_{1}}{\left|\boldsymbol{\beta}_{1}\right|}, \boldsymbol{\gamma}_{2}=\frac{\boldsymbol{\beta}_{2}}{\left|\boldsymbol{\beta}_{2}\right|}, \boldsymbol{\gamma}_{3}=\frac{\boldsymbol{\beta}_{3}}{\left|\boldsymbol{\beta}_{3}\right|}$



\section{求解特征值和特征向量}\index{求解特征值和特征向量}



\subsection{矩阵形式-求解方案一}\index{求解特征值和特征向量!矩阵形式-求解方案一}

1、由$|\lambda E - A|=0$求出所有的特征值

2、对每一个特征值建立齐次线性方程组$(\lambda _i E - A)x = 0$,求出对应特征值的特征向量

技巧:由于一个特征值必有一个特征向量,系数矩阵$(\lambda _i E - A)$初等行变换后必有一行值为零,所以可以将任意一行改成0然后求解



\subsection{矩阵形式-求解方案二}\index{求解特征值和特征向量!矩阵形式-求解方案二}

1、利用定义$A\alpha=\lambda \alpha$的数值$\lambda$即是A的特征值,$\alpha$是A对应的特征向量(一般应用于抽象矩阵,或者元素为文字的矩阵)

备注:若$A\alpha_i=0=0\alpha_i$,则$\alpha_i$是对应$\lambda=0$的特征向量,若$\alpha_i$是齐次方程组$Ax=0$的基础解系,则$\alpha_i$进一步还是线性无关的特征向量(故当$|A|=0$时,$Ax=0$有解,$A$有特征值$\lambda=0$)



\subsection{证明具有相同特征值}\index{求解特征值和特征向量!证明具有相同特征值}

1、证明具有相同的特征方程$|\lambda E-A|=|\lambda E-B|$

2、



\subsection{抽象矩阵的特征值}\index{求解特征值和特征向量!抽象矩阵的特征值}

1、若矩阵A满足某个条件,则应该联想到特征值$\lambda$满足的条件,获取到特征值$\lambda$的范围(矩阵关系对应到特征值关系)

2、矩阵变换后的特征值:$\begin{array}{c|c|c|c|c|c|c}\hline \text { 矩阵 } & \boldsymbol{A} & k \boldsymbol{A}+\boldsymbol{E} & \boldsymbol{A}+k \boldsymbol{E} & \boldsymbol{A}^{n} & \boldsymbol{A}^{-1}(\boldsymbol{A} \text { 可逆 }) & \boldsymbol{A}^{*}(\boldsymbol{A} \text { 可逆 }) \\\hline \text { 特征值 } & \lambda & k \lambda+1 & \lambda+k & \lambda^{n} & \frac{1}{\lambda} & \frac{|\boldsymbol{A}|}{\lambda} \\\hline\end{array}$

3、



注意:对于特征向量的表示,全体特征向量为带有常系数$k$的一个向量,即其个数是无限的;如果一个特征值具有两个特征向量,则用$k_1$和$k_2$表示常系数,并且需要在后边说明$k_1$和$k_2$不同时为0(即先求出来一个特征向量,然后加上常数就扩展到了全体特征向量上)

注意:对于特征值和特征向量的描述,要一一对应,并且特征向量描述为全体特征向量

\section{矩阵相似}\index{矩阵相似}



\subsection{矩阵相似的必要条件:(快速判定是否相似)(并不能由这些条件得到矩阵相似)}\index{矩阵相似!矩阵相似的必要条件:(快速判定是否相似)(并不能由这些条件得到矩阵相似)}

1、类特征多项式:$\lambda E-A\sim \lambda E-B$,其中$\lambda$为实数,特别的当$\lambda$是特征值时也成立;$|\lambda E-A|=|\lambda E-B|$(特征多项式相等)

2、秩的关系:$r(A)=r(B)$

3、特征值的关系:$A,B$具有相同的特征值(利用定义来证明)(反之具有相同的特征值不一定相似(就是非充分条件)(重根的情况导致的(<复习全书 P414 例 5 评注>)))

4、特征值与特征向量判定:

4.1、实对称矩阵的相似判定:$A,B$是同阶实对称矩阵,且$A \sim B \Leftrightarrow A,B$具有相同的特征值及重数 

4.2、不能化为相似对角阵的矩阵判定:对于三阶矩阵来说,若两个矩阵的特征值相同,并且特征值对应的线性无关的特征向量个数相同,则它们相似(对于四阶和四阶以上的不成立)(三阶矩阵相似充要条件)(即不能化为相似对角阵的两个矩阵也可能相似)

5、行列式的关系:$|A|=|B|$,行列式值等于特征值之积

6、矩阵的特征:$A,B$对角线的和相等(迹相等):$tr(A)=tr(B)$

7、相似传递性:$A,B$相似于同一个对角矩阵(也是充分条件)

注意:是必要条件(相似的推论)



\subsection{相似于对角阵的条件:(矩阵可相似对角化的判定)(充分或者充要条件)(单个矩阵)}\index{矩阵相似!相似于对角阵的条件:(矩阵可相似对角化的判定)(充分或者充要条件)(单个矩阵)}

1、实对称矩阵(或者实对称式子组合等)必相似于对角阵(充分)

2、有n个互不相等的特征值(充分)

3、每个r重特征值具有r个线性无关的特征向量(充要)

4、有n个线性无关的特征向量(充要)

5、若$n$阶矩阵$\boldsymbol{A}$有一个非零特征值,且$r(\boldsymbol{A})=1$(两个条件缺一不可,因为如果$r(\boldsymbol{A})=1$不一定会有一个非零特征值),则矩阵$\boldsymbol{A}$一定相似于对角矩阵

注意:条件“A有n个不同的特征值”与“A为实对称矩阵”均能推出“A有n个线性无关的特征向量”和“A的每个特征值对应的线性无关的特征向量的个数等于该特征值的重数”,但它们之间并没有相互蕴含的关系



\subsection{相似对角化A为对角矩阵}\index{矩阵相似!相似对角化A为对角矩阵}

1、求解所有的特征值和特征向量

2、利用特征向量构造可逆矩阵P使得对角矩阵为对应的特征值排列(注意一一对应关系)



\subsection{相似杂例}\index{矩阵相似!相似杂例}

1、证明$AB \sim BA$:$AB = EAB = B^{-1}BAB = B^{-1}(BA)B$(恒成立的,B可逆是前提)

2、

