\chapterimage{chapter_head_2.pdf}
\chapter{无穷级数}

\section{单个级数判断敛散性}\index{单个级数判断敛散性}



\subsection{正项级数判别方法}\index{单个级数判断敛散性!正项级数判别方法}

1、比较判别法(两个级数):若$0 \leqslant u_{n} \leqslant v_{n}$,则$\sum_{n=1}^{\infty} v_{n}$收敛$\Rightarrow \sum_{n=1}^{\infty} u_{n}$收敛;$\sum_{n=1}^{\infty} u_{n}$发散$\Rightarrow \sum_{n=1}^{\infty} v_{n}$发散

2、比较判别法极限形式(两个级数):设$\lim_{n \rightarrow \infty} \frac{u_{n}}{v_{n}}=l \quad(0 \leqslant l \leqslant+\infty)$,若$0<l<+\infty$,则$\sum_{n=1}^{\infty} u_{n}$与$\sum_{n=1}^{\infty} v_{n}$同敛散;若$l=0$,则$\sum_{n=1}^{\infty} v_{n}$收敛$\Rightarrow \sum_{n=1}^{\infty} u_{n}$收敛;若$l=+\infty$,则$\sum_{n=1}^{\infty} v_{n}$发散$\Rightarrow \sum_{n=1}^{\infty} u_{n}$发散

3、比值判别法:设$\lim_{n \rightarrow \infty} \frac{u_{n+1}}{u_{n}}=\rho$,则$\sum_{n=1}^{\infty} u_{n} \begin{cases}\text { 收敛, } & \text { 当 } \rho<1 \text { 时, } \\ \text { 发散, } & \text { 当 } \rho>1 \text { 时, } \\ \text { 不确定, } & \text { 当 } \rho=1 \text { 时. }\end{cases}$

4、根植判别法:设$\lim_{n \rightarrow \infty} \sqrt[n]{u_{n}}=\rho$,则$\sum_{n=1}^{\infty} u_{n} \begin{cases}\text { 收敛, } & \text { 当 } \rho<1 \text { 时, } \\ \text { 发散, } & \text { 当 } \rho>1 \text { 时, } \\ \text { 不确定, } & \text { 当 } \rho=1 \text { 时. }\end{cases}$

5、积分判别法:设$f(x)$是$[c,+\infty)$上单调减,非负的连续函数,且$a_{n}=f(n)$,则$\sum_{n=c}^{\infty} a_{n}$与$\int_{c}^{+\infty} f(x) \mathrm{d} x$同敛散(这里将级数直接转换为积分)

6、使用定义判别:设有数列$\left\{u_{n}\right\}$,则称$\sum_{n=1}^{\infty} u_{n}=u_{1}+u_{2}+\cdots+u_{n}+\cdots$为无穷级数。令$S_{n}=u_{1}+u_{2}+\cdots+u_{n}(n=1,2, \cdots)$,则称数列$\left\{S_{n}\right\}$为级数$\sum_{n=1}^{\infty} u_{n}$的部分和数列,如果部分和数列$ S_{n}$有极限$S$,即$\lim_{n \rightarrow \infty} S_{n}=S$,则称级数$\sum_{n=1}^{\infty} u_{n}$收敛,这时极限$S$叫作级数$\sum_{n=1}^{\infty} u_{n}$的和;如果$\left\{S_{n}\right\}$没有极限,则称级数$\sum_{n=1}^{\infty} u_{n}$发散

注意:比值判别法是充分非必要的



\subsection{正项级数判断关键点}\index{单个级数判断敛散性!正项级数判断关键点}

0、首先判断通项:如果通项不满足极限收敛的必要条件可直接判定为发散

1、方法选择:当通项中出现$a^n$,$n^n$时一般用根值判别法;当通项中出现$n!$时一般使用比值法;如果通项比较复杂可以用比较判别法(放缩);比值法判断不出来的可以用根值法判断

2、化简方法:无穷小的代换(注:等价等于同敛散性(比较判别法);当极限不存在时不能用等价)(即极限的求解,或者利用泰勒公式放缩);去掉前边不规则的项然后进行判断(只考虑n充分大的情况,前边的有限项不影响级数的敛散性);复杂项先进行放缩(观察各项的特征);基本不等式$2ab \le a^2 + b^2$放缩变换(两项相乘或者分式形式均可使用,可以将同一个级数分解为两个级数);

3、与已知对比:常用的p级数($\sum_{n=1}^{\infty} \frac 1{n^p}$,p大于1时收敛,p小于等于1时发散,转为积分时也成立)(利用积分判别法证明)和等比级数($\sum_{n=1}^{\infty} a{q^n}$)(利用级数的求解证明)

4、边界值判断:比值(根值)判别法,当比值(根值)等于1时,根据比值(根值)趋于1的趋势进行判断比值(根值)是大于1还是小于1的<复习全书P270 例2 (2)>



\subsection{交错级数判断敛散性}\index{单个级数判断敛散性!交错级数判断敛散性}

1、莱布尼茨判别准则:若(1)$u_{n} \geqslant u_{n+1}(n=1,2, \cdots)$(单调递减);(2)$\lim_{n \rightarrow \infty} u_{n}=0$(趋于零),则级数$\sum_{n=1}^{\infty}(-1)^{n-1} u_{n}$收敛

2、偶数项和与奇数项和:考虑极限当n趋近于$\infty$时$S_{2N}$是否等于$S_{2N+1}$,相等就是收敛(还没有遇到实际题目)

3、求绝对收敛:如果级数绝对收敛,则级数收敛

4、对级数加括号:加括号以后的级数收敛,则级数收敛



\subsection{任意项级数判断敛散性}\index{单个级数判断敛散性!任意项级数判断敛散性}

1、预处理技巧:求绝对收敛(绝对收敛的级数一定收敛),然后利用正向级数的化简方法(等价,放缩,不等式,等);

注意:如果绝对值级数发散,不能断定原级数发散,此时要判定原级数的性质或者定义

\section{组合级数判断敛散性}\index{组合级数判断敛散性}



\subsection{级数关系的敛散性判别}\index{组合级数判断敛散性!级数关系的敛散性判别}

1、相乘:相乘常用不等式$|u_nv_n| \le \frac{1}{2}(u_n^2+v_n^2)$;

2、相加:两个发散的正项级数相加还是发散的(如果没有提到正项条件则结果是不确定的);两个收敛的级数相加减还是收敛的

3、绝对值:如果$\sum_{n=0}^{\infty}u_n$发散,则$\sum_{n=0}^{\infty}|u_n|$发散(反证法:否则绝对收敛会推出原级数收敛);

4、代入法:特殊级数($\sum_{n=0}^{\infty}\frac{1}{n}$或者$\sum_{n=0}^{\infty}(-1)^n·\frac{1}{n}$或者$\sum_{n=0}^{\infty}\frac{1}{2n}$;$\sum_{n=0}^{\infty}\frac{1}{\sqrt n}$)

5、判别法:利用与现有的级数敛散性进行比较判别法

6、其它:当正项级数$\sum_{n=1}^{\infty} u_{n}$收敛时,有$\lim_{n \rightarrow \infty} u_{n}=0$,则当$n$充分大时,$\left|u_{n}\right|=u_{n}<1$(有界),故$u_{n}^{2}<u_{n}$,由比较审敛法,知$\sum_{n=1}^{\infty} u_{n}^{2}$收敛;若$\sum_{n=1}^{\infty}(-1)^{n-1} u_{n}\left(u_{n}>0\right)$收敛,则$\sum_{n=1}^{\infty}\left(u_{2 n-1}-u_{2 n}\right)$收敛(级数加括号);若$\sum_{n=1}^{\infty} u_{n}^{2}$收 敛,则$\sum_{n=1}^{\infty}(-1)^{n-1} \frac{u_{n}}{n}$绝对收 敛,因$\left|(-1)^{n-1} \frac{u_{n}}{n}\right| \leqslant \frac{1}{2}\left(u_{n}^{2}+\frac{1}{n^{2}}\right)$,而$\sum_{n=1}^{\infty} \frac{1}{n^{2}}$收敛,由比较审敛法,可知$\sum_{n=1}^{\infty}(-1)^{n-1} \frac{u_{n}}{n}$绝对收敛



\subsection{条件收敛和绝对收敛组合判断}\index{组合级数判断敛散性!条件收敛和绝对收敛组合判断}

1、绝对收敛+-绝对收敛=绝对收敛

2、绝对收敛+-条件收敛=条件收敛

3、条件收敛+-条件收敛=条件或绝对收敛



\subsection{已知级数与级数的相乘的极限,求解级数的敛散性}\index{组合级数判断敛散性!已知级数与级数的相乘的极限,求解级数的敛散性}

1、定义找关系:从定义出发,找到两个级数的部分和的关系

2、比较判别法:寻找有界性条件,对待验证的级数进行放缩,得到可以利用已知条件证明的级数,最后利用比较判别法证明放缩前的级数收敛

3、比较极限判别:将已知极限的式子转换成待求式子与已知敛散性式子相除的形式,利用比较判别法(例子:已知$na_n$收敛,求解${a_n}^2$的极限,将$na_n$转换成$\frac{{a_n}^2}{ 1/n^2 }$)



\subsection{已知级数形式收敛证明另一种形式(同一个未知量/级数)敛散性:(看不懂了,待补充)}\index{组合级数判断敛散性!已知级数形式收敛证明另一种形式(同一个未知量/级数)敛散性:(看不懂了,待补充)}

1、求出未知级数/量

2、找出已知级数形式与未知级数形式的(可能需要部分和展开)关系(将已知的级数进行必要的转换(收敛的数列项存在上界->放大))

3、对未知级数进行转换和放缩



备注:区分数列收敛与级数收敛,数列收敛是数列趋近于一个值,级数收敛是数列的级数和趋近于一个值

\section{级数关系}\index{级数关系}



\subsection{不同级数的组合}\index{级数关系!不同级数的组合}

1、递推不等式或者不同级数之间的关系



\subsection{同一个级数项之间的关系}\index{级数关系!同一个级数项之间的关系}

1、不等式放缩、放缩、单调有界

\section{级数综合}\index{级数综合}



\subsection{级数与极限}\index{级数综合!级数与极限}

1、证明极限构成的级数的收敛性

2、由收敛级数的无穷项为0(级数收敛的必要条件),求得极限为0



\subsection{级数与函数}\index{级数综合!级数与函数}

1、看到导数想到中值定理!

2、构造零点问题(零点定理)

3、高阶连续导数想到泰勒公式展开(注意选取导数为0的点)

4、利用函数表达式对级数的通项做估计(不等式运用)



\subsection{级数与常微分方程}\index{级数综合!级数与常微分方程}

1、利用现有关系求解级数的和函数

2、利用级数在特定点的值求解待定系数

\section{幂级数性质}\index{幂级数性质}

1、四则运算:和差,$\sum_{n=0}^{\infty} a_{n} x^{n} \pm \sum_{n=0}^{\infty} b_{n} x^{n}=\sum_{n=0}^{\infty}\left(a_{n} \pm b_{n}\right) x^{n}=S_{1}(x) \pm S_{2}(x), x \in(-R, R)$;积,$\left(\sum_{n=0}^{\infty} a_{n} x^{n}\right)\left(\sum_{n=0}^{\infty} b_{n} x^{n}\right)=\sum_{n=0}^{\infty}\left(a_{0} b_{n}+a_{1} b_{n-1}+\cdots+a_{n} b_{0}\right) x^{n}=S_{1}(x) S_{2}(x), x \in(-R, R) $;商,设$b_0 \ne 0$,$\frac{S_{1}(x)}{S_{2}(x)}=\frac{\sum_{n=0}^{\infty} a_{n} x^{n}}{\sum_{n=0}^{\infty} b_{n} x^{n}}=c_{0}+c_{1} x+\cdots+c_{n} x^{n}+\cdots$,其中$c_{n}$由$\left(\sum_{n=0}^{\infty} b_{n} x^{n}\right) \cdot\left(\sum_{n=0}^{\infty} c_{n} x^{n}\right)=\sum_{n=0}^{\infty} a_{n} x^{n}$来确定(其中收敛半径$R$为参与运算的级数的收敛半径的最小值)

2、分析性质:设幂级数$\sum_{n=0}^{\infty} a_{n} x^{n}$的收敛半径为$R>0$,和函数为$S(x)$,则(1)$S(x)$在$(-R, R)$上连续;(2)$S(x)$在$(-R, R)$上可导,且可逐项求导,即$S^{\prime}(x)=\sum_{n=1}^{\infty} n a_{n} x^{n-1}$(原幂级数含常数项求导会使项数减少);(3)$S(x)$在$(-R, R)$内可积, 且可逐项积分,即$\int_{0}^{x} S(t) \mathrm{d} t=\sum_{n=0}^{\infty} \int_{0}^{x} a_{n} t^{n} \mathrm{~d} t=\sum_{n=0}^{\infty} \frac{a_{n}}{n+1} x^{n+1} \quad(x \in(-R, R))$(注意这里用的定积分,并且下限是$0$,上限是$x$);(积分和求导后均具有与原幂级数相同的收敛半径)(在利用幂级数的分析性质时需要用收敛区间讨论,因为端点上的收敛性可能在求导或者积分时发生变化,在收敛区间上讨论完之后,再由和函数在闭区间上的连续性得到端点上的收敛性)(求导后端点的收敛性可能失去,积分后端点的收敛性可能增加)(使用分析性质时一定要在收敛区间上进行讨论)

\section{幂级数求解}\index{幂级数求解}



\subsection{求解幂级数的收敛域-非缺项级数}\index{幂级数求解!求解幂级数的收敛域-非缺项级数}

1、求解收敛区间(比值法和根值法):比值法,设当$n$充分大时$a_{n} \neq 0$,并设$\lim_{n \rightarrow \infty}\left|\frac{a{n+1}}{a_{n}}\right|=\rho$,则(1)若$\rho=+\infty$,则$R=0$;(2) 若$\rho=0$,则$R=+\infty$;(3) 若$0<\rho<+\infty$,则$R=\frac{1}{\rho}$;根值法,$\lim_{n \rightarrow \infty} \sqrt[n]{\mid a_{n} |}=\rho$,结论同上(当比值法求出不来时可以用根值法(解决一些振荡系数的问题$\sum_{n=1}^{\infty}\left[2-(-1)^{n}\right] x^{n}$),根值法存在推不出来比值法存在,比值法存在可以推出根值法存在)(根据具体级数的情况选择比值法或者根值法,选择方法和正项级数求解敛散性选择方法一致)

2、考虑收敛区间端点处的敛散性,得出收敛域(收敛区间是左右对称的开区间,收敛域是收敛区间加上收敛的端点值组成的区间)

注意:求解收敛区间所用的比值法和根值法是充分非必要的(与级数判断敛散性一致)

备注:幂级数在某一点条件收敛,则该点是收敛域的端点;当某一点在收敛域的内部时,该点绝对收敛;幂级数在对称的两点收敛性不同,则该点就是收敛域的端点



\subsection{求解幂级数的收敛域-缺项级数}\index{幂级数求解!求解幂级数的收敛域-缺项级数}

1、使用级数的比值判别法,求解相邻两项的比值,即$\lim _{n \rightarrow \infty}\left|\frac{u_{n+1}(x)}{u_{n}(x)}\right|$,并使得其比值小于1的$x$的范围(比值法求不出来试试根值法,与非缺项级数的处理类似)

2、考虑收敛区间端点处的敛散性,得出收敛域



\subsection{已知某幂级数的收敛域求解另一级数收敛域}\index{幂级数求解!已知某幂级数的收敛域求解另一级数收敛域}

1、级数乘以x的有限次幂,收敛域不会改变;逐项求导与逐项积分,收敛区间不会改变

2、利用级数在指定区间内的估计值,求解该区间内另一个级数的收敛半径<880 P46 二(2)>



\subsection{求解幂级数的和函数:(可以求解常数项级数)}\index{幂级数求解!求解幂级数的和函数:(可以求解常数项级数)}

1、根据幂级数的代数运算

2、利用变量替换

3、利用常见的函数的麦克劳林展开式

4、先逐项求导, 再利用$S(x)=S(0)+\int_{0}^{x} S^{\prime}(t) \mathrm{d} t$(注意从导数积分回去有一个$S(0)$)

5、先逐项积分,再利用$S(x)=\left(\int_{0}^{x} S(t) \mathrm{d} t\right)^{\prime}$(先积分再求导;区别上述先求导再积分,即常数的问题)

6、先通过幂级数的代数运算、逐项求导、逐项积分等转化为关于和函数$S(x)$的微分方程问题,再求解方程

注意:判断有没有除0造成的简断点(等于0处等);结果中要包含收敛域的全部范围(可能去掉端点)

注意:对于和函数不存在的点需要额外排除,在0点可能有$0^0=1$;

注意:逐项求导和逐项积分均需要在收敛域上讨论

备注:和函数求出来时候利用原级数特殊点对和函数的正确性进行验证



\subsection{求解幂级数的极限}\index{幂级数求解!求解幂级数的极限}

1、常数项级数求和:利用级数定义求解部分和$S_n$,然后求极限得级数和

2、幂级数求和:求解幂级数的和函数(常见得展开式经过一系列的有理运算、逐项求导和逐项积分)(幂级数和函数操作时注意判断函数的定义域),得出和函数在某点处的值(极限)

注意:这里将求数列和极限与级数关联了起来

\section{幂级数展开}\index{幂级数展开}



\subsection{将函数展开为幂级数}\index{幂级数展开!将函数展开为幂级数}

1、间接法:利用常用的几个麦克劳林展开式,通过变量代换和幂级数性质(四则运算、逐项求导和逐项积分)

2、直接法:求解函数在$x_0$处的各阶导数,写出在$x=x_0$处的泰勒级数,并考察泰勒级数的余项极限是否为0

注意:标注幂级数的收敛区间(不是收敛域)



\subsection{常用的麦克劳林展开式}\index{幂级数展开!常用的麦克劳林展开式}

1、$\frac{1}{1-x}=1+x+x^{2}+\cdots+x^{n}+\cdots, x \in(-1,1)$;

2、$\frac{1}{1+x}=1-x+x^{2}+\cdots+(-1)^{n} x^{n}+\cdots, x \in(-1,1)$;

3、$\mathrm{e}^{x}=1+x+\frac{x^{2}}{2 !}+\cdots+\frac{x^{n}}{n !}+\cdots, x \in(-\infty,+\infty)$;

4、$\sin x=x-\frac{x^{3}}{3 !}+\cdots+\frac{(-1)^{n} x^{2 n+1}}{(2 n+1) !}+\cdots, x \in(-\infty,+\infty)$;

5、$\cos x=1-\frac{x^{2}}{2 !}+\cdots+\frac{(-1)^{n} x^{2 n}}{(2 n) !}+\cdots, x \in(-\infty,+\infty)$;

6、$\ln (1+x)=x-\frac{x^{2}}{2}+\cdots+\frac{(-1)^{n-1} x^{n}}{n}+\cdots, x \in(-1,1]$;

7、$(1+x)^{\alpha}=1+\alpha x+\frac{\alpha(\alpha-1)}{2 !} x^{2}+\cdots+\frac{\alpha(\alpha-1) \cdots(\alpha-n+1)}{n !} x^{n}+\cdots, x \in(-1,1)$;

注意:以上1-6后边的区间均为收敛域,7的仅为收敛区间,收敛域要视$\alpha$的情况而定

注意:当求解包含函数与级数的组合时的级数时,先统一转换成函数,再由常用展开式或者分析性质展开

\section{求解函数的傅里叶级数的和函数}\index{求解函数的傅里叶级数的和函数}



\subsection{傅里叶级数的和函数}\index{求解函数的傅里叶级数的和函数!傅里叶级数的和函数}

1、利用敛散性定理处理边界点和间断点(区分求解和函数和求解傅里叶展开式)

2、敛散性定理:狄里克雷定理:除有限个第一类间断点外都连续;只有有限个极值点,则$f(x)$的傅里叶级数在$[-\pi, \pi]$上处处收敛,且收敛于(1)$f(x)$, 当$x$为$f(x)$的连续点;(2)$ \frac{f\left(x^{-}\right)+f\left(x^{+}\right)}{2}$,当$x$为$f(x)$的间断点;(3)$\frac{f\left(-\pi^{+}\right)+f\left(\pi^{-}\right)}{2}$,当$x=\pm \pi$



\subsection{傅里叶级数表示}\index{求解函数的傅里叶级数的和函数!傅里叶级数表示}

1、傅里叶级数三角级数:$f(x) \sim \frac{a_{0}}{2}+\sum_{n=1}^{\infty}\left(a_{n} \cos n x+b_{n} \sin n x\right)$

2、傅里叶级数是一种均方逼近($f(x) - \frac{a_{0}}{2}+\sum_{n=1}^{\infty}\left(a_{n} \cos n x+b_{n} \sin n x\right)$的平方在区间内积分最小)



\subsection{将函数展开为傅里叶级数:(周期为$2\pi$)}\index{求解函数的傅里叶级数的和函数!将函数展开为傅里叶级数:(周期为$2\pi$)}

1、$[-\pi, \pi]$上展开:$\left\{\begin{array}{l}

a_{n}=\frac{1}{\pi} \int_{-\pi}^{\pi} f(x) \cos n x \mathrm{~d} x, \quad n=0,1,2, \cdots \\

b_{n}=\frac{1}{\pi} \int_{-\pi}^{\pi} f(x) \sin n x \mathrm{~d} x, \quad n=1,2, \cdots

\end{array}\right.$

2、$[-\pi, \pi]$上奇偶函数的情况:$f(x)$为奇函数时$\begin{cases}a_{n}=0, & n=0,1, \cdots \\ b_{n}=\frac{2}{\pi} \int_{0}^{\pi} f(x) \sin n x \mathrm{~d} x, & n=1,2, \cdots\end{cases}$,$f(x)$为偶函数时$\begin{cases}a_{n}=\frac{2}{\pi} \int_{0}^{\pi} f(x) \cos n x \mathrm{~d} x, & n=0,1,2, \cdots \\ b_{n}=0, & n=1,2, \cdots\end{cases}$

3、$[0, \pi]$上展为正弦或展为余弦级数:正弦展开为$\begin{cases}a_{n}=0, & n=0,1,2, \cdots \\ b_{n}=\frac{2}{\pi} \int_{0}^{\pi} f(x) \sin n x \mathrm{~d} x, & n=1,2, \cdots\end{cases}$;余弦展开为$\begin{cases}a_{n}=\frac{2}{\pi} \int_{0}^{\pi} f(x) \cos n x \mathrm{~d} x, & n=0,1,2, \cdots \\ b_{n}=0, & n=1,2, \cdots\end{cases}$(相当于将半区间延拓成奇函数或者偶函数形式,延拓之后与上述2公式一致)



\subsection{将函数展开为傅里叶级数:(周期为$2l$)}\index{求解函数的傅里叶级数的和函数!将函数展开为傅里叶级数:(周期为$2l$)}

1、$[-l, l]$上展开:$\left\{\begin{array}{l}

a_{n}=\frac{1}{l} \int_{-l}^{l} f(x) \cos \frac{n \pi x}{l} \mathrm{~d} x, \quad n=0,1,2, \cdots \\

b_{n}=\frac{1}{l} \int_{-l}^{l} f(x) \sin \frac{n \pi x}{l} \mathrm{~d} x, \quad n=1,2, \cdots

\end{array}\right.$

2、$[-l, l]$上奇偶函数的情况:$f(x)$为奇函数时$\begin{cases}a_{n}=0, & n=0,1,2, \cdots \\ b_{n}=\frac{2}{l} \int_{0}^{l} f(x) \sin \frac{n \pi x}{l} \mathrm{~d} x, & n=1,2, \cdots\end{cases}$,$f(x)$为偶函数时$\begin{cases}a_{n}=\frac{2}{l} \int_{0}^{l} f(x) \cos \frac{n \pi x}{l} \mathrm{~d} x, & n=0,1,2, \cdots \\ b_{n}=0, & n=1,2, \cdots\end{cases}$

3、$[0, l]$上展为正弦或展为余弦级数:正弦展开为$\begin{cases}a_{n}=0, & n=0,1,2, \cdots \\ b_{n}=\frac{2}{l} \int_{0}^{l} f(x) \sin \frac{n \pi x}{l} \mathrm{~d} x, & n=1,2, \cdots\end{cases}$(奇延拓);余弦展开为$\begin{cases}a_{n}=\frac{2}{l} \int_{0}^{l} f(x) \cos \frac{n \pi x}{l} \mathrm{~d} x, & n=0,1,2, \cdots \\ b_{n}=0, & n=1,2, \cdots\end{cases}$(偶延拓)

注意:标注定义域

