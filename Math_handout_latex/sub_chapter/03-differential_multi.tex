\chapterimage{chapter_head_2.pdf}
\chapter{多元微分}

\section{函数的特征}\index{函数的特征}



\subsection{连续性}\index{函数的特征!连续性}

1、求解二元函数在一点的极限值(参考求解二重极限)



\subsection{偏导存在与连续}\index{函数的特征!偏导存在与连续}

1、一点处偏导的存在性:定义法判断偏导数存在性

2、偏导在一点的连续性:求偏导在一点的极限,与定义值相比较判断偏导数在此点的偏导连续性



\subsection{可微性}\index{函数的特征!可微性}

1、利用可微的定义:考察$f_{x}^{\prime}\left(x_{0}, y_{0}\right)和f_{y}^{\prime}\left(x_{0}, y_{0}\right)$是否都存在,如果$f_{x}^{\prime}\left(x_{0}, y_{0}\right)$和$f_{y}^{\prime}\left(x_{0}, y_{0}\right)$中至少有一 个不存在,则函数$f(x, y)$在$\left(x_{0}, y_{0}\right)$不可微;如果$f_{x}^{\prime}\left(x_{0}, y_{0}\right)$和$f_{y}^{\prime}\left(x_{0}, y_{0}\right)$都存在,考察$\lim_{\Delta x \rightarrow 0,\Delta y \rightarrow 0}\frac{\left[f\left(x_{0}+\Delta x, y_{0}+\Delta y\right)-f\left(x_{0}, y_{0}\right)\right]-\left[f_{x}^{\prime}\left(x_{0}, y_{0}\right) \Delta x+f_{y}^{\prime}\left(x_{0}, y_{0}\right) \Delta y\right]}{\rho}=0$(可微定义式)是否成立,其中$\rho=\sqrt{(\Delta x)^{2}+(\Delta y)^{2}}$,如果该式成立,则称函数$f(x, y)$在点$(x_0, y_0)$处可微(可微的充要条件)(可微的定义可以将微分的概念($\Delta z=f(x+\Delta x, y+\Delta y)-f(x, y)$或者$\Delta z=A \Delta x+B \Delta y+o(\rho)$)和偏导即可微的必要条件($\mathrm{d} z=\frac{\partial z}{\partial x} \mathrm{~d} x+\frac{\partial z}{\partial y} \mathrm{~d} y$)关联起来,这里好像就有$A=\frac{\partial z}{\partial x},B=\frac{\partial z}{\partial y}$的嫌疑)

2、利用可微的必要条件:可微函数必可导,换言之,不可导的函数一定不可微(点处导数定义不存在)(判断二元函数不可微)

3、利用可微的充分条件:有连续一阶偏导数的函数一定可微(点处偏导数定义值等于偏导在该点的极限值)(判断二元函数可微)

4、利用微分的概念:函数$z=f(x, y)$在点$(x, y)$处的全增量$\Delta z=f(x+\Delta x, y+\Delta y)-f(x, y)$可以表示为$\Delta z=A \Delta x+B \Delta y+o(\rho)$,其中$A,B$不依赖于$\Delta x, \Delta y$,而仅与$x, y$有关,$\rho=\sqrt{(\Delta x)^{2}+(\Delta y)^{2}}$(也可以证明全增量$\Delta z=f(x+\Delta x, y+\Delta y)-f(x, y)$是$\rho=\sqrt{(\Delta x)^{2}+(\Delta y)^{2}}$的无穷小量,即$\lim_{\Delta x \rightarrow 0,\Delta y \rightarrow 0}\frac{f\left(x_{0}+\Delta x, y_{0}+\Delta y\right)-f\left(x_{0}, y_{0}\right)}{\rho}=0$,此时$A=0,B=0$)



\subsection{极值与驻点的定义}\index{函数的特征!极值与驻点的定义}

1、多元函数极值点定义:类似于领域内$f(x, y) \leqslant f\left(x_{0}, y_{0}\right)$(极大值)(注意可以取等号)(极值点除了需要考虑驻点外,还要考虑导数不存在的点)

2、多元函数驻点定义:使$f_{x}^{\prime}(x, y)=0, f_{y}^{\prime}(x, y)=0$同时成立的点$(x, y)$称为函数$f(x, y)$的驻点(驻点不一定是极值点;极值点不一定是驻点)

3、多元函数极值的必要条件:设函数$f(x, y)$在点$M_{0}\left(x_{0}, y_{0}\right)$的一阶偏导数存在,且在$\left(x_{0}, y_{0}\right)$取得极值,则$f_{x}^{\prime}\left(x_{0}, y_{0}\right)=0, f_{y}^{\prime}\left(x_{0}, y_{0}\right)=0$(由此可见具有一阶偏导数的函数的极值点一定是驻点,但驻点不一定是极值点)



\subsection{求解最大最小值}\index{函数的特征!求解最大最小值}

1、求出所有的极值点(求驻点判断极值点)和边界上的最大最小值

2、比较这些极值和边界上的最大最小值,其中最大的为最大值,最小的为最小值(废话)

注意:如果区域内不存在驻点,那么最值在边界上取(转换为条件极值问题,条件为边界)

备注:有的最值问题可以从几何上进行考虑,转换为几何关系

备注:对于闭环内的极值就是对应的最值(例如在一个圆环(线)型定义域上,如果在线上取得极大值或者极小值,那么该极大值或者极小值就是对应的最大值和最小值,这是由于圆线型定义域上函数光滑(处处可导)得到的)

\section{一些几何证明问题}\index{一些几何证明问题}

1、证明曲面为柱面:只需证明曲面上任一点的切平面平行于一条定直线,即证明曲面上任一点的法向量垂直于定向量(设定向量的值,并与曲面法向量相乘,得到利用对应参数相等求解的方程)(即求解这个定向量即可)

\section{泰勒定理}\index{泰勒定理}

1、拉格朗日余项泰勒展开式:$f(x, y)=f\left(x_{0}, y_{0}\right)+f_{x}^{\prime}\left(x_{0}, y_{0}\right)\left(x-x_{0}\right)+f_{y}^{\prime}\left(x_{0}, y_{0}\right)\left(y-y_{0}\right)+R_{1}$,其中余项$R_{1}=\frac{1}{2 !}[\frac{\partial^{2} f\left(P_{1}\right)}{\partial x^{2}}\left(x-x_{0}\right)^{2}+2 \frac{\partial^{2} f\left(P_{1}\right)}{\partial x \partial y}\left(x-x_{0}\right)\left(y-y_{0}\right)+\frac{\partial^{2} f\left(P_{1}\right)}{\partial y^{2}}\left(y-y_{0}\right)^{2}]$,点$P_{1}$为$\left(x_{0}+\theta\left(x-x_{0}\right), y_{0}+\theta\left(y-y_{0}\right)\right)$

2、佩亚诺型余 项的二阶泰勒公式展开式:$f(x, y)=f\left(x_{0}, y_{0}\right)+f_{x}^{\prime}\left(x_{0}, y_{0}\right)\left(x-x_{0}\right)+f_{y}^{\prime}\left(x_{0}, y_{0}\right)\left(y-y_{0}\right)+\frac{1}{2 !}[\frac{\partial^{2} f\left(x_{0}, y_{0}\right)}{\partial x^{2}}\left(x-x_{0}\right)^{2}+2 \frac{\partial^{2} f\left(x_{0}, y_{0}\right)}{\partial x \partial y}\left(x-x_{0}\right)\left(y-y_{0}\right)+\frac{\partial^{2} f\left(x_{0}, y_{0}\right)}{\partial y^{2}}\left(y-y_{0}\right)^{2}]+o\left(\rho^{2}\right)$,其中$\rho=\sqrt{\left(x-x_{0}\right)^{2}+\left(y-y_{0}\right)^{2}}$

\section{隐函数的偏导数}\index{隐函数的偏导数}

1、由一个方程确定的隐函数(一元函数)求导:设$F(x, y)$有连续一阶偏导数,且$F_{y}^{\prime} \neq 0$ ,则由方程$F(x, y)=0$确定的函数$y=y(x)$可导,且$\frac{\mathrm{d} y}{\mathrm{~d} x}=-\frac{F_{x}^{\prime}}{F_{y}^{\prime}}$

2、由一个方程确定的隐函数(二元函数)求导:设$F(x, y, z)$有连续一阶偏导数,且$F_{z}^{\prime} \neq 0, z=z(x, y)$由方程$F(x, y, z)=0$所确定,则$\frac{\partial z}{\partial x}=-\frac{F_{x}^{\prime}}{F_{z}^{\prime}}, \quad \frac{\partial z}{\partial y}=-\frac{F_{y}^{\prime}}{F_{z}^{\prime}}$

3、由方程组确定的隐函数(一元函数)求导:设$u=u(x), v=v(x)$由方程组$\left\{\begin{array}{l}F(x, u, v)=0,  \\ G(x, u, v)=0\end{array}\right. $所确定,$\frac{\mathrm{d} u}{\mathrm{~d} x}$和$\frac{\mathrm{d} v}{\mathrm{~d} x}$, 可通过原方程组两端对$x$求导得到,即$\left\{\begin{array}{l} F_{x}^{\prime}+F_{u}^{\prime} \frac{\mathrm{d} u}{\mathrm{~d} x}+F_{v}^{\prime} \frac{\mathrm{d} v}{\mathrm{~d} x}=0 \\ G_{x}^{\prime}+G_{u}^{\prime} \frac{\mathrm{d} u}{\mathrm{~d} x}+G_{v}^{\prime} \frac{\mathrm{d} v}{\mathrm{~d} x}=0 \end{array}\right.$,从中解出$\frac{\mathrm{d} u}{\mathrm{~d} x}$和$\frac{\mathrm{d} v}{\mathrm{~d} x}$,这里假设由形式解出的$\frac{\mathrm{d} u}{\mathrm{~d} x}$与$\frac{\mathrm{d} v}{\mathrm{~d} x}$中的分母不为零(对$x$求偏导得方程组)

4、由方程组确定的隐函数(二元函数)求导:设$u=u(x, y), v=v(x, y)$由方程组$\left\{\begin{array}{l}F(x, y, u, v)=0, \\ G(x, y, u, v)=0\end{array}\right.$所确定,若要求$\frac{\partial u}{\partial x}$和$\frac{\partial v}{\partial x}$,可先对 原方程组两端对$x$求偏导得到, 即$\left\{\begin{array}{l} F_{x}^{\prime}+F_{u}^{\prime} \frac{\partial u}{\partial x}+F_{v}^{\prime} \frac{\partial v}{\partial x}=0 \\ G_{x}^{\prime}+G_{u}^{\prime} \frac{\partial u}{\partial x}+G_{v}^{\prime} \frac{\partial v}{\partial x}=0 \end{array}\right.$,然后从中解出$\frac{\partial u}{\partial x}$和$\frac{\partial v}{\partial x}$;同理可求得$\frac{\partial u}{\partial y}$和$\frac{\partial v}{\partial y}$,这里假设由形式解出的式子中的分母不为零(对$x$求偏导得方程组求解关于$x$的偏导;对$y$求偏导得方程组求解关于$y$的偏导)

注意:在求解$F^{\prime}_x$时,对$x$求导过程中,即使遇到$z$,也不求$z^{\prime}_x$(不深入求导),仅求导与$x$相关的变量,其它视为常数;求解$F^{\prime}_y$和$F^{\prime}_z$时也是如此,非求导变量均为常数,即使存在$z=z(x, y)$

备注:将函数关系代入可以得到明确的变量关系(循环迭代式关系函数:$z$是$x,y$的函数,$y$是$x,z$的函数,可以得到$z$是$x$的一元函数)

备注:隐函数求导的常用的三种方法,利用隐函数求导公式;对方程两端求导,解出偏导数;利用微分形式不变性,方程两端求微分

\section{极值与条件极值}\index{极值与条件极值}



\subsection{二元极值的判断条件:(充分条件)}\index{极值与条件极值!二元极值的判断条件:(充分条件)}

设函数$z=f(x, y)$在点$\left(x_{0}, y_{0}\right)$的某邻域内有连续的二阶偏导数,且$f_{x}^{\prime}\left(x_{0}, y_{0}\right)=0, f_{y}^{\prime}\left(x_{0}, y_{0}\right)=0$。令$f_{x x}^{\prime \prime}\left(x_{0}, y_{0}\right)=A, f_{x y}^{\prime \prime}\left(x_{0}, y_{0}\right)=B, f_{y y}^{\prime \prime}\left(x_{0}, y_{0}\right)=C$,则

(1)$A C-B^{2}>0$时,$f(x, y)$在点$\left(x_{0}, y_{0}\right)$取极值,且$\left\{\begin{array}{l}\text { 当 } A>0 \text { 时取极小值, } \\ \text { 当 } A<0 \text { 时取极大值. }\end{array}\right.$

(2)$A C-B^{2}<0$时,$f(x, y)$在点$\left(x_{0}, y_{0}\right)$无极值

(3)$A C-B^{2}=0$时,不能确定$f(x, y)$在点$\left(x_{0}, y_{0}\right)$是否有极值,还需进一步讨论(一般用 极值定义)



\subsection{二元极值的判断步骤:(前提:二元函数$z=f(x, y)$有连续二阶偏导数)}\index{极值与条件极值!二元极值的判断步骤:(前提:二元函数$z=f(x, y)$有连续二阶偏导数)}

1、令$f_{x}^{\prime}(x, y)=0, f_{y}^{\prime}(x, y)=0$同时成立,求得所有驻点

2、对每个驻点求出二阶偏导数$A=f_{x x}^{\prime \prime}\left(x_{0}, y_{0}\right), B=f_{x y}^{\prime \prime}\left(x_{0}, y_{0}\right), C=f_{y y}^{\prime \prime}\left(x_{0}, y_{0}\right)$

3、利用极值充分条件,通过$A C-B^{2}$的正负对驻点$\left(x_{0}, y_{0}\right)$作判定



\subsection{二元极值特殊的判断方法}\index{极值与条件极值!二元极值特殊的判断方法}

1、代入特殊路径(反证法/定义法):如果在某个点附近代入曲线($y=x,y=-x$)后,函数值出现了可正可负的情况,则函数不存在极值(常用于极限表示式中)

备注:在求某一点的二阶导时,对于无关的变量可以提前代入化简



\subsection{条件极值求解-化为无条件极值}\index{极值与条件极值!条件极值求解-化为无条件极值}

1、若从条件$\varphi(x, y)=0$中可解出$y=y(x)$或$x=x(y)$

2、代人$z=f(x, y)$化简成无条件极值问题

注意:化为无条件极值适合单条件的情况



\subsection{条件极值求解-拉格朗日乘数法:(多个条件的也是类似的)}\index{极值与条件极值!条件极值求解-拉格朗日乘数法:(多个条件的也是类似的)}

1、函数$f(x, y)$在条件$\varphi(x, y)=0$下的极值的必要条件,先构造拉格朗日函数$F(x, y, \lambda)=f(x, y)+\lambda \varphi(x, y)$

2、解方程组$\left\{\begin{array}{l} \frac{\partial F}{\partial x}=\frac{\partial f}{\partial x}+\lambda \frac{\partial \varphi}{\partial x}=0 \\ \frac{\partial F}{\partial y}=\frac{\partial f}{\partial y}+\lambda \frac{\partial \varphi}{\partial y}=0 \\ \frac{\partial F}{\partial \lambda}=\varphi(x, y)=0 \end{array}\right.$

3、所有满足此方程组的解$(x, y, \lambda)$中$(x, y)$是函数$f(x, y)$在条件$\varphi(x, y)=0$下可能的极值点

注意:求解时注意利用方程中变量的对称性(不要过度依赖因为容易漏解,有时间还是按照标准步骤求解,或者根据不等式验证结果或者根据几何进行预分析)

注意:使用拉格朗日乘数法时对原函数进行简化(相乘的转化为对数和,带根号的可以去掉根号)

备注:对于三元函数的条件极值的求解类似

备注:对于实际条件极值的问题求解,注意区分条件和函数(函数可能需要根据实际情况进行构造,或者做一些问题转化(转化为距离问题等))

备注:对于不等式问题,从不等式中拆出来条件函数(设条件函数为定值)和待求函数,在条件函数为定值的情况下求解待求函数的范围



\section{二元函数的极限和连续}\index{二元函数的极限和连续}



\subsection{定义与性质}\index{二元函数的极限和连续!定义与性质}

1、重极限的定义:定义式为$0<\sqrt{\left(x-x_{0}\right)^{2}+\left(y-y_{0}\right)^{2}}<\delta$(注意:二元函数的重极限是指定义域$D$中的$P(x, y)$以任何方式趋于点$P_{0}\left(x_{0}, y_{0}\right)$时,函数$f(x, y)$都无限趋近于同一常数$A$。换言之,若点$P(x, y)$沿两种不同路径趋向于点$P_{0}\left(x_{0}, y_{0}\right)$时,$f(x, y)$趋于不同常数,或点$P(x, y)$沿某一路径趋于$P_{0}\left(x_{0}, y_{0}\right)$时,$f(x, y)$的极限不存在,则重极限$\lim_{x \rightarrow x_{0},y \rightarrow y_{0}} f(x, y)$不存在。这是证明重极限不存在常用的有效方法)(重极限的极限运算(有理运算,复合运算)和性质(保号性, 夹逼性, 局部有界性,极限 与无穷小的关系)与一元函数完全类似)

2、二元函数连续的概念:设函数$f(x, y)$在开区域(或闭区域)$D$内有定义,$P_{0}\left(x_{0}, y_{0}\right)$是$D$的内点或边界点,且$P_{0} \in D$,如果$\lim_{x \rightarrow x_{0},y \rightarrow y_{0}} f(x, y)$,则称函数$f(x, y)$在点$P_{0}\left(x_{0}, y_{0}\right)$连续

3、连续函数的性质:

3.1、连续函数的和、差、积、商(分母不为零)均是连续函数,连续函数的复合函数仍为连续函数(连续函数的组合)

3.2、在有界闭区域$D$上连续的函数,在该区域$D$上有最大值和小值(最大最小值定理)

3.3、在有界闭区域$D$上连续的函数,可取到它在该区域上的最小值与最大值之间的任何值(介值定理)

3.4、一切多元初等函数在其定义区域内处处连续,这里的定义区域是指包含在定义域内的区域或闭区域

证明二重极限不存在的方法:

1、取不同的趋近路径时极限不相等

2、某一条趋近路径的极限不存在



\subsection{证明二重极限不存在-找趋近路径}\index{二元函数的极限和连续!证明二重极限不存在-找趋近路径}

1、分子分母齐次:令$y=kx$

2、分子分母化简后齐次:例如若$y$是高次方的,令$y_1=y^2$化简为齐次

3、$\frac{xy}{x+y}$:令$y=-x+x^3$,得到趋近于0的结果是无穷,即该路径不存在



\subsection{求解二重极限}\index{二元函数的极限和连续!求解二重极限}

1、利用极限的性质(如四则运算法则,夹逼原理(放缩))

2、消去分母中极限为零的因子(通常采用有理化,等价无穷小代换等)

3、转化为一元函数极限,利用一元函数求极限方法求解(化一元函数是将可微的一部分替换为单一变量,如$t=x^2+y^2$)

4、利用无穷小量与有界变量(利用不等式判断有界)之积为无穷小量

备注:对于多元极限,可以将式子分离(分解)成有界和无穷小相乘的两个部分

\section{偏导数与全微分}\index{偏导数与全微分}



\subsection{二元函数的偏导数与全微分}\index{偏导数与全微分!二元函数的偏导数与全微分}

1、偏导数的定义: 设函数$z=f(x, y)$在点$\left(x_{0}, y_{0}\right)$的某一邻域内有定义,如果$\lim_{\Delta x \rightarrow 0} \frac{f\left(x{0}+\Delta x, y_{0}\right)-f\left(x_{0}, y_{0}\right)}{\Delta x}$存在,则称此极限为函数$z=f(x, y)$在点$\left(x_{0}, y_{0}\right)$处对$x$的偏导数, 记为$f_{x}^{\prime}\left(x_{0}, y_{0}\right)$;类似地可定义$f_{y}^{\prime}\left(x_{0}, y_{0}\right)=\lim_{\Delta y \rightarrow 0} \frac{f\left(x{0}, y_{0}+\Delta y\right)-f\left(x_{0}, y_{0}\right)}{\Delta y}$(偏导数本质上是一元函数的导数,事实上偏导数$f_{x}^{\prime}\left(x_{0}, y_{0}\right)$就是一元函数$\varphi(x)=f\left(x, y_{0}\right)$在$x=x_{0}$处的导数,即$f_{x}^{\prime}\left(x_{0}, y_{0}\right)=\varphi^{\prime}\left(x_{0}\right)=\left.\frac{\mathrm{d}}{\mathrm{d} x} f\left(x, y_{0}\right)\right|_{x=x_{0}}$,而偏导数$f_{y}^{\prime}\left(x_{0}, y_{0}\right)$也是类似的)(从几何上看就是曲面与$x=x_0,y=y_0$平面的交线在交线方向上的导数)

2、全微分的概念:函数$z=f(x, y)$在点$(x, y)$处的全增量$\Delta z=f(x+\Delta x, y+\Delta y)-f(x, y)$可以表示为$\Delta z=A \Delta x+B \Delta y+o(\rho)$,其中$A,B$不依赖于$\Delta x, \Delta y$,而仅与$x, y$有关,$\rho=\sqrt{(\Delta x)^{2}+(\Delta y)^{2}}$,则称函数$z=f(x, y)$在点$(x, y)$处可微,微分记为$\mathrm{d} z=A \Delta x+B \Delta y$

3、可微的必要条件:如果函数$z=f(x, y)$在点$(x, y)$处可微,则该函数在点$(x, y)$处的偏导数$\frac{\partial z}{\partial x}, \frac{\partial z}{\partial y}$必定存在,且$\mathrm{d} z=\frac{\partial z}{\partial x} \mathrm{~d} x+\frac{\partial z}{\partial y} \mathrm{~d} y$(利用该条件求解微分)(偏导数与微分的关系)

4、可微的充分条件:如果函数$z=f(x, y)$的偏导数$\frac{\partial z}{\partial x}$和$\frac{\partial z}{\partial y}$在点$(x, y)$处连续,则称函数$z=f(x, y)$在点$(x, y)$处可微

5、多元函数连续、可导、可微的关系:多元函数可导既不能推得连续,也不能推得可微(原因在于多元的可导是指一阶偏导数存在,而偏导数是用一元函数极限定义的$\left(f_{x}^{\prime}\left(x_{0}\right.\right., \left.\left.y_{0}\right)=\lim_{x \rightarrow x{0}} \frac{f\left(x, y_{0}\right)-f\left(x_{0}, y_{0}\right)}{x-x_{0}}, f_{y}^{\prime}\left(x_{0}, y_{0}\right)=\lim_{y \rightarrow y{0}} \frac{f\left(x_{0}, y\right)-f\left(x_{0}, y_{0}\right)}{y-y_{0}}\right)$,其动点$\left(x, y_{0}\right)$(或$\left.\left(x_{0}, y\right)\right)$是沿$x$(或$y$) 轴方向趋于$\left(x_{0}, y_{0}\right)$,它只与点$\left(x_{0}, y_{0}\right)$邻域内过该点且平行于两坐标的十字方向函数值有关;而连续($\lim_{(x, y) \rightarrow\left(x_{0}, y_{0}\right)} f(x, y)=f\left(x_{0}, y_{0}\right)$)和可微($f(x, y)-f(x_{0},y_{0})=A\left(x-x_{0}\right)+B\left(y-y_{0}\right)+o(\rho)$)都是用重极限定义的, 其动点$(x, y)$是以任意方式趋于$\left(x_{0}, y_{0}\right)$,它与点$\left(x_{0}, y_{0}\right)$邻域内函数值有关)



\subsection{复合函数的偏导数与全微分}\index{偏导数与全微分!复合函数的偏导数与全微分}

1、复合函数求导:利用变量之间的树形图求导(<复习全书 P171>)

2、全微分形式不变性:对中间变量的微分和对自变量的微分具有同样的形式

3、高阶偏导数:$\frac{\partial^{2} z}{\partial x^{2}}=\frac{\partial}{\partial x}\left(\frac{\partial z}{\partial x}\right)=f_{x x}^{\prime \prime}(x, y)$,$\frac{\partial^{2} z}{\partial x \partial y}=\frac{\partial}{\partial y}\left(\frac{\partial z}{\partial x}\right)=f_{x y}^{\prime \prime}(x, y)$(混合偏导),$\frac{\partial^{2} z}{\partial y \partial x}=\frac{\partial}{\partial x}\left(\frac{\partial z}{\partial y}\right)=f_{y x}^{\prime \prime}(x, y)$(混合偏导),$\frac{\partial^{2} z}{\partial y^{2}}=\frac{\partial}{\partial y}\left(\frac{\partial z}{\partial y}\right)=f_{y y}^{\prime \prime}(x, y)$

4、混合偏导数与求导次序无关:若函数$z=f(x, y)$的两个混合偏导数$\frac{\partial^{2} z}{\partial x \partial y}$和$\frac{\partial^{2} z}{\partial y \partial x}$在点$\left(x_{0}, y_{0}\right)$都连续, 则在$\left(x_{0}, y_{0}\right)$点$\frac{\partial^{2} z}{\partial x \partial y}=\frac{\partial^{2} z}{\partial y \partial x}$

注意:如果待求导函数中的自变量不是求导变量,先对函数进行变量替换(化简),转化成有关求导变量为自变量的函数

注意:对于极坐标函数,隐含有$r=\sqrt{x^2+y^2}$,$r$是极坐标的极径

注意:区分整体求导(将一坨式子当作一个整体求导)和部分求导(对某一个自变量求导)

\section{线面的空间特征(切、法)}\index{线面的空间特征(切、法)}



\subsection{曲线的切线和法平面}\index{线面的空间特征(切、法)!曲线的切线和法平面}

形式一:曲线$\Gamma$的方程为参数式$\left\{\begin{array}{l}x=x(t), \\ y=y(t),  \\ z=z(t),\end{array}\right.$

1、该曲线在其上一点$P_{0}\left(x_{0}, y_{0}, z_{0}\right)$处的切向量为$\tau=\left\{x^{\prime}\left(t_{0}\right), y^{\prime}\left(t_{0}\right), z^{\prime}\left(t_{0}\right)\right\}$(利用切向量构造切线或者法平面方程)

2、该点处的切线方程为$\frac{x-x_{0}}{x^{\prime}\left(t_{0}\right)}=\frac{y-y_{0}}{y^{\prime}\left(t_{0}\right)}=\frac{z-z_{0}}{z^{\prime}\left(t_{0}\right)}$

3、该点处的法平面方程为$\left.x^{\prime}\left(t_{0}\right)\left(x-x_{0}\right)+y^{\prime}\left(t_{0}\right)\left(y-y_{0}\right)+z^{\prime}\left(t_{0}\right)\left(z-k_{0}\right) = 0\right\}$

形式二:曲线$\Gamma$的方程为一般式$\left\{\begin{array}{l}F(x, y, z)=0,\\ G(x, y, z)=0,\end{array}\right.$

1、 则该曲线在点$P_{0}\left(x_{0}, y_{0}, z_{0}\right)$处的切向量为曲面$F(x, y, z)=0$和$G(x, y, z)=0$在该点的法向量$\boldsymbol{n}_{1}$和$\boldsymbol{n}_{2}$的向量积,即切向量 $\tau=\boldsymbol{n}_{1} \times \boldsymbol{n}_{2}$,其中$\begin{aligned} &\boldsymbol{n}_{1}=\left\{F{x}^{\prime}\left(x_{0}, y_{0}, z_{0}\right), F_{y}^{\prime}\left(x_{0}, y_{0}, z_{0}\right), F_{z}^{\prime}\left(x_{0}, y_{0}, z_{0}\right)\right\} \\ &\boldsymbol{n}_{2}=\left\{G_{x}^{\prime}\left(x_{0}, y_{0}, z_{0}\right), G_{y}^{\prime}\left(x_{0}, y_{0}, z_{0}\right), G_{z}^{\prime}\left(x_{0}, y_{0}, z_{0}\right)\right\} \end{aligned}$,记$\boldsymbol{n}_{1} \times \boldsymbol{n}_{2}=\{A, B, C\}$(利用切向量构造切线或者法平面方程)

2、$P_{0}\left(x_{0}, y_{0}, z_{0}\right)$点的切线方程为$\frac{x-x_{0}}{A}=\frac{y-y_{0}}{B}=\frac{z-z_{0}}{C}$

3、$P_{0}\left(x_{0}, y_{0}, z_{0}\right)$点的法平面方程为$A\left(x-x_{0}\right)+B\left(y-y_{0}\right)+C\left(z-z_{0}\right)=0$

形式三:平面曲线$y=f(x)$

1、平面曲线在一点的切向量为$(1,y^{\prime}(x))$,可以用于推导曲率圆的中心



\subsection{曲面的切平面与法线}\index{线面的空间特征(切、法)!曲面的切平面与法线}

形式一:$F(x, y, z)=0$

1、该曲面在点$\left(x_{0}, y_{0}, z_{0}\right)$处的法向量为$\boldsymbol{n}=\left\{F_{x}^{\prime}\left(x_{0}, y_{0}, z_{0}\right), F_{y}^{\prime}\left(x_{0}, y_{0}, z_{0}\right), F_{z}^{\prime}\left(x_{0}, y_{0}, z_{0}\right)\right\}$(利用法向量构造切平面或者法线方程)

2、切平面为$F_{x}^{\prime}\left(x_{0}, y_{0}, z_{0}\right)\left(x-x_{0}\right)+F_{y}^{\prime}\left(x_{0}, y_{0}, z_{0}\right)\left(y-y_{0}\right)+F_{z}^{\prime}\left(x_{0}, y_{0}, z_{0}\right)\left(z-z_{0}\right)=0 $

3、法线方程为$\frac{x-x_{0}}{F_{x}^{\prime}\left(x_{0}, y_{0}, z_{0}\right)}=\frac{y-y_{0}}{F_{y}^{\prime}\left(x_{0}, y_{0}, z_{0}\right)}=\frac{z-z_{0}}{F_{z}^{\prime}\left(x_{0}, y_{0}, z_{0}\right)}$

形式二:$z=f(x, y)$

1、该曲面在点$\left(x_{0}, y_{0}, z_{0}\right)$处的法向量为$\left\{f_{x}^{\prime}\left(x_{0}, y_{0}\right),f_{y}^{\prime}\left(x_{0}, y_{0}\right),-1\right\}$(利用法向量构造切平面或者法线方程)

备注:面的法向量可以归一化之后形成方向导数

注意:求解与指定面平行的平面时,注意要求解的点是在曲面上的



\section{方向导数、梯度、散度、旋度}\index{方向导数、梯度、散度、旋度}



\subsection{方向导数}\index{方向导数、梯度、散度、旋度!方向导数}

0、方向导数定义:二维:$\left.\frac{\partial f}{\partial l}\right|_{\left(x_{0}, y_{0}\right)}$=$\lim _{t \rightarrow 0^{+}} \frac{f\left(x_{0}+t \cos \alpha, y_{0}+t \cos \beta\right)-f\left(x_{0}, y_{0}\right)}{t}$;三维:$\left.\frac{\partial f}{\partial l}\right|_{\left(x_{0}, y_{0}, z_{0}\right)}$=$\newline\lim _{t \rightarrow 0^{+}} \frac{f\left(x_{0}+t \cos \alpha, y_{0}+t \cos \beta, z_{0}+t \cos \gamma\right)-f\left(x_{0}, y_{0}, z_{0}\right)}{t}$(注意$t$是趋近于$0^+$的)(求解未说明可微函数的方向导数或者不可微函数的方向导数)

1、方向导数公式:二维:函数在点处可微,则在该点沿任一方向$l$的方向导数存在,且有$\frac{\partial f}{\partial l}|_{\left(x_{0}, y_{0}\right)}= f_{x}^{\prime}(x_{0},y_{0}) \cos \alpha+f_{y}^{\prime}\left(x_{0}, y_{0}\right) \cos \beta$其中$\cos \alpha, \cos \beta$为方向$l$的方向余弦;三维:$\frac{d f}{\partial l}|{\left(x_{0}, y_{0}, z_{0}\right)} $=$f_{x}^{\prime}\left(x_{0}, y_{0}, z_{0}\right) \cos \alpha$+$f_{y}^{\prime}\left(x_{0}, y_{0}, z_{0}\right) \cos \beta$+$f_{y}^{\prime}\left(x_{0}, y_{0}, z_{0}\right) \cos \beta+f_{z}^{\prime}\left(x_{0}, y_{0}, z_{0}\right) \cos \gamma$=$\newline\left\{f_{x}^{\prime}\left(x_{0}, y_{0}, z_{0}\right), f_{y}^{\prime}\left(x_{0}, y_{0}, z_{0}\right), f_{z}^{\prime}\left(x_{0}, y_{0}, z_{0}\right)\right\} \cdot e$,其中$e=\{\cos \alpha, \cos \beta, \cos \gamma\}$是$l$的方向的单位向量(求解可微函数的方向导数求解方向余弦和各个方向的偏导数即可)(利用方向导数公式可以对向量进行求导,设方向导数的单位向量,并利用方向导数公式展开即可,例如对法向量求导$\frac{\partial u}{\partial \boldsymbol{n}} \mathrm{d} s=\frac{\partial u}{\partial x} \cos (\boldsymbol{n}, x) \mathrm{d} s+\frac{\partial u}{\partial y} \cos (\boldsymbol{n}, y) \mathrm{d} s$,其中$\cos (\boldsymbol{n}, x) ,\cos (\boldsymbol{n}, y)$是法向量的方向余弦;或者类似于$\oiint_{\Sigma} \frac{\partial u}{\partial n} \mathrm{~d} S$=$\oiint_{\Sigma}(\frac{\partial u}{\partial x} \cos \alpha+\frac{\partial u}{\partial y} \cos \beta+\frac{\partial u}{\partial z} \cos \gamma) \mathrm{d} S$)

注意:一个点的所有方向的方向导数都存在,但是不一定可微(例如圆锥的顶点,属于尖锐的点)



\subsection{梯度}\index{方向导数、梯度、散度、旋度!梯度}

1、二维梯度公式:$\operatorname{grad} u(x, y)=\frac{\partial u}{\partial x} i+\frac{\partial u}{\partial y} j$

2、三维梯度公式:$\operatorname{grad} u(x, y, z)=\frac{\partial u}{\partial x} i+\frac{\partial u}{\partial y} j+\frac{\partial u}{\partial z} k$

3、方向导数与梯度的关系:$\left.\frac{\partial u}{\partial{l}}\right|_{P}=\left.\operatorname{grad} u\right|_{P} \cdot \boldsymbol{e}_{l}=|\operatorname{grad} u(x_0, y_0, z_0) |cos \theta$,其中$\theta$是梯度方向与$e_l$的夹角(用于求解方向导数)(当方向与梯度方向相同时,函数增长最快;当方向与梯度方向相反时,函数减少最快;当方向与梯度方向正交时,函数在该方向的变化率为0)

注意:梯度是一个向量(注意二维和三维的表示,基向量表示和括号表示),表示下降最快的方向,即梯度方向的方向导数最大,最大值等于该点梯度向量的模

注意:梯度结果不能化简



\subsection{散度}\index{方向导数、梯度、散度、旋度!散度}

1、散度公式:设有向量场$\boldsymbol{A}(x, y, z)=P(x, y, z) \boldsymbol{i}+Q(x, y, z) \boldsymbol{j}+R(x, y, z) \boldsymbol{k}$,其中$P,Q,R$有连续的一阶偏导,则向量场在一点的散度为$\operatorname{div} \boldsymbol{A}=\frac{\partial P}{\partial x}+\frac{\partial Q}{\partial y}+\frac{\partial R}{\partial z}$

注意:散度的表示方法-一个数值



\subsection{旋度}\index{方向导数、梯度、散度、旋度!旋度}

1、旋度公式:设有向量场$\boldsymbol{A}(x, y, z)=P(x, y, z) \boldsymbol{i}+Q(x, y, z) \boldsymbol{j}+R(x, y, z) \boldsymbol{k}$,其中$P,Q,R$有连续的一阶偏导,则向量场在一点的旋度为$\operatorname{rot} \boldsymbol{A}=\left|\begin{array}{ccc}\boldsymbol{i} & \boldsymbol{j} & \boldsymbol{k} \\\frac{\partial}{\partial x} & \frac{\partial}{\partial y} & \frac{\partial}{\partial z} \\P & Q & R\end{array}\right|$

注意:对于具体的点处的各种度求值,记得代入具体点的值

