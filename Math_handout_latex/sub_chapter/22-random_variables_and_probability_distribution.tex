\chapterimage{chapter_head_2.pdf}
\chapter{随机变量及其概率分布}

\section{随机变量与分布函数}\index{随机变量与分布函数}



\subsection{随机变量与分布函数定义}\index{随机变量与分布函数!随机变量与分布函数定义}

1、随机变量:样本空间$ \Omega $上的实值函数$ X=X(\omega), \omega \in \Omega $,则称$ X(\omega) $为随机变量,简记为$ X $(和概率定义一样也是将样本空间映射成一个函数形式,将事件转换成了值(这个值是坐标轴中的横坐标),相当于是为每个$ <font color=purple>\omega \in \Omega</font> $指定了一个实数$ <font color=purple>X(\omega)</font> $,实值函数是根据具体的事件进行定义的,连续随机变量和离散随机变量的区别就是具体的事件可以定义成连续值还是可以定义为离散的值,比如抛几次硬币正面的次数(离散);飞镖到圆心的距离(连续))

2、分布函数:对任意实数x,分布函数定义为$ F(x) = P\{ X \le x\}, -\infty < x < +\infty $(随机变量$ X $是一个值,分布函数为随机变量$ X $小于某一个自变量$ x $值时的概率值)

注意:分布函数是含有等号的



\subsection{分布函数性质:(判定分布函数)}\index{随机变量与分布函数!分布函数性质:(判定分布函数)}

1、负无穷为0,正无穷为1

2、单调非减,右连续($ F(x)=F(x^+) $),左不连续($  F(x^-) \ne F(x) , F(x^-) \le F(x)  $)(左不连续才导致某一个点处概率不为0,这是由点处的概率定义得到的)

3、$ P\{ x_1 < x \le x_2\} = F(x_2)-F(x_1) $

4、$ P\{X=x\} = F(x)-F(x^-) $,$ F(x) $在$ x $点连续时,$ P\{X=x\} = 0 $;即只有左不连续时,说明该点存在概率,否则概率为零



\subsection{离散型}\index{随机变量与分布函数!离散型}

1、离散型随机变量:随机变量取值为有限多个或者可数无穷个

2、概率分布:变量可能取值为$ x_1,x_2,\cdots,x_n,\cdots $,$ X $取各个可能值得概率为$ P\{ X = x_k \} = p_k, k=1,2,\cdots $,可用用列表给出(一维列表、二维列表)

3、性质:每个取值得概率大于等于0,总概率和为1(用来验证求解的结果)



\subsection{连续型}\index{随机变量与分布函数!连续型}

1、对随机变量$ X $得分布函数$ F(x) $存在一个非负的可积函数$ f(x) $使得$ F(x)=\int_{-\infty}^xf(t)dt,-\infty <x<+\infty $,称$ X $为连续型随机变量,$ f(x) $为$ X $的概率密度

2、性质(概率密度的充要条件):概率密度大于等于0;概率密度全域积分为1(用来验证求解的结果)

3、$ P(x_1 < X \le x_2) = \int_{x_1}^{x_2}f(t)dt $(求解指定范围的概率)

4、在$ f(x) $的连续点处有$ F^{'}(x)=f(x) $(可以由分布函数求解概率密度)

5、$ P(x_1 < X < x_2) = P(x_1 < X \le x_2) = P(x_1 \le X < x_2) = P(x_1 \le X \le x_2) $(连续性随机变量范围区间是否闭合不影响变量范围的概率,因为对于连续型每一个具体点的概率都是0)

6、概率密度函数上某一个点的值是概率在该点的变化率,而不是概率的值

注意:连续型随机变量的分布函数一定可以表示成$ <font color=purple>F(x)=\int_{-\infty}^xf(t)dt</font> $,所以这时的$ <font color=purple>**F(x)**</font> $一定是全域上的连续函数,也就是不连续的$ <font color=purple>**F(x)**</font> $一定不是连续随机变量;反之,不能说连续的$ <font color=purple>F(x)</font> $对应的$ <font color=purple>X</font> $一定是连续型随机变量,只有有概率密度的随机变量才称作连续性随机变量(例如:这里需要补充反例)

注意:连续型随机变量的$ <font color=purple>**F(x)**</font> $一定连续,但是$ <font color=purple>**f(x)**</font> $不一定是连续的(例如:分布函数含有尖锐的点)



\section{常见的分布的性质}\index{常见的分布的性质}



\subsection{指数分布的无记忆性}\index{常见的分布的性质!指数分布的无记忆性}

1、$ P\{X>t\}=\int_{t}^{+\infty} \lambda \mathrm{e}^{-\lambda t} \mathrm{~d} t=\mathrm{e}^{-\lambda t}, t>0 $

2、$ P\{X>t+s \mid X>s\}=\frac{P\{X>t+s\}}{P\{X>s\}}=\frac{\mathrm{e}^{-\lambda(t+s)}}{\mathrm{e}^{-\lambda s}}=\mathrm{e}^{-\lambda t}=P\{X>t\}, t, s>0 $(利用定义式推导)

备注:求解指数分布的条件概率时可能用的到



\subsection{正态分布性质}\index{常见的分布的性质!正态分布性质}

1、标准化形式:$ F(x)=\Phi\left(\frac{x-\mu}{\sigma}\right) $

2、范围求解:$ P\{a<X \leqslant b\}=\Phi\left(\frac{b-\mu}{\sigma}\right)-\Phi\left(\frac{a-\mu}{\sigma}\right), a<b $

3、性质:概率密度 $ f(x) $ 关于 $ x=\mu $ 对称, $ \varphi(x) $ 是偶函数(根据图像可以推导)

4、性质:$ \Phi(-x)=1-\Phi(x), \Phi(0)=\frac{1}{2} $(根据图像可以推导)

5、性质:当 $ X \sim N(0,1) $, 有 $ P\{|X| \leqslant a\}=2 \Phi(a)-1 $(根据图像可以推导)

6、性质:若$ X_{1} \sim N\left(\mu_{1}, \sigma_{1}^{2}\right), X_{2} \sim N\left(\mu_{2}, \sigma_{2}^{2}\right) $,$ X_{1} $与$ X_{2} $相互独立,则$ a X_{1}+b X_{2} \sim N\left(a \mu_{1}+b \mu_{2}, a^{2} \sigma_{1}^{2}+b^{2} \sigma_{2}^{2}\right) $(正态分布的组合性质)

\section{离散型随机变量的函数分布}\index{离散型随机变量的函数分布}



\subsection{(离散)函数分布的定义}\index{离散型随机变量的函数分布!(离散)函数分布的定义}

1、设X的分布律为$ P\{X=x_k\}=p_k,k=1,2,\cdots $,则X的函数$ Y=g(X) $的分布律为$ P\{Y=g(x_k)\}=p_k,k=1,2,\cdots $,如果在$ g(x_k) $中有相同的数值,则将它们相应的概率和作为$ Y $取该值的概率(合并相同的值)



\subsection{(离散)函数分布的求解}\index{离散型随机变量的函数分布!(离散)函数分布的求解}

1、将X的值代入到转换函数中,得到新的值Y,并于原X的概率值进行对应

2、将相同的Y值得概率合并

3、得到新的Y和对应得概率值

\section{常用的分布}\index{常用的分布}



\subsection{常见的分布}\index{常用的分布!常见的分布}

1、0-1分布:1重伯努利试验

2、二项分布:n重伯努利试验,$ n $次独立重复试验中成功的次数;$ X \sim B(n,p) $

3、几何分布:独立重复试验,每次试验成功率为$ p $,则在k次试验时首次试验才成功的概率,$ P\{X=k\}=p q^{k-1}, k=1,2, \cdots $

4、超几何分布:N件产品含有M件次品,从中取$ n $件(不放回抽取),令事件$ X $为抽取的$ n $件产品含有的次品个数,则$ X $服从参数为$ n $,$ M $,$ N $的超几何分布,$ P\{X=k\}=\frac{\mathrm{C}_{M}^{k} \mathrm{C}_{N-M}^{n-k}}{\mathrm{C}_{N}^{n}}, k=l_{1}, \cdots, l_{2} $

5、泊松分布:$ P(x) = \frac{\lambda^k}{k!}e^{-\lambda},X \sim P(\lambda) $ :(k的范围是0,1,2...)一段时间内电话总机接到的次数,候车的旅客数,保险索赔次数

6、均匀分布:$ X \sim U(a,b) $

7、指数分布:$ f(x) = \begin{cases} \lambda e^{-\lambda x}, & x > 0, \\[5ex] 0, & x \le 0, \end{cases} \ \ \lambda >0 $;$ X \sim E(\lambda) $

8、正态分布:$ f(x)=\frac{1}{\sqrt{{2\pi}}\times\sigma}e^{-\frac{1}{2}(\frac{x-\mu}{\sigma})^2} $;$ X \sim N(\mu,\sigma^2) $



备注:所有的表达式与表示方法要记牢



\subsection{二项分布和泊松分布和之间的联系:(泊松定理,按照大概了解即可)}\index{常用的分布!二项分布和泊松分布和之间的联系:(泊松定理,按照大概了解即可)}

1、$ \lim_{n \rightarrow \infty}np_n=\lambda, n\ge1000,p\le0.1 \rightarrow \lim _{n \rightarrow \infty} C_{n}^{k} p_{n}^{k}\left(1-p_{n}\right)^{n-k}=\frac{\lambda^{k}}{k !} \mathrm{e}^{-\lambda} $,即二项分布转换成泊松分布的条件和形式(对于$ <font color=purple>\lim_{n \rightarrow \infty}np_n=\lambda</font> $中$ <font color=purple>p_n</font> $是一个事件A在实验中出现的概率,它与实验的总数$ <font color=purple>n</font> $有关)

2、应用条件:n比较大,$ \lambda $比较小($ n\ge1000,p\le0.1 $)

注意:分布之间关联参数的关系



\section{连续型随机变量的函数分布}\index{连续型随机变量的函数分布}



\subsection{(连续)函数分布的定义}\index{连续型随机变量的函数分布!(连续)函数分布的定义}

1、公式法:设$ X $是一个具有概率密度$ f_X(x) $的随机变量,又设$ y=g(x) $是单调,导数不为零的可导函数,$ h(y) $为它的反函数,则$ Y=g(X) $的概率密度为$ f_{Y}(y)=\left\{\begin{array}{cc}\left|h^{\prime}(y)\right| f_{X}(h(y)), & \alpha<y<\beta, \\0, & \text { 其他, }\end{array}\right. $其中,$ (\alpha,\beta) $是函数$ g(x) $在$ x $可能取值的区间上的值域(用公式法时,因要求条件较多:单调,可导,导数不为零,反函数存在等,实际求解比较麻烦;)(推导:$ <font color=orange>F_{X}(x)=P\{X \leqslant x\}=\int_{-\infty}^{x} f_{X}(t) \mathrm{d} t, x=h(y) \Rightarrow F_{Y}(y)=P\{Y \leqslant h(y)\}=\int_{-\infty}^{h(y)} f_{X}(t) \mathrm{d} t</font> $,然后求导得到概率密度(加绝对值,因为$ <font color=orange>h(y)</font> $只有单调递增和单调递减两种情况,加绝对值为了防范单调递减的情况))

2、定义法:先求Y的分布函数$ F_{Y}(y)=P\{Y \leqslant y\}=P\{g(X) \leqslant y\}=\int_{g(x) \leqslant y} f_{X}(x) \mathrm{d} x,  $然后利用$ f_{Y}(y)=F_{Y}^{\prime}(y) $ 求解概率密度(用定义法时,实际上就是求积分$ <font color=purple>\int_{g(x) \leqslant y} f_{X}(x) \mathrm{d} x</font> $,只要掌握好y变化的范围,不同范围和不同积分限的求积就不难求得$ <font color=purple>F_Y(y)</font> $)

备注:公式法在于对上限进行转换;定义法在于对随机变量转换



\subsection{(连续)函数分布的求解(公式法)}\index{连续型随机变量的函数分布!(连续)函数分布的求解(公式法)}

1、将反函数$ h(y) $代入到公式即可,$ f_{Y}(y)=\left\{\begin{array}{cc}\left|h^{\prime}(y)\right| f_{X}(h(y)), & \alpha<y<\beta, \\0, & \text { 其他, }\end{array}\right. $

注意:对于转换函数,需要单调,导数不为0,存在反函数(即转换函数是严格单调递增或者递减函数,这是为了保证$ <font color=orange>x</font> $与$ <font color=orange>y</font> $一一对应)



\subsection{(连续)函数分布的求解:(定义法)(连续转连续)}\index{连续型随机变量的函数分布!(连续)函数分布的求解:(定义法)(连续转连续)}

1、根据$ X $变量的范围判断$ Y $的值域(如果转换函数$ y=g(x) $比较复杂,绘制$ X $与$ Y $的函数图像,进行判断,下面简称图像)

2、求解$ Y $变量的分布函数($ F(y)=P\{ Y \leqslant y \} $,$ y $即为$ y=g(x) $图像上的水平线,代表$ y=y_1 $直线)

3、根据$ Y\leqslant y $判断$ g(x) \leqslant y $对应的$ X $的范围(该范围是用$ **y** $来表示的,至少有一端必须是这样)

4、求解$ X $的概率密度在该范围上的积分,得到关于$ y $的函数,即为$ Y $变量的分布函数

5、对$ Y $的分布函数进行求导即可得到$ Y $变量的概率密度

注意:当题目需要求解$ <font color=purple>Y</font> $变量的概率密度时,可以省略第四步的积分过程,利用变上下限积分的求导公式求导(注意下限有一个负号)



\subsection{(连续)函数分布的求解:(定义法)(连续转离散,或者是依据对应关系)<复习全书 P493 例4>}\index{连续型随机变量的函数分布!(连续)函数分布的求解:(定义法)(连续转离散,或者是依据对应关系)<复习全书 P493 例4>}

1、已知$ X $(已知类型或者分布的随机变量)和$ Y $(未知类型或者分布的随机变量)的对应关系式

2、设待求的$ Y $变量在关系式中是离散的几个点,或者连续加离散

3、求解$ Y $的范围($ P\{Y \leqslant  \} $或者$ P\{Y = \} $(离散))对应的$ X $的范围($ P\{X\leqslant \} $的概率

\section{分布函数或者概率密度(分布律)的求解}\index{分布函数或者概率密度(分布律)的求解}



\subsection{具体事件的概率分布和分布函数}\index{分布函数或者概率密度(分布律)的求解!具体事件的概率分布和分布函数}

1、判断事件所属的分布类型

2、套用分布类型求解

3、如果是正态分布求解一般会将正太分布标准化



\subsection{分布律(概率分布)的未知参数}\index{分布函数或者概率密度(分布律)的求解!分布律(概率分布)的未知参数}

1、连续概率密度性质:在全域上大于0 AND 全域上积分为1(概率密度的充要条件)(可以用来判断复合函数是否是概率密度)

2、离散分布律性质:在全域上大于0 AND 全域上概率和为1(分布律的充要条件)

3、分布函数性质:单调递增,负无穷为0,正无穷为1,右连续($ <font color=purple>F(x)=F(x^+)</font> $,一般适用于分段函数间断位置求解,看清分段函数等号的位置,是否可以用右连续的条件)



\subsection{分布律求分布函数:(离散)}\index{分布函数或者概率密度(分布律)的求解!分布律求分布函数:(离散)}

1、离散情况表示成分段函数的形式(不要忘记两边的两项)

注意:正无穷远为1(即分段函数有一项条件大于指定值的值为1);类似必有一项负无穷远为0;不要忘记这两项



\subsection{概率密度求分布函数:(连续)}\index{分布函数或者概率密度(分布律)的求解!概率密度求分布函数:(连续)}

1、对概率密度求变上限的积分

