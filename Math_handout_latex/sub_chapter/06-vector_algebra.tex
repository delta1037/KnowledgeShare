\chapterimage{chapter_head_2.pdf}
\chapter{向量代数与空间几何}

\section{向量相关}\index{向量相关}



\subsection{基本概念}\index{向量相关!基本概念}

1、向量:

2、向量的模:

3、向量的坐标及坐标表示:

4、零向量:

5、单位向量:$a^{\circ}=\frac{a}{|a|}=\left\{\frac{a_{x}}{\sqrt{a_{x}^{2}+a_{y}^{2}+a_{z}^{2}}}, \frac{a_{y}}{\sqrt{a_{x}^{2}+a_{y}^{2}+a_{z}^{2}}}, \frac{a_{z}}{\sqrt{a_{x}^{2}+a_{y}^{2}+a_{z}^{2}}}\right\}$(向量的单位化)

6、两向量的夹角:夹角的范围是$[0,\pi]$

7、向量的方向余弦:向量与$x,y,z$轴的夹角为$\alpha, \beta, \gamma$,称$\cos \alpha, \cos \beta, \cos \gamma$为$a$的方向余弦,且$a^{\circ}=\{\cos \alpha, \cos \beta, \cos \gamma\}, \cos ^{2} \alpha+\cos ^{2} \beta+\cos ^{2} \gamma=1$(方向余弦是与坐标轴的夹角)(方向余弦是针对单个向量来说的)



\subsection{向量的运算}\index{向量相关!向量的运算}

1、加减和数乘:

2、数量积(点积、内积):$\boldsymbol{a} \cdot \boldsymbol{b}=|\boldsymbol{a}||\boldsymbol{b}| \cos \theta$,其中$\theta$为$\boldsymbol{a}$与$\boldsymbol{b}$的夹角

3、向量积(叉积、外积):$\boldsymbol{a} \times \boldsymbol{b}=\left|\begin{array}{ccc}\boldsymbol{i} & \boldsymbol{j} & \boldsymbol{k} \\a_{x} & a_{y} & a_{z} \\b_{x} & b_{y} & b_{z}\end{array}\right|$,模$|\boldsymbol{c}|=|\boldsymbol{a}||\boldsymbol{b}| \sin \theta$,其中$\theta$为$\boldsymbol{a}$与$\boldsymbol{b}$的夹角,方向由右手规则从$\boldsymbol{a}$转向$\boldsymbol{b}$来确定,垂直于$\boldsymbol{a}$与$\boldsymbol{b}$所决定的平面;向量积具有分配律;$|\boldsymbol{c}|=|\boldsymbol{a}||\boldsymbol{b}| \sin \theta$也是以向量$a,b$为边的平行四边形的面积

4、混合积:$(a b c)=\left|\begin{array}{lll}a_{x} & a_{y} & a_{z} \\b_{x} & b_{y} & b_{z} \\c_{x} & c_{y} & c_{z}\end{array}\right|$;具有轮换对称性(行列式的行交换性)



\subsection{向量运算的应用}\index{向量相关!向量运算的应用}

1、求解向量的模:

2、向量与向量的夹角:

3、平行四边形面积:

4、平行六面体体积:混合积的模

5、向量之间垂直、平行、三向量共面:三向量混合积的值为0(共面)



\subsection{易错点:(易错)}\index{向量相关!易错点:(易错)}

1、同时垂直于两个非平行的单位向量有两种情况(体现在方向上)

\section{面方程}\index{面方程}



\subsection{平面的表示}\index{面方程!平面的表示}

1、一般式:$A x+B y+C z+D=0, n=\{A, B, C\}$为平面法向量,其中$A, B, C$不全为0($A,B,C$确定平面的方向,$D$确定平面的位置)

2、点法式:$A\left(x-x_{0}\right)+B\left(y-y_{0}\right)+C\left(z-z_{0}\right)=0$,其中$\left(x_{0}, y_{0}, z_{0}\right)$为平面上任意一点,$n=\{A, B, C\}$为平面法向量,其中$A, B, C$不全为0($A,B,C$确定平面的方向,$\left(x_{0}, y_{0}, z_{0}\right)$确定平面的位置)

3、截距式:$\frac{x}{a}+\frac{y}{b}+\frac{z}{c}=1$,其中$a, b, c$分别为平面在三个坐标轴上的截距且均不为零(平面与坐标轴的三个交点$a, b, c$确定一个平面)



\subsection{建立平面方程}\index{面方程!建立平面方程}

1、直接:已知平面上的一个点$P_{0}\left(x_{0}, y_{0}, z_{0}\right)$和平面的法向量$\boldsymbol{n}=\{A, B, C\}$可确定平面

2、间接:已知平面$\Pi$上的一个点$P_{0}\left(x_{0}, y_{0}, z_{0}\right)$及与平面$\Pi$平行的两个不共线的向量$\boldsymbol{a}=\left\{a_{x}\right., \left.a_{y}, a_{z}\right\}, b=\left\{b_{x}, b_{y}, b_{z}\right\}$,则可确定平面$\Pi$,此时建立平面方程利用的是以下思路:设$P(x, y, z)$为所求平面$\Pi$上任一点, 则三向量$\overrightarrow{P_{0} P}, a, b$共面, 即有$\left(\overline{P_{0}} \vec{P} a b\right)=0$,所求平面方程为$\left|\begin{array}{ccc} x-x_{0} & y-y_{0} & z-z_{0} \\ a_{x} & a_{y} & a_{z} \\ b_{x} & b_{y} & b_{z} \end{array}\right|=0$(点法式)



\subsection{求解两平面表示的曲线的参数方程:(思路)}\index{面方程!求解两平面表示的曲线的参数方程:(思路)}

1、将三维曲线映射到一个平面上(消去一个变量),先求两个未知量的参数方程

2、将两个未知量的参数方程代入到其中一个平面方程中得到第三个未知量的参数方程

备注:即先有平面方程,再让平面曲线站起来(应用现有的变量关系构成第三个维度)



\section{线方程}\index{线方程}



\subsection{直线的表示}\index{线方程!直线的表示}

1、一般式:$\left\{\begin{array}{l}A_{1} x+B_{1} y+C_{1} z+D_{1}=0, \\ A_{2} x+B_{2} y+C_{2} z+D_{2}=0 .\end{array}\right.$,该直线为两平面的交线,这里假设$\left\{A_{1}, B_{1},C_{1}\right\}$与$\left\{A_{2}, B_{2}, C_{2}\right\}$不共线(两个不平行的平面必存在一条交线)

2、对称式:$\frac{x-x_{0}}{l}=\frac{y-y_{0}}{m}=\frac{z-z_{0}}{n}$,其中$\left(x_{0}, y_{0}, z_{0}\right)$为直线上的任意取定的一点,$s=\{l, m, n\} \neq 0$为直线的方向向量(直线上的一个向量与直线方向向量成比例,可以看出与参数式表示有一定关系)

3、参数式: $\left\{\begin{array}{l}x=x_{0}+l t \\ y=y_{0}+m t \\ z=z_{0}+n t \end{array}\right.$,$\left(x_{0}, y_{0}, z_{0}\right)$为直线上的任意取定的一点$s=\{l, m, n\} \neq \mathbf{0}$是直线的方向向量(参数式表示是一个定点向方向向量延申构成一条直线)



\subsection{空间曲线及其方程}\index{线方程!空间曲线及其方程}

1、参数式:$\left\{\begin{array}{l}x=x(t), \\ y=y(t), \\ z=z(t) .\end{array}\right.$

2、一般式(两曲面方程联立):$\left\{\begin{array}{l}F(x, y, z)=0, \\ G(x, y, z)=0 .\end{array}\right.$



\subsection{建立直线方程}\index{线方程!建立直线方程}

1、已知直线L上的一个点$P\left(x_{0}, y_{0}, z_{0}\right)$和直线$L$的方向向量$s=\{l, m, n\}$就可以确定直线$L$

2、两个不平行的平面相交于一直线(两平面的法向量的叉积确定直线的方向向量)



\subsection{空间曲线的投影}\index{线方程!空间曲线的投影}

1、设有空间曲线$\Gamma:\left\{\begin{array}{l}F(x, y, z)=0, \\ G(x, y, z)=0\end{array}\right.$,先通过$\left\{\begin{array}{l}F(x, y, z)=0, \\ G(x, y, z)=0\end{array}\right.$消去$z$得$\varphi(x, y)=0$,则曲线$\Gamma$在$x O y$面上投影曲线方程包含在方程$\left\{\begin{array}{l}\varphi(x, y)=0, \\ z=0\end{array}\right.$之中(消去$z$意在解除$z$变量与$x,y$变量的关系,相当于$x,y$变量确定了该位置的高度,消去$z$就是把高度部分拿掉了,就得到了平面上$x,y$的关系)

备注:在求解投影时,注意变量的范围,通过利用图形或者原曲线的的方程变量范围进行把握(当消去方程的某一部分时可能会导致范围变大)(易错点)

\section{线面位置关系}\index{线面位置关系}



\subsection{直线间位置关系}\index{线面位置关系!直线间位置关系}

1、直线的相交判断:两直线若相交,则它们共面,根据三向量共面的充要条件知,两直线的方向向量与两直线上各任意取一点(且非交点)所连的向量的混合积为零(这种方法可以对未知参数进行讨论)

2、直线不相交距离的求解:一种方法是利用两不相交直线的距离公式求解;另一种方法是多元函数极值法,利用参数式表示出两直线的方程,求解参数式之间的距离(距离公式),然后将距离分别对两个参数求偏导,导数为0的点就是极小值(因为两直线如果不相交,必定存在一个极小值)<复习全书P152 例6 评注>

3、求解与两直线都相交并且垂直的直线:先利用已知两向量的叉积求解直线的方向向量,由于与两直线的交点不好确定,所以分别求解过直线并且与待求解直线的方向向量平行的平面(平面是由法向量(法向量是由直线方向向量和待求解的直线方向向量叉积决定的)和一个直线上的定点确定的),两个平面的交线就是待求解的直线



\subsection{平面与直线间的位置关系}\index{线面位置关系!平面与直线间的位置关系}

1、平面与平面间的位置关系:设平面$\Pi_{1}: A_{1} x+B_{1} y+C_{1} z+D_{1}=0, \Pi_{2}: A_{2} x+B_{2} y+C_{2} z+D_{2}=0$,平行:$\Pi_{1} / / \Pi_{2} \Leftrightarrow \frac{A_{1}}{A_{2}}=\frac{B_{1}}{B_{2}}=\frac{C_{1}}{C_{2}}$(其中若某分母为零, 理解对应的分子也为零);垂直:$\Pi_{1} \perp \Pi_{2} \Leftrightarrow A_{1} A_{2}+B_{1} B_{2}+C_{1} C_{2}=0$;夹角:$\cos \theta=\frac{\left|A_{1} A_{2}+B_{1} B_{2}+C_{1} C_{2}\right|}{\sqrt{A_{1}^{2}+B_{1}^{2}+C_{1}^{2}} \sqrt{A_{2}^{2}+B_{2}^{2}+C_{2}^{2}}} \quad\left(0 \leqslant \theta \leqslant \frac{\pi}{2}\right)$(平面平行对应法向量平行;平面垂直对应法向量垂直;平面夹角与法向量夹角有关(相加等于$\pi$),法向量夹角的绝对值是平面夹角,注意平面夹角的范围)

2、直线与直线之间的位置关系:设直线$L_{1}: \frac{x-x_{1}}{l_{1}}=\frac{y-y_{1}}{m_{1}}=\frac{z-z_{1}}{n_{1}},L_{2}: \frac{x-x_{2}}{l_{2}}=\frac{y-y_{2}}{m_{2}}=\frac{z-z_{2}}{n_{2}}$,平行:$L_{1} / / L_{2} \Leftrightarrow \frac{l_{1}}{l_{2}}=\frac{m_{1}}{m_{2}}=\frac{n_{1}}{n_{2}}$(其中若某分母为零, 理解对应的分子也为零);垂直:$L_{1} \perp L_{2} \Leftrightarrow l_{1} l_{2}+m_{1} m_{2}+n_{1} n_{2}=0$;夹角:$\cos \theta=\frac{\left|l_{1} l_{2}+m_{1} m_{2}+n_{1} n_{2}\right|}{\sqrt{l_{1}^{2}+m_{1}^{2}+n_{1}^{2}} \sqrt{l_{2}^{2}+m_{2}^{2}+n_{2}^{2}}} \quad\left(0 \leqslant \theta \leqslant \frac{\pi}{2}\right)$(直线平行对应方向向量平行;直线垂直对应方向向量垂直;直线夹角与方向向量夹角有关(相等,或者相加等于$\pi$),方向向量夹角的绝对值是直线夹角,注意直线夹角的范围)

3、平面与直线之间的位置关系:设平面$\Pi: A x+B y+C z+D=0$,直线$L: \frac{x-x_{0}}{l}=\frac{y-y_{0}}{m}=\frac{z-z_{0}}{n}$,平行:$\Pi / / L \Leftrightarrow A l+B m+C n=0$;垂直:$\Pi \perp L \Leftrightarrow \frac{A}{l}=\frac{B}{m}=\frac{C}{n}$(其中若某分母为零, 理解对应的分子也为零);夹角:$\sin \theta=\frac{|A l+B m+C n|}{\sqrt{A^{2}+B^{2}+C^{2}} \sqrt{l^{2}+m^{2}+n^{2}}} \quad\left(0 \leqslant \theta \leqslant \frac{\pi}{2}\right)$(直线与平面平行对应方向向量与法向量垂直;直线与平面垂直对应方向向量与法向量平行;直线与平面夹角与方向向量和法向量夹角有关,取绝对值即可,注意直线夹角的范围)

4、点到平面的距离:点$\left(x_{0}, y_{0}, z_{0}\right)$到平面$A x+B y+C z+D=0$的距离为$d=\frac{\left|A x_{0}+B y_{0}+C z_{0}+D\right|}{\sqrt{A^{2}+B^{2}+C^{2}}}$(点到面距离公式,点带入到面相当于一个新的平面相当于平面的偏离程度,然后在法向量上做了标准化)

5、点到直线的距离:点$\left(x_{0}, y_{0}, z_{0}\right)$到直线$\frac{x-x_{1}}{l}=\frac{y-y_{1}}{m}=\frac{z-z_{1}}{n}$的距离为$d=\frac{\left|\left\{x_{1}-x_{0}, y_{1}-y_{0}, z_{1}-z_{0}\right\} \times\{l, m, n\}\right|}{\sqrt{l^{2}+m^{2}+n^{2}}}$(点到线距离公式,利用点和直线上的一点构造一个向量A,再利用A和方向向量的关系求解,理解即可)

6、两不相交直线的距离:设直线$L_{1}$与$L_{2}$的方向向量分别为$s_{1}=\left\{l_{1}, m_{1}, n_{1}\right\}$与$s_{2}=\left\{l_{2}, m_{2}, n_{2}\right\}$,点$A \in L_{1}$,点$B \in L_{2}$,则$L_{1}$与$L_{2}$间的距离$d=\frac{\left|\left(s_{1} s_{2} \overrightarrow{A B}\right)\right|}{\left|s_{1} \times \boldsymbol{s}_{2}\right|}$(或者写成$d=\frac{\left|\left(\boldsymbol{s}_{1} \times \boldsymbol{s}_{2}\right) \cdot \overrightarrow{A B}\right|}{\left|\boldsymbol{s}_{1} \times \boldsymbol{s}_{2}\right|}$)(其中$s_{1} \times \boldsymbol{s}_{2}$是在求解两直线最短距离的方向(与两直线都垂直),然后求解向量$\overrightarrow{A B}$在该方向上的映射值)

\section{旋转面方程}\index{旋转面方程}

旋转面的定义:由一条平面曲线绕其平面上的一条直线旋转一周所成的曲面叫作旋转曲面。旋转曲线称为旋转面的母线,定直线叫作旋转曲面的轴,即母线按照轴进行旋转构成旋转面

设有$x O y$面上的曲线$L:\left\{\begin{array}{l}f(x, y)=0, \\ z=0,\end{array}\right.$

1、曲线L绕x轴旋转产生旋转面方程为$f\left(x, \pm \sqrt{y^{2}+z^{2}}\right)=0$,其中$\pm$由$L$中$y$所允许的符号而定(绕$X$轴$x$变量不变,将$y$变量变成由$y,z$共同作用,定量为到$x$轴的距离)

2、曲线L绕y轴旋转产生旋转面方程为$f\left(\pm \sqrt{x^{2}+z^{2}}, y\right)=0$,其中$\pm$由$L$中$x$所允许的符号而定(绕$Y$轴$y$变量不变,将$x$变量变成由$x,z$共同作用,定量为到$y$轴的距离)

3、直线绕$z$轴的旋转面(思路):设直线上一点为$\left(x_{0}, y_{0}, z_{0}\right)$,由直线绕$z$轴旋转,半径是不变的,所以就有$x^2+y^2 = x_{0}^2+y_{0}^2$,将$x_{0}, y_{0}$代入到直线方程中得到$z_{0}$的关系式,该关系再代入到上式中可得$x^2+y^2 = f(z_{0})$,又因为绕$z$轴旋转$z$是不改变的,所以将$z=z_{0}$,于是旋转面方程为$x^2+y^2 = f(z)$

备注:关于其它面上的曲线旋转面方程类似

\section{柱面和常见的柱面}\index{柱面和常见的柱面}



\subsection{柱面}\index{柱面和常见的柱面!柱面}

柱面的定义:平行于定直线并沿定曲线$C$移动的直线$L$形成的轨迹叫作柱面。定曲线$C$叫作柱面的准线,动直线$L$叫作柱面的母线,即母线按照准线移动构成了柱面(柱面不一定都是平行于坐标轴的)

1、准线由两平面方程确定:准线为$L:\left\{\begin{array}{l}F(x, y, z)=0, \\ G(x, y, z)=0,\end{array}\right.$,母线的方向向量为$\{l, m, n\}$的柱面方程的建立先在准线$L$上任取一点$\left(x_{0}, y_{0}, z_{0}\right)$,则过点$\left(x_{0}, y_{0}, z_{0}\right)$的母线方程为$\frac{x-x_{0}}{l}=\frac{y-y_{0}}{m}=\frac{z-z_{0}}{n}$,消去方程组$\left\{\begin{array}{l}F\left(x_{0}, y_{0}, z_{0}\right)=0, \\ G\left(x_{0}, y_{0}, z_{0}\right)=0, \\ \frac{x-x_{0}}{l}=\frac{y-y_{0}}{m}=\frac{z-z_{0}}{n}\end{array}\right.$ 中的$x_{0}, y_{0}, z_{0}$(核心),得到关于$x, y, z$的方程即为所求柱面方程(相当于线动成面,准线规定了线动的方向,方向向量规定了线动扫描的位置)

2、准线由参数方程确定:准线为$L:\left\{\begin{array}{l}x=x(t), \\ y=y(t),  \\ z=z(t), \end{array}\right.$ 母线方向向量为$\{l, m, n\}$的柱面方程的建立,柱面方程为$\left\{\begin{array}{l}x=x(t)+l s \\ y=y(t)+m s, \\ z=z(t)+n s,\end{array}\right.$,这里$t, s$均为参数(双参数的面方程)(准线给出了起始位置,由母线方向向量由此位置向周围延申,从而构成了面)

3、其它特例:设柱面的准线为$x O y$平面上的曲线$L:\left\{\begin{array}{l}f(x, y)=0,\\ z=0,\end{array}\right.$ 母线为平行于$z$轴的直线,则该柱面的方程为$S: f(x, y)=0$(其中$z$变量做了延申,跳出平面限制)



\subsection{常见的柱面}\index{柱面和常见的柱面!常见的柱面}

1、圆柱面:$x^{2}+y^{2}=R^{2}, y^{2}+z^{2}=R^{2}, x^{2}+z^{2}=R^{2}$

2、椭圆柱面:$\frac{x^{2}}{a^{2}}+\frac{y^{2}}{b^{2}}=1$

3、抛物柱面:$y^{2}=2 p x$

4、双曲柱面:$\frac{x^{2}}{a^{2}}-\frac{y^{2}}{b^{2}}=1$

注意:均是缺一个变量类型的,相当于把平面曲线在缺的变量的方向的延申

\section{常见的二次曲面}\index{常见的二次曲面}

1、椭球面:$\frac{x^{2}}{a^{2}}+\frac{y^{2}}{b^{2}}+\frac{z^{2}}{c^{2}}=1$

2、单叶双曲面:$\frac{x^{2}}{a^{2}}+\frac{y^{2}}{b^{2}}-\frac{z^{2}}{c^{2}}=1$(可认为旋转轴是z轴,使得曲面连续构成一个单独的叶片)

3、双叶双曲面:$-\frac{x^{2}}{a^{2}}-\frac{y^{2}}{b^{2}}+\frac{z^{2}}{c^{2}}=1$(可认为旋转轴是x轴,使得两边曲面不连续,构成两个单独的叶片)

4、椭圆抛物面:$\frac{x^{2}}{a^{2}}+\frac{y^{2}}{b^{2}}=2 p z(p>0)$

5、双曲抛物面:$\frac{x^{2}}{a^{2}}-\frac{y^{2}}{b^{2}}=2 p z(p>0)$

6、二次锥面:$\frac{x^{2}}{a^{2}}+\frac{y^{2}}{b^{2}}-\frac{z^{2}}{c^{2}}=0$

注意:常见的曲面要了解记忆

