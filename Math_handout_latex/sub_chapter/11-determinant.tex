\chapterimage{chapter_head_2.pdf}
\chapter{行列式}

\section{行列式求值-递归式的推导}\index{行列式求值-递归式的推导}

1、简单类型:$ D_{n}=2D_{n-1}-D_{n-2} $ ⇒ $ D_{n}-D_{n-1} = D_{n-1}-D_{n-2} $

2、复杂类型:不能一眼看出来参数,设变量$ \mu ,k $求解即可$ D_{n}-kD_{n-1} = \mu (D_{n-1}-kD_{n-2}) $<880题 P56 三(4)>

备注:这里可以参考一下具体数学中推导递归式的步骤

\section{矩阵求解余子式或者代数余子式}\index{矩阵求解余子式或者代数余子式}

1、单个余子式或者代数余子式:定义

2、一行(列)的余子式或者代数余子式:构造新的行列式,利用代数余子式和行列式的值的关系,将待求解的那行(列)值重新赋值(注意余子式和代数余子式赋值的差别)

3、斜对角线的代数余子式:等于伴随矩阵的斜对角元素之和(这里可以想到斜对角元素之和等于特征值之和,在知道矩阵的特征值时用这个,否则求特征值也很麻烦)

4、其它不规则的代数余子式:利用伴随矩阵求解(需要求解逆矩阵,$ A^*=|A|A^{-1} $)(利用伴随矩阵时,伴随矩阵与原矩阵的代数余子式的对应关系)

5、代数余子式性质:任意一行(列)元素与另一行(列)的代数余子式的乘积之和为0

注意:如果$ <font color=purple>A^T=A^*</font> $,则$ <font color=purple>a_{ij}=A_{ij}</font> $(证明:应用定义)

注意:余子式与代数余子式的差别,即代数的来源(代数的系数)

\section{行列式是否为0的证明}\index{行列式是否为0的证明}



\subsection{行列式为0的证明}\index{行列式是否为0的证明!行列式为0的证明}

1、矩阵不可逆:

2、矩阵非满秩:

3、齐次方程组有非零解:表示列向量之间有线性关系

4、0是矩阵的特征值:因为特征值之积是行列式值

5、A的行或者列向量线性相关(包含了齐次方程组有非零解):

6、反证法:假设不为0,则说明行列式构成的矩阵可逆,利用可逆性推出不满足现有的条件,得知假设不成立

7、$ |A|=k|A| $,$ k\ne 1 $:代数的方法



\subsection{行列式不为0的证明}\index{行列式是否为0的证明!行列式不为0的证明}

1、矩阵可逆:

2、矩阵满秩:

3、行列式的n个特征值均不为0

4、A的行或者列向量线性无关(包含了齐次方程组只有零解,非齐次方程组有唯一解)

5、$ A^TA $是正定矩阵:对于任意的非零值代入到二次形式子中,式子值恒大于零(注意不包括0),即只含有正系数平方项的式子是正定的

6、矩阵A是两组基的过渡矩阵:基的过渡矩阵是可逆的

7、A的标准型是E:标准型是?二次型化标准型(对称阵)

7、A的行最简形式是E/A的行阶梯形有n个非零行:

8、A的伴随矩阵是可逆矩阵:

\section{一些特殊样例}\index{一些特殊样例}



\subsection{矩阵杂例}\index{一些特殊样例!矩阵杂例}

1、若$ AB=O $,如果$ A $可逆,则$ B=O $,(齐次方程的解,系数矩阵满秩则只有零解;也可以两边同时左乘$ <font color=purple>A^{-1}</font> $)

2、若$ AA^T=A^TA=O $,则$ A=O $(这是证明向量为零的一种方法)(该式子可以由特征值证明,所用的$ <font color=purple>r(A^TA)=r(A)</font> $也可以用特征值证明)



\subsection{伴随与转置矩阵$ **A^T=A^*** $}\index{一些特殊样例!伴随与转置矩阵$ **A^T=A^*** $}

一、若$ **A \ne O** $,则$ **|A|\ne 0** $

1、假设$ A $为三阶矩阵

2、如果$ <font color=red>A^T=A^*</font> $,则$ <font color=red>a_{ij}=A_{ij}</font> $

3、由$ A \ne O $,假设$ a_{11} \ne 0 $(总得有一个不为0的),则$ |A|=a_{11}A_{11}+a_{12}A_{12}+a_{13}A_{13} = a_{11}^2 + a_{12}^2+a_{13}^2\ne 0 $

二、与秩的关系:

1、此时A的秩为0(全零矩阵)($ A = O $)或者满秩(可逆,齐次零解,非齐次唯一解)(若$ <font color=purple>**A \ne O**</font> $,则$ <font color=purple>**|A|\ne 0**</font> $)



\subsection{与自身转置相乘值为1相乘的向量的秩}\index{一些特殊样例!与自身转置相乘值为1相乘的向量的秩}

1、$ \alpha^T\alpha=1 \rightarrow (\alpha\alpha^T)\alpha = 1·\alpha $,即1是$ \alpha\alpha^T $的特征值



\subsection{$ **A^n=O** $的式子的利用(来源880题)}\index{一些特殊样例!$ **A^n=O** $的式子的利用(来源880题)}

1、式子的两边同时加上负号并同时加上$ E $:得到$ E-A^n=E $,左边可以展开为$ (E-A^n)=(E-A)(E+A+A^{2}+\cdots +A^{n-1})=E $,(类似于信号系统中的展开式$ 1-x^6=(1-x)(1+x+x^2+\cdots+x^{n-1}) $),这样可以得到两个式子相乘,可以得到一个式子的逆;左边也可以展开为$ (E-A^n)=(E+A)(E\pm A\pm A^{2}\pm\cdots -A^{n-1})=E $(最后一项一定是减法,所以当n为奇数时,A的偶数次方为负号,当n为偶数时,A的奇数次方为负号,此结论待进一步求证(<880题 P59 二 4>))



\subsection{$ **A^n=E** $的式子的利用(来源880题)}\index{一些特殊样例!$ **A^n=E** $的式子的利用(来源880题)}

1、行列式值特性:两边同时取行列式,得到$ |A|^n=1 \rightarrow |A|=1 $

2、推导$ (A^*)^n=(A^n)^*=E $:$ A^n=E $是已知的,所以右边等式$ (A^n)^*=E $成立,$ (A^*)^n=E $证明可以通过利用(核心公式同时取n次方利用核心公式推导出来的)$ A^* $与$ A $是可随意交换位置的特性得到$ (AA^*)^n=(AA^*)(AA^*)(AA^*)\cdots(AA^*)=A^n(A^*)^n=|A|^nE=E $;也可以由核心公式的导出公式$ A^*=|A|A^{-1} $两边同时取n次方得到$ (A^*)^n=|A|^nA^{-n} \rightarrow A^n(A^*)^n=|A|^nE=E $

\section{行列式求值-求解特征梳理}\index{行列式求值-求解特征梳理}

1、行相等或者列相等:如果每列之和相等把每一行都加到第一行;如果每一行值相等把每一列都加到第一列(常规处理方法)

2、不明显的递归类型,使用$ D_n $表示,行列式中含有$ a_n,a_{n-1},\cdots,a_0 $的,可能有递归操作,可以按照拆掉最后一项($ a_n $之类的)相关的行或列(肯定选择展开简单的(展开时要细心,不要忘记值前面的系数(-1还是1),又一遍,因为忘两次了))()

2、明显递归类型:去掉边边角角与之前的行列式类似,求出递归结构之后还需要对递归式进行推导

3、箭头或爪形:将中间的值加到箭头的一条边上,使得该边除第一个元素外都是0

4、加边法(特征:除主对角线元素外,其它各列或者各行的元素相同或者是与某一个值成比例):添加一层额外的外层,边的一边是1引导一堆零(保证加边之后与加边之前行列式的值相等),剩余元素与的设置为上述其它隔行或者各列的比例数值(元素相同设置为1,与某一个值成比例则设置乘该值);其中剩余元素在行时,每个元素考虑设置值为该列的公约数,剩余元素在列时,每个元素考虑设置值为该行的公约数(确实<880题 P55>)

5、循环类型:(特征:每一列两个两个相邻的值,到最后一列变成分离到两端的值等类似的)按照两端的值展开即可(这种情况并不是递归)

6、范德蒙类型:看到有高次方,每一列都是等比数列(那么矩阵转置与矩阵转置之前的行列式相等的情况下,每一行都是等比数列也可以化作为范德蒙类型来求解)(范德蒙公式:$ <font color=purple>\prod_{1\le j \le i \le n}(x_i-x_j)</font> $,列数大的元素减列数小的元素)

7、正对角线和副对角线:注意副对角线的$ <font color=purple>-1</font> $的次方问题($ <font color=purple>(-1)^{\frac {n(n-1)}{2}}</font> $)

8、十字交叉形:应用递推法求解

9、抽象(非数值)可拆解型:拆解成多个行列式求解(行列式拆开的性质)

10、三对角线:逐行把每一行加到下一行(或者展开试试转化成递归形式,归纳法求解)

11、含有$ O $块(或者简单变换之后):利用拉普拉斯展开式(副对角线的行列式系数为$ <font color=purple>(-1)^{mn}</font> $,$ <font color=purple>m,n</font> $分别为副对角线上两个行列式的阶数)

注意:递归或者递推时行列的变化,会对代数余子式的正负性造成影响

\section{行列式求值-具体求解过程}\index{行列式求值-具体求解过程}

1、数值型行列式:利用行列式的变换法则,利用倍加性质化出1或者-1,以便抵消其它行,变换成易于求解的形式;逐行相加的方式,转换成对角线的形式(列的数值类似,阶梯状分布);当行列式阶数小于等于3时,可以使用对角线法则求解

2、带未知参数的行列式:注意使用变换法则时,未知参数不要做变换的分母

3、抽象型行列式:式子分解,如果是单一的矩阵的变换(在知道原矩阵特征值的情况下)考虑使用特征值(牢记变换后的特征值及其大致推导过程)

4、向量形式的行列式:将向量的组合(注意组合后的向量系数变化,利用行列式性质将系数提出来)分解成已知向量的模的形式(其实是利用行列式的性质:行列式的某行或者某列可以拆成两个行列式之和)(推导:利用代数余子式与行列式值的关系,行列式的变换法则)

5、非满秩⇒ 0:如果行列式对应的矩阵不是满秩的,则行列式值为0

6、利用矩阵特征值:行列式的值等于特征值之积(有的行列式不能通过特征值法求解,是不是需要前提条件(目前初步判定为分解的两个行列式的特征值排布不唯一)<880题 P55 三(2)>)

7、含有逆运算,转置,伴随运算:可以先利用矩阵的基本变换化简式子(特别注意矩阵的逆和矩阵的伴随之间的密切联系)

8、相似矩阵:相似矩阵的行列式值相等(证明:利用相似的定义,$ |B|=|P^{-1}AP|=|A| $),相似与特征值的构思(这是在构思什么?)

9、行列式性质:转置行列式不变;行或列之间互换变号;行或者列的公因子可以提出来(如果行列式两行成比例则值为0);行列式的某行或者某列可以拆成两个行列式之和(注意与$ **|A+B|\ne |A|+|B|** $的区别);某行的k倍可以加到另一行上

10、数学归纳法:初级情况验证,提出中级阶段的假设并假设中级阶段正确,利用假设验证高阶情况时是否正确<详细步骤:复习全书 332页 例 8>

11、展开公式:

注意:行列式展开时,不要忘记值前边的系数(-1还是1)

注意:行列式的某两行或列互换时,行列式值变号(该互换情况也可以复杂一点,做了一系列的行列变换,转换成矩阵与初等矩阵左乘和右乘的关系后,在求解行列式时利用乘积性质展开就相当于初等变换矩阵行列式值与原行列式的乘积关系,初等变换的乘积时注意乘积的顺序(特别是左乘的变换,是是变换的逆序相乘))(在矩阵相关运算过程中没有意识到该点,两次)

\section{行列式求值-复杂式子的分解}\index{行列式求值-复杂式子的分解}

1、伴随运算核心公式:$ AA^*=A^*A=|A|E $

2、逆矩阵的定义和运算:$ AA^{-1}=A^{-1}A=E $,$ |A^{-1}|=|A|^{-1} $

3、矩阵乘积性质:注意乘积顺序

4、伴随矩阵与原矩阵的行列式关系:$ |A^*|=|A|^{n-1} $(伴随矩阵的行列式值和原矩阵的行列式的值的关系)

5、奇数阶反对称阵行列式值为0:$ A=-A^T $(证明两边同时取行列式即可证)

6、正交矩阵的行列式为1或者-1:$ AA^T=A^TA=E $(证明两边同时取行列式即可证)

注意:分解原则$ <font color=purple>|AB|=|A||B|</font> $

