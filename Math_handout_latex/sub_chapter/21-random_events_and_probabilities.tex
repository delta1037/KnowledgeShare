\chapterimage{chapter_head_2.pdf}
\chapter{随机事件和概率}

\section{事件}\index{事件}



\subsection{事件与样本空间}\index{事件!事件与样本空间}

1、随机试验:1、同条件可重复;2、实验开始前无法预知结果;3、结果可能有多个

2、样本空间:1、随机试验的每一个结果为样本点;所有的样本点组成样本空间

3、随机事件:1、样本空间的子集称为随机事件,简称事件;2、一个样本点组成的事件为基本事件;3、必然事件和不可能事件的定义



\subsection{事件的基本运算}\index{事件!事件的基本运算}

1、事件的并、交、差、包含、相等

2、运算规律:交换律、结合律、分配律、对偶律

3、运算过程中对一个等式两边与同一事件相交,等式仍成立;反之不能成立(没有消去律:等式两边同时消去同一个交事件等式不能成立)

备注:可以使用图形(文氏图)求解或对结果进行验证



\subsection{文字叙述:(正确理解至少、最少、最多、恰有的含义)}\index{事件!文字叙述:(正确理解至少、最少、最多、恰有的含义)}

1、至少(最少):求反向事件(一个都没有,几个都没有)

2、最多:求反向事件(多于最多时的情况,大概率没有等号,即不包含最多)

3、恰有:直接求

注意:同一事件会有多种不同的表达形式



\subsection{事件相互独立}\index{事件!事件相互独立}

1、定义:相互独立是指事件集合的任意的子集都互相独立

2、独立事件的充要条件:$ P(AB)=P(A)P(B) $(备注:常用用判断两个事件是否是相互独立的)

3、两事件$ A,B $相互独立的充要条件是$ A $与$ \overline B $,或$ \overline A $与$ B $,或$ \overline A $与$ \overline B $相互独立(事件独立得到的推论,$ <font color=purple>A,B</font> $独立并不意味着$ <font color=purple>A</font> $与$ <font color=purple>\overline B</font> $就不是独立的,这是个双重否定,害,真拗口,总而言之就是$ <font color=purple>A,B</font> $相互独立则所有相关的都互相独立)

4、当$ 0<P(A)<1 $时,$ A \  B $独立等价于$ P(B|A)=P(B) $(一个事件在另一个事件条件下的概率与事件单独发生的概率一致,则这两个事件独立)或者$ P(B|A)=P(B|\overline A) $成立(证明:使用条件概率公式)

5、相互独立的n个事件中,任何几个事件换成对立事件,新组成的n个事件也相互独立(利用了上述3的推论)

备注:独立是指两个事件互不影响

\section{常见的事件模型}\index{常见的事件模型}



\subsection{古典模型}\index{常见的事件模型!古典模型}

1、实验结果为n个样本点,每个样本点的发生有相同的可能性,事件A由$ n_A $个样本点组成,则事件A的概率为$ P(A) = \frac{n_A}{n} = \frac{A包含的样本点数}{样本点总数} $

注意:特点为有限等可能试样

注意:计算时,尽可能选择较小并且简单的样本空间

如何判断和发现最简单的样本空间,或者发现样本空间?整理一下技巧



\subsection{几何型概率}\index{常见的事件模型!几何型概率}

1、试验的样本空间为区域,以$ L(\Omega) $表示其几何度量,$ L(\Omega) $有限,且试验结果出现在$ \Omega $中的任何区域只与该区域的几何度量成正比,事件A的样本点所表示的区域为$ \Omega _A $,则事件A的概率为$ P(A) = \frac{L(\Omega _A)}{L(\Omega)} = \frac{\Omega _A的几何度量}{\Omega 的几何度量} $

注意:样本点无限,几何度量上是等可能的



\subsection{独立重复实验与$ **n** $重伯努利实验}\index{常见的事件模型!独立重复实验与$ **n** $重伯努利实验}

1、独立重复试验:一随机实验做若干次,各次试验所联系的事件相互独立,且同一事件在各个事件在各个试验中出现的概率相同

2、伯努利试验:独立重复试验中只有两个结果

3、n重伯努利实验:伯努利试验重复$ **n** $次

4、n重伯努利试验中事件A($ P(A) = p \ (0<p<1) $)发生$ **k** $次的概率$ C_n^kp^k(1-p^{n-k}),k=0,1,2,\cdots ,n $;即二项式的展开式中的项

思路:独立重复实验中如果有多个结果,可以将结果进行归成两类,转换成伯努利实验



\section{概率计算公式}\index{概率计算公式}



\subsection{加法公式}\index{概率计算公式!加法公式}

1、$ P(A\cup B) = P(A)+P(B)-P(AB) $

2、$ P(A\cup B \cup C) = P(A)+P(B)+P(C)-P(AB)-P(AC)-P(BC)+P(ABC) $



\subsection{减法公式}\index{概率计算公式!减法公式}

1、$ P(A - B) = P(A)-P(AB) $

2、$ A-B=A\overline B \rightarrow P(A-B) = P(A\overline B) $ (注意:如果AB互相独立,则$ P(A-B) = P(A\overline B)=P(A)P(\overline B) $,利用了相互独立的公式)



\subsection{乘法公式}\index{概率计算公式!乘法公式}

1、$ P(A)>0 \rightarrow P(AB)=P(A)P(B|A) $(非独立时乘法运算)

2、$ P(A_1A_2\cdots A_n) > 0 \rightarrow P(A_1A_2\cdots A_n)=P(A_1)P(A_2|A_1)\cdots P(A_n|A_1A_2\cdots A_{n-1}) $

3、推论:$ P(AB) \le P(A) $

备注:事件的相乘是以前提条件为基础的,所以转化成了条件概率



\subsection{全概率公式}\index{概率计算公式!全概率公式}

1、条件:$ \mathop{ \bigcup }\limits_{{k=1}}^{{n}}\mathop{{B}}\nolimits_{{k}}=\Omega \  $,$ \mathop{{B}}\nolimits_{{i}}\mathop{{B}}\nolimits_{{j}}= \emptyset { \left( {i \neq j,i,j=1,2,3, \cdots n} \right) } $,$ P{ \left( {\mathop{{B}}\nolimits_{{k}}} \right) } > 0{ \left( {k=1,2,3, \cdots n} \right) } $;(称$ B_1,B_2,\cdots B_n $为$ \Omega $的一个完备事件组)

2、结论:$ P{ \left( {A} \right) }=\mathop{ \sum }\limits_{{k=1}}^{{n}}P{ \left( {\mathop{{B}}\nolimits_{{k}}} \right) } \cdot P{ \left( {A \left| \mathop{{B}}\nolimits_{{k}}\right. } \right) } $

备注:事件$ <font color=orange>B_i</font> $瓜分了整个样本空间,相当于A样本空间的基底?



\subsection{贝叶斯公式}\index{概率计算公式!贝叶斯公式}

1、条件:$ \mathop{ \bigcup }\limits_{{k=1}}^{{n}}\mathop{{B}}\nolimits_{{k}}=\Omega $,$ \mathop{{B}}\nolimits_{{i}}\mathop{{B}}\nolimits_{{j}}= \emptyset { \left( {i \neq j,i,j=1,2,3, \cdots n} \right) } $,$ P(A) > 0,P{ \left( {\mathop{{B}}\nolimits_{{k}}} \right) } > 0{ \left( {k=1,2,3, \cdots n} \right) } $

2、结论:$ P{ \left( {\mathop{{B}}\nolimits_{{k}} \left| A\right. } \right) }=\frac{{P \left( {\mathop{{B}}\nolimits_{{k}}} \left) \cdot P{ \left( {A \left| {\mathop{{B}}\nolimits_{{k}}}\right. } \right) }\right. \right. }}{{\mathop{ \sum }\limits_{{i=1}}^{{n}}P{ \left( {\mathop{{B}}\nolimits_{{i}}} \right) } \cdot P{ \left( {A \left| \mathop{{B}}\nolimits_{{i}}\right. } \right) }}}=\frac{{P \left( {\mathop{{B}}\nolimits_{{k}}} \left) \cdot P{ \left( {A \left| {\mathop{{B}}\nolimits_{{k}}}\right. } \right) }\right. \right. }}{P(A)} $,换算得到$ P(B|A)=P(B)\cdot \frac{P(A|B)}{P(A)} $

3、作用:新信息出现后的B概率=B概率 X 新信息带来的调整(理解调整式子$ <font color=purple>\frac{P(A|B)}{P(A)}</font> $的计算,依据代数来理解,无法理解,记住公式)

问题:贝叶斯公式能用来干什么?



\subsection{其它公式(AB独立时)}\index{概率计算公式!其它公式(AB独立时)}

1、$ P(A-B) = P(A\overline B)=P(A)P(\overline B) $(事件的减法公式与独立时的进一步推广)

2、$ P(A\cup B) = 1-P(\overline A \  \overline B) = 1-P(\overline A)P(  \overline B) $(利用事件的图形关系可以解释)



备注:可以利用事件的图形关系进行辅助理解上述公式

\section{概率}\index{概率}



\subsection{概率的概念和基本运算}\index{概率!概率的概念和基本运算}

1、概率定义:对应样本空间的实值函数P(即使事件存在也可以使事件概率为0,使用图形法计算事件的关系时,注意区分事件和概率)

2、任意事件概率:$ 0 \leqslant P(A)\leqslant 1 $

3、必然事件概率为1($ P(\Omega)=1 $);空集的概率为0($ P(\emptyset)=0 $)

4、两两互斥(不相容)的事件并等于对应每一个事件概率的和($ P\left(A_{1} \cup A_{2} \cup \cdots \cup A_{n}\right)=\sum_{i=1}^{n} P\left(A_{i}\right) $)

5、$ P(A)=1-P(\overline A) $ 常用

6、$ A \subset B\rightarrow P(A) \leqslant P(B) $:事件的包含与概率的大小的关系比较(事件关系与概率关系对应)



\subsection{区分概率与事件}\index{概率!区分概率与事件}

1、概率的定义为对应样本空间的实值函数(每一个样本都有一个概率值,即使样本存在概率可能也是零)

2、区分概率和事件的运算的区别,事件相等概率是相等的,但是概率相等事件不一定等同

文氏图法:通过图的相交或者分离表示事件事件的关系,对每一个区域标注变量表示事件区域的概率



\section{事件独立、不相容(也叫互斥)、对立的对比}\index{事件独立、不相容(也叫互斥)、对立的对比}

1、事件独立:依据事件独立公式$ P(AB)=P(A)P(B) $,$ A,B $事件的发生互不影响(独立性要求两个变量完全不相关;不相关要求两个变量没有线性关系)

2、事件对立:两个事件$ A,B $占满了同一个样本空间(对立事件组成并占满整个样本空间),并且彼此之间没有交集

3、事件不相容(互斥):两个事件没有交集(事件不能同时发生,没有公共样本点),但是不一定占满了整个样本空间

注意:互斥事件与对立事件的区别是互斥可能没有占据整个样本空间

