\chapterimage{chapter_head_2.pdf}
\chapter{函数极限连续}

\section{函数极限连续的注意点}\index{函数极限连续的注意点}

1、单调不一定是连续的:如果出现间断点,但间断点处符合单调趋势,则整体也是单调的

2、振荡的无穷不是无穷大:函数在某个点的去心邻域虽然无界,但是函数在该点的极限不是无穷大

3、导数的定义式与洛必达:对于导数的定义式使用洛必达(即函数在去心邻域可导),则在心处函数是连续的,左右导数相等(这两个条件也可以反向说明去心邻域可导)

4、极限趋近方向:取极限时注意趋近的值的符号,在对变量进行平方和合并到根号内部时特别注意

5、对于两个函数乘积的极限:如果两个函数的极限都不存在,乘积的极限可能存在(某些影响因子抵消了);如果两个函数其中一个极限存在,另一个极限不存在,乘积的极限可能存在,或者不存在(主要看极限不存在的影响因子是否能被极限存在的多余的影响因子抵消)

6、去心邻域无穷个点无定义不存在极限:如果在去心邻域内有无穷个点没有定义,则谈不上取极限,极限也是不存在(极限不存在的情况之一(其它有无穷,振荡))

\section{已知极限的求解}\index{已知极限的求解}



\subsection{已知极限求另一有关的极限}\index{已知极限的求解!已知极限求另一有关的极限}

1、方法一:利用极限与无穷小的关系,由题设条件去掉极限符号,解出来$f(x)$,代入到欲求的极限式并求解

2、方法二:由已知极限出发,推导出欲求极限的一些有用的结果,或者将欲求的极限凑成已知极限的表示形式

注意:去掉极限符号时注意添加无穷小量并将无穷小量跟随计算(大题需要添加对无穷小量的说明)



\subsection{已知极限求参数}\index{已知极限的求解!已知极限求参数}

1、泰勒公式法:利用佩亚诺余项泰勒公式将函数展开,(有时需要用求解极限的方式化简)直接求解待定参数(当泰勒公式不是现成的,必要时可以推导函数的泰勒级数展开式)

2、洛必达法则法:利用洛必达法则,每一步需要对参数进行讨论,满足一定的条件才可以继续用洛必达,未知参数的值可以用满足的条件进行求解(因为极限存在,所以洛必达法则条件必定满足)

3、先化简再求值:利用求解极限的方式将极限进行化简,根据极限的值的类型求解参数

注意:有时对未知参数的值进行讨论

\section{极限存在性证明与求解}\index{极限存在性证明与求解}



\subsection{单调有界定理证明极限的存在性}\index{极限存在性证明与求解!单调有界定理证明极限的存在性}

1、数列(函数)单调减少且有下界,则极限存在并大于下界(界限可能太靠下了)

2、数列(函数)单调增加且有上界,则极限存在并小于下界(界限可能太靠上了)



\subsection{利用夹逼定理求解极限}\index{极限存在性证明与求解!利用夹逼定理求解极限}

1、使用准则:不要改变起主要作用的n的最高次项,放大缩小之后的极限相等(难点)

\section{函数的有界性讨论}\index{函数的有界性讨论}



\subsection{有界的判定}\index{函数的有界性讨论!有界的判定}

1、有界判定定理:如果函数在闭区间上连续,则函数在闭区间内有界(推论:如果函数在开区间上连续,并且两端的内侧的极限都存在,则函数在开区间内有界,开区间为无穷时也成立)(运算:有界之和、积为有界函数)

2、函数与导数有界的关系:闭区间上,导函数和原函数均有界:$f^{\prime}(x)$在区间上有界 ⇒ $f(x)$在区间上有界,其余有界情况均不成立(拉格朗日中值定理与绝对值放缩来证明)

3、函数与导数无界的关系:闭区间上,导函数和原函数均无界:$f(x)$在区间上无界 ⇒ $f^{\prime}(x)$在区间上无界,其余无界情况均不成立

4、极限表述形式:邻域内有界(指定点的极限);大于指定范围时有界(无穷远处的极限);极限趋近于无穷,则无界;极限无穷振荡,也是无界,不等同于无穷



\subsection{判定特殊情况}\index{函数的有界性讨论!判定特殊情况}

1、三角函数取特殊趋近值(不同的趋近方式):三角函数等于$(-1,1)$之间的值时有无穷个点,导致取0值时或者取非0值时得到的极限有不同的结果

注意:无穷小与无穷大的描述(无穷大与无穷小均为单一的值)

注意:区分无穷振荡型与无穷大(无穷振荡型并非是极限无穷大)

\section{函数的间断与连续}\index{函数的间断与连续}

1、连续:区间连续表示区间内的每一点都连续

2、间断:第一类间断点(可去间断点<左右极限相等但是在点处不连续>、跳跃间断点<左右极限存在但不相等>),第二类间断点(无穷间断点<极限为无穷>、振荡间断点<无限的振荡>)

3、判断函数或者复合函数(先复合)的间断点类型:求解点处两侧的极限值和点处的值

4、极限定义式形式:判断由极限形式定义的函数的连续性或者间断点类型,先求出函数的表达式

5、找间断点(特殊点):分母为0的点,分母是三角函数为0的点,定义域的边界;化简式子;判断间断点的类型

注意:点连续邻域内不一定连续(比如狄利克雷函数)

注意:前提都是去心邻域有定义(去心邻域是一个双边的概念)

备注:复合函数的间断点消失的情况(函数的平方,绝对值都可以抹掉值域的正负性;复合内层的函数的值域不包含外层函数的间断点定义域部分,啪,间断点没了)

\section{求分段复合函数}\index{求分段复合函数}

1、分段过渡:写出中间变量的值域范围,按照外层函数的定义域范围,分段过渡到外层函数的自变量变化范围(定义域范围)

\section{极限特例}\index{极限特例}



\subsection{分类讨论的情况}\index{极限特例!分类讨论的情况}

1、含有$|x|$,$x$趋近于0,区分$0^+$和$0^-$,进行讨论

2、含有$e^{\frac{1}{x}}$,$x$趋近于0,区分$0^+$和$0^-$,进行讨论(同除某一个项,或者提公因式判断)

3、含有取整函数$[x]$,$x$趋近于整数



\subsection{常用的极限等式}\index{极限特例!常用的极限等式}

设$\alpha>0, \beta>0$

1、$\lim _{x \rightarrow+\infty} \frac{\ln ^{\beta} x}{x^{\alpha}}=0$

2、$\lim _{x \rightarrow+\infty} \frac{x^{\alpha}}{\mathrm{e}^{\beta x}}=0$

3、$\lim _{x \rightarrow 0^{+}} x^{\alpha} \ln ^{\beta} x=0$

\section{极限表示的函数的表达式}\index{极限表示的函数的表达式}

1、找影响因子(高次项等):先判断在定义域上其中哪个项在极限的求解中占主要成分

2.1、提影响因子:将其中起作用的影响因子提取出来,求解剩余部分的极限

2.2、将影响因子以外的进行放缩:利用夹逼定理(夹逼定理放缩就是不改变起主要作用的项),整个函数大于该单独的项,并小于其它所有项等于该项的值(依据实际的条件进行放缩)

放缩样例:$\sqrt[n]{|x|^{3 n}}<\sqrt[n]{1+|x|^{3 n}} \leqslant \sqrt[n]{2|x|^{3 n}}=\sqrt[n]{2}|x|^{3}$

备注:当函数的表达式是用极限的形式表示时,需要先求出函数的表达式(must),再根据题目进行别的求解(判断函数的性质等)

\section{复合函数极限}\index{复合函数极限}

1、复合函数极限运算定理:设$\lim_{x \rightarrow x{0}} \varphi(x)=u_{0}$, 且存在$x=x_{0}$的某去心邻域$\dot{U}\left(x_{0}\right)$, 当$x \in \dot{U}\left(x_{0}\right)$时$\varphi(x) \neq u_{0}$(附近没有别的等于极限$u_{0}$的点,排除振荡情况;如果$u_{0}$的值为无穷大,则可认为可排除振荡的情况,即无条件存在$\varphi(x) \neq u_{0}$), 又设$\lim_{u \rightarrow u{0}} f(u)=A$, 则$\lim_{x \rightarrow x{0}} f(\varphi(x))=A$

2、复合函数的单调性与复合函数的极限存在性,按照有关的收敛、单调去考虑即可(单调有界数列必存在极限)(考虑内层函数振荡)

备注:收敛是指收敛于有限值,趋近于无穷不叫收敛

\section{大题求解极限注意事项}\index{大题求解极限注意事项}

1、判断极限的类型

2、使用洛必达时,分析洛必达的使用条件(大题中每一步使用洛必达都需要验证洛必达条件)

3、验证积分为无穷:对于积分为无穷的验证表达形式,求解积分内函数的极限值,当极限存在时,设存在一个$0$到该极限的中间值$f$,当积分的上限大于某个充分大的$T$时,积分内函数的值大于中间值$f$,将积分的区域分解成$0→T$和$T→无穷$两部分,前一部分为有限值,后一部分利用放缩,求解出来值为无穷,从而证明积分的值为无穷<复习全书 P17完整步骤>

4、洛必达法则无穷/无穷型:当分母满足为无穷值时,无论分子是什么(存在、不存在),洛必达法则的结论仍然成立(推广式)

\section{奇偶性的判断}\index{奇偶性的判断}

1、利用定义:利用奇偶性的定义进行讨论

2、导函数的奇偶性:导函数的奇偶性根据被求导的函数的奇偶性进行讨论(奇导为偶,偶导为奇,偶导的零点可能不存在)

3、原函数的奇偶性:原函数的奇偶性通过变上限积分$F(x)=F(0)+\int_{0}^{x} f(t) \mathrm{d} t$讨论(奇函数的原函数一定是偶函数,偶函数的原函数不一定是奇函数,借用$F(x)=F(0)+\int_{0}^{x} f(t) \mathrm{d} t$来进行讨论,可以看出$F(0)$的值不会影响偶函数的特性,但是会影响奇函数的特性(因为0处必须等于0))

4、积分的奇偶性判断:对积分求导

\section{求函数的极限方法}\index{求函数的极限方法}

1、化简:不完全平方形式(有理化;分母补1);无穷多项时先对无穷多项求和(无穷多项相加可能会出现更高次项);幂指函数指数化(整体部分都可用,即可以对部分进行指数化)(不要把指数的底给丢了);三角函数化简(升幂,降幂,合并);提公因式(无穷减无穷形式,指数函数,幂函数)(属于整体的因式);提出为有限值的乘积项(极限存在且不为0的因式按照乘积法则提取出来);极限四则运算(和差商积的极限两个极限都要存在,且商的分母极限值不为零);有界函数与无穷小乘积为无穷小

2、等价无穷小替换:对整体的因式操作,等价无穷小的条件是$x$趋于0

3、洛必达法则:需要验证是否可以洛必达(去心邻域必须可导,0/0型,无穷/无穷型,洛必达的分母的导数不恒为零,求导过后极限应为有限值或者无穷就算洛必达成功(不包括振荡形式),当洛必达的求解结果是不是无穷的不存在,不能说原极限不存在,应该用别的方法),特别是大题要进行验证(带有反三角函数的大概率需要用洛必达)(对抽象函数使用洛必达时,注意验证题目是否给出了可用洛必达的条件)

4、佩亚诺余项泰勒公式:注意展开的阶数

5、夹逼定理:常用的放缩形式

6、拉格朗日函数法:判别规则(存在基形式一致的函数(或者通过加减项凑形式);复合函数特别复杂,但是复合的外层一致的情况,可以使用拉格朗日法将最外层去掉(外层函数类似$tanx$,$e^x$之类的))(去掉的最外层的函数的导数的极限应该存在)

7、数列和的极限:数列转积分和式(利用积分的定义)($dx=\frac{1}{n}$,$x=\frac{i}{n}$)

8、带变限积分的式子:规范变量(对于积分号里面的变量$x$,使用拆项的办法放到积分外面;对于被积分函数里的函数$f$含有变量$x$使用变量代换的方式换到积分限中),使用洛必达法则去掉积分符号

9、单调有界定理:

10、导数的定义的运用:

11、部分项求解代入:对其中类似的项拎出来单独求极限,求出来如果极限存在再代入求解

备注:极限是一个两边趋近概念,讨论范围都是去心邻域;求解极限时除了佩亚诺余项泰勒公式和部分项求解代入法,不允许对一部分进行求解;

注意:求解极限注意符号问题(根据大致的值进行判断);无穷比无穷时,一般要进一步化简(同时除以某一个无穷大);使用式子替换趋于某个值的变量时,需要判断式子是不是只能趋近一边,即只能求解单边的极限

问题:单边极限的求解,也可以用无穷小替换,泰勒公式等?(待验证,如果有看到有人这么做了,可能就是对的)

\section{无穷小的比较}\index{无穷小的比较}



\subsection{无穷小比较的方法}\index{无穷小的比较!无穷小比较的方法}

1、利用商的极限的值进行比较

2、将式子使用泰勒公式展开

3、洛必达法则

4、等价无穷小的充要条件



\subsection{无穷小的分类}\index{无穷小的比较!无穷小的分类}

1、同阶无穷小

2、等价无穷小

3、高阶无穷小、低阶无穷小

4、无穷小的阶数

\section{极限的性质与应用}\index{极限的性质与应用}

1、极限的保号性:在极限的一个范围,表达式与极限的值接近(对于数列,比如当$n$无穷时极限为$A$,则存在一个充分大的值$N$,使得大于$N$的时候,数列表达式与$A$的值接近(做一下放缩,变成确定的值$A_0$))

2、极限的唯一性:如果极限存在,极限值必唯一

3、去极限符号:极限定义为$\lim \frac{g(x)}{f(x)}=A$,则可以将极限符号去掉,得到$g(x) = Af(x)+o(f(x))$,其中$o(f(x))$是$f(x)$的无穷小量(注意:对于无穷小量的描述和分解,如果$o(\Delta x)$为$\Delta x$的无穷小量,则$o(\Delta x) \sim \varepsilon·\Delta x$,其中$\varepsilon$为无穷小量)

\section{数列极限(和式或者递推式)}\index{数列极限(和式或者递推式)}



\subsection{求解数列的极限-递推形式}\index{数列极限(和式或者递推式)!求解数列的极限-递推形式}

1、单调有界准则(递推式包含两项):先利用已知条件证明数列是有界的(可以试用夹逼定理,或者放缩,归纳法),然后利用放缩证明数列两项之间的关系(单调性,除了$u_{n+1}-u_{n}$,也可以用$\frac{u_{n+1}}{u_{n}}$),由单调有界性证明界限存在,然后假设极限的值,代入到已知关系式中求解到极限的值(先证明存在,然后设值求解)

2、猜测极限并证明(递推式包含两项):在可以事先猜出极限值的前提下计算出极限$\lambda$,然后证明极限确实是这个值(验证$\lambda_n$是否是收敛于$\lambda$,可以转化为相减趋近于0,中间可以用夹逼定理之类的求解方式,因为$\lambda$是个抽象值,而结果需要明确的趋近于0,所以需要将式子放缩,将$\lambda$放到可以明确求解的部分)<参考 复习全书 P29>(特征:使用递推式或者迭代式表示出数列的各个项的值,极限式类似为$lim_{n \rightarrow \infty}\lambda_n$)

3、逐项相加抵消求解通项公式(递推式包含三项):求解数列的递推式,如果为$x_{n+1}-x_{n}=q(x_{n}-x_{n-1})$的形式,则将$x_{n}-x_{n-1}$逐项相加,由于$x_{n}-x_{n-1}$是公比为q的等比数列,所以$x_{n}-x_{n-1}$的值是知道的,逐项相加可以互相抵消得到$x_{n}$的式子

4、证明项的正负性:将递推的两项分离到两边,然后利用归纳法证明



\subsection{求解数列的极限-其它形式}\index{数列极限(和式或者递推式)!求解数列的极限-其它形式}

1、夹逼定理:(可放缩形式)

2、数列转函数讨论(一项表示通项公式):对于一般的数列,如果可以明确写出使得$u_{n}=f(n)$的函数时,可以利用函数来转换成讨论函数的极限(这是离散转连续的一种方式,如果是连续转离散,不同的路径可能会导致结果不一致,如在三角函数$sin(1/x),x \rightarrow 0$中)



\subsection{求解数列和的极限-积分和式形式或者可转化成积分和式形式}\index{数列极限(和式或者递推式)!求解数列和的极限-积分和式形式或者可转化成积分和式形式}

特征(用来替换和发现积分和式形式):$dx=\frac{1}{n}$,$x=\frac{i}{n}$

样例:$\lim _{n \rightarrow \infty} \sum_{i=1}^{n} f\left(\frac{i}{n}\right) \cdot \frac{1}{n}=\int_{0}^{1} f(x) \mathrm{d} x$

形式及方案:

1、和式形式:直接以和式进行表示的极限,并且有积分和式特征

2、乘积形式:n个因式的积,常指数化之后指数部分转换成和式求解

注意:序列变量$i$(也是和式的限制变量)的变化从$1-n$,或者$0-(n-1)$

备注:有时需要进行放缩(夹逼定理)才能转换成积分和式(比如两项相乘而不是n项相乘)

备注:有时和式的两项是分离的,需要进行并项处理



