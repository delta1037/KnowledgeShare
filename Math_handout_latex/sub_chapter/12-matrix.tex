\chapterimage{chapter_head_2.pdf}
\chapter{矩阵}

\section{矩阵的相关基本定义}\index{矩阵的相关基本定义}

1、矩阵、n阶矩阵、O矩阵、矩阵相等:矩阵是符号,行列式是数

2、同型矩阵:形状一致

3、单位阵、数量阵:数量阵是单位阵乘了个常数

4、对角阵、上三角、下三角阵:上三角矩阵的乘积仍是上三角矩阵

5、对称阵、反对称阵:对称($ A=A^T $),反对称($ <font color=purple>A=-A^T</font> $,对角线元素为0);当A为对称矩阵时,$ A^* $也是对称矩阵;当A为反对称矩阵时且A的阶数为奇数,$ A^* $是对称矩阵,A的阶数为偶数时,$ A^* $是反对称矩阵

6、正交阵:$ AA^T=A^TA=E $ → $ A^T = A^{-1} $(与自身转置相乘值为1)

7、初等矩阵:一次初等变换(倍乘、互换、倍加),初等矩阵的简单表示方法(在大题中用该简单表示法时作一下注明)(表示方法:倍乘:$ <font color=purple>E_{i}(k)</font> $第$ <font color=purple>i</font> $行乘$ <font color=purple>k</font> $倍;互换:$ <font color=purple>E_{ij}</font> $第$ <font color=purple>ij</font> $行互换;倍加:$ <font color=purple>E_{ij}(k)</font> $第$ <font color=purple>**j**</font> $行的$ <font color=purple>**k**</font> $倍加到第$ <font color=purple>**i**</font> $行),初等矩阵的逆与初等矩阵的关系($ <font color=purple>E_{i}^{-1}(k)=E_{i}(\frac 1 k)</font> $;$ <font color=purple>E_{ij}^{-1}=E_{ij}</font> $;$ <font color=purple>E_{ij}^{-1}(k)=E_{ij}(-k)</font> $)

8、行阶梯矩阵:零行都在矩阵底部;非零行的主元(该行最左边的第一个非0元素)所在的列下面都是0

9、行最简矩阵:行阶梯矩阵进一步满足,非零行主元是1,主元所在的其它列元素都是0

10、等价矩阵:一个矩阵经过有限次初等变换变成另一个矩阵($ B=PAQ $),则这两个矩阵等价,等价标准形(分块之后只有零块和单位阵块,且单位块在左上角)是矩阵的所有等价矩阵的最简矩阵

11、伴随矩阵:伴随矩阵具有$ AA^*=A^*A=|A|E $关系成立(并不要求矩阵可逆)

12、可逆矩阵(非奇异矩阵):$ AA^{-1}=A^{-1}A=E $,一个矩阵如果可逆则可逆矩阵是唯一的

13、相似矩阵:$ A \sim B $,$ A=P^{-1}BP $,此时$ |A|=|B| $

14、矩阵的迹:对角线的和

15、可交换矩阵:两个矩阵的相乘可以交换$ AB=BA $

16、正交矩阵:列向量两两正交,且是单位向量,定义$ AA^T=A^TA=E $,当A为正交矩阵时,$ A^* $也是正交矩阵

17、幂等阵:$ A^2=A $

18、顺序主子式:

\section{矩阵方程求解}\index{矩阵方程求解}



\subsection{求解基本步骤}\index{矩阵方程求解!求解基本步骤}

1、利用矩阵的基本变换来凑成或用已知量表示出未知量的表达式(利用逆运算时判断式子可逆性)(常用于求解逆运算,而直接求逆不方便)

2、如果已知方程,需要求解方程中的矩阵每个元素的具体值,并且如果待求解的矩阵较小,可对矩阵元素设值求解

注意:在进行逆运算时,注意判断式子是否具有可逆性;当不可逆时,转换成求解非齐次线性方程组的问题(非齐次线性方程组具有特解和齐次解,最终的式子里具有未知量并需要对未知量进行说明)

注意:表示时要用题目中所给的已知量来表示



\subsection{求解构造技巧}\index{矩阵方程求解!求解构造技巧}

1、由$ A^2=A $凑$ A+E $逆运算的形式,将$ A^2=A $中右边的A挪到左边得到$ A^2-A=O $,为了凑出$ A+E $的形式,$ A+E $需要与$ A-2E $相乘才能得到待凑式子的一部分$ A^2-A $和一个单位矩阵的和或差的形式,所以依照相乘后的结果来对原带待凑的式子$ A^2-A=O $进行变换

2、如果给出了多个提示,需要想办法消去提示式子中的中间未知量(例如$ A $的与$ A^{-1} $相乘可以消去$ A $)

3、由$ A^2=A $和$ AB+BA=O $(1式)推导$ AB=BA $,1式分别左乘$ A $和右乘$ A $得到两个式子,就可以看出来了

\section{矩阵可逆性证明}\index{矩阵可逆性证明}

证明的构造方法:

1、利用已知条件构建(利用可逆的定义)方程(平方):若$ AB=E $,则$ A,B $互为逆矩阵

2、已知式子的提因式分析:提取因式辅助

3、对现有的式子进行加项或者减项,构造可分解因式的形式

\section{矩阵-基本变换公式}\index{矩阵-基本变换公式}

1、逆:$ (AB)^{-1}=B^{-1} A^{-1} $,$ (A+B)^{-1}\ne A^{-1} + B^{-1} $

2、转置:$ (AB)^T=B^TA^T $,$ (A+B)^T=A^T + B^T $

3、伴随运算:$ (AB)^*=B^*A^* $,核心公式:$ AA^*=A^*A=|A|E $(该核心公式可以直接看出$ <font color=purple>A^*</font> $与$ <font color=purple>A</font> $是可随意交换位置的;也可以衍生出与逆运算的公式$ <font color=purple>A^*=|A|A^{-1}</font> $,此时A一定是可逆的,因为如果A不可逆那么$ <font color=purple>|A|=0</font> $会导致$ <font color=purple>A^*A=O</font> $),将$ A $挪到右边可得到$ A^*=|A|A^{-1} $(式子1),将$ A^* $挪到右边得到$ (A^*)^{-1}=\frac{A}{|A|} $,再由可交换运算可以得到$ (A^{-1})^*=\frac{A}{|A|} $(也可以令$ <font color=purple>B=A^{-1}</font> $代入到式子1或者代入到伴随核心公式中推导);伴随矩阵的行列式:$ |A^*|=|A|^{n-1} $

4、伴随矩阵乘系数的运算:$ (kA)^*=k^{n-1}A^* $(推导:依据伴随定义)

5、可交换运算:$ "*","-1","T" $是可交换的(仅限于指数运算中),并且对运算$ (AB)^x=B^xA^x $将这三种符号代入到$ x $均成立;相同的$ T $运算合并成1次方运算,相同的$ -1 $运算合并成$ -2 $次方运算;相同的*符号可以由$ (A^*)^{*}=|A|^{n-2}A $转换成数值与原矩阵的积(推导(用到了伴随矩阵的行列式运算):$ <font color=purple>(A^*)^{*}=|A^*|(A^*)^{-1}=|A^*|\frac{A}{|A|}=|A|^{n-2}A</font> $,$ <font color=purple>|(A^*)^{*}|=|A|^{(n-1)^2}</font> $)

注意:在进行矩阵相乘相关的计算时,注意乘积的顺序,特别是次方与次方之间的合并时

注意:忘记过一次$ <font color=purple>E</font> $前边的系数

\section{分块矩阵-运算定义}\index{分块矩阵-运算定义}

1、矩阵基本运算:加法、乘法、转置、次方(正对角线,没有副对角线)、逆运算(正对角线与逆对角线,有区别,注意其中是两块$ <font color=purple>**O**</font> $区域);没有交换和消去律

2、按行或者按列分块:若$ AB=O $,则$ B $的列向量($ B $列分块)是方程组$ Ax=0 $的解;若$ AB=C $,对$ B、C $按行分块,则$ C $的行向量可以由$ B $的行向量线性表出(系数是$ A $的行向量),对$ A、C $按列分块,则$ C $的列向量可以由A的列向量线性表出(系数是$ B $的列向量)

3、矩阵的行列式运算(拉普拉斯展开式):含有一个$ O $块,注意正对角线和副对角线的系数区别(注意与不含$ <font color=purple>O</font> $块仅含副对角线的行列式求解公式的系数的区别)

\section{矩阵的秩的计算}\index{矩阵的秩的计算}



\subsection{矩阵的秩的计算}\index{矩阵的秩的计算!矩阵的秩的计算}

0、求解矩阵的秩:对矩阵做初等行变换或者列变换均可(列秩和行秩相等);

1、矩阵的乘积:$ r(AB) \leqslant min\{r(A),r(B)\} $;$ r(AB) \leqslant r(A)+r(B) $(积的秩可能(等号)会变小)(对于两个列向量,有$ <font color=purple>r\left(\boldsymbol{\beta} \boldsymbol{\alpha}^{\mathrm{T}}\right) \leqslant r(\boldsymbol{\beta})=1</font> $,可以使用方程组来解释,或者向量的相关性判断)

2、矩阵的乘积且结果为$ O $:$ AB=O \rightarrow r(A)+r(B) \leqslant n $,其中n为中间阶数($ A_{mn} $与$ B_{ns} $)(推导:将矩阵相乘转换为方程组的解的问题,$ <font color=purple>B</font> $是$ <font color=purple>A</font> $的解构成的矩阵,即可得到$ <font color=purple>r(B)\leqslant n - r(A) </font> $,其中n是A的列,也是未知数的个数)

3、矩阵的乘积且其中一个可逆:$ r(AB)=r(B) $,其中$ A $可逆

4、矩阵的和差:$ r(A)-r(B)\leqslant r(A\pm B) \leqslant r(A)+r(B) $(矩阵积或者和的秩可能(等号)会小于两个矩阵的秩和)

5、两个矩阵互为转置:$ r(A^TA)= r(A) $($ <font color=purple>A</font> $不可逆也行:可以利用特征值来证明,$ <font color=purple>**A^TA**</font> $与$ <font color=purple>**A**</font> $的特征值为平方关系)($ <font color=purple>A</font> $不是方阵也成立,如果$ <font color=purple>A</font> $不是方阵,那么可以得出$ <font color=purple>A^TA \ne E</font> $,用秩的关系证明)

6、矩阵的组合:$ max\{r(A),r(B)\}  \leqslant r(A,B) \leqslant r(A)+r(B) $

7、单个矩阵$ A_{mn} $性质为$ 0 \leqslant A_{mn} \leqslant min\{ m,n \} $(矩阵的秩小于行数或者列数的最小值)

8、分块矩阵:其中包含两块$ O $区域,则秩为另两块区域的秩的和

9、运算不变性:矩阵乘不为0的常系数或者转置之后的秩不变



\subsection{伴随矩阵与原矩阵秩的关系}\index{矩阵的秩的计算!伴随矩阵与原矩阵秩的关系}

0、对$ n(n \geqslant 2) $阶矩阵$ \boldsymbol{A}, r\left(\boldsymbol{A}^{*}\right)= \begin{cases}n, & r(\boldsymbol{A})=n, \\ 1, & r(\boldsymbol{A})=n-1, \\ 0, & r(\boldsymbol{A}) \leqslant n-2 .\end{cases} $

1、原矩阵秩为n:伴随矩阵秩为n

2、原矩阵秩为n-1:伴随矩阵秩为1(考虑特殊情况,矩阵的最后一列和最后一行是0,这时所有的代数余子式只有第$ <font color=purple>A_{nn}</font> $的不是零,其它位置的代数余子式都包含最后一行和最后一列0,导致其它的代数余子式为0)

3、原矩阵秩为小于n-1:伴随矩阵秩为0(考虑特殊情况,矩阵的最后两列和最后两行是0,这时所有的代数余子式都是0,任意个位置的代数余子式都至少包含一行或者至少一列0,导致所有代数余子式为0)

\section{求解矩阵的n次方问题}\index{求解矩阵的n次方问题}

1、矩阵$ A $可以分解成$ \alpha\beta^T $的情况:常用于行或列有倍数关系的矩阵或者明确告知该关系的矩阵,$ \beta^T\alpha $为具体的数值,则可以将数值$ X $提取出来,做一次$ \alpha\beta^T $的运算即可,注意此时数值$ <font color=red>X</font> $的指数应该是$ <font color=red>n-1</font> $次方而不是$ <font color=red>n</font> $次方

2、$ (A+E)^n $的情况:利用二项式进行分解,一般只需要分解到A的某一个次方开始为0即可(可能是副对角线在内的其中一边全是0,这时高次方导致矩阵值为0;或者A是第一种情况(矩阵$ <font color=purple>A</font> $可以分解成$ <font color=purple>\alpha\beta^T</font> $的情况));或者找到A的次方的规律

3、可以分块的矩阵:如果存在$ \alpha,\alpha^T $的形式,分块之后,试乘就变成了第一种情况(矩阵$ A $可以分解成$ \alpha\beta^T $的情况),注意对偶次方和奇数次方的分类讨论,这时需要乘到四次方才能找得到规律;如果分成了对角线的方块,可以分块计算

4、试乘找规律:乘一项看看,与原矩阵有没有什么规律可循(通常是常数倍)

5、求解相似对角阵:矩阵$ A $分解成$ A=P^{-1}BP,A=PBP^{-1} $的情况,转换成了求$ B $的$ n $次方问题

注意:求得结果之后代入1次方的情况进行验证(前边系数的次方应该为$ <font color=red>**n-1**</font> $而不是$ <font color=red>**n**</font> $,错两次了)

\section{矩阵的行列初等变换}\index{矩阵的行列初等变换}

1、初等矩阵左乘B时,相当于对$ B $做相应的行初等变换(理解:如果初等变换是第$ <font color=purple>i</font> $行加到第$ <font color=purple>j</font> $行,将矩阵B做横向量切分,由该初等矩阵左乘时,当乘到第$ <font color=purple>j</font> $行时,就是B的第$ <font color=purple>i</font> $个行向量与第$ <font color=purple>j</font> $个行向量相加并放到第$ <font color=purple>j</font> $行,表现为$ <font color=purple>B</font> $的第$ <font color=purple>i</font> $行数据加到了第$ <font color=purple>j</font> $行,与初等矩阵的行变换一致)

2、初等矩阵右乘B时,相当于对B做相应的列初等变换(理解:如果初等变换是第$ <font color=purple>i</font> $列加到第j列,将矩阵B做列向量切分,由该初等矩阵右乘时,当乘到第j列时,就是B的第i个列向量与第j个列向量相加并放到第j列,表现为B的第i列数据加到了第j列,与初等矩阵的列变换一致)

备注:其实一个初等矩阵的行变换也可以认为是列变换,决定矩阵B是做行初等变换还是做列初等变换取决于初等矩阵的位置

备注:简记为前乘行变换,后乘列变换

注意:注意做初等变换的顺序,总是离B最近的变换先做,逆向求初等 变换时特别注意,列式求解

\section{矩阵-相关矩阵求解}\index{矩阵-相关矩阵求解}



\subsection{求解伴随矩阵}\index{矩阵-相关矩阵求解!求解伴随矩阵}

1、计算逆矩阵(可逆的条件下):利用逆矩阵与伴随矩阵的关系,计算逆矩阵

2、 计算代数余子式:利用代数余子式构造伴随矩阵,注意排列顺序

注意:伴随矩阵与原矩阵的代数余子式的对应关系



\subsection{求解逆矩阵}\index{矩阵-相关矩阵求解!求解逆矩阵}

1、初等行变换法:常用(可以适当扩展一下行变换的原理)

2、分块矩阵:注意分块形式的利用

3、定义法:设未知量,求解(分块求解也可以)

4、利用伴随矩阵:参考伴随矩阵的求解

备注:初等矩阵的逆矩阵($ <font color=orange>E_{i}^{-1}(k)=E_{i}(\frac 1 k)</font> $;$ <font color=orange>E_{ij}^{-1}=E_{ij}</font> $;$ <font color=orange>E_{ij}^{-1}(k)=E_{ij}(-k)</font> $)



\subsection{求解可交换矩阵}\index{矩阵-相关矩阵求解!求解可交换矩阵}

1、设矩阵各个元素,求解

2、可交换定义:$ AB=BA $

\section{矩阵-是否可逆的证明}\index{矩阵-是否可逆的证明}



\subsection{可逆的证明(充要条件)}\index{矩阵-是否可逆的证明!可逆的证明(充要条件)}

1、行列式法:矩阵的行列式不为0(等价于矩阵可逆)

2、秩法:矩阵为满秩

3、方程组法:齐次方程组只有零解;或者非齐次方程组只有唯一解(注意此时不包括齐次值全为0的情况)

4、反证法:设矩阵不可逆,即存在不为0的向量使得与矩阵相乘为0(齐次方程有非零解),解得系数后与现有条件矛盾,则可逆

5、定义法:存在$ n $阶矩阵$ B $,使得$ AB=E $或者$ BA=E $成立

6、特征值法:特征值全不为0



\subsection{不可逆的证明}\index{矩阵-是否可逆的证明!不可逆的证明}

1、行列式:行列式值为0,转换成了行列式的求解问题

2、秩法:矩阵为非满秩

3、方程组法:齐次方程组有非零解

4、特征值法:特征值有一个是0

