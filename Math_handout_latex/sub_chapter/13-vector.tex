\chapterimage{chapter_head_2.pdf}
\chapter{向量}

\section{矩阵等价与向量组等价}\index{矩阵等价与向量组等价}



\subsection{矩阵等价与向量组等价区别}\index{矩阵等价与向量组等价!矩阵等价与向量组等价区别}

1、向量组等价:向量组等价可以推出向量组的秩相等;向量组的秩相等无法推出向量组等价(两个向量组的秩和组合后的向量组的秩三者相等可以得到向量组等价)

2、矩阵等价:矩阵等价与秩相等是可以互相推出的(充要条件:两个矩阵的秩相等),前提是矩阵为同型矩阵

备注:原因是矩阵的等价变换是两个维度的;向量组的变换只有一维的组合



\subsection{矩阵的向量组等价推论}\index{矩阵等价与向量组等价!矩阵的向量组等价推论}

1、矩阵$ A、B $的列向量组等价:存在$ P $使得$ AP=B $,存在$ Q $使得$ BQ=A $,进一步可以推知$ APQ=A $,但这时不可以认为$ PQ=E $

2、矩阵$ A、B $的行向量组等价:存在$ P $使得$ PA=B $,存在$ Q $使得$ QB=A $,此时有$ Ax=0 $时,$ PAx=0 \rightarrow Bx=0 $;进一步可以推知$ PQB=B $,但这时不可以认为$ PQ=E $

备注:如果$ <font color=purple>Ax=0</font> $与$ <font color=purple>Bx=0</font> $同解,此时$ <font color=purple>A、B</font> $的行向量组等价(直观上来说,齐次方程组的线性组合之后的方程的解与方程组的解一致)(推导路径是什么样的?:通过AB的竖式等于AO的竖式的解,所以B可以由A的线性组合消掉)



\subsection{向量组等价的判定}\index{矩阵等价与向量组等价!向量组等价的判定}

1、秩:两个向量组的秩和组合后的向量组的秩三者相等

2、线性表出:构建两个向量组的矩阵,做初等行变换(相当于可逆变换),判断两个向量组是否能互相线性表出(非齐次方程求解问题)

3、秩与线性表示:两个向量组的秩相等,并且其中一个向量组可以由另一个表示 

4、向量组转换方程:证明转换系数矩阵可逆(满秩:转换系数矩阵的秩的上限(可以由矩阵的大小决定)和下限$ <font color=purple>r(AB) \le r(A)</font> $),即可得到可以互相转换

\section{向量空间的基}\index{向量空间的基}



\subsection{不同基下有相同坐标的向量}\index{向量空间的基!不同基下有相同坐标的向量}

1、设基坐标,联立矩阵方程求解(联立求解的条件是基组成的矩阵是可逆的;联立的中心是不同基下的该向量)

注意:求向量时,求出向量在基下的坐标之后需要用基将向量表示出来



\subsection{证明向量组是向量空间的一组基}\index{向量空间的基!证明向量组是向量空间的一组基}

1、由一组基(向量)判断另一组基(向量)(过渡来的):过渡矩阵满秩(过渡矩阵不为0)

2、只有一组向量:行列式满秩,向量组线性无关(证明向量组线性无关即可,而且向量组组成方阵,即秩等于向量组的向量维度)

\section{求解矩阵的极大线性无关组}\index{求解矩阵的极大线性无关组}

1、定义:向量组中的一组向量线性无关,再添加任一个向量后线性相关

2、求解:做初等行变换,求出阶梯矩阵

备注:只有零向量的向量组没有极大线性无关组

\section{向量组相关性}\index{向量组相关性}



\subsection{已知线性相关或者线性无关求解未知参数}\index{向量组相关性!已知线性相关或者线性无关求解未知参数}

1、构建齐次方程:转换为考虑齐次方程是否具有非零解



\subsection{同一组向量的相关性判断}\index{向量组相关性!同一组向量的相关性判断}

1、根据向量组值:当向量组为方阵时,向量组的行列式值不为0(或者说非满秩n阶矩阵)则线性无关

2、构建齐次方程:组成的齐次方程只有零解,则向量组线性无关;否则线性相关



\subsection{两组向量的相关性的判断}\index{向量组相关性!两组向量的相关性的判断}

1、少表多:向量组个数多的能够由向量组少的线性表出,则向量组个多的必线性相关(因为多的肯定有重复的)

2、多表少:如果向量组1能够由向量组2线性表出,且向量组1线性无关 ⇒ 则向量组2的向量个数大于等于向量组1的向量个数(相当于少的向量组是从多的向量组里面抽取出来的线性无关向量,有可能抽不完, 再抽可能抽到相对于少的向量组来说线性无关的或者线性相关的向量)



\subsection{判断线性无关的向量构成的新的向量组是否线性相关}\index{向量组相关性!判断线性无关的向量构成的新的向量组是否线性相关}

1、定义证明线性相关(观察法):挑选一个线性无关向量(在新向量组中只出现在两个新向量中组合),先确定这两个向量的系数,逐步地确定其它的系数

2、定义证明线性无关(反证法):假设存在一组为0的系数使得向量组为0,然后证明系数只能是0(转换成线性无关向量组成的方程,构建系数为0的方程,证明齐次方程组只有零解)

3、构建过渡矩阵:构建向量的过渡矩阵,如果过度矩阵行列式不为0,则就是线性无关的;反之线性相关



\subsection{将某个向量由已知向量线性表出}\index{向量组相关性!将某个向量由已知向量线性表出}

1、构建非齐次方程组:转换为非齐次方程组的求解问题(注意利用秩与解的关系,系数矩阵和增广矩阵的秩相等时非齐次方程有解,即表出)

\section{由向量组求解过度矩阵}\index{由向量组求解过度矩阵}

1、由$ [\alpha_1,\alpha_2,\cdots,\alpha_n] $向$ [\beta_1,\beta_2,\cdots,\beta_n] $过渡(列变换):过渡方程为:$ [\alpha_1,\alpha_2,\cdots,\alpha_n]C=[\beta_1,\beta_2,\cdots,\beta_n] $,其中$ C $为过渡矩阵(注意:过渡的方向)

2、求解过渡矩阵:利用过渡方程转换你为求逆运算(在$ [\alpha_1,\alpha_2,\cdots,\alpha_n] $为可逆矩阵的条件下),即$ C=[\alpha_1,\alpha_2,\cdots,\alpha_n]^{-1}[\beta_1,\beta_2,\cdots,\beta_n] $

3、坐标的过渡矩阵:$ [\alpha_1,\alpha_2,\cdots,\alpha_n] $下坐标为$ [x_1,x_2,\cdots,x_n] $,$ [\beta_1,\beta_2,\cdots,\beta_n] $下坐标为$ [y_1,y_2,\cdots,y_n] $,$ C $为过渡矩阵,则$ x=Cy $(证明:1中的过度方程两边同时右乘$ <font color=purple>y</font> $,与$ <font color=purple>[\alpha_1,\alpha_2,\cdots,\alpha_n][x_1,x_2,\cdots,x_n]^T=[\beta_1,\beta_2,\cdots,\beta_n][y_1,y_2,\cdots,y_n]^T</font> $对比)

\section{向量空间的定义}\index{向量空间的定义}

1、维数:维数等于向量组的秩

2、基与基坐标:这两个概念是绑定的,基坐标是相对于基来说的;基是满秩的

3、自然基:

4、解空间:解空间维数与矩阵秩的关系($ n-r(\boldsymbol{A}) $)

5、极大线性无关组:向量组中的一组向量线性无关,再添加任一个向量后线性相关

6、正交向量组:两两正交

7、正交规范向量组:两两正交且是单位向量

8、规范正交基(标准正交基):任意的不同两个基的内积为0,相同的基内积为1;变换不变性:一组规范正交基经过正交矩阵变换之后还是一组规范正交基(变换矩阵为正交是充要条件)(可以用正交矩阵定义式证明)

9、秩:定义为极大线性无关组中向量的个数

\section{正交矩阵}\index{正交矩阵}



\subsection{正交矩阵相关定义}\index{正交矩阵!正交矩阵相关定义}

1、定义:$ AA^T=A^TA=E $

2、推论:$ A^T=A^{-1} $,(根据可逆的定义推导)

3、推论:$ A $的行或者列向量是正交规范向量组

4、推论:行列式的值$ |A|=1 $或者$ |A|=-1 $(证明:对定义取行列式即可)



\subsection{正交矩阵的证明}\index{正交矩阵!正交矩阵的证明}

1、利用定义:$ AA^T=A^TA=E $

2、

