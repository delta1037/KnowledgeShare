\chapterimage{chapter_head_2.pdf}
\chapter{二次型}

\section{矩阵合同的判断}\index{矩阵合同的判断}



\subsection{两个矩阵合同的充要条件:(推导合同性)}\index{矩阵合同的判断!两个矩阵合同的充要条件:(推导合同性)}

1、矩阵$ A,B $具有相同的正负惯性指数



\subsection{两个矩阵合同的必要条件:(合同性的推论)}\index{矩阵合同的判断!两个矩阵合同的必要条件:(合同性的推论)}

1、矩阵$ A,B $具有相同的正负惯性指数

2、矩阵$ A,B $的秩相等

\section{二次型相关概念}\index{二次型相关概念}

1、二次型:$ n $个变量的二次齐次多项式,当系数均为实数时,称为n元实二次型

2、二次型矩阵表示:二次型表示矩阵为对称阵形式;$ r(f) $是求解二次型系数矩阵的秩(注意求解二次型的秩时一定要是对实对称矩阵求解的)

3、二次型的标准形:二次形式子中只有平方项,没有混合项

4、二次型的规范性:二次形的标准形式子中的系数只有0,-1,1(注意规范型的表示是由只有平方项的二次型的式子表示的)

5、合同矩阵和合同二次型:(合同的定义)合同为存在可逆阵$ C $,使得$ C^TAC=B $,记为$ A \simeq B $(合同的表示)

6、正惯性指数,负惯性指数和符号差:正惯性指数为标准形中正平方项的项数;负惯性指数为标准形中负平方项的项数;符号差是正惯性指数减去负惯性指数;秩就等于正惯性指数加上负惯性指数

7、正定与正定二次形:(正定的定义)对于任意的非零值代入到二次形式子中,式子值恒大于零(注意不包括0),即只含有正系数平方项的式子是正定的(注意判断正定的首要条件是矩阵是否是实对称矩阵)

8、惯性定理:二次型经过可逆线性变换,正负惯性指数不变,秩不变

\section{二次型与曲面关联}\index{二次型与曲面关联}

判断方式:将二次型式子化成标准形式,与常见的曲面类型进行对比($ f(x,y,z)=1 $)

1、椭球或者球:系数全正(从图像上看是一个包围的闭曲面)

2、单叶双曲面:两正一负(负号个数决定有几个”叶“)(从图像上看是一个连在一起的面)(双曲线旋转轴问题)

3、双叶双曲面:两负一正(负号个数决定有几个”叶“)(从图像上看是两个不相连接的面)(双曲线旋转轴问题)

4、柱面:两正一零(椭圆在一个方向上延伸)

5、双曲柱面:一正一负一零(双曲线在一个方向上延伸)

6、两个平行平面:一正两零(单个轴上的两个值在平面上延伸)

注意:曲面类型与二次型的秩有联系

备注:当求解二次型式子的标准型时(获得标准形或者判断曲面类型),注意化简之后的平方项里构成的变换矩阵需要是满秩的(或行列式不为0,或向量组线性无关),否则就需要进一步化简(被坑两次,这里又写一次,下一次不被坑就删掉)

\section{二次型、标准型、规范型}\index{二次型、标准型、规范型}



\subsection{二次型化为标准型:(重点)}\index{二次型、标准型、规范型!二次型化为标准型:(重点)}

定理:对于任意一个n元二次型,必定存在(因为二次型对应矩阵是实对称矩阵)一个正交变换$ x=Qy $,化二次型为标准形(从$ x^TAx \Rightarrow y^TQ^TAQy $),即$ Q^{-1}AQ=Q^TAQ=\Lambda $

1、配方法:若二次型中含有某变量的平方,则先集中处理含该变量的项,对含该变量的项配方,直至全部配成完全平方,再利用同样的方法处理含其它变量的项;若二次型中只含有混合项,则先选取其中一个,利用平方差公式将混合项转化为平方项,再进行配方

注意:在求解可逆线性变换时,如果表示的方程个数不够,注意补充(即所作的线性变换一定是可逆的)

2、正交变换:利用正交变换$ x=Py $将二次型$ x^TAx $化为标准形与利用正交矩阵$ P $把实对称矩阵$ A $化为对角矩阵的过程一致。此时二次型变为$ y^TP^TAPy $,又因为$ P $是正交矩阵,其中$ P^TAP =P^{-1}AP= \Lambda $(注意表示格式),矩阵$ A $既相似于对角矩阵,又合同于对角矩阵,二次型的标准形的系数为$ A $的特征值(当求解出二次型得特征值之后,就可以得到标准型得格式)

备注:当求解二次型式子的标准型时(获得标准形或者判断曲面类型),注意化简之后的平方项里构成的变换矩阵需要是满秩的(行列式不为0,向量组线性无关),否则就需要进一步化简(被坑两次)



\subsection{标准型化成规范性}\index{二次型、标准型、规范型!标准型化成规范性}

1、分析正惯性指数和负惯性指数的个数(或者将对角阵形式进一步变换)

2、构造新的二次式(注意并非是矩阵的形式)



\subsection{规范型的求解}\index{二次型、标准型、规范型!规范型的求解}

1、正定二次型:正定二次型的规范型是系数全为1的平方和

2、秩求解项个数:利用秩判断总的项的个数,如果还能判断只有正惯性指数或者负惯性指数就可以直接写出来规范型

3、计算特征值:由特征值的正负情况得到规范型的系数

4、配方法:将二次型进行配方,得到向规范型的转换方程(要写出来这个方程),从而得到规范型



\subsection{二次型不同的正交变换关系}\index{二次型、标准型、规范型!二次型不同的正交变换关系}

1、关联:二次型正交变换成标准形形式,即标准形的每一个值对应正交变换对应的列(特征值与特征向量的对应关系);当正交矩阵的对角阵发生列的交换时,对应的特征向量也会发生响应的列变换,即列之间的变换关系

\section{矩阵正定判断}\index{矩阵正定判断}



\subsection{正定的充要条件:(判定条件与推论)}\index{矩阵正定判断!正定的充要条件:(判定条件与推论)}

1、特征值法:矩阵$ **A** $的全部特征值大于0;矩阵的正惯性指数等于未知量的个数(实际上还是特征值大于0)

2、定义法:仅在平方项的值为0时等号才成立(平方项为0当且仅当构成的齐次方程组只有零解(各个项线性无关);若存在非零解,则各个项之间线性相关,导致值非零时也等于零就不满足正定的定义)(反证法);若方阵$ A $各列线性无关,则$ A^TA $正定:$ A $各列线性无关,则$ Ax=0 $只有零解,所以当$ x\ne 0 $时,$ x^TA^TAx = (Ax)^TAx \ne 0 $

3、矩阵合同于$ E $:对于矩阵$ A $有$ A \simeq E $,即存在可逆矩阵$ C $,使得$ C^TAC=E $,则$ A $正定;对于矩阵$ A $有$ A=D^TD $,其中$ D $是可逆阵(由$ <font color=purple>A=D^TD</font> $可以得到$ <font color=purple>(D^{-1})^TAD^{-1}=E</font> $),则$ A $正定

4、行列式法:$ A $的全部顺序主子式大于0,$ a_{11}>0,\left|\begin{array}{ll}a_{11} & a_{12} \\a_{21} & a_{22}\end{array}\right|>0, \cdots,\left|\begin{array}{ccc}a_{11} & \cdots & a_{1 n} \\\vdots & & \vdots \\a_{n 1} & \cdots & a_{n n}\end{array}\right|>0 $

性质:可逆线性变换不改变矩阵的正定性(作可逆线性变换成标准形)



\subsection{正定的必要条件:(推论)}\index{矩阵正定判断!正定的必要条件:(推论)}

1、矩阵$ A $的主对角元素大于0:$ a_{ii} > 0 $

2、矩阵$ A $的行列式$ |A| $大于0



\subsection{判断非正定}\index{矩阵正定判断!判断非正定}

1、不满足正定的必要条件

2、不满足正定的定义

