\chapterimage{chapter_head_2.pdf}
\chapter{随机变量的数字特征}

\section{数学期望定义与性质}\index{数学期望定义与性质}



\subsection{离散型随机变量}\index{数学期望定义与性质!离散型随机变量}

1、一维:$\sum_{k=1}^{\infty}x_kp_k$

2、一维函数:$E(Y)=E[g(X)]=\sum_{k=1}^{+\infty} g\left(x_{k}\right) p_{k}$(分布律还是$X$变量的分布律,相当于套了一个函数上去,但是函数值对应的概率没有变,所以还是用的原来的分布律)

3、二维函数:$E(Z)=E[g(X, Y)]=\sum_{i=1}^{+\infty} \sum_{j=1}^{+\infty} g\left(x_{i}, y_{j}\right) p_{i j}$(分布律还是$X,Y$变量的分布律,相当于套了一个函数上去,但是函数值对应的概率没有变,所以还是用的原来的分布律)

注意:所有级数绝对收敛



\subsection{连续型随机变量}\index{数学期望定义与性质!连续型随机变量}

1、一维:$E(X)=\int_{-\infty}^{+\infty} x f(x) \mathrm{d} x$

2、一维函数:$E(Y)=E[g(X)]=\int_{-\infty}^{+\infty} g(x) f(x) \mathrm{d} x$(分布函数还是$X$变量的分布函数,相当于套了一个函数上去,但是函数值对应的概率没有变,所以还是用的原来的概率分布)

3、二维函数:$E(Z)=E[g(X, Y)]=\int_{-\infty}^{+\infty} \int_{-\infty}^{+\infty} g(x, y) f(x, y) \mathrm{d} x \mathrm{~d} y$(分布函数还是$X,Y$变量的分布函数,相当于套了一个函数上去,但是函数值对应的概率没有变,所以还是用的原来的概率分布)

注意:所有积分绝对收敛



\subsection{期望的性质}\index{数学期望定义与性质!期望的性质}

1、性质1:$E(a_1X_1+a_2X_2+\cdots+a_nX_n+b_1+b_2+\cdots+b_m)=a_1E(X_1)+a_2E(X_2)+\cdots+a_nE(X_n)+b_1+b_2+\cdots+b_m$(无论变量之间是否独立都成立!这点很重要,可以在有些变量相关的条件下简化计算)

2、性质2:若$X$、$Y$不相关,则(相关系数计算的一部分)$C o v ( X , Y ) = E ( X Y ) - E ( X ) E ( Y ) =0$,即可以得到$E ( X Y ) = E ( X ) E ( Y )$

\section{方差的定义与求解}\index{方差的定义与求解}



\subsection{定义}\index{方差的定义与求解!定义}

1、方差定义:$D(X)=E\left\{[X-E(X)]^{2}\right\}$

1、标准差/均方差表示:$\sigma(X)=\sqrt{D(X)}$

2、推论1:$D(X)=E\left(X^{2}\right)-[E(X)]^{2}$(定义公式展开)

3、推论2:$E\left(X^{2}\right) \geqslant[E(X)]^{2}$(因为对任何随机变量$X, D(X) \geqslant 0$)



\subsection{性质}\index{方差的定义与求解!性质}

1、性质1(可推广):$D(a X+b)=a^{2} D(X) $

2、性质2(前提条件:$X,Y$相互独立/不相关):$D(X \pm Y)=D(X)+D(Y)$

3、性质2(前提条件:$X,Y$关系未知):$D ( X \pm Y ) = D ( X ) + D ( Y ) \pm 2 Cov ( X , Y )  $(方差展开公式)(其中协方差可以用相关系数与协方差的关系替换)



\subsection{根据已知概率密度求解方差}\index{方差的定义与求解!根据已知概率密度求解方差}

1、先想到方差公式:$D(X)=E\left(X^{2}\right)-[E(X)]^{2}$,因为该公式只需要求一阶和二阶原点矩,已知概率密度利用原点矩的定义,很好求

2、如果方差变量是已知变量的函数形式,利用方差的性质公式进行展开

\section{矩的定义与求解}\index{矩的定义与求解}

1、$k$阶原点矩:$E\left(X^{k}\right), \quad k=1,2, \cdots $

2、$k$阶中心距:$E\left\{[X-E(X)]^{k}\right\}, \quad k=1,2, \cdots$

3、$k+l$阶混合原点矩:$E\left(X^{k} Y^{l}\right), \quad k, l=1,2, \cdots$

4、$k+l$阶混合中心矩:$E\left\{[X-E(X)]^{k}[Y-E(Y)]^{\iota}\right\}, \quad k, l=1,2, \cdots$

\section{独立与不相关}\index{独立与不相关}



\subsection{定义}\index{独立与不相关!定义}

1、协方差定义:$\operatorname{Cov}(X, Y)=E\{[X-E(X)][Y-E(Y)]\}$

2、相关系数定义:$\rho_{X Y}=\frac{\operatorname{Cov}(X, Y)}{\sqrt{D(X)} \sqrt{D(Y)}}$

注意:协方差要求$E\{[X-E(X)][Y-E(Y)]\}$存在(不存在是什么情况)

注意:相关系数要求$D(X) D(Y) \neq 0$;如果$D(X) D(Y)=0$, 则$\rho_{X Y}=0$



\subsection{性质}\index{独立与不相关!性质}

1、协方差公式1:$C o v ( X , Y ) = E ( X Y ) - E ( X ) E ( Y ) $(公式展开)

2、协方差公式2:$D ( X \pm Y ) = D ( X ) + D ( Y ) \pm 2 Cov ( X , Y )  $(方差展开公式)

3、协方差性质1:$Cov(X, Y)=Cov(Y, X)$

4、协方差性质2:$\operatorname{Cov}(a X, b Y)=a b \operatorname{Cov}(X, Y)$,其中$a,b$为常数(协方差简化计算)

5、协方差性质3:$\operatorname{Cov}\left(X_{1}+X_{2}, Y\right)=\operatorname{Cov}\left(X_{1}, Y\right)+\operatorname{Cov}\left(X_{2}, Y\right)$(协方差简化计算)



\subsection{相关系数性质}\index{独立与不相关!相关系数性质}

1、相关系数性质1:$\left|\rho_{X Y}\right| \leqslant 1$

2、相关系数性质2:$\mid \rho_{X Y} \mid=1$的充分必要条件是存在常数$a$和$b$,其中$a \neq 0$, 使得$P\{Y=a X+b\}=1$(即相关是两个变量存在线性关系,相反如果完全没有线性关系,那么相关系数就是0)



\subsection{独立与不相关的关系}\index{独立与不相关!独立与不相关的关系}

1、如果随机变量$X$和$Y$相互独立,则$X$和$Y$必不相关;反之,$X$和$Y$不相关时,$X$和$Y$却不一定相互独立(相关系数$\rho_{X Y}=0$只能说明变量之间是不相关的,不能说是互相独立的(没有线性关系不一定没有其它类型关系))

2、对二维正态随机变量$(X,Y)$,$X$和$Y$相互独立的充分必要条件是$\rho=0$(二维正态随机变量相互独立与不相关是等价的)



\subsection{相关性求解:(协方差和相关系数)(相关系数可以判断变量是否相关)}\index{独立与不相关!相关性求解:(协方差和相关系数)(相关系数可以判断变量是否相关)}

0、先求解期望和方差(可利用期望和二阶原点矩求解)

1、利用协方差的定义求解协方差

2、运用相关系数的定义求解相关系数

技巧1、对于组合的变量求解协方差时,利用协方差性质分解成相关和不相关的两部分(随机变量和的分解,级数的分解),不相关的一部分为0,可以简化协方差的计算

技巧2、事件的对称性的应用:多个事件除了符号表示,内涵是一样的,有效利用(具有对称性)可以简化求解协方差中的具体参数(中心距之类的)<复习全书P517 例9>



\subsection{判断两个随机变量是否独立}\index{独立与不相关!判断两个随机变量是否独立}

1、利用变量独立的定义:独立时$P\{X \le a, Y \le a\} = P\{X \le a\}P\{Y \le a\}$,如果需要证明变量不独立利用事件包含的关系证明这个式子不能取等号即可(反证法)

2、二维正态分布如果相关系数为0,则两个变量也是相互独立的

注意:如果事件之间有关联,包含关系,则肯定是相关的,利用关联或者包含的事件不满足变量独立的定义求解,包含关系会导致$P\{X \le a, Y \le a\} = P\{X \le a\}$这种类似的情况(随机变量的区间存在包含关系)

\section{求解期望}\index{求解期望}



\subsection{简单的事件期望(单一)}\index{求解期望!简单的事件期望(单一)}

1、求解事件的分布律

2、根据定义计算期望



\subsection{复杂的事件期望(事件的组合、重复类型,独立或者不独立,不影响期望的和差计算)}\index{求解期望!复杂的事件期望(事件的组合、重复类型,独立或者不独立,不影响期望的和差计算)}

1、将事件分解为简单的事件(分解的是事件,组合的是期望)

2、简单的事件利用期望的定义求解

3、将简单事件的期望进行组合



\subsection{组合与重复求解期望样例}\index{求解期望!组合与重复求解期望样例}

1、筛子独立的抛三次,拆分成每一次

2、从N件产品取出n件产品,求解次品个数的期望,拆分成每一件是次品的期望,因为是求解次数,所以需要将取出是次品的变量的值设置为1即可,如果是求取出良品的个数的期望,则设置取出是良品的变量的值为1<复习全书P505 9>

3、独立重复实验n次,拆分成每一次

4、将n个球放入到N个盒子,拆分成对每一个盒子含有球的概率,对于每一个盒子,n个球过来,求解这n个球放入的概率(可以先求对立事件n个球都没有放入的概率),并将盒子有球变量设置为1(对应着这一个盒子有球)这样可以得到一个盒子有球的期望,虽然各个盒子不是独立的,但是期望的性质并没有要求独立性(妙蛙),所以可以求解

注意:对于复杂事件求解期望无论事件是否独立都可以,因为期望的性质并没有限制事件是否独立



\subsection{求解独立同分布的$X,Y$,$Z=min(X,Y)$的数学期望}\index{求解期望!求解独立同分布的$X,Y$,$Z=min(X,Y)$的数学期望}

1、对于类似的复杂的函数都是用二维函数定义求解:$E(Z)=E[g(X, Y)]=\int_{-\infty}^{+\infty} \int_{-\infty}^{+\infty} g(x, y) f(x, y) \mathrm{d} x \mathrm{~d} y$

2、可以先对变量进行归一化处理(将两个变量表示的函数转换成一个变量表示的函数?),归一化的函数可以利用期望的定义进行求解

3、非标准正态分布的函数转换成标准正态分布求解(正态分布标准化),可以简化计算



\subsection{从概率密度求解期望:(或与变量函数进行组合)}\index{求解期望!从概率密度求解期望:(或与变量函数进行组合)}

1、求解原始变量的期望:依据连续型随机变量期望的一维定义计算

2、求解原变量一维函数的期望:依据连续型随机变量期望的一维函数定义计算

3、求解原变量二维函数的期望:依据连续型随机变量期望的二维函数定义计算(注意该情况下可能会以实际的例子出题,概率密度可能需要根据题意进行求解)(如果求解某个与另外两个连续变量相关的变量的期望,则就是这种二维函数期望类型的,即如果出现多变量组合,就组合成函数形式,求函数期望)

4、验证泊松分布的期望时,注意利用泊松分布的级数和为1的性质

