\chapterimage{chapter_head_2.pdf}
\chapter{概率论公式}

\section{大数定律}\index{大数定律}

(变量自解释,不赘述)(注意有些公式需要变量独立性,一般用这种公式题目自然会给定)

1、概率估计:$P\{|X-\mu| \geqslant \varepsilon\} \leqslant \frac{\sigma^{2}}{\varepsilon^{2}}$(切比雪夫不等式)

2、依概率收敛:$\lim_{n \rightarrow \infty} P\{|X_n-A| \le \varepsilon \}=1$,对任意$\varepsilon > 0$成立

3、切比雪夫大数定律:$\lim_{n \rightarrow \infty} P\left\{\left|\frac{1}{n} \sum_{i=1}^{n} X_{i}-\frac{1}{n} \sum_{i=1}^{n} E\left(X_{i}\right)\right|<\varepsilon\right\}=1$

4、辛钦大数定律:$\lim_{n \rightarrow \infty} P\left\{\left|\frac{1}{n} \sum_{k=1}^{n} X_{k}-\mu\right|<\varepsilon\right\}=1$(独立同分布的序列)

5、伯努利大数定律:$\lim_{n \rightarrow \infty} P\left\{\left|\frac{X_n}{n}-p\right|<\varepsilon\right\}=1$或者$\lim_{n \rightarrow \infty} P\left\{\left|\frac{X_n}{n}-p\right| \geqslant \varepsilon\right\}=0$,任意$\varepsilon>0$($n$次独立重复试验中事件发生的次数$X_n$)

\section{求解分布函数}\index{求解分布函数}

0、分布函数公式:$F(x) = P\{ X \le x\}=\int_{-\infty}^xf(t)dt,-\infty <x<+\infty$

1、离散型(一维和二维):离散的点代入,分情况求解即可(注意相同值的合并)

2、一般型(二维):$F_{Z}(z)=P\{Z \leqslant z\}=P\{g(X, Y) \leqslant z\}=\iint_{g(x, y) \leqslant z} f(x, y) \mathrm{d} x \mathrm{~d} y$

3、一般型(一维):$F_{Y}(y)=P\{Y \leqslant y\}=P\{g(X) \leqslant y\}=\int_{g(x) \leqslant y} f_{X}(x) \mathrm{d} x, $

3、离散连续组合:$F_{Z}(z)=P\{Z \leqslant z\}=P\{g(X, Y) \leqslant z\} =\sum_{i} P\left\{X=x_{i}\right\} P\left\{g(X, Y) \leqslant z \mid X=x_{i}\right\}=\sum_{i} p_{xi} P\left\{g\left(x_{i}, Y\right) \leqslant z \mid X=x_{i}\right\}$

4、相加的函数(变量不独立):$F_{Z}(z)=P\{Z \leqslant z\}=P\{X+Y \leqslant z\}$=$\int_{-\infty}^{+\infty} \mathrm{d} x \int_{-\infty}^{z-x} f(x, y) \mathrm{d} y$或者$\int_{-\infty}^{+\infty} \mathrm{d} y \int_{-\infty}^{z-y} f(x, y) \mathrm{d} x$

5、相加的函数(变量独立): $f(x, y)=f_{X}(x) f_{Y}(y)$,则$f_{Z}(z)=\int_{-\infty}^{+\infty} f_{X}(x) f_{Y}(z-x) \mathrm{d} x$或者$f_{Z}(z)=\int_{-\infty}^{+\infty} f_{X}(z-y) f_{Y}(y) \mathrm{d} y$,这两个公式称为卷积公式,记为$f_{X} * f_{Y}$

6、独立的正态分布:$X \sim{N}\left(\mu_{1}, \sigma_{1}^{2}\right), Y \sim N\left(\mu_{2}, \sigma_{2}^{2}\right)$,$Z \sim N\left(a\mu_{1}+b\mu_{2}, a^2\sigma_{1}^{2}+b^2\sigma_{2}^{2}\right)$

\section{概率计算}\index{概率计算}



\subsection{加法公式}\index{概率计算!加法公式}

1、$P(A\cup B) = P(A)+P(B)-P(AB)$

2、$P(A\cup B \cup C) = P(A)+P(B)+P(C)-P(AB)-P(AC)-P(BC)+P(ABC)$



\subsection{减法公式}\index{概率计算!减法公式}

1、$P(A - B) = P(A)-P(AB)$

2、$A-B=A\overline B \rightarrow P(A-B) = P(A\overline B)$ (注意:如果AB互相独立,则$P(A-B) = P(A\overline B)=P(A)P(\overline B)$,利用了相互独立的公式)



\subsection{乘法公式}\index{概率计算!乘法公式}

1、$P(A)>0 \rightarrow P(AB)=P(A)P(B|A)$(非独立时乘法运算)

2、$P(A_1A_2\cdots A_n) > 0 \rightarrow P(A_1A_2\cdots A_n)=P(A_1)P(A_2|A_1)\cdots P(A_n|A_1A_2\cdots A_{n-1})$

3、推论:$P(AB) \le P(A)$



\subsection{全概率公式}\index{概率计算!全概率公式}

1、条件:$\mathop{ \bigcup }\limits_{{k=1}}^{{n}}\mathop{{B}}\nolimits_{{k}}=\Omega \ \& \ \mathop{{B}}\nolimits_{{i}}\mathop{{B}}\nolimits_{{j}}= \emptyset { \left( {i \neq j,i,j=1,2,3, \cdots n} \right) } \ \& \ P{ \left( {\mathop{{B}}\nolimits_{{k}}} \right) } > 0{ \left( {k=1,2,3, \cdots n} \right) }$;(称$B_1,B_2,\cdots B_n$为$\Omega$的一个完备事件组)

2、结论:$P{ \left( {A} \right) }=\mathop{ \sum }\limits_{{k=1}}^{{n}}P{ \left( {\mathop{{B}}\nolimits_{{k}}} \right) } \cdot P{ \left( {A \left| \mathop{{B}}\nolimits_{{k}}\right. } \right) }$



\subsection{贝叶斯公式}\index{概率计算!贝叶斯公式}

1、条件:$\mathop{ \bigcup }\limits_{{k=1}}^{{n}}\mathop{{B}}\nolimits_{{k}}=\Omega \ \& \ \mathop{{B}}\nolimits_{{i}}\mathop{{B}}\nolimits_{{j}}= \emptyset { \left( {i \neq j,i,j=1,2,3, \cdots n} \right) } \ \& \ P(A) > 0,P{ \left( {\mathop{{B}}\nolimits_{{k}}} \right) } > 0{ \left( {k=1,2,3, \cdots n} \right) }$

2、结论:$P{ \left( {\mathop{{B}}\nolimits_{{k}} \left| A\right. } \right) }=\frac{{P \left( {\mathop{{B}}\nolimits_{{k}}} \left) \cdot P{ \left( {A \left| {\mathop{{B}}\nolimits_{{k}}}\right. } \right) }\right. \right. }}{{\mathop{ \sum }\limits_{{i=1}}^{{n}}P{ \left( {\mathop{{B}}\nolimits_{{i}}} \right) } \cdot P{ \left( {A \left| \mathop{{B}}\nolimits_{{i}}\right. } \right) }}}=\frac{{P \left( {\mathop{{B}}\nolimits_{{k}}} \left) \cdot P{ \left( {A \left| {\mathop{{B}}\nolimits_{{k}}}\right. } \right) }\right. \right. }}{P(A)}$,换算得到$P(B|A)=P(B)\cdot \frac{P(A|B)}{P(A)}$

3、作用:新信息出现后的B概率=B概率 X 新信息带来的调整(理解调整式子$\frac{P(A|B)}{P(A)}$的计算,依据代数来理解,无法理解,记住公式)

\section{独立判断}\index{独立判断}

1、独立定义:$P\{X \le a, Y \le a\} = P\{X \le a\}P\{Y \le a\}$,$a$、$b$可以取具体值

2、正态分布相关系数为0

\section{一维分布}\index{一维分布}



\subsection{分布}\index{一维分布!分布}

1、0-1分布:1重伯努利试验

2、二项分布:n重伯努利试验,$n$次独立重复试验中成功的次数;$X \sim B(n,p)$

3、几何分布:独立重复试验,每次试验成功率为$p$,则在k次试验时首次试验才成功的概率,$P\{X=k\}=p q^{k-1}, k=1,2, \cdots$

4、超几何分布:N件产品含有M件次品,从中取$n$件(不放回抽取),令事件$X$为抽取的$n$件产品含有的次品个数,则$X$服从参数为$n$,$M$,$N$的超几何分布,$P\{X=k\}=\frac{\mathrm{C}_{M}^{k} \mathrm{C}_{N-M}^{n-k}}{\mathrm{C}_{N}^{n}}, k=l_{1}, \cdots, l_{2}$

5、泊松分布:$P(x) = \frac{\lambda^k}{k!}e^{-\lambda},X \sim P(\lambda)$ :($k$的范围是$0,1,2...$)一段时间内电话总机接到的次数,候车的旅客数,保险索赔次数

6、均匀分布:$X \sim U(a,b)$

7、指数分布:$f(x) = \begin{cases} \lambda e^{-\lambda x}, & x > 0, \\[5ex] 0, & x \le 0, \end{cases} \ \ \lambda >0$;$X \sim E(\lambda)$

8、正态分布:$f(x)=\frac{1}{\sqrt{{2\pi}}\times\sigma}e^{-\frac{1}{2}(\frac{x-\mu}{\sigma})^2}$;$X \sim N(\mu,\sigma^2)$



\subsection{指数分布的无记忆性}\index{一维分布!指数分布的无记忆性}

1、$P\{X>t\}=\int_{t}^{+\infty} \lambda \mathrm{e}^{-\lambda t} \mathrm{~d} t=\mathrm{e}^{-\lambda t}, t>0$

2、$P\{X>t+s \mid X>s\}=\frac{P\{X>t+s\}}{P\{X>s\}}=\frac{\mathrm{e}^{-\lambda(t+s)}}{\mathrm{e}^{-\lambda s}}=\mathrm{e}^{-\lambda t}=P\{X>t\}, t, s>0$(利用定义式推导)

备注:求解指数分布的条件概率时可能用的到



\subsection{正态分布性质}\index{一维分布!正态分布性质}

1、标准化形式:$F(x)=\Phi\left(\frac{x-\mu}{\sigma}\right)$

2、范围求解:$P\{a<X \leqslant b\}=\Phi\left(\frac{b-\mu}{\sigma}\right)-\Phi\left(\frac{a-\mu}{\sigma}\right), a<b$

3、性质:概率密度 $f(x)$ 关于 $x=\mu$ 对称, $\varphi(x)$ 是偶函数(根据图像可以推导)

4、性质:$\Phi(-x)=1-\Phi(x), \Phi(0)=\frac{1}{2}$(根据图像可以推导)

5、性质:当 $X \sim N(0,1)$, 有 $P\{|X| \leqslant a\}=2 \Phi(a)-1$(根据图像可以推导)

6、性质:若$X_{1} \sim N\left(\mu_{1}, \sigma_{1}^{2}\right), X_{2} \sim N\left(\mu_{2}, \sigma_{2}^{2}\right)$,$X_{1}$与$X_{2}$相互独立,则$a X_{1}+b X_{2} \sim N\left(a \mu_{1}+b \mu_{2}, a^{2} \sigma_{1}^{2}+b^{2} \sigma_{2}^{2}\right)$(正态分布的组合性质)



\subsection{二项分布转换成泊松分布:(泊松定理)}\index{一维分布!二项分布转换成泊松分布:(泊松定理)}

1、转换关系:$\lim_{n \rightarrow \infty}np_n=\lambda, n\ge1000,p\le0.1 \rightarrow \lim _{n \rightarrow \infty} C_{n}^{k} p_{n}^{k}\left(1-p_{n}\right)^{n-k}=\frac{\lambda^{k}}{k !} \mathrm{e}^{-\lambda}$

\section{协方差与相关系数}\index{协方差与相关系数}



\subsection{定义}\index{协方差与相关系数!定义}

1、协方差定义:$\operatorname{Cov}(X, Y)=E\{[X-E(X)][Y-E(Y)]\}$

2、相关系数定义:$\rho_{X Y}=\frac{\operatorname{Cov}(X, Y)}{\sqrt{D(X)} \sqrt{D(Y)}}$



\subsection{协方差性质}\index{协方差与相关系数!协方差性质}

1、协方差公式1:$C o v ( X , Y ) = E ( X Y ) - E ( X ) E ( Y ) $(公式展开)

2、协方差公式2:$D ( X \pm Y ) = D ( X ) + D ( Y ) \pm 2 Cov ( X , Y )  $(方差展开公式)

3、协方差性质1:$Cov(X, Y)=Cov(Y, X)$

4、协方差性质2:$\operatorname{Cov}(a X, b Y)=a b \operatorname{Cov}(X, Y)$,其中$a,b$为常数(协方差简化计算)

5、协方差性质3:$\operatorname{Cov}\left(X_{1}\pm X_{2}, Y\right)=\operatorname{Cov}\left(X_{1}, Y\right)\pm \operatorname{Cov}\left(X_{2}, Y\right)$(协方差简化计算)



\subsection{相关系数性质}\index{协方差与相关系数!相关系数性质}

1、相关系数性质1:$\left|\rho_{X Y}\right| \leqslant 1$

2、相关系数性质2:$\mid \rho_{X Y} \mid=1$的充分必要条件是存在常数$a$和$b$,其中$a \neq 0$, 使得$P\{Y=a X+b\}=1$(即相关是两个变量存在线性关系,相反如果完全没有线性关系,那么相关系数就是0)

\section{零碎公式}\index{零碎公式}



\subsection{其它公式(AB独立时)}\index{零碎公式!其它公式(AB独立时)}

1、$P(A-B) = P(A\overline B)=P(A)P(\overline B)$(事件的减法公式与独立时的进一步推广)

2、$P(A\cup B) = 1-P(\overline A \  \overline B) = 1-P(\overline A)P(  \overline B)$(利用事件的图形关系可以解释)



\subsection{值的构造}\index{零碎公式!值的构造}

1、最大值变量表示:$U=\max (X, Y)=\frac{X+Y+|X-Y|}{2}$



2、最小值变量表示:$V=\min (X, Y)=\frac{X+Y-|X-Y|}{2}$



\subsection{其它公式}\index{零碎公式!其它公式}

1、独立公式$P(AB)=P(A)P(B)$

2、$P(A)=1-P(\overline A)$

3、不相容计算:$P\left(A_{1} \cup A_{2} \cup \cdots \cup A_{n}\right)=\sum_{i=1}^{n} P\left(A_{i}\right)$

4、$A \subset B\rightarrow P(A) \leqslant P(B)$

\section{数字特征}\index{数字特征}

1、样本均值:$\bar{X}=\frac{1}{n} \sum_{i=1}^{n} X_{i}$

2、样本方差:$S^{2}=\frac{1}{n-1} \sum_{i=1}^{n}\left(X_{i}-\bar{X}\right)^{2}=\frac{1}{n-1}\left(\sum_{i=1}^{n} X_{i}^{2}-n \bar{X}^{2}\right)$

3、标准差:$S$

4、样本$k$阶(原点) 矩:$A_{k}=\frac{1}{n} \sum_{i=1}^{n} X_{i}^{k}, k=1,2, \cdots$

5、样本k阶中心矩:$B_{k}=\frac{1}{n} \sum_{i=1}^{n}\left(X_{i}-\bar{X}\right)^{k}, k=2,3, \cdots$(中心是根据样本均值来定的)

\section{抽样分布公式}\index{抽样分布公式}

1、卡方:$\chi^{2} = X_1^2+X_2^2+\cdots+X_n^2 \sim \chi^{2}(n)$,$X_1,X_2,\cdots ,X_n$相互独立且服从标准正态分布

2、T分布:$t=\frac{X}{\sqrt{Y/n}} \sim t(n)$,其中$X \sim N(0,1)$,$Y\sim \chi^{2}\left(n\right)$,且$X,Y$相互独立

3、F分布:$F=\frac{U / n_{1}}{V / n_{2}} \sim F\left(n_{1}, n_{2}\right)$,其中$U \sim \chi^{2}\left(n_{1}\right)$,$ V \sim \chi^{2}\left(n_{2}\right)$,且$U,V$相互独立(delta1037)

\section{边缘相关}\index{边缘相关}

1、边缘概率密度:$f_{X}(x)=\int_{-\infty}^{+\infty} f(x, y) \mathrm{d} y$);关于$Y$的边缘概率密度类似

2、边缘分布函数:$F_{X}(x)=F(x,+\infty)=\int_{-\infty}^{x}\left[\int_{-\infty}^{+\infty} f(x, y) \mathrm{d} y\right] \mathrm{d} x$);关于$Y$的边缘分布函数类似

\section{期望}\index{期望}



\subsection{离散}\index{期望!离散}

1、一维:$\sum_{k=1}^{\infty}x_kp_k$

2、一维函数:$E(Y)=E[g(X)]=\sum_{k=1}^{+\infty} g\left(x_{k}\right) p_{k}$

3、二维函数:$E(Z)=E[g(X, Y)]=\sum_{i=1}^{+\infty} \sum_{j=1}^{+\infty} g\left(x_{i}, y_{j}\right) p_{i j}$



\subsection{连续}\index{期望!连续}

1、一维:$E(X)=\int_{-\infty}^{+\infty} x f(x) \mathrm{d} x$

2、一维函数:$E(Y)=E[g(X)]=\int_{-\infty}^{+\infty} g(x) f(x) \mathrm{d} x$

3、二维函数:$E(Z)=E[g(X, Y)]=\int_{-\infty}^{+\infty} \int_{-\infty}^{+\infty} g(x, y) f(x, y) \mathrm{d} x \mathrm{~d} y$



\subsection{矩的定义}\index{期望!矩的定义}

1、$k$阶原点矩:$E\left(X^{k}\right), \quad k=1,2, \cdots $

2、$k$阶中心距:$E\left\{[X-E(X)]^{k}\right\}, \quad k=1,2, \cdots$

3、$k+l$阶混合原点矩:$E\left(X^{k} Y^{l}\right), \quad k, l=1,2, \cdots$

4、$k+l$阶混合中心矩:$E\left\{[X-E(X)]^{k}[Y-E(Y)]^{\iota}\right\}, \quad k, l=1,2, \cdots$



\subsection{性质}\index{期望!性质}

1、性质1:$E(a_1X_1+a_2X_2+\cdots+a_nX_n+b_1+b_2+\cdots+b_m)=a_1E(X_1)+a_2E(X_2)+\cdots+a_nE(X_n)+b_1+b_2+\cdots+b_m$(无论变量之间是否独立都成立!这点很重要,可以在有些变量相关的条件下简化计算)

2、若$X$、$Y$不相关,则$E ( X Y ) = E ( X ) E ( Y )$(推导:$C o v ( X , Y ) = E ( X Y ) - E ( X ) E ( Y ) =0$)

\section{条件概率}\index{条件概率}

1、连续:$P\{X>x|Y>y\} = \frac {P\{X>x,Y>y\}}{P\{Y>y\}}$

2、离散:$P\{X=x|Y=y\} = \frac {P\{X=x,Y=y\}}{P\{Y=y\}} = \frac {p_{ij}}{p_{\cdot j}}$

3、离散条件密度:$P\{X=x|Y=y\} = \frac {P\{X=x,Y=y\}}{P\{Y=y\}}$

4、连续条件分布定义:$F_{X \mid Y}(x \mid y)=\int_{-\infty}^{x} \frac{f(s, y)}{f_{Y}(y)} \mathrm{d} s$

5、连续条件密度定义:$f_{X \mid Y}(x \mid y)=\frac{f(x, y)}{f_{Y}(y)}, f_{Y}(y)>0$

\section{方差}\index{方差}



\subsection{定义}\index{方差!定义}

1、方差定义:$D(X)=E\left\{[X-E(X)]^{2}\right\}$

1、标准差/均方差表示:$\sigma(X)=\sqrt{D(X)}$

2、推论1:$D(X)=E\left(X^{2}\right)-[E(X)]^{2}$(定义公式展开)(方差的一般求解方法)

3、推论2:$E\left(X^{2}\right) \geqslant[E(X)]^{2}$(因为对任何随机变量$X, D(X) \geqslant 0$)



\subsection{性质}\index{方差!性质}

1、性质1(可推广):$D(a X+b)=a^{2} D(X) $

2、性质2(前提条件:$X,Y$相互独立/不相关):$D(X \pm Y)=D(X)+D(Y)$

3、性质2(前提条件:$X,Y$关系未知):$D ( X \pm Y ) = D ( X ) + D ( Y ) \pm 2 Cov ( X , Y )  $(方差展开公式)(其中协方差可以用相关系数与协方差的关系替换)

\section{估计公式}\index{估计公式}



\subsection{单样本构造}\index{估计公式!单样本构造}

0、前提:设总体$X \sim N\left(\mu, \sigma^{2}\right), X_{1}, X_{2},\cdots, X_{n}$是来自$X$的样本,样本均值$\bar{X}=\frac{1}{n} \sum_{i=1}^{n} X_{i}$, 样本方差$S^{2}=\frac{1}{n-1} \sum_{i=1}^{n}\left(X_{i}-\bar{X}\right)^{2}$

1、(单)样本均值分布:$\bar{X} \sim N\left(\mu, \frac{\sigma^{2}}{n}\right)$,标准化$U=\frac{\bar{X}-\mu}{\sigma / \sqrt{n}} \sim N(0,1)$

2、(单)样本方差构造卡方分布:$\chi^{2}=\frac{(n-1) S^{2}}{\sigma^{2}} \sim \chi^{2}(n-1)$

3、(单)均值和样本方差构造$T$分布:$T=\frac{\bar{X}-\mu}{S / \sqrt{n}} \sim t(n-1)$

4、(单)中心距分布:$\chi^{2}=\frac{1}{\sigma^{2}} \sum_{i=1}^{n}\left(X_{i}-\mu\right)^{2} \sim \chi^{2}(n)$



\subsection{双样本构造}\index{估计公式!双样本构造}

0、前提:$X_{1}, X_{2}, \cdots, X_{n_{1}}$与$Y_{1}, Y_{2}, \cdots,Y_{n_{2}}$分别是来自正态总体$N\left(\mu_{1}, \sigma_{1}^{2}\right)$和$N\left(\mu_{2}, \sigma_{2}^{2}\right)$的样本,且这两个样本相互独立. 设$\bar{X}=\frac{1}{n_{1}} \sum_{i=1}^{n_{1}} X_{i}$,$\bar{Y}=\frac{1}{n_{2}} \sum_{i=1}^{n_{2}} Y_{i}$分别是这两个样本的样本均值; $S_{1}^{2}=\frac{1}{n_{1}-1} \sum_{i=1}^{n_{1}}\left(X_{i}-\bar{X}\right)^{2}$, $S_{2}^{2}=\frac{1}{n_{2}-1} \sum_{i=1}^{n_{2}}\left(Y_{i}-\bar{Y}\right)^{2}$分别是这两个样本的样本方差

1、(双)均值差构造正态分布:$\bar{X}-\bar{Y} \sim N\left(\mu_{1}-\mu_{2}, \frac{\sigma_{1}^{2}}{n_{1}}+\frac{\sigma_{2}^{2}}{n_{2}}\right)$

2、(双)均值和样本方差构造$T$分布:$\frac{(\bar{X}-\bar{Y})-\left(\mu_{1}-\mu_{2}\right)}{S_{w} \sqrt{\frac{1}{n_{1}}+\frac{1}{n_{2}}}} \sim t\left(n_{1}+n_{2}-2\right)$,$S_{w}^{2}=\frac{\left(n_{1}-1\right) S_{1}^{2}+\left(n_{2}-1\right) S_{2}^{2}}{n_{1}+n_{2}-2}, \quad S_{w}=\sqrt{S_{w}^{2}}$

3、(双)样本方差构造$F$分布:$\frac{S_{1}^{2} / S_{2}^{2}}{\sigma_{1}^{2} / \sigma_{2}^{2}} \sim F\left(n_{1}-1, n_{2}-1\right)$

\section{中心极限定理}\index{中心极限定理}

(变量自解释,不赘述)(注意公式需要变量独立性,一般用这种公式题目自然会给定)

1、棣莫弗一拉普拉斯定理:$\lim _{n \rightarrow \infty} P\left\{\frac{X_{n}-n p}{\sqrt{n p(1-p)}} \leqslant x\right\}=\int_{-\infty}^{x} \frac{1}{\sqrt{2 \pi}} \mathrm{e}^{-\frac{t^{2}}{2}} \mathrm{~d} t=\Phi(x)$(二项分布)

2、列维一林德伯格定理:$\lim_{n \rightarrow \infty} F_{n}(x)=\lim_{n \rightarrow \infty} P\left\{\frac{\sum_{k=1}^{n} X_{k}-n \mu}{\sqrt{n} \sigma} \leqslant x\right\}=\int_{-\infty}^{x} \frac{1}{\sqrt{2 \pi}} \mathrm{e}^{-\frac{t^{2}}{2}} \mathrm{~d} t=\Phi(x)$(独立同分布)

\section{二维分布}\index{二维分布}



\subsection{二维均匀分布}\index{二维分布!二维均匀分布}

1、公式:$f(x, y)=\left\{\begin{array}{lc}

\frac{1}{A}, & (x, y) \in G, \\

0, & \text { 其他 },

\end{array}\right.$其中A是平面有界区域$G$的面积,则称$(X,Y)$服从区域$G$上的均匀分布



\subsection{二维正态分布}\index{二维分布!二维正态分布}

公式:$f(x, y)=\frac{1}{2 \pi \sigma_{1} \sigma_{2} \sqrt{1-\rho^{2}}} \exp \left\{-\frac{1}{2\left(1-\rho^{2}\right)}\left[\frac{\left(x-\mu_{1}\right)^{2}}{\sigma_{1}^{2}}-\frac{2 \rho\left(x-\mu_{1}\right)\left(y-\mu_{2}\right)}{\sigma_{1} \sigma_{2}}+\frac{\left(y-\mu_{2}\right)^{2}}{\sigma_{2}^{2}}\right]\right\},-\infty<x<+\infty,-\infty<y<+\infty$,其中$\mu_{1}, \mu_{2}, \sigma_{1}>0, \sigma_{2}>0,-1<\rho<1$ 均为常数, 则称 $(X, Y) $服从参数为$\mu_{1}, \mu_{2}, \sigma_{1}, \sigma_{2}$和$\rho$的二维正态分布,记作$(X, Y) \sim N\left(\mu_{1}, \mu_{2} ; \sigma_{1}^{2}, \sigma_{2}^{2} ; \rho\right)$



\subsection{二维正态分布性质}\index{二维分布!二维正态分布性质}

1、$\left(X, Y\right) \sim N\left(\mu_{1}, \mu_{2} ; \sigma_{1}^{2}, \sigma_{2}^{2} ; \rho\right)$时, $X$与$Y$均服从一维正态:$X \sim{N}\left(\mu_{1}, \sigma_{1}^{2}\right), Y \sim N\left(\mu_{2}, \sigma_{2}^{2}\right)$

2、$(X, Y) \sim N\left(\mu_{1}, \mu_{2} ; \sigma_{1}^{2}, \sigma_{2}^{2} ; \rho\right)$时, $X$与$Y$相互独立的充分必要条件是$\rho=0$(正态的独立与不相关是一致的)

3、$(X, Y)$服从二维正态时,行列式$\left|\begin{array}{ll}a & b \\ c & d\end{array}\right| \neq 0$时$(a X+b Y, c X+d Y)$也服从二维正态(二维正态的满足特定条件(即两个新变量没有常数倍关系,如果有会发生什么)的组合还是二维正态); $ a X+b Y$服从一维正态$\left(a^{2}+b^{2} \neq 0\right)$(这是由性质1得出的,需要相互独立条件)

4、约定: 当$X$与$Y$均服从一维正态,且相互独立,就是指$(X, Y)$服从二维正态

5、如果$X$与$Y$均服从一维正态, 不能保证$(X, Y)$服从二维正态, 也不能保证$a X+b Y$服从一维正态(两个一维正态推不出来组合是一维的,也不能保证能组合出来二维的(需要相互独立的条件))

