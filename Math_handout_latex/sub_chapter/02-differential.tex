\chapterimage{chapter_head_2.pdf}
\chapter{一元微分}

\section{特殊函数形式的求导}\index{特殊函数形式的求导}

1、参数式函数的求导:对于一阶导,$ x $和$ y $分别对$ t $求导,然后相除;对于二阶导,将一阶导对$ t $求导,除以$ x $对$ t $求导(理解推导过程)

2、极坐标函数的求导:将极坐标形式转换成$ **\theta** $的参数式,采用参数式函数的求导方式(参数方程为$ <font color=purple>r=r(\theta)</font> $;转换成$ <font color=purple>\theta</font> $为参数的参数方程是$ <font color=purple>\left\{\begin{array}{l}x=r(\theta) \cos \theta \\y=r(\theta) \sin \theta\end{array},(\theta \text { 为参数 })\right.</font> $)

3、隐函数的求导:对于一阶导,将隐函数中$ y $视为$ x $的函数,将函数的两边同时对$ x $求导,得到$ dx/dy $的式子,解出来即可(也可以用隐函数求导法则,不要忘记负号即可);对于二阶导,将求得的一阶导继续对x求导,并将求解出来的一阶导代入,得到不含导数的二阶导

4、反函数的求导:一阶导将微分形式颠倒即是$ x $对$ y $的导数;二阶导对一阶导继续求对$ y $的导数,转换成求对$ x $的导数,并与$ x $对$ y $的导数相乘(链式法则,当求导变量与函数的自变量不对等时,使用链式法则将不对等变量转移,例如将x的函数对y求导,先将x的函数对x求导,然后与x对y求导相乘)(可以由已知函数的导数求解反函数的导数,能求解的反函数的导数最高阶数与已知函数导数的最高阶数一致)

5、幂函数的求导:转换成指数形式求导

6、绝对值的导数:

6.1、去绝对值分类讨论:讨论绝对值内的正负性,分类讨论,去掉绝对值;或者将整体收缩到绝对值内,如下的函数在一点为0的特殊情况

6.2、绝对值的处理情况:当函数在一点为$ f(a)=0 $时,有$ |f(x)|-|f(a)|=|f(x)|=|f(x)-f(a)| $,将导数定义全部转入到绝对值内,然后分成左右趋近两种情况讨论

\section{求渐近线}\index{求渐近线}



\subsection{XY坐标系式子的求解:(备注:一共需要验证6种情况,但是在无穷的位置水平和斜渐近线只能存在一个(同一边的无穷))}\index{求渐近线!XY坐标系式子的求解:(备注:一共需要验证6种情况,但是在无穷的位置水平和斜渐近线只能存在一个(同一边的无穷))}

1、水平渐近线:正无穷和负无穷的极限值,$ \lim _{x \rightarrow \infty} f(x)=b $(注意正无穷和负无穷如果极限值一样只能算一条)(注意:包含正无穷和负无穷)

2、铅直渐近线:趋近于某个值时,极限不存在(表现为存在$ x_{0} $,  使$ \lim_{x \rightarrow x{0}^{-}} f(x)=\infty $ (或$ \lim_{x \rightarrow x{0}^{+}} f(x)=\infty $);先由观察法得知特殊点(分母为0,对数的真数为0处)(注意:包含左趋近和右趋近)

3、斜渐近线:($ \lim _{x \rightarrow+\infty} \frac{f(x)}{x}=a $)趋近于无穷的极限为斜率a,($ \lim _{x \rightarrow+\infty} ({f(x)}-{ax})=b $)趋近于无穷的极限即为截距b(获得y=ax+b,斜渐近线方程)(注意:包含正无穷和负无穷)(当只有斜率存在并且截距存在时才有斜渐近线,如果仅有斜率存在但是截距为无穷或者振荡形式,则渐近线是不存在的)



\subsection{参数方程的求解:(找$ x $无穷的点和$ y $无穷的点)}\index{求渐近线!参数方程的求解:(找$ x $无穷的点和$ y $无穷的点)}

1、水平渐近线:找一个使得$ x $趋近于无穷时的$ t_0 $点(特定值或者无穷),求解$ t $趋近于$ t_0 $时$ y $的值

2、铅直渐近线:找使$ y $趋近于无穷并且$ x $是有限值的点$ t_0 $

3、斜渐近线:找一个使得$ x $趋近于无穷时的$ t_0 $点(特定值或者无穷),求解$ \lim _{t \rightarrow t_0} \frac{y(t)}{x(t)}=a $,$ \lim _{t \rightarrow t_0} ({y(t)}-{ax(t)})=b $,(获得$ <font color=purple>y=ax+b</font> $,斜渐近线方程)(当只有斜率存在并且截距存在时才有斜渐近线,如果仅有斜率存在但是截距为无穷,则渐近线是不存在的)

\section{导数与微分}\index{导数与微分}



\subsection{导数的定义}\index{导数与微分!导数的定义}

1、导数定义式1:$ \lim _{x \rightarrow x_{0}} \frac{f(x)-f\left(x_{0}\right)}{x-x_{0}}=f^{\prime}\left(x_{0}\right) $

2、导数定义式2:$ \lim _{\Delta x \rightarrow 0} \frac{f\left(x_{0}+\Delta x\right)-f\left(x_{0}\right)}{\Delta x}=f^{\prime}\left(x_{0}\right)=\left.\frac{\mathrm{d} f(x)}{\mathrm{d} x}\right|_{x=x_{0}} $

3、高阶导数定义:高阶导数的定义式与导数的定义式类似

4、在点处可导:左右导数均存在且相等

注意:定义中的$ <font color=purple>\Delta x</font> $是可正可负的,是从两边趋近的,如果区分了正负就是左右导数(如果需要对$ <font color=purple>\Delta x</font> $进行函数形式替换,需要判断函数形式的正负性,判断替换之后得出的是左右导数中的一个,还是点处的导数)

注意:在一点导数存在是指左右导数均存在且相等



\subsection{不可导}\index{导数与微分!不可导}

1、判断不可导点:绝对值函数与非绝对值函数下相乘,求解出所有的零点,判断哪些是只有在绝对值函数内的(只有绝对值函数内的零点才是不可导点,其余的会和绝对值函数外的零点合并,变成可导点)



\subsection{微分和增量的关系}\index{导数与微分!微分和增量的关系}

1、增量与微分相差一个变量的一阶无穷小,可以用拉格朗日余项泰勒公式关联起来$ \Delta y-\mathrm{d} y=\frac{1}{2} f^{\prime \prime}(\xi)(\Delta x)^{2} $(大于零或者小于0取决于二阶导的正负性,可以使用拉格朗日余项泰勒公式证明)

\section{中值、不等式、零点问题的基本定理}\index{中值、不等式、零点问题的基本定理}

1、费马定理:可导条件下的极值的必要条件,即可导时,极值位置的导数为0

2、罗尔定理:函数$ f(x) $在闭区间$ [a,b] $上连续,在开区间$ (a,b) $内可导,并且两端值相等$ f(a)=f(b) $,则至少存在一点$ \xi \in(a, b) $,使得$ f^{\prime}(\xi)=0 $(相同高度的点中间存在水平切线)

3、拉格朗日中值定理:函数$ f(x) $在闭区间$ [a,b] $上连续,在开区间$ (a,b) $内可导,则至少存在一点$ \xi \in(a, b) $,使得$ f(a)-f(b)=f^{\prime}(\xi)(a-b) $(另一种常用形式:在定理条件下$ x_0 $和$ x $是区间$ [a,b] $上的任意两点,至少存在一点$ \xi $在$ x_0 $和$ x $之间,使得$ f(x)=f\left(x_{0}\right)+f^{\prime}(\xi)\left(x-x_{0}\right) $成立,令$ \theta = \frac{\xi - x_0}{x-x_0} $,其中$ 0< \theta < 1 $,则拉定又可以写成$ f(x)=f\left(x_{0}\right)+f^{\prime}\left(x_{0}+\theta\left(x-x_{0}\right)\right)\left(x-x_{0}\right) $,其中$ <font color=purple>f^{\prime}\left(x_{0}+\theta\left(x-x_{0}\right)\right)</font> $是$ <font color=purple>f^{\prime}(\xi)</font> $的另一种表达)

4、柯西中值定理:函数$ f(x),g(x) $在闭区间$ [a,b] $上连续,在开区间$ (a,b) $内可导,且$ g(x)\ne 0,x \in(a, b) $,,则至少存在一点$ \xi \in(a, b) $,使得$ \frac{f(b)-f(a)}{g(b)-g(a)}=\frac{f^{\prime}(\xi)}{g^{\prime}(\xi)} $

5、泰勒定理:函数$ f(x) $在闭区间$ [a,b] $上有$ **n** $阶连续导数,在开区间$ (a,b) $内具有$ n+1 $阶导数($ <font color=purple>**n+1**</font> $阶是为了表示$ <font color=purple>**R_{n}(x)**</font> $),$ x_0 $和$ x $是区间$ [a,b] $上的任意两点,至少存在一点$ \xi $在$ x_0 $和$ x $之间,使得$ f(x)=f\left(x_{0}\right)+\frac{f^{\prime}\left(x_{0}\right)}{1 !}\left(x-x_{0}\right)+\frac{f^{\prime \prime}\left(x_{0}\right)}{2 !}\left(x-x_{0}\right)^{2}+\cdots+\frac{f^{(n)}\left(x_{0}\right)}{n !}\left(x-x_{0}\right)^{n}+R_{n}(x) $,其中$ R_{n}(x)=\frac{f^{(n+1)}(\xi)}{(n+1) !}\left(x-x_{0}\right)^{n+1} $(如果将条件减弱为函数$ <font color=purple>f(x)</font> $在$ <font color=purple>x=x_0</font> $处(只有一点)有$ <font color=purple>**n**</font> $阶导数(这意味着在$ <font color=purple>x=x_0</font> $的邻域内具有连续的$ <font color=purple>n-1</font> $阶导数,即$ <font color=purple>f^{(n-1)}(x)</font> $在$ <font color=purple>x=x_0</font> $处连续,$ <font color=purple>x</font> $为$ <font color=purple>x_0</font> $邻域内充分小的邻域内的任意一点,有$ <font color=purple>f(x)=f\left(x_{0}\right)+\frac{f^{\prime}\left(x_{0}\right)}{1 !}\left(x-x_{0}\right)+\cdots+\frac{f^{(n)}\left(x_{0}\right)}{n !}\left(x-x_{0}\right)^{n}+R_{n}(x)</font> $,其中$ <font color=purple>R_{n}(x)=o\left(\left(x-x_{0}\right)^{n}\right)\left(\lim_ {x \rightarrow x{0}} \frac{o\left(\left(x-x_{0}\right)^{n}\right)}{\left(x-x_{0}\right)^{n}}=0\right)</font> $,这就是佩亚诺余项泰勒公式)

6、积分中值定理:

注意:如果泰勒公式中展开点为0点,则称该公式为麦克劳林公式

注意:拉格朗日中值定理时罗尔定理证明的,不能用拉定去证明罗定

注意:具有拉格朗日余项的0阶泰勒公式就是拉格朗日中值公式

注意:具有拉格朗日余项的1阶泰勒公式,就是函数的微分与增量之间的关系公式

注意:泰勒定理在不同的位置展开,$ <font color=purple>**\xi**</font> $要用不同的值$ <font color=purple>**\xi_n**</font> $表示

\section{函数特征判断}\index{函数特征判断}



\subsection{驻点}\index{函数特征判断!驻点}

1、驻点为一阶导为0的点(极值点不一定是驻点,因为极值点可能不存在导数)

备注:驻点是一个横向坐标($ <font color=orange>x=x_0</font> $)

注意:驻点一定是平滑的;驻点处可能没有方向的变换(即单调)



\subsection{极值点}\index{函数特征判断!极值点}

0、极值点定义:邻域内存在一点的值大于等于或者小于等于其它的值(邻域内需要有定义,无穷值不算有定义;邻域是双边的,如果只有一边有定义,则也不叫有定义)

1、第一充分条件:函数在点处连续,点处去心邻域内可导(点处可能不可导:点处为尖锐的点),两侧导数数值反号(这种类型包括尖锐点)(备注:若函数连续,且左右导数存在,$ <font color=orange>f(x_0)</font> $为极值,未必存在左右导数反号,例如常函数的极值)

2、第二充分条件:函数在点处去心邻域存在二阶导,点处一阶导为0,二阶导不为0(注意二阶导可能只有一点,点处二阶导存在可以推知一阶导点邻域内连续,函数在该点邻域内也连续)(二阶导为0,不一定不是极值点,比如x的三次方)

3、可导点极值必要条件:如果函数在一点处取得极值,并且该点导数存在,则导数一定为0,反之不成立

备注:极值点是一个横向坐标($ <font color=orange>x=x_0</font> $);极值是纵向坐标($ <font color=orange>y=f(x_0)</font> $)

注意:极值点也可能是尖锐的(连续但是不可导的点)

注意:极值点是去心邻域内最大的点,所以即使函数不连续也可以,只要是最大的就行了;单调函数在闭区间内是没有极值点的,因为不存在在去心邻域内的最大点



\subsection{最值}\index{函数特征判断!最值}

最值定义:区间上,存在一点的值大于等于或者小于等于其它区间上所有的值(区间上需要有定义,无穷值不算有定义)

1、找到所有的驻点和不可导的点(尖锐的点和端点)

2、比较所有的点的值

备注:最值是一个纵向坐标($ <font color=orange>f(x_0)</font> $)

注意:若区间内只有一个极值点,该极值点必是最大或最小值

注意:讨论应用问题最值时注意判定函数在区间内是否可导(注意应用的实际情况)



\subsection{拐点}\index{函数特征判断!拐点}

1、二阶导判定:去心邻域二阶可导(注意:点处可能不可导会有出现尖锐点的情况),且点处左右二阶导数反号→是拐点(点的两边不连续也行,只要凹凸性相反就行了)

2、三阶导判定:二阶导为0,三阶导不为0的点也是拐点(三阶导不为0是为了确定二阶导在该点邻域是反号)

3、邻近域判断:如果某个点的临近域的凹凸性相反,则该点是拐点

4、更高阶导判断:如果函数某一点处n阶导不为零,n阶一下都是零,则当n为奇数时,是拐点,偶数时不是拐点

备注:拐点是一个二维坐标($ <font color=orange>(x_0,y_0)</font> $)

注意:拐点处二阶导为0或者不存在

注意:一点是极值点就不是拐点,是拐点就不是极值点



\subsection{凹凸性}\index{函数特征判断!凹凸性}

1、二阶导判断:区间上二阶导大于等于0(小于等于0)(注意:不在某一个区间上取等)→凹的(凸的);可以理解为二阶导大于0,导数增加越来越快导致图像下凹,或者二阶导小于0,导数增加越来越慢导致图形上凸

2、三阶导判断:画图转成二阶导判定(需要有一些辅助定量的点才能画图)



\subsection{曲率、曲率圆、曲率半径}\index{函数特征判断!曲率、曲率圆、曲率半径}

1、曲率计算公式(注意:曲率与二阶导符号无关):$ k=\frac{\left|y^{\prime \prime}\right|}{\left(1+y^{\prime 2}\right)^{3 / 2}} $

2、曲率半径:$ R=\frac{1}{k}=\frac{\left(1+y^{\prime 2}\right)^{3 / 2}}{\left|y^{\prime \prime}\right|} $

3、曲率中心坐标:$ (\alpha=x-\frac{y^{\prime}\left(1+y^{\prime 2}\right)}{y^{\prime \prime}}, \beta=y+\frac{1+y^{\prime 2}}{y^{\prime \prime}}) $(曲率中心由点处的法线方程和曲率半径求得,相应的可以得到曲率圆方程)

注意:如果题目给出了曲率圆方程,可以通过求解曲率圆方程的导数获得曲线的导数(一阶和二阶导都是相等的)(可以避免讨论)

注意:如果使用曲线求导,会因为曲率计算公式中有绝对值的存在而导致参数需要讨论(讨论绝对值)



\section{零点问题的证明(等式的证明)}\index{零点问题的证明(等式的证明)}



\subsection{至多有几个零点的证明}\index{零点问题的证明(等式的证明)!至多有几个零点的证明}

当函数的各阶导数存在时(原函数的零点存在性)

1、如果$ f'(x) $没有零点,则$ f(x) $至多有$ 1 $个零点

2、如果$ f'(x) $至多有1个零点,则$ f(x) $至多有$ 2 $个零点

3、如果$ f'(x) $至多有k个零点,则$ f(x) $至多有$ k+1 $个零点

4、如果$ f''(x) $没有零点,则f'(x)至多有$ 1 $个零点,$ f(x) $至多有$ 2 $个零点,$ … $,依此类推

注意:该规则的方向与罗尔定理相反,规定了导数的原函数至多有多少零点(导函数正负切换与原函数跨y轴次数的关系。导函数正负切换不一定会导致原函数跨y轴(出现零点),但是如果导函数正负切换全导致了原函数跨$ <font color=purple>**y**</font> $轴变化,那么这就是原函数零点最多的情况)



\subsection{零点存在性证明}\index{零点问题的证明(等式的证明)!零点存在性证明}

1、连续函数的介值定理:连续函数在区间上有最大值M和最小值m,则在区间上必存在一点使得函数值在最大值和最小值之间(闭区间)(注意:如果题目未设定函数连续,则不能用连续函数的介值定理来说明有零点或者存在某一点)

2、连续函数的零点定理:如果函数的两端的值相乘小于0,则在开区间上至少存在一点,使得函数值为0

3、利用罗尔定理(导函数的零点存在性):以下设所提到的导数存在,则有结论:如果$ f(x) $有$ k(k≥2) $个零点,则$ f'(x) $至少有$ (k-1) $个零点;$ … $,$ f^{(k-1)}(x) $至少有$ 1 $个零点(求导之后,会减少零点的个数)(该规则规定了求导之后至少应该有多少个零点)(跨y轴的零点与增减性(导函数正负切换的次数)的关系,原函数零点的出现,使得两个零点之间必有一次导函数的正负切换(即导函数的零点),由于两个零点之间的导函数切换会出现大于一次的情况,所以这就是导函数零点最少的情况)

4、两种不同函数形式的存在性证明等式:想到用柯西中值定理证明;需要找到其中一个函数的导数形式

备注:罗尔定理也可以看作是导函数的零点定理,构造积分式,积分两端的值相等,即存在导函数为0,即被积分函数存在为0的点,即存在零点



\subsection{零点问题证明细节}\index{零点问题的证明(等式的证明)!零点问题证明细节}

1、当需要证明积分式中被积分函数的零点:用罗尔定理,证明积分两个端点值相同,则存在导数即被积分函数存在零点

2、有极大值大于0的曲线,判断与$ x $轴的交点,需要判断两边延伸部分的正负性;极小值同理

3、零点的个数问题,可以用至多存在几个零点和至少存在几个零点进行框定(夹逼)得到

4、利用奇偶性:利用奇偶性,只讨论定义域的一边!!!简化计算

5、证明连续函数内有最小值点(也就是极值点):如果函数在一点处取得极值,并且该点导数存在,则导数一定为0(可导点极值必要条件)

6、两个函数最大值相等,中间必有一个零点,构造相减的式子,用零点定理证明

7、如果讨论整个范围上的零点,先根据函数的大概的正负性,判定零点的位置,然后根据导数,讨论别的区间的增减性,将零点的范围缩小

注意:如果题目未设定函数连续,则不能用连续函数的介值定理来说明有零点或者存在某一点



\section{导数的注意点}\index{导数的注意点}



\subsection{概念区分问题}\index{导数的注意点!概念区分问题}

1、区分点可导与邻域可导:高一阶的导数在某个点存在,说明低一阶的导数在该点邻域连续;当没有说函数的(去心)邻域可导时,不能对函数求导数,也就是不能对该函数使用洛必达法则,求解极限只能凑成导数的定义形式

2、点可导可推出邻域连续:已知某一点可导可以获得的隐含条件,包括函数在该点的邻域内连续(左右极限相等并等于该点的值),函数在该点的左右导数相等

3、点处连续不一定可导:尖锐的点,左右导数不相等

4、点处可导但是在邻域不一定可导:在点处可导但是邻域的导数在该点的极限不一定存在,因为邻域的导数可能会出现振荡的成分,导致导数不存在(即点处可导,邻域导数在点处的极限不一定等于点处的导数(振荡起来了))(一个点处的导数不可以推导出邻域的单调性,但是可以根据极限的保号性(导数的定义与极限的定义)推出邻域的值与导数点值的大小关系)

5、两个函数和的导数存在推不出两个函数的导数分别都存在:函数相加会损失掉一部分信息(损失掉无穷或者振荡的部分)

6、极值点:是去心邻域内最大的点,所以即使函数不连续也可以,只要是最大的就行了

7、一元函数点处可导可以推知点处可微,且微分和导数相关联:$ dy=f^{\prime}(x)dx $

8、大题规范:描述参数方程的单调性和拐点时,$ t $的范围和对应$ x $的范围一起描述

9、反函数存在的条件:函数严格单调

10、单调函数在闭区间内是没有极值点的,因为不存在在去心邻域内的最大点



\subsection{求导容易出现的问题}\index{导数的注意点!求导容易出现的问题}

1、求导未求尽:对于$ arctan(f(x)) $之类的函数求导忘记对$ f(x) $求导;对于$ (1+f(x))^{a} $之类的函数求导忘记对$ f(x) $求导;对于$ \sqrt{(1+f(x))} $之类的函数求导忘记对$ f(x) $求导(与幂形式是一样的,可以转换)(不要忘记指数部分带来的阶乘)

2、幂指函数求导:转换成指数函数求导,不要忘记指数项,求导要求完整

\section{不等式的证明方法细节}\index{不等式的证明方法细节}

1、将不等式中的一个量转换成变量:转换成一个函数,转为用函数单调性的方法求解(当某一个端点是开区间时,求解该端点的极限值,构造新的补充端点值得函数讨论函数在区间上得正负性)(注意将1转换成变量时,可能需要有隐藏的变量需要转换)

2、对称性的应用:证明含有$ 1/u $在$ (0,\infty) $区间上的单调性,先证明$ (0,1) $,再利用对称性证明$ (1,\infty) $,减小计算量(用变量替换。替换后形式与原式子形式一致)

3、有高阶导:用拉格朗日余项泰勒公式推导函数的值,再所证的范围上的其中一个端点展开(该点的高阶导数要尽可能地多;尽可能让不需要的项为0;容易约去的点,用于多个函数泰勒组合)

4、拉格朗日中值公式与单调性的证法是相通的:单调性可以将拉格朗日中值公式化简成$ f(b)-Ab>f(a)-Aa $,构造函数$ f(x)+Ax $,求导即是证明$ f^{\prime}(x)-A>0 $类似的问题,与拉格朗日中值公式的证明类似

5、两端值未定的规范化:假设$ f(a),f(b) $的关系未定,构造函数$ \varphi(x)=f(x)-\left[f(a)+\frac{f(b)-f(a)}{b-a}(x-a)\right] $,这样就有$ \varphi(a)=\varphi(b) $

6、式子预处理:如果是复杂的乘式,利用求对数,将乘积形式变为和差的形式

7、罗尔定理是证明导数零点的(等式),如果题目中给出一个区间两个端点值为0的情况,用拉定在$ x $(区间的中间值)和这两个值联立得到两个式子,

8、数值不等式也可以通过一些定理的形式转换,例如如果具有拉格朗日中值定理的形式,就将原数值不等式转换成了函数导数不等式的形式

注意:当不等式两边同时乘一个负值时,不等式要变号

\section{不等式证明}\index{不等式证明}



\subsection{证明不等式方法}\index{不等式证明!证明不等式方法}

1、单调性的方法:首先将函数相减得到新函数(构造),如果该函数在区间内是单调的,判断两个端点的值是大于零还是小于0的,从而得到该函数的值在区间内的正负性;如果该函数在区间内分成两个区间,分别单调递增和单调递减,考虑两个端点的值和中间分界点的值,从而得到该函数在区间内的正负性(最值的方法:如果构造的函数的最小值大于等于0,则函数整体大于0;如果构造的函数的最大值小于0,则函数整体小于0)

2、拉格朗日中值公式的方法(一阶导用一次,二阶导需要拆分开用两次拉定,二阶导关键是找到一个中间值):(题目特征:$ <font color=purple>f(b)-f(a)>A(b-a)(or \ f(b)-f(a)<A(b-a))</font> $),由$ f(a)-f(b)=f^{\prime}(\xi)(a-b) $,只要证明$ f^{\prime}(\xi)>A\left(\text { 或 } f^{\prime}(\xi)<A\right), \text { 当 } \xi \in(a, b) $(注意其中A是一个定值,不包括参数a、b!)(拉格朗日中值定理可以认为是对函数利用导数值进行区间(利用区间的端点)上的一种放缩)

3、拉格朗日余项泰勒公式的方法(二阶导数):(题目特征:函数二阶导存在,并且大于或者小于0),将$ f(x) $在合适的$ x=x_0 $处展开,有$ f(x)=f\left(x_{0}\right)+\frac{1}{1 !} f^{\prime}\left(x_{0}\right)\left(x-x_{0}\right)+\frac{1}{2 !} f^{\prime \prime}(\xi)\left(x-x_{0}\right)^{2} $,证明的关键是合适的$ **x=x_0** $处展开;这种特征情况下也可能用两次拉格朗日中值定理证明



\subsection{存在某个$ **\xi** $满足某不等式:(存在性证明,对函数特征分析找到一点即可)}\index{不等式证明!存在某个$ **\xi** $满足某不等式:(存在性证明,对函数特征分析找到一点即可)}

1、肯定不能用的:罗尔定理,极值的必要条件(这两个定理都是求相等的,不能证不等式),单调性也不能用(单调性证明区间的值的大小,而这是要证明$ <font color=purple>\xi</font> $满足不等式)

2、可以用的:拉格朗日中值定理(存在一阶导),拉格朗日泰勒余项公式(展开到第$ <font color=purple>n</font> $阶就需要存在$ <font color=purple>**n**</font> $阶连续导数,即开区间存在$ <font color=purple>**n+1**</font> $阶导数)

函数构造:构造函数$ \varphi(x)=f(x)-\left[f(a)+\frac{f(b)-f(a)}{b-a}(x-a)\right] $,$ \varphi^{\prime}(x)=f^{\prime}(x)-\frac{f(b)-f(a)}{b-a} $(得到了拉格朗日中值定理的中值导数和两端斜率的式子,通过判断拉格朗日定理中$ <font color=purple>\varphi^{\prime}(x)</font> $因题目条件出现大于0或者小于0,就可以得到拉格朗日中值定理的$ <font color=purple>\xi</font> $的不等式)



\section{按照定义求解一点的导数}\index{按照定义求解一点的导数}



\subsection{明确的由已知可得需要用定义求解}\index{按照定义求解一点的导数!明确的由已知可得需要用定义求解}

1、先讨论连续性:不连续的函数一定不可导

2、利用定义求解该点处的左右导数:如果两边函数相同,则利用一次定义式即可求解,定义式极限存在则导数存在;如果两边函数不同,则需要分别对两边进行定义式求左右导数,左右导数都存在且相等则导数存在



\subsection{隐含的需要已知式子凑成包含定义式的式子}\index{按照定义求解一点的导数!隐含的需要已知式子凑成包含定义式的式子}

1、确定求导位置:凑成在指定点的导数定义式

2、凑定义式:通过对式子进行变换,凑出定义式(分离出来导数的定义式和另一个简单的形式相乘,转换成求解两个极限式相乘)

注意:将极限进行拆分时,拆分后的两个极限需要都存在

\section{常用的导数恒等式变换(证明构造)}\index{常用的导数恒等式变换(证明构造)}

1、$ \xi f^{\prime}(\xi)+f(\xi)=0 $:$ [x f(x)]^{\prime}=x f^{\prime}(x)+f(x) $

2、$ \xi f^{\prime}(\xi)+n f(\xi)=0 $:$ \left[x^{n} f(x)\right]^{\prime}=x^{n-1}\left[x f^{\prime}(x)+n f(x)\right] $

3、$ \xi f^{\prime}(\xi)-f(\xi)=0 $:$ \left[\frac{f(x)}{x}\right]^{\prime}=\frac{x f^{\prime}(x)-f(x)}{x^{2}} $

4、$ \xi f^{\prime}(\xi)-n f(\xi)=0 $:$ \left[\frac{f(x)}{x^{n}}\right]^{\prime}=\frac{x f^{\prime}(x)-n f(x)}{x^{n+1}} $

5、$ f^{\prime}(\xi)+\lambda f(\xi)=0 $:$ \left[\mathrm{e}^{\lambda x} f(x)\right]^{\prime}=\mathrm{e}^{\lambda x}\left[f^{\prime}(x)+\lambda f(x)\right] $

6、$ f^{\prime}(\xi)-\lambda f(\xi)=0 $:$ \left[\mathrm{e}^{-\lambda x} f(x)\right]^{\prime}=\mathrm{e}^{-\lambda x}\left[f^{\prime}(x)-\lambda f(x)\right] $

7、$ f(\xi) f^{\prime \prime}(\xi)+\left[f^{\prime}(\xi)\right]^{2}=0 $:$ \left[f(x) f^{\prime}(x)\right]^{\prime}=f(x) f^{\prime \prime}(x)+\left[f^{\prime}(x)\right]^{2} $

备注:把上述导数恒等式中的$ <font color=orange>f</font> $换成$ <font color=orange>f^{\prime}</font> $,$ <font color=orange>f^{\prime}</font> $换成$ <font color=orange>f^{\prime \prime}</font> $,$ <font color=orange>f^{\prime \prime}</font> $换成$ <font color=orange>f^{\prime \prime \prime}</font> $,可以用于证明含高阶导数的等式

注意:非以上类型,如果出现跨阶的情况,先进行补阶,函数替换后转换成以上形式(补阶操作类似于求解递推公式的方法)

注意:如果上述的值不为0,需要将非零值挪到左边,右边继续保存零值,此时左边常数部分的构造要与需要构造的形式综合考虑

\section{可导与极限}\index{可导与极限}



\subsection{讨论由极限式定义的函数的导数}\index{可导与极限!讨论由极限式定义的函数的导数}

1、先求解函数的表达式

2、再讨论函数导数的情况,利用定义或者求导



\subsection{已知函数在某一点可导,求解与此相关的极限或者参数}\index{可导与极限!已知函数在某一点可导,求解与此相关的极限或者参数}

1、当分数形式存在极限时的可推导内容:当分数形式存在极限时,如果分子或者分母一边为含有未知函数的式子,则含有未知函数的一边与另一边同阶,根据分数极限的值,可以推断含有未知函数的一边的等价形式(系数*阶数幂+无穷小,即去极限符号);特别的,不含未知参数的一边为0,且整体极限不为0,则含参数的一边极限也是0

\section{求解n阶导数}\index{求解n阶导数}

1、简单的有理式直接拆项(简单拆项):对幂函数求导时,注意指数部分会导致求导之后出现阶乘(出现$ <font color=purple>n!</font> $)

2、泰勒级数+0处$ n $阶导(0点展开):如果式子可以套用常用的泰勒级数(都是0点处的展开)展开,则可以直接展开,然后求解第$ n $次幂的系数即可(注意泰勒级数的系数中除了导数之外分母还有一个$ <font color=purple>n!</font> $,在对级数进行求导时,与$ <font color=purple>x</font> $的幂阶数进行抵消,当求导至$ <font color=purple>x</font> $的幂为0阶时,式子就剩下了泰勒级数展开点的$ <font color=purple>n</font> $阶导数值)

3、幂级数与泰勒级数综合(任一点展开):将式子展开为幂级数(在题目要求求解某一点的导数时):幂级数展开式($ \varphi(x)=\sum_{n=0}^{\infty} a_{n}\left(x-x_{0}\right)^{n} $),泰勒级数展开($ \varphi(x)=\sum_{n=0}^{\infty} \frac{1}{n !} \varphi^{(n)}\left(x_{0}\right)\left(x-x_{0}\right)^{n} $),对应关系($ \varphi^{(n)}\left(x_{0}\right)=n ! a_{n}, \quad(n=0,1, \cdots) $)(注意不要忘记幂级数的系数需要乘$ <font color=purple>n!</font> $)

4、预处理:如果是复杂的函数(arctan),先求导一次之后,展开成幂级数,再将幂级数积分得到原函数的幂级数(求解原函数时注意$ <font color=purple>f(x)=f(0)+\int_{0}^{x} f^{\prime}(t) \mathrm{d} t</font> $,补充一点是为了让0处改式子也成立)

