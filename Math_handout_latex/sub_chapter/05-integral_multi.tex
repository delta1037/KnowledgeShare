\chapterimage{chapter_head_2.pdf}
\chapter{多元积分}

\section{求解三重积分}\index{求解三重积分}



\subsection{直角坐标}\index{求解三重积分!直角坐标}

1、先一后二:对于柱形区域,先积分柱的母线方向

2、先二后一:对于某个平面的面积比较好算的情况,先积分这个平面方向

3、特殊情况(先二后一):如果从图形上可以看出可以有圆的积分(切圆片),并且圆的半径可以由另一个坐标表示,先积分圆片(就是圆面积),再将各个圆片整合(积分另一个方向)



\subsection{柱坐标}\index{求解三重积分!柱坐标}

1、柱坐标变换:$\begin{cases} x=r \cos \theta, & 0 \leqslant r < +\infty \\ y=r \sin \theta,& 0 \leqslant \theta \leqslant 2 \pi \\ z=z, &  -\infty<z<+\infty \\ \end{cases}$

2、柱坐标微元:$\mathrm{d} V=r \mathrm{drd} \theta \mathrm{d} z$

3、柱坐标积分:$\iiint_{\Omega} f(x, y, z) \mathrm{d} V=\iiint_{\Omega}(r \cos \theta, r \sin \theta, z) r \mathrm{drd} \theta \mathrm{d} z$

4、柱坐标适用范围:应用于柱、锥或柱锥组合等类型区域(被积分函数的形式是否有利于柱坐标计算$f(x, y, z)=\varphi(z) g\left(x^{2}+y^{2}\right)$)



\subsection{球坐标}\index{求解三重积分!球坐标}

1、球坐标变换:$\begin{cases}x=r \sin \varphi \cos \theta, & 0 \leqslant r<+\infty \\ y=r \sin \varphi \sin \theta, & 0 \leqslant \varphi \leqslant \pi \\ z=r \cos \varphi, & 0 \leqslant \theta \leqslant 2 \pi\end{cases}$

2、球坐标微元:$\mathrm{d} V=r^{2} \sin \varphi \mathrm{drd} \varphi \mathrm{d} \theta$

3、球坐标积分:$\iiint_{\Omega} f(x, y, z) \mathrm{d} V=\iiint_{\Omega} f(r \sin \varphi \cos \theta, r \sin \varphi \sin \theta, r \cos \varphi) r^{2} \sin \varphi \mathrm{drd} \varphi \mathrm{d} \theta$

4、球坐标适用范围:球坐标应用于球、半球、锥与球的组合类型区域(被积分函数的形式是否有利于柱坐标计算$f(x, y, z)=\varphi\left(x^{2}+y^{2}+z^{2}\right)$)

\section{空间内第二类积分恒成立问题}\index{空间内第二类积分恒成立问题}



\subsection{空间内任意有向闭曲线都有第二类曲线积分方程成立,求解方程未知量:(利用路径无关的条件(四个互相等价的条件))}\index{空间内第二类积分恒成立问题!空间内任意有向闭曲线都有第二类曲线积分方程成立,求解方程未知量:(利用路径无关的条件(四个互相等价的条件))}

1、积分曲线C为区域D上任意一段分段光滑闭曲线,且第二类曲线积分恒为0

2、转换成偏导相等(即路径无关,或者格林公式被积分函数为0),从而可得一个求未知函数的方程



\subsection{空间内任意有向闭曲面都有第二类曲面积分方程成立,求解方程未知量}\index{空间内第二类积分恒成立问题!空间内任意有向闭曲面都有第二类曲面积分方程成立,求解方程未知量}

1、想到高斯公式求三维区域积分

2、因为是任意的区域,所以只能是被积分函数为0,从而求得未知量



注意:如果未知量是函数可能用到常微分方程

\section{求解第一类曲面积分}\index{求解第一类曲面积分}



\subsection{步骤}\index{求解第一类曲面积分!步骤}

1、利用对称性和奇偶性化简,直接法进行计算(应用公式将曲面映射到某一个平面上,转换成二重积分)

key:

1、注意选取合适的平面(映射平面)

2:曲面方程可以代入到被积分函数中(因为映射需要去掉第三个维度,并且曲面的积分是对于面上的一点来说的,曲面方程可以表示被积分函数需要的所有的点)



\subsection{奇偶性对称性应用}\index{求解第一类曲面积分!奇偶性对称性应用}

1、对称性仅对积分曲面有要求(积分曲面变量互换后方程不变,变量就可以互换)

2、奇偶性对积分曲面和被积分函数都有要求(积分曲面对称、被积分函数关于对称方向有奇偶性)



\subsection{直接法计算}\index{求解第一类曲面积分!直接法计算}

1、计算公式:积分曲面$\Sigma$由方程$z=z(x, y)$给出,$\Sigma$在$x O y$面上的投影域为$D$,函数$z(x, y)$在$D$上有连续一阶偏导数,$f(x, y, z)$在$\Sigma$上连续,则$\iint_{\Sigma} f(x, y, z) \mathrm{d} S=\newline\iint_{D} f(x, y, z(x, y)) \sqrt{1+(z_{x}^{\prime})^{2}+(z_{y}^{\prime})^{2}} \mathrm{~d} x \mathrm{~d} y$(曲面积分转成平面积分)(对于$\sqrt{1+\left(z_{x}^{\prime}\right)^{2}+\left(z_{y}^{\prime}\right)^{2}}$的化简是可以代入曲面的方程进行化简的,因为这里是在利用曲面的指定位置求解曲面向平面转换的系数)(对于曲面的投影,如果是一个球类的曲面或者是关于投影面对称,注意是分割成两部分投影,最终结果乘2)(具有规范性的图形一般先使用平移法进行化简)



\subsection{图形判别}\index{求解第一类曲面积分!图形判别}

1、凭借图形选取合适的dS(例如圆筒形,可以按照纵向分割成圆环形dS)

\section{求解第二类曲线积分(二维)}\index{求解第二类曲线积分(二维)}



\subsection{步骤}\index{求解第二类曲线积分(二维)!步骤}

1、如果曲线封闭,则考虑使用格林公式(注意区域是否有不连续一阶偏导的点)

2、如果不封闭,考虑路径无关(改变积分路径或者求解原函数)

3、如果直接法方便则使用直接法,否则考虑补线(注意补线的路径包含的区域需要绕过不连续一阶导的点!)使用格林公式



\subsection{直接法(适用于参数方程)}\index{求解第二类曲线积分(二维)!直接法(适用于参数方程)}

1、参数式表示为$L:\left\{\begin{array}{l}x=x(t), \\ y=y(t),\end{array} t \in[\alpha, \beta]\right.$

2、计算公式:$\int_{L} P \mathrm{~d} x+Q \mathrm{~d} y=\int_{a}^{\beta}\left[P(x(t), y(t)) x^{\prime}(t)+Q(x(t), y(t)) y^{\prime}(t)\right] \mathrm{d} t$,这里$\alpha$对应$L$的起点,$\beta$对应曲线$L$的终点



\subsection{xy变量之间有简单关系式}\index{求解第二类曲线积分(二维)!xy变量之间有简单关系式}

1、将变量代入转换成定积分:例如x与y的方程是一条直线,可以直接代入到第二类曲面方程中化成定积分

注意:第二类曲面积分没有对称性和奇偶性



\subsection{格林公式法:(连接平面上第二类曲线积分与二重积分)(是斯托克公式的在平面上的特殊情形)}\index{求解第二类曲线积分(二维)!格林公式法:(连接平面上第二类曲线积分与二重积分)(是斯托克公式的在平面上的特殊情形)}

1、前提条件:区域$D$由分段光滑闭曲线$L$围成;$P(x, y)$及$Q(x, y)$在区域$D$连续一阶偏导

2、计算公式:$\oint_{L} P \mathrm{~d} x+Q \mathrm{~d} y=\iint_{D}\left(\frac{\partial Q}{\partial x}-\frac{\partial P}{\partial y}\right) \mathrm{d} x \mathrm{~d} y$,其中$L$是$D$的取正向的边界曲线(正向是指沿$L$的某一方向前进时,区域$D$始终在左侧)(根据曲线方向和包围面的关系确定积分的正负性)

3、对于不连续点的处理:添加曲线将不连续的点排除在外,并且该新加的曲线能够抵消掉被积分函数中不连续的点(化解掉分母)<复习全书P241>

注意:格林公式仅有分段光滑曲线和区域内连续一阶偏导的条件,并没有路径无关的$\frac{\partial P}{\partial y}=\frac{\partial Q}{\partial x}, \forall(x, y) \in D$

备注:对于包含不连续一阶导点的区域,沿任何不包含该点在内的分段光滑曲线积分为0;沿任何包含该点在内的分段光滑曲线积分相等

备注:曲线为闭曲线(或通过补线形成闭区域);在积分区域上P、Q具有连续一阶偏导(注意判断是否有不存在的点:如果xy在分母上会有<0,0>点导数不存在的情况);如果区域内部含有不存在的点的情况,通过补上一条规则曲线(曲线的选择为去掉被积分函数的分母为准)(目的是将不存在的点圈出来),然后转换为求补上的曲线内部的积分(这时通过曲线的特殊选择将分母的不可导点去掉了,可以进行内部积分了)



\subsection{补线用格林公式(不封闭曲线)}\index{求解第二类曲线积分(二维)!补线用格林公式(不封闭曲线)}

1、计算公式:$\int_{L(\widehat{A B})} P \mathrm{~d} x+Q \mathrm{~d} y=\oint_{L(\widehat{A B})+L_{1}(\widehat{B A})} P \mathrm{~d} x+Q \mathrm{~d} y-\int_{L_{1}(\widehat{BA})} P \mathrm{~d} x+Q \mathrm{~d} y$(如果右边第一项满足格林公式条件就用格林公式,否则适用直接法)



\subsection{路径无关法}\index{求解第二类曲线积分(二维)!路径无关法}

1、判定路径无关:设$P(x, y), Q(x, y)$在单连通域$D$上有连续一阶偏导数,则以下四条等价 (1) 线积分$\int P \mathrm{~d} x+Q \mathrm{~d} y$与路径无关;(2)$\oint_{C} P \mathrm{~d} x+Q \mathrm{~d} y=0$,其中$C$为$D$中任一分段光滑闭曲线;(3)$\frac{\partial P}{\partial y}=\frac{\partial Q}{\partial x}, \forall(x, y) \in D$;(4)存在可微函数$F(x, y)$,使$P(x, y) \mathrm{d} x+Q(x, y) \mathrm{d} y=\mathrm{d} F(x, y)$(路径无关的P、Q一般具有明显的变量对称性)

2、路径无关的计算:改换路径,通常取平行于坐标轴的折线;利用原函数,设$F(x, y)$是$P \mathrm{~d} x+Q \mathrm{~d} y$的原函数, 即$P \mathrm{~d} x+Q \mathrm{~d} y=\mathrm{d} F(x, y)$,则$\int_{(A)}^{(B)} P \mathrm{~d} x+Q \mathrm{~d} y=F\left(x_{2}, y_{2}\right)-F\left(x_{1}, y_{1}\right)$,其中L的起点为$A\left(x_{1}, y_{1}\right)$,终点为$B\left(x_{2}, y_{2}\right)$(求解原函数可以用偏积分和凑微分)

\section{函数中含有对函数的二重积分}\index{函数中含有对函数的二重积分}



\subsection{积分区域与函数变量无关}\index{函数中含有对函数的二重积分!积分区域与函数变量无关}

1、将二重积分设置成一个变量A(需要认识到二重积分实际上是一个常数)

2、将设置变量后函数代入到函数的二重积分中得到等于A的一个方程,解出A即可



\subsection{积分区域与函数变量相关}\index{函数中含有对函数的二重积分!积分区域与函数变量相关}

1、将二重积分简化为与函数自变量相关的积分形式(例如将自变量转换到积分上限)

2、然后对两边同时求导数(一边是函数求导,一边是变上限积分求导)

3、可能需要利用常微分方程求解函数

\section{重积分与几何}\index{重积分与几何}



\subsection{求解柱面的侧面积}\index{重积分与几何!求解柱面的侧面积}

1、直线(柱向)沿着指定路径的积分,以柱的底部为线路径(<复习全书 P262 例2 方法2>)



\subsection{求解函数对外法向量的第一类曲面积分}\index{重积分与几何!求解函数对外法向量的第一类曲面积分}

1、函数对外法向量的求导,利用一二类曲面积分的关系转换成第二类曲面积分



\subsection{第一类曲线积分,被积函数包含函数对向量(外法线向量)求偏导数}\index{重积分与几何!第一类曲线积分,被积函数包含函数对向量(外法线向量)求偏导数}

1、函数关于法向量求导的关系式:$\frac{\partial u}{\partial \boldsymbol{n}} \mathrm{d} s=\frac{\partial u}{\partial x} \cos (\boldsymbol{n}, x) \mathrm{d} s+\frac{\partial u}{\partial y} \cos (\boldsymbol{n}, y) \mathrm{d} s$(这是由方向导数推导出来的)

2、找到法向量表示与曲线切线方向向量(即方向余弦)的关系并进行转换

3、将第一类曲线积分转换成第二类曲线积分(第二类曲线积分的方向向量是曲线的正向的切线方向)

\section{曲线、曲面积分的定义与性质}\index{曲线、曲面积分的定义与性质}



\subsection{第一类曲线积分(对弧长的积分)的定义和性质}\index{曲线、曲面积分的定义与性质!第一类曲线积分(对弧长的积分)的定义和性质}

1、定义:$\int_{L} f(x, y) \mathrm{d} s\frac{\Delta}{=} \lim _{\lambda \rightarrow 0} \sum_{i=1}^{n} f\left(\xi_{i}, \eta_{i}\right) \Delta s_{i}$

2、性质(与积分方向无关):$\int_{L(\widehat{A B})} f(x, y) \mathrm{d} s=\int_{L(\widehat{B A})} f(x, y) \mathrm{d} s$



\subsection{第二类曲线积分(对坐标的积分)的定义和性质}\index{曲线、曲面积分的定义与性质!第二类曲线积分(对坐标的积分)的定义和性质}

1、定义式:$\int_{L(\widehat{A B})} P(x, y) \mathrm{d} x+Q(x, y) \mathrm{d} y \Longrightarrow \lim_{\lambda \rightarrow 0} \sum_{i=1}^{n}\left[P\left(\xi_{i}, \eta_{i}\right) \Delta x_{i}+Q\left(\xi_{i}, \eta_{i}\right) \Delta y_{i}\right]$

2、性质(与积分方向有关):$\int_{L(\widehat{A B})} P \mathrm{~d} x+Q \mathrm{~d} y=-\int_{L(\widehat{B A})} P \mathrm{~d} x+Q \mathrm{~d} y$(负号出现是因为方向余弦反向了)

3、与第一类曲线积分的关系:$\int_{L} P \mathrm{~d} x+Q \mathrm{~d} y=\int_{L}(P \cos \alpha+Q \cos \beta) \mathrm{ds}$,其中$\cos \alpha, \cos \beta$为有向曲线$L$的切线的方向余弦(注意:方向余弦的方向是否是正向的,即积分路径的方向)



\subsection{第一类面积分定义与性质}\index{曲线、曲面积分的定义与性质!第一类面积分定义与性质}

1、定义:$\iint_{\Sigma} f(x, y, z) \mathrm{d} S=\lim_{\lambda \rightarrow 0} \sum_{i=1}^{n} f\left(\xi_{i}, \eta_{i}, \zeta_{i}\right) \Delta S_{i}$

2、性质(与曲面的侧无关):即$\iint_{\Sigma} f(x, y, z) \mathrm{d} S=\iint_{-\Sigma} f(x, y, z) \mathrm{d} S$



\subsection{第二类面积分定义与性质}\index{曲线、曲面积分的定义与性质!第二类面积分定义与性质}

1、定义:$\iint_{\Sigma} P \mathrm{~d} y \mathrm{~d} z+Q \mathrm{~d} z \mathrm{~d} x+R \mathrm{~d} x \mathrm{~d} y=\lim _{\lambda \rightarrow 0} \sum_{i=1}^{n}[P\left(\xi_{i}, \eta_{i}, \zeta_{i}\right)\left(\Delta S_{i}\right)_{y z}+Q\left(\xi_{i}, \eta_{i}, \zeta_{i}\right)\left(\Delta S_{i}\right)_{z x}+R\left(\xi_{i}, \eta_{i}, \zeta_{i}\right)\left(\Delta S_{i}\right)_{x y}]$

2、性质(与曲面的侧有关):$\iint_{\Sigma} P \mathrm{~d} y \mathrm{~d} z+Q \mathrm{~d} z \mathrm{~d} x+R \mathrm{~d} x \mathrm{~d} y=-\iint_{-\Sigma} P \mathrm{~d} y \mathrm{~d} z+Q \mathrm{~d} z \mathrm{~d} x+R \mathrm{~d} x \mathrm{~d} y$(负号的出现是因为取面侧时法线向量的方向反向了)

3、与第一类面积分的关系:$\iint_{\Sigma} P \mathrm{~d} y \mathrm{~d} z+Q \mathrm{~d} z \mathrm{~d} x+R \mathrm{~d} x \mathrm{~d} y=\iint_{\Sigma}(P \cos \alpha+Q \cos \beta+R \cos \gamma) \mathrm{d} S$,其中$\cos \alpha, \cos \beta, \cos \gamma$为曲面$\Sigma$上点$P(x, y, z)$处指定侧的法线向量的方向余弦(转换关系类似于$\mathrm{~d} y \mathrm{~d} z = |\cos \alpha| \ \mathrm{d} S$,对坐标的积分转为对曲面的积分,所以从曲面转换到平面上的积分时,会损失方向余弦的正负性,这时就出现了曲面方向与坐标轴夹角与积分正负性的问题)



\subsection{第一类曲面积分和第一类曲线积分化简:(平移法或形心公式法)(重要思路)}\index{曲线、曲面积分的定义与性质!第一类曲面积分和第一类曲线积分化简:(平移法或形心公式法)(重要思路)}

1、待积分曲面或者积分曲线具有规范性(圆、球)具有关于某个变量(也就是积分变量)对称性质的,利用了平移之后的对称性(结果为0)

2、积分变量是一次的(一般奇数次幂变量平移都可以,因为需要利用奇偶性抵消,但是三次之上就比较麻烦了),并且变量可以是多个,可以分别平移

\section{求解二重积分}\index{求解二重积分}



\subsection{直角坐标}\index{求解二重积分!直角坐标}

1、先积分不规则的方向(使用函数框定的方向)

2、再积分规则的方向(使用变量值框定的方向)

3、复杂区域的处理:使用平行于坐标轴的方向的直线将积分区域划分(尽量划分成利于对称性和奇偶性应用的区域)

注意:如果某个方向的积分不利于求出来,可以转换积分次序(累次积分交换次序)



\subsection{极坐标}\index{求解二重积分!极坐标}

方法一、先积分径向,再积分角向

1、判断径向范围:径向是由角度表示的函数,由曲线方程给出(由方程的定义域值域之类的确定)或者图像确定(即由图像求解,从图像上看径向范围有极点在积分区域外部$\iint_{D} f(x, y) \mathrm{d} \sigma=\int_{a}^{\beta} \mathrm{d} \theta \int_{r_{1}(\theta)}^{r_{2}(\theta)} f(r \cos \theta, r \sin \theta) r \mathrm{~d} r$、在积分区域边界$\iint_{D} f(x, y) \mathrm{d} \sigma=\int_{a}^{\beta} \mathrm{d} \theta \int_{0}^{r(\theta)} f(r \cos \theta, r \sin \theta) r \mathrm{~d} r$、在积分区域内部$\iint_{D} f(x, y) \mathrm{d} \sigma=\int_{0}^{2 \pi} \mathrm{d} \theta \int_{0}^{r(\theta)} f(r \cos \theta, r \sin \theta) r \mathrm{~d} r$、在环形区域内部$\iint_{D} f(x, y) \mathrm{d} \sigma=\int_{0}^{2 \pi} \mathrm{d} \theta \int_{r_{1}(\theta)}^{r_{2}(\theta)} f(r \cos \theta, r \sin \theta) r \mathrm{~d} r$)

2、角度的范围确定:即$r > 0 $的范围(判断角度范围:转换成极坐标形式判断角度范围,根据r=0求解Θ的值,根据Θ的象限来判断是属于哪个范围(判断临界的切线角度的方法))

方法二、先积分角向,再积分径向<复习全书P221>

1、判断角度范围:角度用关于径$r$的方程确定(即角度的边界表示),即将待积分图像分割成环状,环的边界由关于径$r$的方程确定(如果关于径$r$的方程在尖锐点发生了变化,需要将积分区域拆分(当环的边界发生变化时注意分段))

2、判断径向的范围:径向范围就是积分区域最内到最外的范围(最好通过图像确定)



\subsection{参数坐标(边界曲线为参数方程)}\index{求解二重积分!参数坐标(边界曲线为参数方程)}

1、判断x、y的范围:利用直角坐标的形式(注意上下限的判断,画出来大概的图形,或者根据参数式判断x、y的范围)

2、先积分一个未知量,上下限可以先用函数表示,转换成一重积分

3、将参数方程代入一重积分即可(<复习全书P218>)



注意:对于含有绝对值、min、max被积分函数的先划分积分区域然后去掉这些符号

\section{重积分累次积分交换次序}\index{重积分累次积分交换次序}



\subsection{直角坐标-二重积分}\index{重积分累次积分交换次序!直角坐标-二重积分}

1、根据积分确定积分区域并画出草图

2、按另一种积分次序确定积分的上下限



\subsection{直角坐标-三重积分}\index{重积分累次积分交换次序!直角坐标-三重积分}

1、确定积分区域

2、然后进行两两交换(交换感觉容易出错)



\subsection{极坐标-二重积分}\index{重积分累次积分交换次序!极坐标-二重积分}

1、根据积分确定积分区域并画出草图

2、极坐标下交换,变换成先θ后r的形式(即先分割成环状(角向积分),然后进行环的积分(径向积分))<复习全书P221>

备注:极坐标下交换成先θ后r的形式时,需要由原先的$r=f(\theta)$形式转换为$\theta=g(r)$形式,即由角度求值变成由值确定角度,对于$f(\theta)$具有三角函数表示的这类问题先作三角函数图像,然后确定自变量$\theta$的范围(由积分确定)并框定三角函数图像中的有效范围(有自变量的范围),然后根据$r=f(\theta)$求解$sin(\theta)=A$(或类似的三角函数形式)的形式,然后画出y=A与三角函数有效范围的交点,然后由交点确定角度的范围(反三角函数图像就是$x,y$轴交换之后的图像,三角函数图像关于$y=x$对称,注意与三角函数倒数的图像区别)

注意:如果题目所给的二重积分不好算,则一般需要交换次序,如果仍然不好算,则试着转换成极坐标形式

\section{二重积分不等式}\index{二重积分不等式}



\subsection{二重积分不等式处理}\index{二重积分不等式!二重积分不等式处理}

1、积分的乘积转换成二重积分

2、幂级数展开对被积分函数进行放缩

3、柯西-施瓦兹积分不等式(积分乘积与二重积分的关系)



\subsection{二重积分的比较}\index{二重积分不等式!二重积分的比较}

1、比较被积分函数在同一个积分区域上的大小(比较内部函数大小(指定定义域函数比较))

2、比较被积分函数在不同积分区域上的大小(分割区域,比较内部函数大小(指定同一个定义域函数比较),不同函数在额外的定义域上的正负性啥的)

注意:利用积分区域的对称型和奇偶性去掉无关的被积分函数



\section{求解第一类曲线积分}\index{求解第一类曲线积分}



\subsection{奇偶性对称性应用}\index{求解第一类曲线积分!奇偶性对称性应用}

1、对称性仅对积分曲线有要求(积分曲线变量互换后方程不变,变量就可以互换)

2、奇偶性对积分曲线和被积分函数都有要求(积分曲线对称、被积分函数关于对称方向有奇偶性)



\subsection{二维曲线积分}\index{求解第一类曲线积分!二维曲线积分}

1、参数方程:$\int_{L} f(x, y) \mathrm{d} s=\int_{a}^{\beta} f(x(t), y(t)) \sqrt{x^{\prime 2}(t)+y^{\prime 2}(t)} \mathrm{d} t$

2、直角坐标方程:$\int_{L} f(x, y) \mathrm{d} s=\int_{a}^{b} f(x, y(x)) \sqrt{1+y^{\prime 2}(x)} \mathrm{d} x$

3、极坐标方程:$\int_{L} f(x, y) \mathrm{d} s=\int_{a}^{\beta} f(r(\theta) \cos \theta, r(\theta) \sin \theta) \sqrt{r^{2}+r^{\prime 2}} \mathrm{~d} \theta$



\subsection{三维曲线积分}\index{求解第一类曲线积分!三维曲线积分}

1、转参数方程:空间曲线一般转换为参数方程(注意参数方程的求解$ds$化为$dt$,关系为$ds = \sqrt{x^{\prime 2}(t)+y^{\prime 2}(t)+z^{\prime 2}(t)} \mathrm{d} t$)

2、简化式子:利用变量的对称性将被积分函数转换为常量

\section{二重积分的定义和性质}\index{二重积分的定义和性质}



\subsection{二重积分的定义和性质}\index{二重积分的定义和性质!二重积分的定义和性质}

1、定义:$\iint_{D} f(x, y) \mathrm{d} \sigma=\frac{\Delta}{=} \lim _{d \rightarrow 0} \sum_{k=1}^{n} f\left(\xi_{k}, \eta_{k}\right) \Delta \sigma_{k}$

2、几何意义:侧面以$D$的边界为准线,顶为$f(x, y)$的曲顶柱体的体积

3、性质-比较定理:如果$f(x, y) \leqslant g(x, y)$,则$\iint_{D} f(x, y) \mathrm{d} \sigma \leqslant \iint_{D} g(x, y) \mathrm{d} \sigma$

4、性质-估值定理:设$M, m$分别为连续函数$f(x, y)$在闭区域$D$上的最大值和最小值,$S$表示$D$的面积,则$m S \leqslant \iint_{D} f(x, y) \mathrm{d} \sigma \leqslant M S$

5、性质-中值定理:设函数$f(x, y)$在闭区域$D$上连续,$S$为$D$的面积,则在$D$上至少存在一点$(\xi, \eta)$,使$\iint_{D} f(x, y) \mathrm{d} \sigma=f(\xi, \eta) S$



\subsection{三重积分的定义和性质}\index{二重积分的定义和性质!三重积分的定义和性质}

1、定义:$\iiint_{\Omega} f(x, y,z) \mathrm{d} V=\frac{\Delta}{=} \lim _{\lambda \rightarrow 0} \sum_{k=1}^{n} f\left(\xi_{k}, \eta_{k},\zeta_{k} \right) \Delta v_{k}$

2、几何意义:空间体的体积或者质量,取决于$f(x, y,z) $

3、性质-比较定理:与二重积分相似

4、性质-估值定理:与二重积分相似

5、性质-中值定理:与二重积分相似

\section{方向导数、梯度、散度、旋度-题型}\index{方向导数、梯度、散度、旋度-题型}



\subsection{函数梯度的旋度为0}\index{方向导数、梯度、散度、旋度-题型!函数梯度的旋度为0}

1、代入公式求解即可,对任意函数均成立



\subsection{函数梯度的散度为定值,求解函数}\index{方向导数、梯度、散度、旋度-题型!函数梯度的散度为定值,求解函数}

1、代入公式建立方程

注意:可能会用到常微分方程



\section{求解第二类曲面积分}\index{求解第二类曲面积分}



\subsection{步骤}\index{求解第二类曲面积分!步骤}

1、如果是闭曲面,先考虑高斯公式;如果不是闭曲面,考虑用直接法,如果直接法不方便时,补面使用高斯公式



\subsection{求解方法}\index{求解第二类曲面积分!求解方法}

1、高斯公式:(连接空间上第二类曲面积分与三重积分)设空间闭区域$\Omega$是由分片光滑的闭曲面$\Sigma$所围成, 函数$P(x, y, z), Q(x, y, z), R(x, y, z)$在$\Omega$上有连续一阶偏导数,闭曲面$\Sigma$取外侧,则$\oiint_{\Sigma} P \mathrm{~d} y \mathrm{~d} z+Q \mathrm{~d} z \mathrm{~d} x+R \mathrm{~d} x \mathrm{~d} y=\iiint_{\Omega}\left(\frac{\partial P}{\partial x}+\frac{\partial Q}{\partial y}+\frac{\partial R}{\partial z}\right) \mathrm{d} V$(备注:一定要注意区域内是否有不连续的点<分母为0等>!!!)(对闭曲面的积分转换为对体积的三重积分)

2、补面使用高斯公式:$\iint_{\Sigma}=\oiint_{\Sigma+\Sigma_{1}}-\iint_{\Sigma_{1}}$

3、直接法:设有向曲面$\Sigma: z=z(x, y),(x, y) \in D_{x y}$,则$\iint_{\Sigma} R(x, y, z) \mathrm{d} x \mathrm{~d} y=\pm \iint_{D_{x y}} R(x, y, z(x, y)) \mathrm{d} x \mathrm{~d} y$,若有向曲面$\Sigma$的法向量与$z$轴正向夹角为锐角,即上侧,上式中取“+" 号,否则取“-"号,按照类似的求解$\iint_{\Sigma} P(x, y, z) \mathrm{d} y \mathrm{~d} z+Q(x, y, z) \mathrm{d} z \mathrm{~d} x+R(x, y, z) \mathrm{d} x \mathrm{~d} y$(将每一项第二类曲面积分都化为二重积分计算,计算三项计算量特别大,通常在对三项归一化之后再化为二重积分计算)

4、直接法(一次性):如果整个曲面$\Sigma$可用方程$z=z(x, y)$(或$x=x(y, z), y=y(x, z)$)表示,则可一次将以上面积分化为一个重积分计算,例如有向曲面$\Sigma$由方程$z=z(x, y)$给出,$\Sigma$在$x O y$面上的投影域为$D_{x y}$,并且$z(x, y)$在$D_{x y}$上有连续一阶偏导数,$P(x, y, z), Q(x, y, z), R(x, y, z)$在$\Sigma$上连续,则$\iint_{\Sigma} P(x, y, z) \mathrm{d} y \mathrm{~d} z+Q(x, y, z) \mathrm{d} z \mathrm{~d} x+R(x, y, z) \mathrm{d} x \mathrm{~d} y = \pm \iint_{D_{x y}}[P(x, y, z(x, y))\left(-\frac{\partial z}{\partial x}\right)+Q(x, y, z(x, y))\left(-\frac{\partial z}{\partial y}\right)+R(x, y, z(x, y))] \mathrm{d} x \mathrm{~d} y$,若有向曲面$\Sigma$的法向量与$z$轴正向夹角为锐角,即上侧,上式中取“+" 号,否则取“-"号(备注:注意曲面的方向)(该式子是根据第一类曲面积分和第二类曲面积分的关系推导的,无论曲面是什么方向,曲面的归一化式子始终是$\iint_{\Sigma} P(x, y, z) \mathrm{d} y \mathrm{~d} z+Q(x, y, z) \mathrm{d} z \mathrm{~d} x+R(x, y, z) \mathrm{d} x \mathrm{~d} y =  \iint_{\Sigma}[P(x, y, z(x, y))\left(-\frac{\partial z}{\partial x}\right)+Q(x, y, z(x, y))\left(-\frac{\partial z}{\partial y}\right)+R(x, y, z(x, y))] \cos \gamma \ \mathrm{d} S$,再转换到平面上时,由$\mathrm{~d} x \mathrm{~d} y = |\cos \gamma | \ \mathrm{d} S$(注意此时是平面的$\mathrm{~d} x \mathrm{~d} y$,如果是曲面的$\mathrm{~d} x \mathrm{~d} y$表示为$\mathrm{~d} x \mathrm{~d} y = \cos \gamma  \ \mathrm{d} S$,曲面的正负性还在曲面方向表示上),最终化为了一个第二类曲面积分,由$\cos \gamma$的正负性,即与$z$轴的夹角带来正负号,即归一化,然后再利用直接法化为平面上的二重积分)(总的来说分两步走,先归一化,再用直接法转为二重积分)



\subsection{化简方法}\index{求解第二类曲面积分!化简方法}

1、当待积分区域在某一个平面上投影为一条曲线时,该面上的第二类曲面积分为0



注意:没有对称性和奇偶性

备注:高斯公式条件是函数具有连续一阶偏导数,注意确认题目是否满足

\section{二重与三重积分-求解步骤和化简}\index{二重与三重积分-求解步骤和化简}



\subsection{步骤}\index{二重与三重积分-求解步骤和化简!步骤}

1、先考虑对称性和奇偶性化简

2、根据积分区域和被积分函数的方程选择合适的坐标系进行积分

3、选择合适的积分次序,必要时对区域进行分块



\subsection{奇偶性对称性应用}\index{二重与三重积分-求解步骤和化简!奇偶性对称性应用}

1、对称性:仅对积分平面有要求(积分平面关于$y=x$对称,变量就可以互换)(当区域对称时,无论是在什么情况下,都要考虑到这还有对称性可以用)

2、奇偶性:对积分平面和被积函数都有要求(积分平面在积分方向上对称、被积函数在积分方向上有奇偶性,积分可以为0或折叠)

3、可平移:可平移前提是积分平面平移后出现某一个方向的奇偶性,平移方法为被积函数向积分平面平移后出现奇偶性的方向的反向进行同样大小的变换(相当于平移坐标轴使得新的坐标轴下有关于积分平面的奇偶性)(可平移是为了得到奇偶性简化计算,奇偶性能否应用还是要看奇偶性的应用条件:积分平面+被积函数的要求)

\section{重积分与函数、微分结合}\index{重积分与函数、微分结合}



\subsection{求解带有二重积分的极限}\index{重积分与函数、微分结合!求解带有二重积分的极限}

1、交换积分上下限(有可能),利用洛必达法则

2、利用积分中值定理,转换成面积与中值的乘积



\subsection{二重积分-变上限累次积分求导}\index{重积分与函数、微分结合!二重积分-变上限累次积分求导}

1、累次积分交换次序,变换成先积分与表示的函数无关的变量<复习全书P223>

\section{求解第二类曲线积分(三维)}\index{求解第二类曲线积分(三维)}

三维空间积分定义:$\int_{L} P(x, y, z) \mathrm{d} x+Q(x, y, z) \mathrm{d} y+R(x, y, z) \mathrm{d} z$

1、参数式法:曲线$L:\left\{\begin{array}{l}x=x(t) \\ y=y(t), t \in[\alpha, \beta] \text { 或 } t \in[\beta, \alpha]\\ z=z(t),\end{array}\right.$,则$\int_{L} P \mathrm{~d} x+Q \mathrm{~d} y+R \mathrm{~d} z= \int_{a}^{\beta}\left[P(x(t), y(t), z(t)) x^{\prime}(t)+Q(x(t), y(t), z(t)) y^{\prime}(t)+R(x(t), y(t), z(t)) z^{\prime}(t)\right] \mathrm{d} t$(计算简单,不容易出错)

2、斯托克斯公式法:(连接空间上第二类曲线积分与第一类曲面积分或第二类曲面积分)设$\Gamma$为分段光滑的空间有向闭曲线,$\Sigma$是以$\Gamma$为边界的分段光滑有向曲面,$\Gamma$的方向与$\Sigma$的方向符合右手法则,$P, Q, R$在$\Sigma$上有连续一阶偏导数,则$\oint_{\Gamma} P \mathrm{~d} x+Q \mathrm{~d} y+R \mathrm{~d} z=\iint_{\Sigma}\left(\frac{\partial R}{\partial y}-\frac{\partial Q}{\partial z}\right) \mathrm{d} y \mathrm{~d} z+\left(\frac{\partial P}{\partial z}-\frac{\partial R}{\partial x}\right) \mathrm{d} z \mathrm{~d} x+\left(\frac{\partial Q}{\partial x}-\frac{\partial P}{\partial y}\right) \mathrm{d} x \mathrm{~d} y$(也可以记为$\iint_{\Sigma}\left|\begin{array}{ccc}\cos \alpha & \cos \beta & \cos \gamma \\\frac{\partial}{\partial x} & \frac{\partial}{\partial y} & \frac{\partial}{\partial z} \\P & Q & R\end{array}\right| d S=\oint_{\Gamma} P d x+Q d y+R d z$)(将第二类曲线积分转换为第一类曲面积分,其中正负符号是在曲面法线向量中确定的)

3、将$z$关于$x,y$的方程直接代入,转换为二维第二类曲线积分<复习全书P247>

