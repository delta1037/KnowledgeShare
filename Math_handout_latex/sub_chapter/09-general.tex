\chapterimage{chapter_head_2.pdf}
\chapter{微积分通识}

\section{计算细节问题}\index{计算细节问题}

1、$ arcsin(sin\theta)=\theta $时,$ \theta $的范围必须是$ [-\pi/2,\pi/2] $,否则需要利用$ sin(\pi-\theta)=sin(\theta) $或者其它三角恒等式对内层函数进行预变换(变换的原因是内层的$ <font color=purple>sin\theta</font> $会把值限制在$ <font color=purple>[-1,1]</font> $的范围之内,也就抹除掉了$ <font color=purple>\theta</font> $超出$ <font color=purple>[-\pi/2,\pi/2]</font> $的特性)(反三角函数图像就是$ <font color=purple>x,y</font> $轴交换之后的图像,三角函数图像关于$ <font color=purple>y=x</font> $对称,注意与三角函数倒数的图像区别)

2、$ {(x^2)}^{\frac{3}{2}}=|x|^{3} $:变量先平方再开方之后,负值特性就失去了

3、$ \sqrt{sin^2x}=|sinx| $:三角函数求根号值时加绝对值

4、变量并入根号:$ xy=\sqrt{x^2+y^2} $可得到$ y=\sqrt{1+\frac{y^2}{x^2}}(x > 0) $和$ y=-\sqrt{1+\frac{y^2}{x^2}}(x < 0) $两种情况

5、$ \int_{0}^{1} e^x dx $的计算,不要漏下限的计算结果$ -1 $,$ e^0=1 $!!!

6、$ arctan\ x + arctan\ \frac{1}{x}=\frac{\pi}{2} $,反三角函数的值的特性

7、反三角函数的合并公式:$ arctan\ a - arctan\ b=arctan\ \frac{a-b}{1+ab} $(通过$ tan(a-b) $的展开式进行推导)



\section{通识公式}\index{通识公式}

1、海伦公式(求三角形面积):$ S=\sqrt{p(p-x)(p-y)(p-z)}=\frac{1}{2} y h $(其中$ p $是三角形半周长)

2、$ e $的不等式:$ \left(1+\frac{1}{n}\right)^{n}<\mathrm{e}<\left(1+\frac{1}{n}\right)^{n+1} $(n为正整数)

3、伽马函数积分:伽马函数定义为$ \Gamma(z)=\int_{0}^{\infty} t^{z-1} \mathrm{e}^{-t} \mathrm{~d} \mathrm{t} $,伽马函数的递推公式$ \Gamma(z+1)=z \Gamma(z) $(成立条件为z是非负整数和非0以外的所有值),并且$ \Gamma\left(\frac{1}{2}\right) =\sqrt{\pi} $;$ \Gamma(1) =0 ! $(对于其它的值利用递推公式递推即可)



