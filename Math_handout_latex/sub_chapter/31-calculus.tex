\chapterimage{chapter_head_2.pdf}
\chapter{高等数学公式}

\section{零碎公式}\index{零碎公式}

1、海伦公式(求三角形面积):$S=\sqrt{p(p-x)(p-y)(p-z)}=\frac{1}{2} y h$(其中$p$是三角形半周长)

2、$e$的不等式:$\left(1+\frac{1}{n}\right)^{n}<\mathrm{e}<\left(1+\frac{1}{n}\right)^{n+1}$(n为正整数)

3、伽马函数积分:伽马函数定义为$\Gamma(z)=\int_{0}^{\infty} t^{z-1} \mathrm{e}^{-t} \mathrm{~d} \mathrm{t}$,伽马函数的递推公式$\Gamma(z+1)=z \Gamma(z)$(成立条件为z是非负整数和非0以外的所有值),并且$\Gamma\left(\frac{1}{2}\right) =\sqrt{\pi}$;$\Gamma(1) =0 !$(对于其它的值利用递推公式递推即可)

\section{展开式}\index{展开式}

1、$\frac{1}{1-x}=1+x+x^{2}+\cdots+x^{n}+\cdots, x \in(-1,1)$;

2、$\frac{1}{1+x}=1-x+x^{2}+\cdots+(-1)^{n} x^{n}+\cdots, x \in(-1,1)$;

3、$\mathrm{e}^{x}=1+x+\frac{x^{2}}{2 !}+\cdots+\frac{x^{n}}{n !}+\cdots, x \in(-\infty,+\infty)$;

4、$\sin x=x-\frac{x^{3}}{3 !}+\cdots+\frac{(-1)^{n} x^{2 n+1}}{(2 n+1) !}+\cdots, x \in(-\infty,+\infty)$;

5、$\cos x=1-\frac{x^{2}}{2 !}+\cdots+\frac{(-1)^{n} x^{2 n}}{(2 n) !}+\cdots, x \in(-\infty,+\infty)$;

6、$\ln (1+x)=x-\frac{x^{2}}{2}+\cdots+\frac{(-1)^{n-1} x^{n}}{n}+\cdots, x \in(-1,1]$;

7、$(1+x)^{\alpha}=1+\alpha x+\frac{\alpha(\alpha-1)}{2 !} x^{2}+\cdots+\frac{\alpha(\alpha-1) \cdots(\alpha-n+1)}{n !} x^{n}+\cdots, x \in(-1,1)$;

\section{傅里叶级数}\index{傅里叶级数}



\subsection{将函数展开为傅里叶级数:(周期为$2\pi$)}\index{傅里叶级数!将函数展开为傅里叶级数:(周期为$2\pi$)}

1、$[-\pi, \pi]$上展开:$\left\{\begin{array}{l} a_{n}=\frac{1}{\pi} \int_{-\pi}^{\pi} f(x) \cos n x \mathrm{~d} x, \quad n=0,1,2, \cdots \\ b_{n}=\frac{1}{\pi} \int_{-\pi}^{\pi} f(x) \sin n x \mathrm{~d} x, \quad n=1,2, \cdots \end{array}\right.$

2、$[-\pi, \pi]$上奇偶函数的情况:$f(x)$为奇函数时$\begin{cases}a_{n}=0, & n=0,1, \cdots \\ b_{n}=\frac{2}{\pi} \int_{0}^{\pi} f(x) \sin n x \mathrm{~d} x, & n=1,2, \cdots\end{cases}$,$f(x)$为偶函数时$\begin{cases}a_{n}=\frac{2}{\pi} \int_{0}^{\pi} f(x) \cos n x \mathrm{~d} x, & n=0,1,2, \cdots \\ b_{n}=0, & n=1,2, \cdots\end{cases}$

3、$[0, \pi]$上展为正弦或展为余弦级数:正弦展开为$\begin{cases}a_{n}=0, & n=0,1,2, \cdots \\ b_{n}=\frac{2}{\pi} \int_{0}^{\pi} f(x) \sin n x \mathrm{~d} x, & n=1,2, \cdots\end{cases}$;余弦展开为$\begin{cases}a_{n}=\frac{2}{\pi} \int_{0}^{\pi} f(x) \cos n x \mathrm{~d} x, & n=0,1,2, \cdots \\ b_{n}=0, & n=1,2, \cdots\end{cases}$(相当于将半区间延拓成奇函数或者偶函数形式,延拓之后与上述2公式一致)



\subsection{将函数展开为傅里叶级数:(周期为$2l$)}\index{傅里叶级数!将函数展开为傅里叶级数:(周期为$2l$)}

1、$[-l, l]$上展开:$\left\{\begin{array}{l} a_{n}=\frac{1}{\pi} \int_{-\pi}^{\pi} f(x) \cos n x \mathrm{~d} x, \quad n=0,1,2, \cdots \\ b_{n}=\frac{1}{\pi} \int_{-\pi}^{\pi} f(x) \sin n x \mathrm{~d} x, \quad n=1,2, \cdots \end{array}\right.$

2、$[-l, l]$上奇偶函数的情况:$f(x)$为奇函数时$\begin{cases}a_{n}=0, & n=0,1,2, \cdots \\ b_{n}=\frac{2}{l} \int_{0}^{l} f(x) \sin \frac{n \pi x}{l} \mathrm{~d} x, & n=1,2, \cdots\end{cases}$,$f(x)$为偶函数时$\begin{cases}a_{n}=\frac{2}{l} \int_{0}^{l} f(x) \cos \frac{n \pi x}{l} \mathrm{~d} x, & n=0,1,2, \cdots \\ b_{n}=0, & n=1,2, \cdots\end{cases}$

3、$[0, l]$上展为正弦或展为余弦级数:正弦展开为$\begin{cases}a_{n}=0, & n=0,1,2, \cdots \\ b_{n}=\frac{2}{l} \int_{0}^{l} f(x) \sin \frac{n \pi x}{l} \mathrm{~d} x, & n=1,2, \cdots\end{cases}$(奇延拓);余弦展开为$\begin{cases}a_{n}=\frac{2}{l} \int_{0}^{l} f(x) \cos \frac{n \pi x}{l} \mathrm{~d} x, & n=0,1,2, \cdots \\ b_{n}=0, & n=1,2, \cdots\end{cases}$(偶延拓)

\section{线性方程的解}\index{线性方程的解}



\subsection{二阶常系数齐次线性方程($y^{\prime \prime}+p y^{\prime}+q y=0$)的通解:对于特征方程$r^{2}+p r+q=0$的两个根$ r_{1}, r_{2} $}\index{线性方程的解!二阶常系数齐次线性方程($y^{\prime \prime}+p y^{\prime}+q y=0$)的通解:对于特征方程$r^{2}+p r+q=0$的两个根$ r_{1}, r_{2} $}

1、两个不相等的实根$r_{1}, r_{2}$:方程通解为$ y=C_{1} \mathrm{e}^{r_1 x}+C_{2} \mathrm{e}^{r_2 x}$

2、两个相等的实根$r_{1}=r_{2}$:方程通解为$y=\left(C_{1}+C_{2} x\right) \mathrm{e}^{r_1 x}$

3、一对共轭复根$r_{1,2}=\alpha \pm \mathrm{i} \beta$:方程通解为$y=\mathrm{e}^{a x}\left(C_{1} \cos \beta x+C_{2} \sin \beta x\right)$



\subsection{$n$阶常系数齐次线性方程($y^{(n)}+p_{1} y^{(n-1)}+p_{2} y^{(n-2)}+\cdots+p_{n-1} y^{\prime}+p_{n} y=0$)的通解:对于特征方程的根}\index{线性方程的解!$n$阶常系数齐次线性方程($y^{(n)}+p_{1} y^{(n-1)}+p_{2} y^{(n-2)}+\cdots+p_{n-1} y^{\prime}+p_{n} y=0$)的通解:对于特征方程的根}

1、单实根$r$:方程通解对应一项$ C \mathrm{e}^{rx}$

2、$k$重实根$r$:方程通解对应k项${e}^{r x}\left(C_{1}+C_{2} x+\cdots+C_{k} x^{k-1}\right)$

3、一对单复根$r_{1,2}=\alpha \pm \mathrm{i} \beta$:方程通解对应两项${e}^{a x}\left(C_{1} \cos \beta x+C_{2} \sin \beta x\right)$

4、一对$k$重单复根$r_{1,2}=\alpha \pm \mathrm{i} \beta$:方程通解为对应$2k$项${e}^{{ax}}[(C_{1}+C_{2} x+\cdots+C_{k} x^{k-1}) \cos \beta x+(D_{1}+D_{2} x+\cdots+D_{k} x^{k-1}) \sin \beta x]$



\subsection{线性常系数非齐次方程特解}\index{线性方程的解!线性常系数非齐次方程特解}

1、$f(x)=\mathrm{e}^{\lambda x} P_{m}(x)$型($\lambda$为已知常数,$P_{m}(x)$为$x$的$m$次已知多项式):待定特解设为$y^{}=x^{k} \mathrm{e}^{\lambda x} Q_{m}(x)$($k$是特征方程根$\lambda$的重数,$Q_{m}(x)$为系数待定的$x$的$m$次多项式)

2、$f(x)=\mathrm{e}^{\lambda x}\left[P_{l}(x) \cos w x+P_{n}(x) \sin w x\right]$型($\lambda$为已知常数,$P_{l}(x)$与$P_{n}(x)$分别为$x$的$l$次、$x$的$n$次的已知多项式):待定特解设为$y^{}=x^{k} \mathrm{e}^{\lambda x}\left[R_{m}^{(1)}(x) \cos w x+R_{m}^{(2)}(x) \sin w x\right]$($k$是特征方程根$\lambda+\mathrm{i} w\left(\right.$或$\lambda-\mathrm{i} w$) 的重数,$R_{m}^{(1)}(x)$与$R_{m}^{(2)}(x)$为系数待定的$m$次多项式,$m=\max \{l, n\}$)

\section{坐标代换}\index{坐标代换}



\subsection{柱坐标}\index{坐标代换!柱坐标}

1、柱坐标变换:$\begin{cases} x=r \cos \theta, & 0 \leqslant r < +\infty \\ y=r \sin \theta,& 0 \leqslant \theta \leqslant 2 \pi \\ z=z, &  -\infty<z<+\infty \\ \end{cases}$

2、柱坐标微元:$\mathrm{d} V=r \mathrm{drd} \theta \mathrm{d} z$

3、柱坐标积分:$\iiint_{\Omega} f(x, y, z) \mathrm{d} V=\iiint_{\Omega}(r \cos \theta, r \sin \theta, z) r \mathrm{drd} \theta \mathrm{d} z$

4、柱坐标适用范围:应用于柱、锥或柱锥组合等类型区域(被积分函数的形式是否有利于柱坐标计算$f(x, y, z)=\varphi(z) g\left(x^{2}+y^{2}\right)$)



\subsection{球坐标}\index{坐标代换!球坐标}

1、球坐标变换:$\begin{cases}x=r \sin \varphi \cos \theta, & 0 \leqslant r<+\infty \\ y=r \sin \varphi \sin \theta, & 0 \leqslant \varphi \leqslant \pi \\ z=r \cos \varphi, & 0 \leqslant \theta \leqslant 2 \pi\end{cases}$

2、球坐标微元:$\mathrm{d} V=r^{2} \sin \varphi \mathrm{drd} \varphi \mathrm{d} \theta$

3、球坐标积分:$\iiint_{\Omega} f(x, y, z) \mathrm{d} V=\iiint_{\Omega} f(r \sin \varphi \cos \theta, r \sin \varphi \sin \theta, r \cos \varphi) r^{2} \sin \varphi \mathrm{drd} \varphi \mathrm{d} \theta$

\section{空间积分}\index{空间积分}



\subsection{核心关系}\index{空间积分!核心关系}

1、第一第二线积分关系:$\int_{L} P \mathrm{~d} x+Q \mathrm{~d} y=\int_{L}(P \cos \alpha+Q \cos \beta) \mathrm{ds}$,其中$\cos \alpha, \cos \beta$为有向曲线$L$的切线的方向余弦

2、第一第二面积分关系::$\iint_{\Sigma} P \mathrm{~d} y \mathrm{~d} z+Q \mathrm{~d} z \mathrm{~d} x+R \mathrm{~d} x \mathrm{~d} y=\iint_{\Sigma}(P \cos \alpha+Q \cos \beta+R \cos \gamma) \mathrm{d} S$,其中$\cos \alpha, \cos \beta, \cos \gamma$为曲面$\Sigma$上点$P(x, y, z)$处指定侧的法线向量的方向余弦



\subsection{第一类二维曲线积分}\index{空间积分!第一类二维曲线积分}

1、参数方程:$\int_{L} f(x, y) \mathrm{d} s=\int_{a}^{\beta} f(x(t), y(t)) \sqrt{x^{\prime 2}(t)+y^{\prime 2}(t)} \mathrm{d} t$

2、直角坐标方程:$\int_{L} f(x, y) \mathrm{d} s=\int_{a}^{b} f(x, y(x)) \sqrt{1+y^{\prime 2}(x)} \mathrm{d} x$

3、极坐标方程:$\int_{L} f(x, y) \mathrm{d} s=\int_{a}^{\beta} f(r(\theta) \cos \theta, r(\theta) \sin \theta) \sqrt{r^{2}+r^{\prime 2}} \mathrm{~d} \theta$



\subsection{第一类三维曲线积分}\index{空间积分!第一类三维曲线积分}

1、转参数方程:$ds$化为$dt$,关系为$ds = \sqrt{x^{\prime 2}(t)+y^{\prime 2}(t)+z^{\prime 2}(t)} \mathrm{d} t$)



\subsection{第二类曲线积分:(二维)}\index{空间积分!第二类曲线积分:(二维)}

1、参数式:$\int_{L} P \mathrm{~d} x+Q \mathrm{~d} y=\int_{a}^{\beta}\left[P(x(t), y(t)) x^{\prime}(t)+Q(x(t), y(t)) y^{\prime}(t)\right] \mathrm{d} t$

2、格林公式:$\oint_{L} P \mathrm{~d} x+Q \mathrm{~d} y=\iint_{D}\left(\frac{\partial Q}{\partial x}-\frac{\partial P}{\partial y}\right) \mathrm{d} x \mathrm{~d} y$

3、路径无关:$\frac{\partial P}{\partial y}=\frac{\partial Q}{\partial x}, \forall(x, y) \in D$



\subsection{第二类曲线积分:(三维)}\index{空间积分!第二类曲线积分:(三维)}

1、参数式法:曲线$L:\left\{\begin{array}{l}x=x(t) \\ y=y(t), t \in[\alpha, \beta] \text { 或 } t \in[\beta, \alpha]\\ z=z(t),\end{array}\right.$,则$\int_{L} P \mathrm{~d} x+Q \mathrm{~d} y+R \mathrm{~d} z= \int_{a}^{\beta}\left[P(x(t), y(t), z(t)) x^{\prime}(t)+Q(x(t), y(t), z(t)) y^{\prime}(t)+R(x(t), y(t), z(t)) z^{\prime}(t)\right] \mathrm{d} t$(计算简单,不容易出错)

2、斯托克斯公式法:$\oint_{\Gamma} P \mathrm{~d} x+Q \mathrm{~d} y+R \mathrm{~d} z=\iint_{\Sigma}\left(\frac{\partial R}{\partial y}-\frac{\partial Q}{\partial z}\right) \mathrm{d} y \mathrm{~d} z+\left(\frac{\partial P}{\partial z}-\frac{\partial R}{\partial x}\right) \mathrm{d} z \mathrm{~d} x+\left(\frac{\partial Q}{\partial x}-\frac{\partial P}{\partial y}\right) \mathrm{d} x \mathrm{~d} y$(也可以记为$\iint_{\Sigma}\left|\begin{array}{ccc}\cos \alpha & \cos \beta & \cos \gamma \\\frac{\partial}{\partial x} & \frac{\partial}{\partial y} & \frac{\partial}{\partial z} \\P & Q & R\end{array}\right| d S=\oint_{\Gamma} P d x+Q d y+R d z$),$\Gamma$的方向与$\Sigma$的方向符合右手法则(正负符号是在曲面法线向量中确定的)



\subsection{第一类曲面积分}\index{空间积分!第一类曲面积分}

1、公式法:$\iint_{\Sigma} f(x, y, z) \mathrm{d} S=\iint_{D} f(x, y, z(x, y)) \sqrt{1+\left(z_{x}^{\prime}\right)^{2}+\left(z_{y}^{\prime}\right)^{2}} \mathrm{~d} x \mathrm{~d} y$



\subsection{第二类曲面积分}\index{空间积分!第二类曲面积分}

1、高斯公式:$\oiint_{\Sigma} P \mathrm{~d} y \mathrm{~d} z+Q \mathrm{~d} z \mathrm{~d} x+R \mathrm{~d} x \mathrm{~d} y=\iiint_{\Omega}\left(\frac{\partial P}{\partial x}+\frac{\partial Q}{\partial y}+\frac{\partial R}{\partial z}\right) \mathrm{d} V$,闭曲面$\Sigma$取外侧(一定要注意区域内是否有不连续的点<分母为0等>!!!)

4、直接法(一次性):$\iint_{\Sigma} P(x, y, z) \mathrm{d} y \mathrm{~d} z+Q(x, y, z) \mathrm{d} z \mathrm{~d} x+R(x, y, z) \mathrm{d} x \mathrm{~d} y = \pm \iint_{D_{x y}}[P(x, y, z(x, y))\left(-\frac{\partial z}{\partial x}\right)+Q(x, y, z(x, y))\left(-\frac{\partial z}{\partial y}\right)+R(x, y, z(x, y))] \mathrm{d} x \mathrm{~d} y$,(注意曲面的方向)若有向曲面$\Sigma$的法向量与$z$轴正向夹角为锐角,即上侧,上式中取“+" 号,否则取“-"号

\section{偏导数与全微分}\index{偏导数与全微分}

1、偏导数: $f_{x}^{\prime}\left(x_{0}, y_{0}\right)=\lim_{\Delta x \rightarrow 0} \frac{f\left(x{0}+\Delta x, y_{0}\right)-f\left(x_{0}, y_{0}\right)}{\Delta x}$,;类似地可定义$f_{y}^{\prime}\left(x_{0}, y_{0}\right)=\lim_{\Delta y \rightarrow 0} \frac{f\left(x{0}, y_{0}+\Delta y\right)-f\left(x_{0}, y_{0}\right)}{\Delta y}$(从几何上看就是曲面与$x=x_0,y=y_0$平面的交线在交线方向上的导数)

2、全微分:$\Delta z=f(x+\Delta x, y+\Delta y)-f(x, y)=A \Delta x+B \Delta y+o(\rho)$,$\rho=\sqrt{(\Delta x)^{2}+(\Delta y)^{2}}$,微分记为$\mathrm{d} z=A \Delta x+B \Delta y$

3、微分表示:$\mathrm{d} z=\frac{\partial z}{\partial x} \mathrm{~d} x+\frac{\partial z}{\partial y} \mathrm{~d} y$(求解微分)(偏导数与微分关系)

4、可微定义式:$\lim_{\Delta x \rightarrow 0,\Delta y \rightarrow 0}\frac{\left[f\left(x_{0}+\Delta x, y_{0}+\Delta y\right)-f\left(x_{0}, y_{0}\right)\right]-\left[f_{x}^{\prime}\left(x_{0}, y_{0}\right) \Delta x+f_{y}^{\prime}\left(x_{0}, y_{0}\right) \Delta y\right]}{\rho}=0$,其中$\rho=\sqrt{(\Delta x)^{2}+(\Delta y)^{2}}$

5、可微定义式(偏导为0的特殊情况):$\lim_{\Delta x \rightarrow 0,\Delta y \rightarrow 0}\frac{f\left(x_{0}+\Delta x, y_{0}+\Delta y\right)-f\left(x_{0}, y_{0}\right)}{\rho}=0$,其中$\rho=\sqrt{(\Delta x)^{2}+(\Delta y)^{2}}$

\section{隐函数的偏导数}\index{隐函数的偏导数}

1、一个方程(一元函数):$F(x, y)=0$$\frac{\mathrm{d} y}{\mathrm{~d} x}=-\frac{F_{x}^{\prime}}{F_{y}^{\prime}}$

2、一个方程(二元函数):$F(x, y, z)=0$,$\frac{\partial z}{\partial x}=-\frac{F_{x}^{\prime}}{F_{z}^{\prime}}, \quad \frac{\partial z}{\partial y}=-\frac{F_{y}^{\prime}}{F_{z}^{\prime}}$

3、方程组(一元函数):方程组$\left\{\begin{array}{l}F(x, u, v)=0,  \\ G(x, u, v)=0\end{array}\right. $,对x求偏导,$\left\{\begin{array}{l} F_{x}^{\prime}+F_{u}^{\prime} \frac{\mathrm{d} u}{\mathrm{~d} x}+F_{v}^{\prime} \frac{\mathrm{d} v}{\mathrm{~d} x}=0 \\ G_{x}^{\prime}+G_{u}^{\prime} \frac{\mathrm{d} u}{\mathrm{~d} x}+G_{v}^{\prime} \frac{\mathrm{d} v}{\mathrm{~d} x}=0 \end{array}\right.$($u=u(x), v=v(x)$)

3、方程组(二元函数):方程组$\left\{\begin{array}{l}F(x, y, u, v)=0, \\ G(x, y, u, v)=0\end{array}\right.$,对$x$求偏导得到, 即$\left\{\begin{array}{l} F_{x}^{\prime}+F_{u}^{\prime} \frac{\partial u}{\partial x}+F_{v}^{\prime} \frac{\partial v}{\partial x}=0 \\ G_{x}^{\prime}+G_{u}^{\prime} \frac{\partial u}{\partial x}+G_{v}^{\prime} \frac{\partial v}{\partial x}=0 \end{array}\right.$,然后从中解出$\frac{\partial u}{\partial x}$和$\frac{\partial v}{\partial x}$;同理可求得$\frac{\partial u}{\partial y}$和$\frac{\partial v}{\partial y}$

\section{二元极值}\index{二元极值}



\subsection{二元极值的判断条件:(充分条件)}\index{二元极值!二元极值的判断条件:(充分条件)}

设函数$z=f(x, y)$在点$\left(x_{0}, y_{0}\right)$的某邻域内有连续的二阶偏导数,且$f_{x}^{\prime}\left(x_{0}, y_{0}\right)=0, f_{y}^{\prime}\left(x_{0}, y_{0}\right)=0$。令$f_{x x}^{\prime \prime}\left(x_{0}, y_{0}\right)=A, f_{x y}^{\prime \prime}\left(x_{0}, y_{0}\right)=B, f_{y y}^{\prime \prime}\left(x_{0}, y_{0}\right)=C$,则

(1)$A C-B^{2}>0$时,$f(x, y)$在点$\left(x_{0}, y_{0}\right)$取极值,且$\left\{\begin{array}{l}\text { 当 } A>0 \text { 时取极小值, } \\ \text { 当 } A<0 \text { 时取极大值. }\end{array}\right.$

(2)$A C-B^{2}<0$时,$f(x, y)$在点$\left(x_{0}, y_{0}\right)$无极值

(3)$A C-B^{2}=0$时,不能确定$f(x, y)$在点$\left(x_{0}, y_{0}\right)$是否有极值,还需进一步讨论(一般用 极值定义)



\subsection{条件极值求解-拉格朗日乘数法:(多个条件的也是类似的)}\index{二元极值!条件极值求解-拉格朗日乘数法:(多个条件的也是类似的)}

1、函数$f(x, y)$在条件$\varphi(x, y)=0$下的极值的必要条件,先构造拉格朗日函数$F(x, y, \lambda)=f(x, y)+\lambda \varphi(x, y)$

2、解方程组$\left\{\begin{array}{l} \frac{\partial F}{\partial x}=\frac{\partial f}{\partial x}+\lambda \frac{\partial \varphi}{\partial x}=0 \\ \frac{\partial F}{\partial y}=\frac{\partial f}{\partial y}+\lambda \frac{\partial \varphi}{\partial y}=0 \\ \frac{\partial F}{\partial \lambda}=\varphi(x, y)=0 \end{array}\right.$

3、所有满足此方程组的解$(x, y, \lambda)$中$(x, y)$是函数$f(x, y)$在条件$\varphi(x, y)=0$下可能的极值点

\section{空间特征}\index{空间特征}

1、曲线$\Gamma$表示为$\left\{\begin{array}{l}x=x(t), \\ y=y(t),  \\ z=z(t),\end{array}\right.$的切向量:$\tau=\left\{x^{\prime}\left(t_{0}\right), y^{\prime}\left(t_{0}\right), z^{\prime}\left(t_{0}\right)\right\}$

2、曲线$\Gamma$表示为$\left\{\begin{array}{l}F(x, y, z)=0,\\ G(x, y, z)=0,\end{array}\right.$的切向量: $\tau=\boldsymbol{n}_{1} \times \boldsymbol{n}_{2}$,其中$\newline\begin{aligned} &\boldsymbol{n}_{1}=\left\{F{x}^{\prime}\left(x_{0}, y_{0}, z_{0}\right), F_{y}^{\prime}\left(x_{0}, y_{0}, z_{0}\right), F_{z}^{\prime}\left(x_{0}, y_{0}, z_{0}\right)\right\} \\ &\boldsymbol{n}_{2}=\left\{G_{x}^{\prime}\left(x_{0}, y_{0}, z_{0}\right), G_{y}^{\prime}\left(x_{0}, y_{0}, z_{0}\right), G_{z}^{\prime}\left(x_{0}, y_{0}, z_{0}\right)\right\} \end{aligned}$

3、曲面$F(x, y, z)=0$法向量:$\boldsymbol{n}=\left\{F_{x}^{\prime}\left(x_{0}, y_{0}, z_{0}\right), F_{y}^{\prime}\left(x_{0}, y_{0}, z_{0}\right), F_{z}^{\prime}\left(x_{0}, y_{0}, z_{0}\right)\right\}$

4、曲面$z=f(x, y)$法向量:$\boldsymbol{n}=\left\{f_{x}^{\prime}\left(x_{0}, y_{0}\right),f_{y}^{\prime}\left(x_{0}, y_{0}\right),-1\right\}$

\section{泰勒}\index{泰勒}



\subsection{泰勒与级数}\index{泰勒!泰勒与级数}

1、关系:幂级数展开式$\varphi(x)=\sum_{n=0}^{\infty} a_{n}\left(x-x_{0}\right)^{n}$,泰勒级数展开$\varphi(x)=\sum_{n=0}^{\infty} \frac{1}{n !} \varphi^{(n)}\left(x_{0}\right)\left(x-x_{0}\right)^{n}$,对应关系$\varphi^{(n)}\left(x_{0}\right)=n ! a_{n}, \quad(n=0,1, \cdots)$



\subsection{基本没用的}\index{泰勒!基本没用的}

1、拉格朗日余项泰勒展开式:$f(x, y)=f\left(x_{0}, y_{0}\right)+f_{x}^{\prime}\left(x_{0}, y_{0}\right)\left(x-x_{0}\right)+f_{y}^{\prime}\left(x_{0}, y_{0}\right)\left(y-y_{0}\right)+R_{1}$,其中余项$R_{1}=\frac{1}{2 !}\left[\frac{\partial^{2} f\left(P_{1}\right)}{\partial x^{2}}\left(x-x_{0}\right)^{2}+2 \frac{\partial^{2} f\left(P_{1}\right)}{\partial x \partial y}\left(x-x_{0}\right)\left(y-y_{0}\right)+\frac{\partial^{2} f\left(P_{1}\right)}{\partial y^{2}}\left(y-y_{0}\right)^{2}\right]$,点$P_{1}$为$\left(x_{0}+\theta\left(x-x_{0}\right), y_{0}+\theta\left(y-y_{0}\right)\right)$

2、佩亚诺型余 项的二阶泰勒公式展开式:$f(x, y)=f\left(x_{0}, y_{0}\right)+f_{x}^{\prime}\left(x_{0}, y_{0}\right)\left(x-x_{0}\right)+f_{y}^{\prime}\left(x_{0}, y_{0}\right)\left(y-y_{0}\right)+\frac{1}{2 !}[\frac{\partial^{2} f\left(x_{0}, y_{0}\right)}{\partial x^{2}}\left(x-x_{0}\right)^{2}+2 \frac{\partial^{2} f\left(x_{0}, y_{0}\right)}{\partial x \partial y}\left(x-x_{0}\right)\left(y-y_{0}\right)+\frac{\partial^{2} f\left(x_{0}, y_{0}\right)}{\partial y^{2}}\left(y-y_{0}\right)^{2}]+o\left(\rho^{2}\right)$,其中$\rho=\sqrt{\left(x-x_{0}\right)^{2}+\left(y-y_{0}\right)^{2}}$

\section{积分公式}\index{积分公式}



\subsection{基本积分公式}\index{积分公式!基本积分公式}

1、$\int a^{x} \mathrm{~d} x=\frac{a^{x}}{\ln a}+C(a>0, a \neq 1)$

2、$\int \tan x \mathrm{~d} x=-\ln |\cos x|+C$

3、$\int \cot x \mathrm{~d} x=\ln |\sin x|+C$

4、$\int \csc x \mathrm{~d} x=\ln |\csc x-\cot x|+C$

5、$\int \sec x \mathrm{~d} x=\ln |\sec x+\tan x|+C$

6、$\int \sec ^{2} x \mathrm{~d} x=\tan x+C$

7、$\int \csc ^{2} x \mathrm{~d} x=-\cot x+C$

8、$\int \frac{1}{a^{2}+x^{2}} \mathrm{~d} x=\frac{1}{a} \arctan \frac{x}{a}+C$

9、$\int \frac{1}{a^{2}-x^{2}} \mathrm{~d} x=\frac{1}{2 a} \ln \mid \frac{a+x}{a-x} \mid+C$

10、$\int \frac{1}{\sqrt{a^{2}-x^{2}}} \mathrm{~d} x=\arcsin \frac{x}{a}+C$

11、$\int \frac{\mathrm{d} x}{\sqrt{x^{2} \pm a^{2}}}=\ln \mid x+\sqrt{x^{2} \pm a^{2}} \mid+ C$



\subsection{常见的换元法}\index{积分公式!常见的换元法}

1、$\int R\left(x, \sqrt{a^{2}-x^{2}}\right) \mathrm{d} x, \int R\left(x, \sqrt{x^{2} \pm a^{2}}\right) \mathrm{d} x \text { 型, } a>0$(根号型的处理,或者相似的,根号中配成平方之后的)

1.1、$\sqrt{a^{2}-x^{2}}$型,令$x=a \sin t, \mathrm{~d} x=a \cos t \mathrm{~d} t$

1.2、$\sqrt{x^{2}+a^{2}}$型,令$x=a \tan t, \mathrm{~d} x=a \sec ^{2} t \mathrm{~d} t$

1.3、$\sqrt{x^{2}-a^{2}}$型,令$x=a \sec t, \mathrm{~d} x=a \sec t \tan t \mathrm{~d} t$

2、$\int R(x, \sqrt[n]{a x+b}, \sqrt[m]{a x+b}) \mathrm{d} x$型,令$\sqrt[n n]{a x+b}=t, x=\frac{t^{m n}-b}{a}, \mathrm{~d} x=\frac{m n}{a} t^{m n-1} \mathrm{~d} t$(分数幂,无分数)

3、$\int R\left(x, \sqrt{\frac{a x+b}{c x+d}}\right) \mathrm{d} x$型,令$\sqrt{\frac{a x+b}{c x+d}}=t, x=\frac{d t^{2}-b}{a-c t^{2}}, \mathrm{~d} x=\frac{2(a d-b c) t}{\left(a-c t^{2}\right)^{2}} \mathrm{~d} t .$其中设$a d-b c \neq 0$(分数幂,有分数)

4、$\int R(\sin x, \cos x) \mathrm{d} x$型,令$\tan({\frac{x}{2}})=t$,则$\sin x=\frac{2 t}{1+t^{2}}, \cos x=\frac{1}{1+t^{2}}, \mathrm{~d} x=\frac{2}{1+t^{2}} \mathrm{~d} t$(万能代换)



\subsection{其它定积分公式}\index{积分公式!其它定积分公式}

1、周期函数的积分:$\int_{a}^{a+T} f(x) \mathrm{d} x=\int_{0}^{T} f(x) \mathrm{d} x$

2、华里士公式:$\int_{0}^{\frac{\pi}{2}} \sin ^{n} x \mathrm{~d} x=\int_{0}^{\frac{\pi}{2}} \cos ^{n} x \mathrm{~d} x=\left\{\begin{array}{l}\frac{n-1}{n} \cdot \frac{n-3}{n-2} \cdots \cdots \frac{1}{2} \cdot \frac{\pi}{2}, \quad \text { 当 } n \text { 为正偶数 } \\\frac{n-1}{n} \cdot \frac{n-3}{n-2} \cdots \cdots \cdot \frac{2}{3} \cdot 1, \quad \text { 当 } n \text { 为大于 } 1 \text { 的正奇数 }\end{array}\right.$



\subsection{分部积分:(不定积分和定积分)(重点)}\index{积分公式!分部积分:(不定积分和定积分)(重点)}

1、不定积分:$\int u(x) v^{\prime}(x) \mathrm{d} x=u(x) v(x)-\int v(x) u^{\prime}(x) \mathrm{d} x$或者$\int u(x) \mathrm{d} v(x)=u(x) v(x)-\int v(x) \mathrm{d} u(x)$

2、定积分:$\int_{a}^{b} u(x) v^{\prime}(x) \mathrm{d} x=\left.u(x) v(x)\right|_{a} ^{b}-\int_{a}^{b} v(x) u^{\prime}(x) \mathrm{d} x$或者$\int_{a}^{b} u(x) \mathrm{d} v(x)=\left.u(x) v(x)\right|_{a} ^{b}-\int_{a}^{b} v(x) \mathrm{d} u(x)$



\subsection{用分部积分法的题型}\index{积分公式!用分部积分法的题型}

1、幂与指数,三角函数相乘的:$\int x^{n} \mathrm{e}^{x} \mathrm{~d} x=\int x^{n} \mathrm{~d} \mathrm{e}^{x}$,$\int x^{n} \sin x \mathrm{~d} x=-\int x^{n} \mathrm{~d} \cos x$,$\int x^{n} \cos x \mathrm{~d} x=\int x^{n} \mathrm{~d} \sin x$

2、幂与对数、反三角函数相乘的:$\int x^{n} \ln x \mathrm{~d} x=\frac{1}{n+1} \int \ln x \mathrm{~d} x^{n+1}$,$\int x^{n} \arctan x \mathrm{~d} x=\frac{1}{n+1} \int \arctan x \mathrm{~d} x^{n+1}$,$\int x^{n} \arcsin x \mathrm{~d} x=\frac{1}{n+1} \int \arcsin x \mathrm{~d} x^{n+1}$

3、连用两次的(指数与三角相乘):$\int \mathrm{e}^{x} \sin x \mathrm{~d} x$,$\int \mathrm{e}^{x} \cos x \mathrm{~d} x$

\section{反常积分}\index{反常积分}

1、正态分布:$\int_{-\infty}^{+\infty} \mathrm{e}^{-x^{2}} \mathrm{~d} x=2 \int_{0}^{+\infty} \mathrm{e}^{-x^{2}} \mathrm{~d} x=\sqrt{\pi}$

2、幂与对数:设$a$与$p$都是常数,且$a>1$则$\int_{a}^{+\infty} \frac{\mathrm{d} x}{x \ln ^{p} x} \begin{cases}\text { 收敛, } & \text { 当 } p>1 \\ \text { 发散, } & \text { 当 } p \leqslant 1\end{cases}$(直接积分即可证明)

\section{积分应用}\index{积分应用}

1、平面图形的面积:极坐标系下方程$r=r(\theta)$,在两射线$\theta=\alpha$与$\theta=\beta(0<\beta-\alpha \leqslant 2 \pi)$之间面积为$A=\frac{1}{2} \int_{a}^{\beta} r^{2}(\theta) \mathrm{d} \theta$

2、平面曲线的弧长:参数坐标系下的弧长为$s=\int_{a}^{\beta} \sqrt{x^{\prime 2}(t)+y^{\prime 2}(t)} \mathrm{d} t$;直角坐标系下的弧长为$s=\int_{a}^{b} \sqrt{1+y^{\prime 2}(x)} \mathrm{d} x$(弧长上一小段微元的长度为$\sqrt{1+y^{\prime 2}(x)} \mathrm{d} x$);极坐标系下的弧长为$s=\int_{a}^{\beta} \sqrt{r^{2}(\theta)+r^{\prime 2}(\theta)} \mathrm{d} \theta$(注意:弧长的计算有时需要用到弧线的对称性,否则计算结果为0)

3、旋转体的体积:曲线$y=y(x)$与$x=a, x=b, x$轴围成的曲边梯形绕$x$轴旋转一周所围成的体积为$V=\pi \int_{a}^{b} y^{2}(x) \mathrm{d} x, a<b$(圆面的微元,圆面积乘厚度微元);曲线$y=y_{2}(x), y=y_{1}(x), x=a, x=b\left(y_{2}(x) \geqslant y_{1}(x) \geqslant 0\right)$绕$x$轴旋转围成的旋转体体积为$V=\pi \int_{a}^{b}\left[y_{2}^{2}(x)-y_{1}^{2}(x)\right] \mathrm{d} x, a<b$(两圆相减的面的微元,两圆面相减的面积乘厚度微元);曲线$y=y_{2}(x), y=y_{1}(x), x=a, x=b\left(b>a \geqslant 0, y_{2}(x) \geqslant y_{1}(x)\right)$绕$y$轴旋转所成的旋转体体积为$V=2 \pi \int_{a}^{b} x\left(y_{2}(x)-y_{1}(x)\right) \mathrm{d} x$(圆筒状的微元,周长乘高乘厚度微元)

4、旋转曲面面积:在直角坐标系上曲线$y=y(x)$绕$x$轴旋转,所形成的曲面面积为$S=2 \pi \int_{a}^{b}|y| \sqrt{1+f^{\prime 2}(x)} \mathrm{d} x, a<b$(圆周与倾斜度的微元,对曲面纵向切分,先求解圆环(注意圆环是带斜率的,所以圆环宽度微元为$\sqrt{1+f^{\prime 2}(x)} \mathrm{d} x$),圆周乘圆环宽度微元);在参数坐标系下有$S=2 \pi \int_{a}^{\beta}|y(t)| \sqrt{x^{\prime 2}(t)+y^{\prime 2}(t)} \mathrm{d} t$

5、在区间上平行截面面积为已知$A(x)$的立体体积:$V=\int_{a}^{b} A(x) \mathrm{d} x, a<b$

6、函数的平均值:$\bar{f}=\frac{1}{b-a} \int_{a}^{b} f(x) \mathrm{d} x$

7、质心:平面:$\bar{x}=\frac{\iint_{D} x \rho(x, y) \mathrm{d} x \mathrm{~d} y}{\iint_{D} \rho(x, y) \mathrm{d} x \mathrm{~d} y}$;空间体:$\bar{x}=\frac{\iiint_{\Omega} x \rho(x, y, z) \mathrm{d} v}{\iiint_{\Omega} \rho(x, y, z) \mathrm{d} v}$;曲线:$\bar{x}=\frac{\int_{C} x \rho(x, y, z) \mathrm{d} s}{\int_{C} \rho(x, y, z) \mathrm{d} s}$;曲面:$\bar{x}=\frac{\iint_{\Sigma} x \rho(x, y, z) \mathrm{d} S}{\iint_{\Sigma} \rho(x, y, z) \mathrm{d} S}$(几何体上质量点的矩与质心矩的关系)

8、转动惯量:平面:$I_{x}=\iint_{D} y^{2} \rho(x, y) \mathrm{d} \sigma$;空间体:$I_{x}=\iiint_{\Omega}\left(y^{2}+z^{2}\right) \cdot \rho(x, y, z) \mathrm{d} v$;曲线:$I_{x}=\int_{C}\left(y^{2}+z^{2}\right) \cdot \rho(x, y, z) \mathrm{d} s$;曲面:$I_{x}=\iint_{\Sigma}\left(y^{2}+z^{2}\right) \cdot \rho(x, y, z) \mathrm{d} S$(注意对应轴的转动惯量的转动轴)

9、质量:平面:$m=\iint_{D} \rho(x, y) \mathrm{d} x \mathrm{~d} y$;空间体:$m=\iiint_{\Omega} \rho(x, y, z) \mathrm{d} v$;曲线:$m=\int_{C} \rho(x, y, z) \mathrm{d} s$;曲面:$m=\iint_{\Sigma} \rho(x, y, z) \mathrm{d} S$

10、几何度量:平面面积:$S=\iint_{D} \mathrm{~d} x \mathrm{~d} y$;空间体体积:$V=\iiint_{\Omega} \mathrm{d} v$;曲弧长线:$L=\int_{C} \mathrm{~d} s$;曲面面积:$S=\iint_{\Sigma} \mathrm{d} S$

11、变力做功:设有力场$\boldsymbol{F}(x, y, z)=P \boldsymbol{i}+Q \boldsymbol{j}+R \boldsymbol{k}$,则力$F$沿曲线$\widehat{A B}$从$A$到$B$所做的功为$W=\int_{\widehat{A B}} P \mathrm{~d} x+Q \mathrm{~d} y+R \mathrm{~d} z$(做功的求解在于将力场进行分解,对每一个方向的功分开计算)

12、通量:设有向量场$\boldsymbol{A}(x, y, z)=P(x, y, z) \boldsymbol{i}+Q(x, y, z) \boldsymbol{j}+R(x, y, z) \boldsymbol{k}, S$为有向曲面且$\mathrm{d} \boldsymbol{S}=\mathrm{d} y \mathrm{~d} x \boldsymbol{i}+\mathrm{d} z \mathrm{~d} x \boldsymbol{j}+\mathrm{d} x \mathrm{~d} y \boldsymbol{k}$, 则向量场$\boldsymbol{A}(x, y, z)$穿过曲面$S$的指定侧的通量为$\Phi=\iint_{\Sigma} P \mathrm{~d} y \mathrm{~d} z+Q \mathrm{~d} z \mathrm{~d} x+R \mathrm{~d} x \mathrm{~d} y$(可理解为向量场在各个方向上的分量,在该方向上曲面的投影面积的积分)

13、液体压强公式:一点的压强为$p=\rho gh$

\section{方向导数、梯度、散度、旋度}\index{方向导数、梯度、散度、旋度}



\subsection{方向导数:(数值)}\index{方向导数、梯度、散度、旋度!方向导数:(数值)}

0、方向导数定义:二维:$\lim _{t \rightarrow 0^{+}} \frac{f\left(x_{0}+t \cos \alpha, y_{0}+t \cos \beta\right)-f\left(x_{0}, y_{0}\right)}{t}$;三维::$\left.\frac{\partial f}{\partial l}\right|_{\left(x_{0}, y_{0}, z_{0}\right)}$=$\newline\lim _{t \rightarrow 0^{+}} \frac{f\left(x_{0}+t \cos \alpha, y_{0}+t \cos \beta, z_{0}+t \cos \gamma\right)-f\left(x_{0}, y_{0}, z_{0}\right)}{t}$(注意$t$是趋近于$0^+$的)

1、方向导数公式(可微):二维:$\frac{\partial f}{\partial l}|_{\left(x_{0}, y_{0}\right)}= f_{x}^{\prime}(x_{0},y_{0}) \cos \alpha+f_{y}^{\prime}\left(x_{0}, y_{0}\right) \cos \beta$,其中$\cos \alpha, \cos \beta$为方向$l$的方向余弦;三维:$\frac{d f}{\partial l}|{\left(x_{0}, y_{0}, z_{0}\right)} =f_{x}^{\prime}\left(x_{0}, y_{0}, z_{0}\right) \cos \alpha+f_{y}^{\prime}\left(x_{0}, y_{0}, z_{0}\right) \cos \beta+f_{z}^{\prime}\left(x_{0}, y_{0}, z_{0}\right) \cos \gamma =\left\{f_{x}^{\prime}\left(x_{0}, y_{0}, z_{0}\right), f_{y}^{\prime}\left(x_{0}, y_{0}, z_{0}\right), f_{z}^{\prime}\left(x_{0}, y_{0}, z_{0}\right)\right\} \cdot e$,其中$e=\{\cos \alpha, \cos \beta, \cos \gamma\}$是$l$的方向的单位向量

2、平面曲线法向量求导:$\frac{\partial u}{\partial \boldsymbol{n}} \mathrm{d} s=\frac{\partial u}{\partial x} \cos (\boldsymbol{n}, x) \mathrm{d} s+\frac{\partial u}{\partial y} \cos (\boldsymbol{n}, y) \mathrm{d} s$,其中$\cos (\boldsymbol{n}, x) ,\cos (\boldsymbol{n}, y)$是法向量对$x$轴和$y$轴的方向余弦;

3、三维曲面法向量求导:$\oiint_{\Sigma} \frac{\partial u}{\partial n} \mathrm{~d} S=\oiint_{\Sigma}\left(\frac{\partial u}{\partial x} \cos \alpha+\frac{\partial u}{\partial y} \cos \beta+\frac{\partial u}{\partial z} \cos \gamma\right) \mathrm{d} S$



\subsection{梯度:(向量)}\index{方向导数、梯度、散度、旋度!梯度:(向量)}

1、二维梯度公式:$\operatorname{grad} u(x, y)=\frac{\partial u}{\partial x} i+\frac{\partial u}{\partial y} j$

2、三维梯度公式:$\operatorname{grad} u(x, y, z)=\frac{\partial u}{\partial x} i+\frac{\partial u}{\partial y} j+\frac{\partial u}{\partial z} k$

3、方向导数与梯度的关系:$\left.\frac{\partial u}{\partial{l}}\right|_{P}=\left.\operatorname{grad} u\right|_{P} \cdot \boldsymbol{e}_{l}=|\operatorname{grad} u(x_0, y_0, z_0) |cos \theta$,其中$\theta$是梯度方向与$e_l$的夹角(梯度方向的方向导数最大,最大值等于该点梯度向量的模)



\subsection{散度:(数值)}\index{方向导数、梯度、散度、旋度!散度:(数值)}

1、散度公式:设有向量场$\boldsymbol{A}(x, y, z)=P(x, y, z) \boldsymbol{i}+Q(x, y, z) \boldsymbol{j}+R(x, y, z) \boldsymbol{k}$,其中$P,Q,R$有连续的一阶偏导,则向量场在一点的散度为$\operatorname{div} \boldsymbol{A}=\frac{\partial P}{\partial x}+\frac{\partial Q}{\partial y}+\frac{\partial R}{\partial z}$



\subsection{旋度:(向量)}\index{方向导数、梯度、散度、旋度!旋度:(向量)}

1、旋度公式:设有向量场$\boldsymbol{A}(x, y, z)=P(x, y, z) \boldsymbol{i}+Q(x, y, z) \boldsymbol{j}+R(x, y, z) \boldsymbol{k}$,其中$P,Q,R$有连续的一阶偏导,则向量场在一点的旋度为$\operatorname{rot} \boldsymbol{A}=\left|\begin{array}{ccc}\boldsymbol{i} & \boldsymbol{j} & \boldsymbol{k} \\\frac{\partial}{\partial x} & \frac{\partial}{\partial y} & \frac{\partial}{\partial z} \\P & Q & R\end{array}\right|$

\section{导数恒等式变换(构造)}\index{导数恒等式变换(构造)}

1、$\xi f^{\prime}(\xi)+f(\xi)=0$:$[x f(x)]^{\prime}=x f^{\prime}(x)+f(x)$

2、$\xi f^{\prime}(\xi)+n f(\xi)=0$:$\left[x^{n} f(x)\right]^{\prime}=x^{n-1}\left[x f^{\prime}(x)+n f(x)\right]$

3、$\xi f^{\prime}(\xi)-f(\xi)=0$:$\left[\frac{f(x)}{x}\right]^{\prime}=\frac{x f^{\prime}(x)-f(x)}{x^{2}}$

4、$\xi f^{\prime}(\xi)-n f(\xi)=0$:$\left[\frac{f(x)}{x^{n}}\right]^{\prime}=\frac{x f^{\prime}(x)-n f(x)}{x^{n+1}}$

5、$f^{\prime}(\xi)+\lambda f(\xi)=0$:$\left[\mathrm{e}^{\lambda x} f(x)\right]^{\prime}=\mathrm{e}^{\lambda x}\left[f^{\prime}(x)+\lambda f(x)\right]$

6、$f^{\prime}(\xi)-\lambda f(\xi)=0$:$\left[\mathrm{e}^{-\lambda x} f(x)\right]^{\prime}=\mathrm{e}^{-\lambda x}\left[f^{\prime}(x)-\lambda f(x)\right]$

7、$f(\xi) f^{\prime \prime}(\xi)+\left[f^{\prime}(\xi)\right]^{2}=0$:$\left[f(x) f^{\prime}(x)\right]^{\prime}=f(x) f^{\prime \prime}(x)+\left[f^{\prime}(x)\right]^{2}$

\section{曲率公式}\index{曲率公式}

1、曲率计算公式(注意:曲率与二阶导符号无关):$k=\frac{\left|y^{\prime \prime}\right|}{\left(1+y^{\prime 2}\right)^{3 / 2}}$

2、曲率半径:$R=\frac{1}{k}=\frac{\left(1+y^{\prime 2}\right)^{3 / 2}}{\left|y^{\prime \prime}\right|}$

3、曲率中心坐标:$(\alpha=x-\frac{y^{\prime}\left(1+y^{\prime 2}\right)}{y^{\prime \prime}}, \beta=y+\frac{1+y^{\prime 2}}{y^{\prime \prime}})$

\section{向量相关}\index{向量相关}

1、向量积(叉积、外积):$\boldsymbol{a} \times \boldsymbol{b}=\left|\begin{array}{ccc}\boldsymbol{i} & \boldsymbol{j} & \boldsymbol{k} \\a_{x} & a_{y} & a_{z} \\b_{x} & b_{y} & b_{z}\end{array}\right|$,模$|\boldsymbol{c}|=|\boldsymbol{a}||\boldsymbol{b}| \sin \theta$,其中$\theta$为$\boldsymbol{a}$与$\boldsymbol{b}$的夹角,方向由右手规则从$\boldsymbol{a}$转向$\boldsymbol{b}$来确定,垂直于$\boldsymbol{a}$与$\boldsymbol{b}$所决定的平面;向量积具有分配律;$|\boldsymbol{c}|=|\boldsymbol{a}||\boldsymbol{b}| \sin \theta$也是以向量$a,b$为边的平行四边形的面积

2、混合积:$(a b c)=\left|\begin{array}{lll}a_{x} & a_{y} & a_{z} \\b_{x} & b_{y} & b_{z} \\c_{x} & c_{y} & c_{z}\end{array}\right|$;具有轮换对称性(行列式的行交换性)

