\chapterimage{chapter_head_2.pdf}
\chapter{线性代数公式}

\section{相似与合同}\index{相似与合同}



\subsection{相似}\index{相似与合同!相似}

1、关系:$A \sim B$

2、说明:存在可逆矩阵$P$,使得$A=P^{-1}BP$

3、对角化:$P^{-1}AP=\Lambda$;反求$A$,$A=P\Lambda P^{-1}$(根据特征值或者特征向量反求矩阵问题)



\subsection{合同}\index{相似与合同!合同}

1、关系:$A \simeq B$

2、说明:存在可逆阵$C$,使得$C^TAC=B$

3、对角化:正交对角化$Q^{-1}AQ=Q^TAQ=\Lambda$;反求$A$,$A=Q\Lambda Q^{-1}=Q\Lambda Q^{T}$(根据特征值或者特征向量反求矩阵问题)

\section{求正交向量组}\index{求正交向量组}

1、设向量组$\boldsymbol{\alpha}_{1}, \boldsymbol{\alpha}_{2}, \boldsymbol{\alpha}_{3}$线性无关

2、令$\boldsymbol{\beta}_{1}=\boldsymbol{\alpha}_{1}$;$\boldsymbol{\beta}_{2}=\boldsymbol{\alpha}_{2}-\frac{\left(\boldsymbol{\alpha}_{2}, \boldsymbol{\beta}_{1}\right)}{\left(\boldsymbol{\beta}_{1}, \boldsymbol{\beta}_{1}\right)} \boldsymbol{\beta}_{1}$;$\boldsymbol{\beta}_{3}=\boldsymbol{\alpha}_{3}-\frac{\left(\boldsymbol{\alpha}{3}, \boldsymbol{\beta}_{1}\right)}{\left(\boldsymbol{\beta}_{1}, \boldsymbol{\beta}_{1}\right)} \boldsymbol{\beta}_{1}-\frac{\left(\boldsymbol{\alpha}_{3}, \boldsymbol{\beta}_{2}\right)}{\left(\boldsymbol{\beta}_{2}, \boldsymbol{\beta}_{2}\right)} \boldsymbol{\beta}_{2}$

3、则$\boldsymbol{\beta}_{1}, \boldsymbol{\beta}_{2}, \boldsymbol{\beta}_{3}$两两正交

4、再将$\boldsymbol{\beta}_{1}, \boldsymbol{\beta}_{2}, \boldsymbol{\beta}_{3}$单位化, 取$\boldsymbol{\gamma}_{1}=\frac{\boldsymbol{\beta}_{1}}{\left|\boldsymbol{\beta}_{1}\right|}, \boldsymbol{\gamma}_{2}=\frac{\boldsymbol{\beta}_{2}}{\left|\boldsymbol{\beta}_{2}\right|}, \boldsymbol{\gamma}_{3}=\frac{\boldsymbol{\beta}_{3}}{\left|\boldsymbol{\beta}_{3}\right|}$

\section{秩计算公式}\index{秩计算公式}

矩阵的秩的计算:

1、矩阵的乘积:$r(AB) \leqslant min\{r(A),r(B)\}$;$r(AB) \leqslant r(A)+r(B)$(积的秩可能(等号)会变小)(对于两个列向量,有$r\left(\boldsymbol{\beta} \boldsymbol{\alpha}^{\mathrm{T}}\right) \leqslant r(\boldsymbol{\beta})=1$,可以使用方程组来解释,或者向量的相关性判断)

2、矩阵的乘积且结果为$O$:$AB=O \rightarrow r(A)+r(B) \leqslant n$,其中n为中间阶数($A_{mn}$与$B_{ns}$)(推导:将矩阵相乘转换为方程组的解的问题,$B$是$A$的解构成的矩阵,即可得到$r(B)\leqslant n - r(A) $,其中n是A的列,也是未知数的个数)

3、矩阵的乘积且其中一个可逆:$r(AB)=r(B)$,其中$A$可逆

4、矩阵的和差:$r(A)-r(B)\leqslant r(A\pm B) \leqslant r(A)+r(B)$(矩阵积或者和的秩可能(等号)会小于两个矩阵的秩和)

5、两个矩阵互为转置:$r(A^TA)= r(A)$($A$不可逆也行:可以利用特征值来证明,$A^TA$与$A$的特征值为平方关系)($A$不是方阵也成立,如果$A$不是方阵,那么可以得出$A^TA \ne E$,用秩的关系证明)

6、矩阵的组合:$max\{r(A),r(B)\}  \leqslant r(A,B) \leqslant r(A)+r(B)$

7、单个矩阵$A_{mn}$性质为$0 \leqslant A_{mn} \leqslant min\{ m,n \}$(矩阵的秩小于行数或者列数的最小值)

8、伴随矩阵与原矩阵秩的关系:对$n(n \geqslant 2)$阶矩阵$\boldsymbol{A}, r\left(\boldsymbol{A}^{*}\right)= \begin{cases}n, & r(\boldsymbol{A})=n, \\ 1, & r(\boldsymbol{A})=n-1, \\ 0, & r(\boldsymbol{A}) \leqslant n-2 .\end{cases}$

\section{过渡矩阵}\index{过渡矩阵}

1、由$[\alpha_1,\alpha_2,\cdots,\alpha_n]$向$[\beta_1,\beta_2,\cdots,\beta_n]$过渡(列变换):过渡方程为:$[\alpha_1,\alpha_2,\cdots,\alpha_n]C=[\beta_1,\beta_2,\cdots,\beta_n]$,其中$C$为过渡矩阵(注意:过渡的方向)

2、求解过渡矩阵:利用过渡方程转换你为求逆运算(在$[\alpha_1,\alpha_2,\cdots,\alpha_n]$为可逆矩阵的条件下),即$C=[\alpha_1,\alpha_2,\cdots,\alpha_n]^{-1}[\beta_1,\beta_2,\cdots,\beta_n]$

3、坐标的过渡矩阵:$[\alpha_1,\alpha_2,\cdots,\alpha_n]$下坐标为$[x_1,x_2,\cdots,x_n]$,$[\beta_1,\beta_2,\cdots,\beta_n]$下坐标为$[y_1,y_2,\cdots,y_n]$,$C$为过渡矩阵,则$x=Cy$(证明:1中的过度方程两边同时右乘$y$,与$[\alpha_1,\alpha_2,\cdots,\alpha_n][x_1,x_2,\cdots,x_n]^T=[\beta_1,\beta_2,\cdots,\beta_n][y_1,y_2,\cdots,y_n]^T$对比)

\section{矩阵公式}\index{矩阵公式}



\subsection{单个矩阵}\index{矩阵公式!单个矩阵}

1、伴随核心公式:$AA^*=A^*A=|A|E$

2、逆矩阵的定义:$AA^{-1}=A^{-1}A=E$,$|A^{-1}|=|A|^{-1}$

4、伴随矩阵行列式:$|A^*|=|A|^{n-1}$

5、奇数阶反对称阵:$A=-A^T$

6、正交矩阵:$AA^T=A^TA=E$

7、初等矩阵求逆:($E_{i}^{-1}(k)=E_{i}(\frac 1 k)$;$E_{ij}^{-1}=E_{ij}$;$E_{ij}^{-1}(k)=E_{ij}(-k)$)(倍乘:$E_{i}(k)$第$i$行乘$k$倍;互换:$E_{ij}$第$ij$行互换;倍加:$E_{ij}(k)$第$j$行的$k$倍加到第$i$行)



\subsection{多个矩阵}\index{矩阵公式!多个矩阵}

1、逆:$(AB)^{-1}=B^{-1} A^{-1}$,$(A+B)^{-1}\ne A^{-1} + B^{-1}$

2、转置:$(AB)^T=B^TA^T$,$(A+B)^T=A^T + B^T$

3、伴随运算:$(AB)^*=B^*A^*$,核心公式:$AA^*=A^*A=|A|E$,将$A$挪到右边可得到$A^*=|A|A^{-1}$(式子1),将$A^*$挪到右边得到$(A^*)^{-1}=\frac{A}{|A|}$,再由可交换运算可以得到$(A^{-1})^*=\frac{A}{|A|}$(也可以令$B=A^{-1}$代入到式子1或者代入到伴随核心公式中推导);伴随矩阵的行列式:$|A^*|=|A|^{n-1}$

4、伴随矩阵乘系数的运算:$(kA)^*=k^{n-1}A^*$(推导:依据伴随定义)

5、可交换运算:$"*","-1","T"$是可交换的(仅限于指数运算中),并且对运算$(AB)^x=B^xA^x$将这三种符号代入到$x$均成立;相同的$T$运算合并成1次方运算,相同的$-1$运算合并成$-2$次方运算;相同的*符号可以由$(A^*)^{*}=|A|^{n-2}A$转换成数值与原矩阵的积

\section{特征值}\index{特征值}

1、矩阵变换后的特征值:$\begin{array}{c|c|c|c|c|c|c}\hline \text { 矩阵 } & \boldsymbol{A} & k \boldsymbol{A}+\boldsymbol{E} & \boldsymbol{A}+k \boldsymbol{E} & \boldsymbol{A}^{n} & \boldsymbol{A}^{-1}(\boldsymbol{A} \text { 可逆 }) & \boldsymbol{A}^{*}(\boldsymbol{A} \text { 可逆 }) \\\hline \text { 特征值 } & \lambda & k \lambda+1 & \lambda+k & \lambda^{n} & \frac{1}{\lambda} & \frac{|\boldsymbol{A}|}{\lambda} \\\hline\end{array}$

2、特征方程:$|\lambda E-A|$

3、

