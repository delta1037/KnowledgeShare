\chapterimage{chapter_head_2.pdf}
\chapter{数理统计的基本概念}

\section{常用统计抽样分布-卡方、T、F}\index{常用统计抽样分布-卡方、T、F}



\subsection{卡方分布}\index{常用统计抽样分布-卡方、T、F!卡方分布}

0、定义:$ \chi^{2} = X_1^2+X_2^2+\cdots+X_n^2 \sim \chi^{2}(n) $,自由度为$ n $,$ X_1,X_2,\cdots ,X_n $相互独立且服从标准正态分布

1、上侧$ \alpha $分位数定义:$ P\left\{\chi^{2}>\chi_{\alpha}^{2}(n)\right\}=\int_{\chi_{\alpha}^{2}(n)}^{+\infty} f(t) \mathrm{d} t=\alpha $

2、卡方均值:$ E\left(\chi^{2}\right)=n $

3、卡方方差:$ D\left(\chi^{2}\right)=2 n $

4、可加性:$ \chi_{1}^{2}+\chi_{2}^{2} \sim \chi^{2}\left(n_{1}+n_{2}\right) $,其中$ \chi_{1}^{2}, \chi_{2}^{2} $相互独立,$ \chi_{1}^{2} \sim \chi^{2}\left(n_{1}\right) $,$ \chi_{2}^{2} \sim \chi^{2}\left(n_{2}\right) $

注意:分位数是将面积(概率)与x轴的坐标对应了起来,卡方分布只有右边



\subsection{T分布}\index{常用统计抽样分布-卡方、T、F!T分布}

0、定义:$ t=\frac{X}{\sqrt{Y/n}} \sim t(n) $,其中$ X \sim N(0,1) $,$ Y\sim \chi^{2}\left(n\right) $,且$ **X,Y** $相互独立

1、上侧$ \alpha $分位数定义:$ P\left\{t>t_{\alpha}(n)\right\}=\int_{t_{\alpha}(n)}^{+\infty} f(t) \mathrm{d} t=\alpha $,

2、分位数性质:$ t $分布具有对称性,$ t_{1-\alpha}(n)=t_{\alpha}(n) $

注意:分位数是将面积(概率)与x轴的坐标对应了起来,T分布是关于X轴对称的



\subsection{F分布}\index{常用统计抽样分布-卡方、T、F!F分布}

0、定义:$ F=\frac{U / n_{1}}{V / n_{2}} \sim F\left(n_{1}, n_{2}\right) $,其中$ U \sim \chi^{2}\left(n_{1}\right) $,$  V \sim \chi^{2}\left(n_{2}\right) $,且$ **U,V** $相互独立;并且由定义可知$ \frac{1}{F} \sim F\left(n_{2}, n_{1}\right) $

1、上侧$ \alpha $分位数定义:$ P\left\{F>F_{\alpha}\left(n_{1}, n_{2}\right)\right\}=\int_{F_{\alpha}\left(n_{1}, n_{2}\right)}^{+\infty} f(t) \mathrm{d} t=\alpha $

2、分位数性质:$ F_{1-\alpha/2}(n_{2},n_{1})=\frac{1}{F_{{\alpha}/{2}}\left(n_{1}, n_{2}\right)} $(三变性质)

注意:分位数是将面积(概率)与x轴的坐标对应了起来,F分布只有右边

\section{样本的数字特征}\index{样本的数字特征}



\subsection{数字特征}\index{样本的数字特征!数字特征}

1、样本均值:$ \bar{X}=\frac{1}{n} \sum_{i=1}^{n} X_{i} $

2、样本方差:$ S^{2}=\frac{1}{n-1} \sum_{i=1}^{n}\left(X_{i}-\bar{X}\right)^{2}=\frac{1}{n-1}\left(\sum_{i=1}^{n} X_{i}^{2}-n \bar{X}^{2}\right) $(样本方差和方差是具有区别的)

3、标准差:$ S $

4、样本$ k $阶(原点) 矩:$ A_{k}=\frac{1}{n} \sum_{i=1}^{n} X_{i}^{k}, k=1,2, \cdots $(原点矩的和形式)

5、样本k阶中心矩:$ B_{k}=\frac{1}{n} \sum_{i=1}^{n}\left(X_{i}-\bar{X}\right)^{k}, k=2,3, \cdots $(原点矩的和形式)(中心是根据样本均值来定的)



\subsection{数字特征性质}\index{样本的数字特征!数字特征性质}

1、样本均值期望:$ E(\bar{X}) = \mu $(样本均值公式代入证明)

2、样本均值方差:$ D(\bar{X}) = \frac{\sigma^2}{n} $(样本均值公式代入证明)

3、样本方差均值:$ E(S^2) = \sigma^2 $(样本方差公式代入证明)

4、样本均值期望推广:若总体$ X $的$ k $阶原点矩$ E(X^k)=\mu_k $存在,当$ n $趋近于无穷时,样本的$ k $阶原点矩$ \frac{1}{n}\sum_{i=1}^{n}X_i^k $依概率收敛于$ \mu_k $(样本数量较多时,样本均值收敛于总体均值)

\section{常用统计抽样分布-正态(区间估计和假设检验的核心)}\index{常用统计抽样分布-正态(区间估计和假设检验的核心)}



\subsection{单个正态总体}\index{常用统计抽样分布-正态(区间估计和假设检验的核心)!单个正态总体}

1、定义:设总体$ X \sim N\left(\mu, \sigma^{2}\right), X_{1}, X_{2},\cdots, X_{n} $是来自$ X $的样本,样本均值$ \bar{X}=\frac{1}{n} \sum_{i=1}^{n} X_{i} $, 样本方差$ S^{2}=\frac{1}{n-1} \sum_{i=1}^{n}\left(X_{i}-\bar{X}\right)^{2} $

2、性质

2.1 样本均值分布:$ \bar{X} \sim N\left(\mu, \frac{\sigma^{2}}{n}\right) $,标准化之后为$ U=\frac{\bar{X}-\mu}{\sigma / \sqrt{n}} \sim N(0,1) $

2.2 样本方差构造卡方分布:$ \chi^{2}=\frac{(n-1) S^{2}}{\sigma^{2}} \sim \chi^{2}(n-1) $(推导:$ <font color=purple>\frac{(n-1) s^{2}}{\sigma^{2}} =\frac{\sum_{i=1}^{n}\left(x_{i}-\bar{x}\right)^{2}}{\sigma^{2}} =\sum_{i=1}^{n}\left(\frac{x_{i}}{\sigma}-\frac{\bar{x}}{\sigma}\right)^{2} =\sum_{i=1}^{n}\left(\frac{x_{i}-\mu}{\sigma}-\frac{\bar{x}-\mu}{\sigma}\right)^{2} =\sum_{i=1}^{n}\left(Z_{i}-\bar{Z}\right)^{2} =\sum_{i=1}^{n} Z_{i}^{2}-n \bar{Z}^{2}</font> $)

2.3 样本均值和样本方差构造$ **T** $分布:$ T=\frac{\bar{X}-\mu}{S / \sqrt{n}} \sim t(n-1) $(推导:利用标准化的$ <font color=purple>U</font> $和满足卡方分布的$ <font color=purple>\frac{(n-1) S^{2}}{\sigma^{2}} \sim \chi^{2}(n-1)</font> $构造)

2.4 中心矩分布:$ \chi^{2}=\frac{1}{\sigma^{2}} \sum_{i=1}^{n}\left(X_{i}-\mu\right)^{2} \sim \chi^{2}(n) $(注意与样本方差构造卡方分布的区别)(推导:就是将样本标准化成标准正态分布后的平方和)



\subsection{两个正态总体}\index{常用统计抽样分布-正态(区间估计和假设检验的核心)!两个正态总体}

1、定义:设$ X_{1}, X_{2}, \cdots, X_{n_{1}} $与$ Y_{1}, Y_{2}, \cdots,Y_{n_{2}} $分别是来自正态总体$ N\left(\mu_{1}, \sigma_{1}^{2}\right) $和$ N\left(\mu_{2}, \sigma_{2}^{2}\right) $的样本,且这两个样本相互独立. 设$ \bar{X}=\frac{1}{n_{1}} \sum_{i=1}^{n_{1}} X_{i} $,$ \bar{Y}=\frac{1}{n_{2}} \sum_{i=1}^{n_{2}} Y_{i} $分别是这两个样本的样本均值; $ S_{1}^{2}=\frac{1}{n_{1}-1} \sum_{i=1}^{n_{1}}\left(X_{i}-\bar{X}\right)^{2} $, $ S_{2}^{2}=\frac{1}{n_{2}-1} \sum_{i=1}^{n_{2}}\left(Y_{i}-\bar{Y}\right)^{2} $分别是这两个样本的样本方差

2、性质

2.1 均值差构造的正态的分布:$ \bar{X}-\bar{Y} \sim N\left(\mu_{1}-\mu_{2}, \frac{\sigma_{1}^{2}}{n_{1}}+\frac{\sigma_{2}^{2}}{n_{2}}\right) $(即正态分布的组合),标准化$  U=\frac{(\bar{X}-\bar{Y})-\left(\mu_{1}-\mu_{2}\right)}{\sqrt{\frac{\sigma_{1}^{2}}{n_{1}}+\frac{\sigma_{2}^{2}}{n_{2}}}} \sim N(0,1) $

2.2 均值和样本方差构造$ **T** $分布(需要$ <font color=purple>**\sigma_{1}^{2}=\sigma_{2}^{2}=\sigma^{2}**</font> $):$ \frac{(\bar{X}-\bar{Y})-\left(\mu_{1}-\mu_{2}\right)}{S_{w} \sqrt{\frac{1}{n_{1}}+\frac{1}{n_{2}}}} \sim t\left(n_{1}+n_{2}-2\right) $,$ S_{w}^{2}=\frac{\left(n_{1}-1\right) S_{1}^{2}+\left(n_{2}-1\right) S_{2}^{2}}{n_{1}+n_{2}-2}, \quad S_{w}=\sqrt{S_{w}^{2}} $(推导:利用上述标准化的$ <font color=purple>U</font> $和满足卡方分布的$ <font color=purple>\frac{(n_1-1) S_1^{2}}{\sigma_1^{2}} \sim \chi^{2}(n_1-1)</font> $和$ <font color=purple>\frac{(n_2-1) S_2^{2}}{\sigma_2^{2}} \sim \chi^{2}(n_2-1)</font> $,以及卡方分布具有可加性,以及$ <font color=purple>U</font> $和相加后的卡方分布之间的独立性)

2.3 样本方差构造$ **F** $分布:$ \frac{S_{1}^{2} / S_{2}^{2}}{\sigma_{1}^{2} / \sigma_{2}^{2}} \sim F\left(n_{1}-1, n_{2}-1\right) $(推导:满足卡方分布的$ <font color=purple>\frac{(n_1-1) S_1^{2}}{\sigma_1^{2}} \sim \chi^{2}(n_1-1)</font> $和$ <font color=purple>\frac{(n_2-1) S_2^{2}}{\sigma_2^{2}} \sim \chi^{2}(n_2-1)</font> $,以及两个卡方分布之间的独立性)

疑问:对于构造T分布时,标准化的U和相加后的卡方分布为什么具有独立性

\section{总体的样本的相关计算}\index{总体的样本的相关计算}



\subsection{样本满足条件的概率:(或已知概率求解未知参数)}\index{总体的样本的相关计算!样本满足条件的概率:(或已知概率求解未知参数)}

1、将来自总体的样本依照常用的统计抽样分布的条件转换,得到标准分布(概率论思想:标准化)

2、依照标准分布求解概率



\subsection{样本满足某种分布:(证明)}\index{总体的样本的相关计算!样本满足某种分布:(证明)}

1、依照常用的统计抽样分布的性质(各部分的独立性)对样本函数的各个部分进行证明验证



\subsection{样本构成的函数的期望:(类似于矩估计量的进一步求解)}\index{总体的样本的相关计算!样本构成的函数的期望:(类似于矩估计量的进一步求解)}

1、单个正态总体的统计抽样分布性质

2、函数的拆分组合,依据概率分布的定义求解:$ P\{ X_n > max(X_1,X_2,\cdots,X_{n-1})\} = P\{ X_n > X_1,X_n >X_2,\cdots,X_n >X_{n-1})\} $



\subsection{样本的联合概率密度或概率分布}\index{总体的样本的相关计算!样本的联合概率密度或概率分布}

1、相互独立的随机变量的联合概率密度:连续情况下,若已知样本概率密度,则来自总体的样本的联合概率密度为各个样本的概率密度相乘

2、先求得分布函数:在题目已知条件下,可以求得联合概率密度的分布函数,由分布函数求导得到概率密度

3、相互独立的随机变量的联合概率分布:离散情况下,来自总体的样本的联合概率分布为各个样本的概率分布乘积

4、离散变量均值的分布:离散情况下,求解均值(一种样本函数)的分布(如果总体分布为泊松分布,注意泊松分布具有可加性)

\section{总体、样本、统计量定义}\index{总体、样本、统计量定义}

1、总体:研究对象的某项数量指标$ X $的全体

2、样本:$ X_1,X_2,\cdots X_n $相互独立且同分布,则称该序列是总体$ X $的简单随机样本(样本)

3、统计量:由样本构造的不含未知数的函数

