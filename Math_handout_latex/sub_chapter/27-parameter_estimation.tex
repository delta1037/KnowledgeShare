\chapterimage{chapter_head_2.pdf}
\chapter{参数估计}

\section{求估计量-区间估计}\index{求估计量-区间估计}



\subsection{区间估计}\index{求估计量-区间估计!区间估计}

1、理解置信区间:当估计未知量时,利用已有的东西(均值,方差,均值差之类的)构成一个分布(正态分布,卡方分布,T分布,F分布)并将分布标准化(标准化后的正态分布,卡方分布,T分布,F分布),置信区间是该标准分布的一个区间内时(该区间的概率值为$1-\alpha$,并叫置信度为$1-\alpha$的置信区间),待估计的未知量的范围(基本公式为$P\{ \theta_1 < \theta < \theta_2 \} = 1-\alpha$,则置信区间为$(\theta_1 , \theta_2)$,也就是标准分布在置信区间内的概率等于置信度)(两侧或者单侧的概率就是$\alpha$)



\subsection{有关置信区间的未知量求解}\index{求估计量-区间估计!有关置信区间的未知量求解}

1、利用置信区间的性质(正态分布和$T$分布的对称性,$F$分布的三变性质)

注意:对称轴的位置可能不在原点,可绘制坐标图辅助理解

\section{区间估计-一个正态总体的区间估计}\index{区间估计-一个正态总体的区间估计}

1、一个正态总体的区间估计利用了单个正态总体的抽样分布

2、如果要估计$\mu$,$\sigma^{2}$已知,需要用样本均值构造的正态分布($\bar{X} \sim N\left(\mu, \frac{\sigma^{2}}{n}\right)$,标准化$U=\frac{\bar{X}-\mu}{\sigma / \sqrt{n}} \sim N(0,1)$,因为这个分布中除了$\mu$其它值($\sigma^{2}$,样本个数$n$,样本均值$\bar{X}$)都是已知的,可以套用基本公式得出待估计量的置信区间),找一个以$1−α$概率涵盖标准正态分布观测的区间,自然会找到$(−u_{\alpha/2},u_{\alpha/2})$,则待估计量$\mu$的估计区间为$\left(\bar{X}-u_{{\alpha}/{2}} \frac{\sigma}{\sqrt{n}}, \quad \bar{X}+u_{{\alpha}/{2}} \frac{\sigma}{\sqrt{n}}\right) $(因为正态分布是双侧的,所以对两侧的概率每一边都是$\alpha/2$,即让标准化的$U$落在$(-u_{{\alpha}/{2}},u_{{\alpha}/{2}})$区间,从而求得$\mu$的估计区间)

3、如果要估计$\mu$,$\sigma^{2}$未知,需要用样本均值和样本方差构造的$T$分布($T=\frac{\bar{X}-\mu}{S / \sqrt{n}} \sim t(n-1)$,因为这个分布中除了$\mu$其它值(样本方差$S^2$,样本个数$n$,样本均值$\bar{X}$)都是已知的,可以套用基本公式得出待估计量的置信区间),找一个以$1−α$概率涵盖标准$T$分布观测的区间,自然会找到$(−t_{\alpha/2}(n-1),t_{\alpha/2}(n-1))$,则待估计量$\mu$的估计区间为$\left(\bar{X}-t_{{\alpha}/{2}}(n-1) \frac{S}{\sqrt{n}}, \quad \bar{X}+t_{{\alpha}/{2}}(n-1) \frac{S}{\sqrt{n}}\right)$(因为T分布是双侧的,所以对两侧的概率每一边都是$\alpha/2$,即让标准化的$T$落在$(−t_{\alpha/2}(n-1),t_{\alpha/2}(n-1))$区间,从而求得$\mu$的估计区间)

4、如果要估计$\sigma^{2}$值,需要用样本方差构造的卡方分布($\chi^{2}=\frac{(n-1) S^{2}}{\sigma^{2}} \sim \chi^{2}(n-1)$,因为这个分布中除了$\sigma^{2}$其它值(样本方差$S^2$,样本个数$n$)都是已知的,可以套用基本公式得出待估计量的置信区间),找一个以$1−α$概率涵盖标准卡方分布观测的区间,自然会找到$(\chi_{{\alpha}/{2}}^{2}(n-1),\chi_{1-{\alpha}/{2}}^{2}(n-1))$,则待估计量$\sigma^{2}$的估计区间为$\left(\frac{(n-1) S^{2}}{\chi_{{\alpha}/{2}}^{2}(n-1)}, \frac{(n-1) S^{2}}{\chi_{1-{\alpha}/{2}}^{2}(n-1)}\right)$(因为卡方分布是单侧的,所以使单侧的两边的概率每一边都是$\alpha/2$,即让标准化的卡方变量落在$(\chi_{{\alpha}/{2}}^{2}(n-1),\chi_{1-{\alpha}/{2}}^{2}(n-1))$区间,从而求得$\mu$的估计区间)

\section{求估计量-矩估计}\index{求估计量-矩估计}



\subsection{矩估计法}\index{求估计量-矩估计!矩估计法}

1、未知参数表示:列出未知参数用矩表示的表示式(列出未知参数与总体的矩的关系式)(当一阶矩与未知参数没有关系时(比如一阶矩为0),求解二阶矩)(备注:其中矩可以根据题目情况选择原点矩或者中心矩,看未知参数用总体的什么矩表示更方便)

2、样本矩替换矩得到估计量:由样本的等于总体矩得到未知参数的估计量(将关系式中的总体矩替换成样本矩,未知参数就需要转换为估计量的说法,变成了估计量与样本矩的表示式,样本矩是可以从样本计算的,所以估计量就可以求出来了)

3、进一步求解估计量的数字特征:当需要求解估计量或者估计量表达式的期望时,将求解的估计量(用样本表示的)代入到待求解的期望表达式,转换成求解样本的数字特征,由于样本具有与总体一样的数字特征,所以使用总体的概率密度或者相关表达式进行求解(求解的结果等于未知参数或者未知参数的表达式(与估计量一样的表达式))

备注:矩估计含义是以已知的样本的矩来估计总体的矩中的未知参数

注意:在矩估计中是用不到样本方差的表达式的(因为都是用矩估计的)



\subsection{求解参数的矩估计量}\index{求估计量-矩估计!求解参数的矩估计量}

1、一个待估计参数只需要一个方程,一阶矩(如果一阶矩与未知参数没有关系,需要求二阶矩)

2、两个待估计参数需要两个方程,一阶矩和二阶矩方程

3、列出的方程的关系是样本的矩和总体的矩相等或者是样本中心矩和总体中心矩相等

注意:参数的估计量中的未知变量只能有样本相关的量(如果样本是具体的数值,则估计量也是数值)

\section{求估计量-最大似然估计法}\index{求估计量-最大似然估计法}

1、列出关于未知变量的似然函数(似然函数是根据样本点的值对应的概率密度相乘得到的)

2、有驻点:求解方程达到最大值的点(直接求导或者取对数后求导等于0的点),对应的未知变量的值就是待求解的值

3、没有驻点:从题目中找到关于未知参数的限制关系,根据最大似然函数单调性找到使得最大似然函数最大点的值

注意:写出似然函数后就是求解最大值的问题,如果似然函数存在驻点则在驻点中找;如果由导数没有驻点或不可导,用其它方法找使得似然函数最大值的点

\section{无偏估计}\index{无偏估计}

证明随机变量函数是某个值的无偏估计:(或者已知是无偏估计求解未知参数)

方案一:直接求解

1、随机变量函数(平方形式,指数形式)的拆分组合

2、利用特定分布的性质

方案二:求核心部分的分布(随机变量整体是正态分布的)(一定要注意核心分布的求解需要其中的各个变量互相独立,无法验证是否独立时用别的方法)

1、依据:方差的性质$D(a X+b)=a^{2} D(X) $;期望的性质$E(a_1X_1+a_2X_2+\cdots+a_nX_n+b_1+b_2+\cdots+b_m)=a_1E(X_1)+a_2E(X_2)+\cdots+a_nE(X_n)+b_1+b_2+\cdots+b_m$;正态分布变量组合性质(当$X$与$Y$均服从一维正态,且相互独立,就是指$(X, Y)$服从二维正态,$a X+b Y$服从一维正态,其中$\left(a^{2}+b^{2} \neq 0\right)$,这是从多维分布里面拖过来的)

2、步骤:求解核心部分的分布函数(分别求解核心部分的期望和方差,从而得到正态分布);利用新的正态分布求解即可

注意:样本与总体是同分布的

\section{估计定义}\index{估计定义}

1、点估计:用样本$X_1,X_2,\ldots,X_n$构造统计量$\hat \theta(X_1,X_2,\ldots,X_n)$来估计未知参数$\theta$称为点估计,统计量$\hat \theta(X_1,X_2,\ldots,X_n)$称为估计量

2、无偏估计量:$\hat \theta$是$\theta$的无偏估计量,如果$E(\hat \theta)=\theta$,则称$\hat \theta=\hat \theta(X_1,X_2,\ldots,X_n)$是未知参数$\theta$的无偏估计量

3、更有效估计量:设$\hat \theta_1$和$\hat \theta_2$都是$\theta$的无偏估计量,且$D(\hat \theta_1) \le D(\hat \theta_2)$,则称$\hat \theta_1$比$\hat \theta_2$更有效,或$\hat \theta_1$比$\hat \theta_2$更有效估计量

4、一致估计量:设$\hat \theta(X_1,X_2,\ldots,X_n)$是$\theta$的估计值,如果估计值$\hat \theta$依概率收敛于未知参数$\theta$,则称$\hat \theta(X_1,X_2,\ldots,X_n)$是$\theta$的一致估计量

\section{区间估计-两个正态总体的区间估计}\index{区间估计-两个正态总体的区间估计}

1、两个正态总体的区间估计利用了两个正态总体的抽样分布

2、如果要估计$\mu_1-\mu_2$,$\sigma_{1}^{2}$和$\sigma_{2}^{2}$已知,需要用均值差构造的正态的分布($\bar{X}-\bar{Y} \sim N\left(\mu_{1}-\mu_{2}, \frac{\sigma_{1}^{2}}{n_{1}}+\frac{\sigma_{2}^{2}}{n_{2}}\right)$,标准化$ U=\frac{(\bar{X}-\bar{Y})-\left(\mu_{1}-\mu_{2}\right)}{\sqrt{\frac{\sigma_{1}^{2}}{n_{1}}+\frac{\sigma_{2}^{2}}{n_{2}}}} \sim N(0,1)$,因为这个分布中除了$\mu_{1}-\mu_{2}$其它值($\sigma_{1}^{2},\sigma_{2}^{2}$,样本个数$n_1,n_2$,样本均值$\bar{X},\bar{Y}$)都是已知的,可以套用基本公式得出待估计量的置信区间),找一个以$1−α$概率涵盖标准正态分布观测的区间,自然会找到$(−u_{\alpha/2},u_{\alpha/2})$,则待估计量$\mu_1-\mu_2$的估计区间为$ \left(\bar{X}-\bar{Y}-u_{{\alpha}/{2}} \sqrt{\frac{\sigma_{1}^{2}}{n_{1}}+\frac{\sigma_{2}^{2}}{n_{2}}}, \bar{X}-\bar{Y}+u_{{\alpha}/{2}} \sqrt{\frac{\sigma_{1}^{2}}{n_{1}}+\frac{\sigma_{2}^{2}}{n_{2}}}\right)$

3、如果要估计$\mu_1-\mu_2$,$\sigma_{1}^{2}$和$\sigma_{2}^{2}$值未知,但是已知$\sigma_{1}^{2}=\sigma_{2}^{2}$,需要用均值和样本方差构造$T$分布((需要$\sigma_{1}^{2}=\sigma_{2}^{2}=\sigma^{2}$):$\frac{(\bar{X}-\bar{Y})-\left(\mu_{1}-\mu_{2}\right)}{S_{w} \sqrt{\frac{1}{n_{1}}+\frac{1}{n_{2}}}} \sim t\left(n_{1}+n_{2}-2\right)$,$S_{w}^{2}=\frac{\left(n_{1}-1\right) S_{1}^{2}+\left(n_{2}-1\right) S_{2}^{2}}{n_{1}+n_{2}-2}, \quad S_{w}=\sqrt{S_{w}^{2}}$,因为这个分布中除了$\mu_{1}-\mu_{2}$其它值($S_{1}^{2},S_{2}^{2}$,样本个数$n_1,n_2$,样本均值$\bar{X},\bar{Y}$,$\sigma_{1}^{2}=\sigma_{2}^{2}=\sigma^{2}$是T分布的要求)都是已知的,可以套用基本公式得出待估计量的置信区间),找一个以$1−α$概率涵盖标准$T$分布观测的区间,自然会找到$(−t_{\alpha/2}(n_{1}+n_{2}-2),t_{\alpha/2}(n_{1}+n_{2}-2))$,则待估计量$\mu_1-\mu_2$的估计区间为$\left(\bar{X}-\bar{Y}-t_{{\alpha}/{2}}\left(n_{1}+n_{2}-2\right) S_{w} \sqrt{\frac{1}{n_{1}}+\frac{1}{n_{2}}}\right. ,\left.\bar{X}-\bar{Y}+t_{{\alpha}/{2}}\left(n_{1}+n_{2}-2\right) S_{\omega} \sqrt{\frac{1}{n_{1}}+\frac{1}{n_{2}}}\right)$

4、如果要估计$\frac{\sigma_{1}^{2}}{\sigma_{2}^{2}}$值,需要用样本方差构造$F$分布($\frac{S_{1}^{2} / S_{2}^{2}}{\sigma_{1}^{2} / \sigma_{2}^{2}} \sim F\left(m-1, n-1\right)$,因为这个分布中除了$\frac{\sigma_{1}^{2}}{\sigma_{2}^{2}}$其它值($S_{1}^{2},S_{2}^{2}$,样本个数$n_1,n_2$)都是已知的,可以套用基本公式得出待估计量的置信区间),找一个以$1−α$概率涵盖标准卡方分布观测的区间,自然会找到$(F_{1-\alpha/2}(m−1,n−1)=\frac{1}{F_{{\alpha}/{2}}\left(n−1, m−1\right)},F_{\alpha/2}(m−1,n−1))$,则待估计量$\frac{\sigma_{1}^{2}}{\sigma_{2}^{2}}$的估计区间为$\left(\frac{S_{1}^{2}}{S_{2}^{2}} \cdot \frac{1}{F_{{\alpha}/{2}}\left(n_{1}-1, n_{2}-1\right)}, \frac{S_{1}^{2}}{S_{2}^{2}} F_{{\alpha}/{2}}\left(n_{2}-1, n_{1}-1\right.\right)$(因为F分布是单侧的,所以使单侧的两边的概率每一边都是$\alpha/2$,即让标准化的$F$变量落在$(F_{1-\alpha/2}(m−1,n−1)=\frac{1}{F_{{\alpha}/{2}}\left(n−1, m−1\right)},F_{\alpha/2}(m−1,n−1))$区间,从而求得$\mu$的估计区间)

