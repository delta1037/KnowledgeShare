\chapterimage{chapter_head_2.pdf}
\chapter{假设检验}

\section{假设检验}\index{假设检验}



\subsection{一般步骤}\index{假设检验!一般步骤}

1、根据实际问题的要求,提出原假设$H_0$及备择假设$H_1$

2、给定显著性水平$\alpha$及样本容量$n$

3、确定检验统计量以及拒绝域的形式

4、按犯第一类错误的概率等于$\alpha$求出拒绝域$W$

5、取样,根据样本计算检验统计量的观测值$t$,当$t$处于拒绝域$W$内拒绝$H_0$,否则接受原假设

注意:常用的统计量构造(单样本、双样本)

注意:如果假设是等号,则拒绝域是双边的;如果假设是小于号,则拒绝域在右边;如果假设是大于号,则拒绝域在左边



\subsection{统计量构造类型:(注意与区间估计的区别是区间估计是双边的,统计量检验只有原假设为等号时是双边的,其它都是单边的)($\mu_0$和$\sigma_0$都是检验中的临界值)(所有的统计量可以认为是在观测值按照已有的条件标准化)}\index{假设检验!统计量构造类型:(注意与区间估计的区别是区间估计是双边的,统计量检验只有原假设为等号时是双边的,其它都是单边的)($\mu_0$和$\sigma_0$都是检验中的临界值)(所有的统计量可以认为是在观测值按照已有的条件标准化)}

1、(单样本)检验$\mu$,$\sigma^{2}$已知:$N(0,1) \sim U=\frac{\bar{X}-\mu_{0}}{\sigma / \sqrt{n}}$

2、(单样本)检验$\mu$,$\sigma^{2}$未知: $t(n-1) \sim T=\frac{\bar{X}-\mu_{0}}{S / \sqrt{n}}$

3、(单样本)检验$\sigma^{2}$,$\mu$已知:$X^2(n) \sim \chi^{2}=\frac{1}{\sigma_{0}^{2}} \sum_{i=1}^{n}\left(X_{i}-\mu\right)^{2}$

4、(单样本)检验$\sigma^{2}$,$\mu$未知:$X^2(n-1) \sim \chi^{2}=\frac{(n-1) S^{2}}{\sigma^{2}}$

5、(双样本)检验$\mu_1-\mu_2$,$\sigma_{1}^{2}$和$\sigma_{2}^{2}$已知:$N(0,1) \sim U=\frac{\bar{X}-\bar{Y}-\mu_{0}}{\sqrt{\frac{\sigma_{1}^{2}}{n_{1}}+\frac{\sigma_{2}^{2}}{n_{2}}}}$

6、(双样本)检验$\mu_1-\mu_2$,$\sigma_{1}^{2}$和$\sigma_{2}^{2}$未知但相等:$t(n_1+n_2-2) \sim T=\frac{\bar{X}-\bar{Y}-\mu_{0}}{S_{\omega} \sqrt{\frac{1}{n_{1}}+\frac{1}{n_{2}}}}$

7、(双样本)检验$\sigma_{1}^{2} = \sigma_{2}^{2}$,$\mu_1,\mu_2$已知:$F(n_1,n_2) \sim F=\frac{n_{2} \sum_{i=1}^{n_{1}}\left(X_{i}-\mu_{1}\right)^{2}}{n_{1} \sum_{j=1}^{n_{2}}\left(Y_{j}-\mu_{2}\right)^{2}}$

8、(双样本)检验$\sigma_{1}^{2} = \sigma_{2}^{2}$,$\mu_1,\mu_2$未知:$F(n_1-1,n_2-1) \sim F=\frac{S_{1}^{2}}{S_{2}^{2}}$

\section{统计量的构造}\index{统计量的构造}

1、统计量的构造需要考虑需要检验的参数,并且考虑是单样本还是双样本(两个样本之间的比较)

2、考虑一个或者两个正态总体的分布,其中有哪些参数是已知和未知的,确定检验范围(一个正态总体待检验的参数是否等于,小于,大于某个值;两个正态总体的检验参数差值大于,小于等情况),去确定未知参数的情况,从而构造合适的统计量

\section{假设检验的两类错误}\index{假设检验的两类错误}

1、第一类错误(弃真)$\alpha$:可认为原假设$H_0$是正确的,但是样本却在拒绝域里导致拒绝了本是正确的假设(弃真,是指放弃真的原假设),$P\{ 原假设拒绝域| H_0 \}$,解释为原假设正确条件下样本在拒绝域的概率,导致拒绝了真的原假设

2、第二类错误(纳伪)$\beta$:可认为备选假设$H_1$是正确的,但是样本却在原假设的非拒绝域内导致无法拒绝(错误的)原假设从而发生的错误(纳伪,是指接受错误的原假设),$P\{ 原假设拒绝域的对立情况| H_1 \}$,解释为备选假设正确的条件下,样本却在原假设的非拒绝域的概率,导致无法拒绝错误的原假设

