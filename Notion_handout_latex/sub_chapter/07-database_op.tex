\chapterimage{chapter_head_2.pdf}
\chapter{数据库操作}

\section{类型互联}\index{类型互联}

1、数据库的分类:数据库分类型只是对同一个数据库不同的展示方式

2、获取其它展示:数据库名称旁边的“+ Add View”,选择一种方式创建即可

\section{基本类型}\index{基本类型}

1、Text:富文本类型

2、Number:数值类型(数值,百分比等等,特别多)

3、Select:单选(从已创建的项里选择其中一项)(点击代替重复编辑,一改均改)

4、Multi-select:多选(从已创建的项里选择其中一项或者多项)

5、Date:日期类型(具体时间,是否包含结尾,日期的格式)

6、Person:人(notion的用户)

7、File\&Media:多媒体文件

8、Checkbox:复选框(代表该项是否完成)

9、URL:超链接

10、Email:邮箱

11、Phone:电话号码

\section{高级类型}\index{高级类型}

1、Formula:基于同一数据库其它属性的函数,具体见https://www.notion.so/help/formulas

2、Relation:基于不同数据库之间的关联

3、Rollup:基于同一数据库的关联的相关计算

4、Created time:该项创建时间

5、Last edited time:该项最后编辑时间

6、Created by:该项创建人

7、Last edited by:该项最后的编辑人

\section{展示部分}\index{展示部分}

1、Properties:只展示部分属性

2、Filter:过滤展示某些条目

3、Sort:排序展示

4、Group:按照指定属性分组展示

5:Turn into page:转换成页面

6、Turn into simple table:转换成表格

\section{常用操作}\index{常用操作}

1、搜索(Search):右上角搜索图标,可搜索数据库内的任意内容

2、复制(Duplicate):将数据库拷贝一份放到现有数据库下面

3、拷贝链接(Copy link):复制数据库的链接在别的地方创建一个数据库链接

4、锁定(Database lock):锁定数据库视图和属性,子页面不受影响

5、内容全显示(Wrap cell):将内容全显示(不隐藏过长部分)

6、作为子页面打开(Open as page):将数据库作为一个子页面打开(右上角的双向箭头也是这个功能)

7、移动(Move to):转移到其它page

8、和CSV合并(Merge with CSV):和导入的CSV文件合并

9、导出(Export):将数据库导出为HTML,PDF,Markdown\&CSV

\section{操作技巧}\index{操作技巧}

1、Text内容换行:Shift + Enter

