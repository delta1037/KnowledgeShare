\chapterimage{chapter_head_2.pdf}
\chapter{数据库公式}

\section{prop}\index{prop}

1、简介:获取属性的值

2、语法:prop(”property“)

3、样例:prop("title")

4、返回类型:属性值

\section{e}\index{e}

1、简介:自然底数,值为2.718281828459045

2、语法:e

3、样例:e == 2.718281828459045

4、返回类型:数值

\section{pi}\index{pi}

1、简介:圆周率, 值为3.14159265359

2、语法:pi

3、样例:2 * pi == 6.283185307179586

4、返回类型:数值

\section{true}\index{true}

1、简介:布尔值,真

2、语法:true

3、样例:true ? "yes" : "no" == "yes"

4、返回类型:布尔值

\section{false}\index{false}

1、简介:布尔值,假

2、语法:false

3、样例:false ? "yes" : "no" == "no"

4、返回类型:布尔值

\section{if}\index{if}

1、简介:判断语句,根据第一个值的布尔值在后边两个选项之间做选择

2、语法:if(boolean, value, value) ;boolean ? value : value

3、操作对象:第一个参数类型是布尔类型,后边两个参数类型必须一致

4、返回类型:与后两个参数类型一致

5、操作样例:if(false, "yes", "no") == "no"

\section{add}\index{add}

1、简介:对两个传入的数值做相加操作,或者对两个传入的字符串拼接

2、语法:add(number, number) ;add(text, text) ;number + number ;text + text

3、操作对象:两个参数均为数值或者两个参数均为字符串

4、返回类型:与后两个参数类型一致

5、操作样例:add(1, 3) == 4

\section{subtract}\index{subtract}

1、简介:对两个传入的参数做相减操作(前-后)

2、语法:subtract(number, number) ;number - number

3、操作对象:两个参数均为数值

4、返回类型:数值类型

5、操作样例:subtract(4, 5) == -1

\section{multiply}\index{multiply}

1、简介:对两个传入的参数做乘法操作

2、语法:multiply(number, number) ;number * number

3、操作对象:两个参数均为数值

4、返回类型:数值类型

5、操作样例:multiply(2, 10) == 20

\section{divide}\index{divide}

1、简介:对两个传入的参数做除法操作(前除后)

2、语法:divide(number, number) ;number / number

3、操作对象:两个参数均为数值

4、返回类型:数值类型

5、操作样例:divide(12, 3) == 4

\section{pow}\index{pow}

1、简介:对两个传入的参数做幂运算($\text{前}^\text{后}$)

2、语法:pow(number1, number2) ;$number1 ^ {number2}$

3、操作对象:两个参数均为数值

4、返回类型:数值类型

5、操作样例:pow(2, 6) == 64

\section{mod}\index{mod}

1、简介:对两个传入的参数做取余运算

2、语法:mod(number, number) ;number % number

3、操作对象:两个参数均为数值

4、返回类型:数值类型

5、操作样例:mod(3, 3) == 0

\section{unaryMinus}\index{unaryMinus}

1、简介:求当前值的负值

2、语法:unaryMinus(number) ;- number

3、操作对象:参数为数值

4、返回类型:数值类型

5、操作样例:unaryMinus(42) == -42

\section{unaryPlus}\index{unaryPlus}

1、简介:将参数转换为数值类型

2、语法:unaryPlus(value) ;+ value

3、操作对象:参数为布尔值或者字符串

4、返回类型:数值类型

5、操作样例:unaryPlus(true) == 1 ;+ "42" == 42

\section{not}\index{not}

1、简介:布尔运算非操作

2、语法:not(boolean);not boolean

3、操作对象:参数为布尔值

4、返回类型:布尔值类型

5、操作样例:not(false) == true ;not true == false

\section{and}\index{and}

1、简介:布尔运算与操作

2、语法:and(boolean, boolean) ;boolean and boolean

3、操作对象:两个参数均为布尔值

4、返回类型:布尔值类型

5、操作样例:and(true, true) == true ;true and false == false

\section{or}\index{or}

1、简介:布尔运算或操作

2、语法:or(boolean, boolean) ;boolean or boolean

3、操作对象:两个参数均为布尔值

4、返回类型:布尔值类型

5、操作样例:or(false, false) == false ;false or true == true

\section{equal}\index{equal}

1、简介:数学运算判断两个参数是否相等

2、语法:equal(value, value) ;value == value

3、操作对象:两个参数为相同类型(数值、布尔值、日期、文本)

4、返回类型:布尔值类型

5、操作样例:equal(false, not true) == true ;(3 * 5 == 15) == true

\section{unequal}\index{unequal}

1、简介:数学运算判断两个参数是否不相等

2、语法:unequal(value, value) ;value != value

3、操作对象:两个参数为相同类型(数值、布尔值、日期、文本)

4、返回类型:布尔值类型

5、操作样例:unequal(true, not false) == false ;(6 * 9 != 42) == false

\section{larger}\index{larger}

1、简介:数学运算判断第一个参数是否大于第二个参数

2、语法:larger(number, number) ;number > number (其它类型类似)

3、操作对象:两个参数为相同类型(数值、布尔值、日期、文本)

4、返回类型:布尔值类型

5、操作样例:larger(number, number) ;5 > 3 == true

\section{largerEq}\index{largerEq}

1、简介:数学运算判断第一个参数是否大于等于第二个参数

2、语法:largerEq(number, number) ;number >= number (其它类型类似)

3、操作对象:两个参数为相同类型(数值、布尔值、日期、文本)

4、返回类型:布尔值类型

5、操作样例:largerEq(date, date) ;5 >= 3 == true

\section{smaller}\index{smaller}

1、简介:数学运算判断第一个参数是否小于第二个参数

2、语法:smaller(number, number) ;number < number (其它类型类似)

3、操作对象:两个参数为相同类型(数值、布尔值、日期、文本)

4、返回类型:布尔值类型

5、操作样例:smaller(boolean, boolean) ;10 < 8 == false

\section{smallerEq}\index{smallerEq}

1、简介:数学运算判断第一个参数是否小于等于第二个参数

2、语法:smallerEq(number, number) ;number <= number(其它类型类似)

3、操作对象:两个参数为相同类型(数值、布尔值、日期、文本)

4、返回类型:布尔值类型

5、操作样例:smallerEq(text, text) ;10 <= 8 == false

\section{concat}\index{concat}

1、简介:连接多个字符串返回一个新的字符串(和add类似)

2、语法:concat(text...)

3、操作对象:参数均为字符串

4、返回类型:字符串类型

5、操作样例:concat("dog", "go") == "doggo" ;"dog" +"go" == "doggo"

\section{join}\index{join}

1、简介:将第一个字符串插入后边的各个参数之间返回一个新的字符串

2、语法:join(text...)

3、操作对象:参数均为字符串

4、返回类型:字符串类型

5、操作样例:join("-", "a", "b", "c") == "a-b-c"

\section{slice}\index{slice}

1、简介:提取一个字符串的子字符串

2、语法:slice(text, number) ;slice(text, number, number)

3、操作对象:第一个参数为字符串,第二个参数切分起始坐标,第三个参数是切分结束坐标(第三个参数可以为空)

4、返回类型:字符串类型

5、操作样例:slice("Hello world", 1, 5) == "ello" ;slice("notion", 3) == "ion"

\section{length}\index{length}

1、简介:获取一个字符串的长度

2、语法:length(text)

3、操作对象:参数为字符串

4、返回类型:数值类型

5、操作样例:length("Hello world") == 11

\section{format}\index{format}

1、简介:将对象转换为字符串

2、语法:format(value)

3、操作对象:参数为数值,时间,布尔值

4、返回类型:字符串类型

5、操作样例:format(42) == "42" ;format(true) == "true" ;

\section{toNumber}\index{toNumber}

1、简介:将对象转为数值

2、语法:toNumber(text) (其它类型类似)

3、操作对象:参数为字符串,数值,时间,布尔值

4、返回类型:数值类型

5、操作样例:toNumber("42") == 42

\section{contains}\index{contains}

1、简介:检查第一个字符串中是否包含第二个字符串

2、语法:contains(text, text)

3、操作对象:两个参数均为字符串

4、返回类型:布尔值类型

5、操作样例:contains("notion", "ion") == true

\section{replace}\index{replace}

1、简介:将第一个匹配到的内容(正则匹配)替换为一个新的值

2、语法:replace(text, text, text)

3、操作对象:所有参数均为字符串(官方给出第一个值可以是布尔类型或者数值类型,我不理解)

4、返回类型:字符串类型

5、操作样例:replace("1-2-3", "-", "!") == "1!2-3"

\section{replaceAll}\index{replaceAll}

1、简介:将所有匹配到的内容(正则匹配)替换为一个新的值

2、语法:replaceAll(text, text, text)

3、操作对象:所有参数均为字符串(官方给出第一个值可以是布尔类型或者数值类型,我不理解)

4、返回类型:字符串类型

5、操作样例:replaceAll("1-2-3", "-", "!") == "1!2!3"

\section{test}\index{test}

1、简介:检查一个字符串是否和正则表达式匹配(看官方样例好像不太对)

2、语法:test(text, text)

3、操作对象:所有参数均为字符串(官方给出第一个值可以是布尔类型或者数值类型,我不理解)

4、返回类型:布尔值类型

5、操作样例:test("1-2-3", "-") == true

\section{empty}\index{empty}

1、简介:检查内容是否为空

2、语法:empty(text) (其它类型类似)

3、操作对象:参数为字符串,数值,布尔值,日期

4、返回类型:布尔值类型

5、操作样例:empty("") == true

\section{abs}\index{abs}

1、简介:返回数值的绝对值

2、语法:abs(number)

3、操作对象:参数为数值

4、返回类型:数值类型

5、操作样例:abs(-3) == 3

\section{cbrt}\index{cbrt}

1、简介:返回数值的立方根

2、语法:cbrt(number)

3、操作对象:参数为数值

4、返回类型:数值类型

5、操作样例:cbrt(8) == 2

\section{exp}\index{exp}

1、简介:指数运算($e^{\text{参数}}$)

2、语法:exp(number)

3、操作对象:参数为数值

4、返回类型:数值类型

5、操作样例:exp(1) == 2.718281828459045

\section{floor}\index{floor}

1、简介:数值向下取整

2、语法:floor(number)

3、操作对象:参数为数值

4、返回类型:数值类型

5、操作样例:floor(2.8) == 2

\section{ceil}\index{ceil}

1、简介:数值向上取整

2、语法:ceil(number)

3、操作对象:参数为数值

4、返回类型:数值类型

5、操作样例:ceil(4.2) == 5

\section{ln}\index{ln}

1、简介:自然数e为底的对数运算

2、语法:ln(number)

3、操作对象:参数为数值

4、返回类型:数值类型

5、操作样例:ln(e) == 1

\section{log10}\index{log10}

1、简介:10为底的对数运算

2、语法:log10(number)

3、操作对象:参数为数值

4、返回类型:数值类型

5、操作样例:log10(1000) == 3

\section{log2}\index{log2}

1、简介:2为底的对数运算

2、语法:log2(number)

3、操作对象:参数为数值

4、返回类型:数值类型

5、操作样例:log2(64) == 6

\section{max}\index{max}

1、简介:返回数值中的最大值

2、语法:max(number...)

3、操作对象:所有参数均为数值类型

4、返回类型:数值类型

5、操作样例:max(5, 2, 9, 3) == 9

\section{min}\index{min}

1、简介:返回数值中的最小值

2、语法:min(number...)

3、操作对象:所有参数均为数值类型

4、返回类型:数值类型

5、操作样例:min(4, 1, 5, 3) == 1

\section{round}\index{round}

1、简介:返回数值的四舍五入值

2、语法:round(number)

3、操作对象:参数为数值

4、返回类型:数值类型

5、操作样例:round(4.4) == 4 ;round(4.5) == 5

\section{sign}\index{sign}

1、简介:返回数值的符号,1是正值,-1是负值,0是0

2、语法:sign(number)

3、操作对象:参数为数值

4、返回类型:数值类型

5、操作样例:sign(4) == 1

\section{sqrt}\index{sqrt}

1、简介:返回数值的开方

2、语法:sqrt(number)

3、操作对象:参数为数值

4、返回类型:数值类型

5、操作样例:sqrt(144) == 12

\section{start}\index{start}

1、简介:返回时间范围对象的开始日期

2、语法:start(date)

3、操作对象:参数为日期

4、返回类型:日期类型

5、操作样例:start(prop("Date")) == Feb 2, 1996

\section{end}\index{end}

1、简介:返回时间范围对象的结束日期

2、语法:end(date)

3、操作对象:参数为日期

4、返回类型:日期类型

5、操作样例:end(prop("Date")) == Feb 2, 1996

\section{now}\index{now}

1、简介:返回当前的日期(页面加载时才会改变)

2、语法:now()

3、操作对象:无参数

4、返回类型:日期类型

5、操作样例:now() == Feb 2, 1996

\section{timestamp}\index{timestamp}

1、简介:返回毫秒级的时间戳

2、语法:timestamp(date)

3、操作对象:参数为日期

4、返回类型:数值类型

5、操作样例:timestamp(now()) == 1512593154718

\section{fromTimestamp}\index{fromTimestamp}

1、简介:毫秒级的时间戳恢复为日期

2、语法:fromTimestamp(number)

3、操作对象:参数为数值

4、返回类型:日期类型

5、操作样例:fromTimestamp(2000000000000) == Tue May 17 2033

\section{dateAdd}\index{dateAdd}

1、简介:对指定的日期做加法操作,第三个参数是增加的粒度,包括:"years", "quarters", "months", "weeks", "days", "hours", "minutes", "seconds", 或者 "milliseconds"

2、语法:dateAdd(date, number, text)

3、操作对象:第一个参数是日期,第二个参数是大小,第三个参数是粒度

4、返回类型:日期类型

5、操作样例:dateAdd(date, amount, "years")

\section{dateSubtract}\index{dateSubtract}

1、简介:对指定的日期做减法操作,第三个参数是减少的粒度,包括:"years", "quarters", "months", "weeks", "days", "hours", "minutes", "seconds", 或者 "milliseconds"

2、语法:dateSubtract(date, number, text)

3、操作对象:第一个参数是日期,第二个参数是大小,第三个参数是粒度

4、返回类型:日期类型

5、操作样例:dateSubtract(date, amount, "years")

\section{dateBetween}\index{dateBetween}

1、简介:返回两个日期之间指定粒度的差值,第三个参数是粒度,包括:"years", "quarters", "months", "weeks", "days", "hours", "minutes", "seconds", 或者 "milliseconds"

2、语法:dateBetween(date, date, text)

3、操作对象:第一第二个参数是日期,第三个参数是粒度

4、返回类型:数值类型

5、操作样例:dateBetween(date, date2, "years")

\section{formatDate}\index{formatDate}

1、简介:按照指定的格式输出时间字符串

2、语法:formatDate(date, text)

3、操作对象:第一个参数是日期,第二个参数是时间字符串格式

4、返回类型:字符串类型

5、操作样例:formatDate(now(), "M/D/YY") == 3/30/10

\section{minute}\index{minute}

1、简介:返回指定日期的分钟值

2、语法:minute(date)

3、操作对象:参数是日期

4、返回类型:数值类型

5、操作样例:minute(now()) == 45

\section{hour}\index{hour}

1、简介:返回指定日期的小时值

2、语法:hour(date)

3、操作对象:参数是日期

4、返回类型:数值类型

5、操作样例:hour(now()) == 17

\section{day}\index{day}

1、简介:返回指定日期的星期值(0是周日,1是周一,2是周二,以此类推)

2、语法:day(date)

3、操作对象:参数是日期

4、返回类型:数值类型

5、操作样例:day(now()) == 3

\section{date}\index{date}

1、简介:返回指定日期的日值

2、语法:date(date)

3、操作对象:参数是日期

4、返回类型:数值类型

5、操作样例:date(now()) == 13

\section{month}\index{month}

1、简介:返回指定日期的月值

2、语法:month(date)

3、操作对象:参数是日期

4、返回类型:数值类型

5、操作样例:month(now()) == 11

\section{year}\index{year}

1、简介:返回指定日期的年值

2、语法:year(date)

3、操作对象:参数是日期

4、返回类型:数值类型

5、操作样例:year(now()) == 1984

\section{id}\index{id}

1、简介:返回一个唯一字符串ID

2、语法:id()

3、操作对象:无参数

4、返回类型:字符串类型

5、操作样例:id() == "c9696b5c72e9462684d0589ba6e4ced7"

